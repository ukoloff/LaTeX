\documentclass[]{interact}

\usepackage[caption=false]{subfig}% Support for small, `sub' figures and tables
%\usepackage[nolists,tablesfirst]{endfloat}% To `separate' figures and tables from text if required

%\usepackage[doublespacing]{setspace}% To produce a `double spaced' document if required
%\setlength\parindent{24pt}% To increase paragraph indentation when line spacing is doubled
%\setlength\bibindent{2em}% To increase hanging indent in bibliography when line spacing is doubled

\usepackage[numbers,sort&compress]{natbib}% Citation support using natbib.sty
\bibpunct[, ]{[}{]}{,}{n}{,}{,}% Citation support using natbib.sty
\renewcommand\bibfont{\fontsize{10}{12}\selectfont}% Bibliography support using natbib.sty

\usepackage{textcomp}

\begin{document}

\title{A novel algorithm for construction of the shortest path
between a finite set of nonintersecting contours on the plane}

\author{
\name{
  A.~A. Petunin\textsuperscript{a}\textsuperscript{b}
  and
  E.~G. Polishchuk\textsuperscript{a}
  and
  S.~S. Ukolov\textsuperscript{a}\thanks{CONTACT S.~S. Ukolov. Email: s.s.ukolov@urfu.ru}}
\affil{
  \textsuperscript{a}Ural Federal University, Yekaterinburg, Russia
  \textsuperscript{b}Institute of Mathematics and Mechanics, UBr RAS, Yekaterinburg, Russia
}}

\maketitle

\begin{abstract}
The article discusses one of the optimization problems
that arise when modeling the tool path for CNC sheet metal
cutting machines for the case when the boundary contours
of the parts are defined by polygons and the pierce points
are located on the boundary contours.
Only closed-loop cutting is used,
i.e. Continuous Cutting Problem (CCP),
hence
the task of minimizing the length of the tool path
is reduced to the problem of finding the minimum air move length,
which is shown to be equivalent to the problem of finding
the shortest polyline with vertices on disjoint contours in the plane,
while these contours do not contain internal contours.
An algorithm for constructing minimal length broken line
for a fixed order of cutting contours is described
and is proved to provide local minimum.
Sufficient conditions for this minimum to be global are given.
Heuristic algorithm is suggested for finding the optimal order for cutting contours.
The results of a computational experiment
and a comparison of the results with an exact solution for the GTSP problem are presented.

\end{abstract}

\begin{keywords}
  Continuous optimization;
  Discrete optimization;
  Fermat principle;
  Variable Neighborhood Search
\end{keywords}

\section{Introduction}

A number of optimization problems arise
during development of control programs for CNC sheet cutting machines.
One of them is
the task of minimizing the tool air move,
which in some special cases can be reduced
to the problem of finding the shortest polyline
with vertices on flat contours.
Contours are interpreted as the boundaries of flat parts.
The location of the contours on the plane is determined
during the solution of the ``nesting'' problem.
Both tasks are generally NP-hard.

In its turn,
the task of minimizing tool air move
is a subtask of another optimization problem --
the task of optimizing the tool path when cutting flat parts.
Its exact solution cannot be obtained for problems
that actually arise in production
(for hundreds of parts / contours) in a reasonable time,
therefore,
various heuristics are typically applied
to get solutions of acceptable quality.
At the same time, the issues of developing algorithms
that provide optimal solutions for some problem cases,
as well as evaluating the quality of their solutions
in comparison with the optimal solution,
remain unresolved and are of significant scientific interest.

The general problem of optimizing the tool path
when cutting 2D objects on CNC machines,
which consists in minimizing cutting time and cost,
includes a whole range of different optimization tasks.
A classification of such problems
can be found
in \cite{bi01,bi02,bi03}.

\section{Conclusion}

\section*{Acknowledgements}

The work was supported by
Act 211 Government of the Russian Federation,
contract \textnumero 02.A03.21.0006,
and by Ministry of Education and Science
of the Russian Federation,
project \textnumero 2.9563.2017/8.9

\bibliographystyle{tfs}
\bibliography{ccp}
\nocite{*}

\end{document}
