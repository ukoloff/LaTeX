\documentclass[14pt]{extarticle}

\usepackage{geometry}

\geometry{
    a4paper,
    top=2cm,
    bottom=2cm,
    left=2.5cm,
    right=1cm,
    nomarginpar,
    % showframe
}

\usepackage[utf8]{inputenc}
\usepackage[T2A]{fontenc}
\usepackage[english,russian]{babel}

\usepackage{amssymb,amsmath}
\usepackage{graphicx}
\usepackage{tikz}
\usetikzlibrary{patterns,intersections}
\usepackage{theoremref}

\graphicspath{{media/}}

\usepackage{amsthm}
\theoremstyle{plain}% Theorem-like structures provided by amsthm.sty
\newtheorem{theorem}{Theorem}[section]
\newtheorem{lemma}[theorem]{Lemma}
\newtheorem{corollary}[theorem]{Corollary}
\newtheorem{proposition}[theorem]{Proposition}

\theoremstyle{definition}
\newtheorem{definition}[theorem]{Definition}
\newtheorem{example}[theorem]{Example}

\theoremstyle{remark}
\newtheorem{remark}{Remark}
\newtheorem{notation}{Notation}
\newtheorem{condition}{Condition}

\usepackage{csquotes}

\usepackage[%
  parentracker=true,
  style=gost-numeric,
  defernumbers=true,
  % sorting=none,
]{biblatex}

\toggletrue{bbx:gostbibliography}

\addbibresource{ccp.bib}

\begin{document}

\thispagestyle{empty}
\begin{center}
{\small
Министерство науки и высшего образования Российской Федерации

Федеральное государственное автономное образовательное учреждение \\
высшего образования

\textbf{<<Уральский федеральный университет\\
имени первого Президента России Б.Н. Ельцина>>}
}

\vspace{0pt plus2fill}
\begin{flushright}
\begin{minipage}{0.5\linewidth}
  \centering
  \textit{На правах рукописи}

  \small {(подпись)}
\end{minipage}
\end{flushright}


\vspace{0pt plus4fill}
Уколов Станислав Сергеевич

\vspace{0pt plus1fill}
\textbf{
Эвристический алгоритм решения задачи непрерывной резки
}

\vspace{0pt plus1fill}
НАУЧНЫЙ ДОКЛАД

по результатам научно-квалификационной работы (диссертации)

\vspace{0pt plus2fill}
\begin{tabular}{l p{10cm}}
  Направление подготовки: &
  09.06.01
  "---
  <<Информатика и вычислительная техника>>
  \\
  Направленность: &
  05.13.12
  "---
  <<Системы автоматизации проектирования (в промышленности)>>
\end{tabular}

\vspace{0pt plus4fill}
\begin{flushright}
Научный руководитель:
профессор,
д.т.н.
\\
Петунин Александр Александрович
\end{flushright}

\vspace{0pt plus6fill}
Екатеринбург
\\
2020
\end{center}
\newpage

\tableofcontents
\newpage

\section{Превед}

См.
\cite{berlin2019}
и
\cite{bi07},
а также
\cite{Miskolc,Sozopol,Obuhovo}.

\printbibliography[heading=bibintoc]%[env=gostbibliography]
\nocite{*}

\end{document}
