\documentclass[14pt]{extarticle}

\usepackage{geometry}

\geometry{
    a4paper,
    top=2cm,
    bottom=2cm,
    left=2.5cm,
    right=1cm,
    nomarginpar,
    % showframe
}

\usepackage[utf8]{inputenc}
\usepackage[T2A]{fontenc}
\usepackage[english,russian]{babel}
\usepackage{indentfirst}

\usepackage{amssymb,amsmath}
\usepackage{graphicx}
\usepackage{tikz}
\usetikzlibrary{patterns,intersections}
\usepackage{theoremref}

\graphicspath{{media/}}

\usepackage{amsthm}
\theoremstyle{plain}% Theorem-like structures provided by amsthm.sty
\newtheorem{theorem}{Theorem}[section]
\newtheorem{lemma}[theorem]{Lemma}
\newtheorem{corollary}[theorem]{Corollary}
\newtheorem{proposition}[theorem]{Proposition}

\theoremstyle{definition}
\newtheorem{definition}[theorem]{Definition}
\newtheorem{example}[theorem]{Example}

\theoremstyle{remark}
\newtheorem{remark}{Remark}
\newtheorem{notation}{Notation}
\newtheorem{condition}{Condition}

\usepackage{csquotes}

\usepackage[%
  parentracker=true,
  style=gost-numeric,
  defernumbers=true,
  % sorting=none,
]{biblatex}

\toggletrue{bbx:gostbibliography}

\addbibresource{ccp.bib}

\begin{document}

\thispagestyle{empty}
\begin{center}
{\small
Министерство науки и высшего образования Российской Федерации

Федеральное государственное автономное образовательное учреждение \\
высшего образования

\textbf{<<Уральский федеральный университет\\
имени первого Президента России Б.Н. Ельцина>>}
}

\vspace{0pt plus2fill}
\begin{flushright}
\begin{minipage}{0.5\linewidth}
  \centering
  \textit{На правах рукописи}

  \small {(подпись)}
\end{minipage}
\end{flushright}


\vspace{0pt plus4fill}
Уколов Станислав Сергеевич

\vspace{0pt plus1fill}
\textbf{
Эвристический алгоритм решения задачи непрерывной резки
}

\vspace{0pt plus1fill}
НАУЧНЫЙ ДОКЛАД

по результатам научно-квалификационной работы (диссертации)

\vspace{0pt plus2fill}
\begin{tabular}{l p{10cm}}
  Направление подготовки: &
  09.06.01
  "---
  <<Информатика и вычислительная техника>>
  \\
  Направленность: &
  05.13.12
  "---
  <<Системы автоматизации проектирования (в промышленности)>>
\end{tabular}

\vspace{0pt plus4fill}
\begin{flushright}
Научный руководитель:
профессор,
д.т.н.
\\
Петунин Александр Александрович
\end{flushright}

\vspace{0pt plus6fill}
Екатеринбург
\\
2020
\end{center}
\newpage

\tableofcontents
\newpage

\section{Введение}

См.
\cite{berlin2019}
и
\cite{bi07},
а также
\cite{Miskolc,Sozopol,Obuhovo}.

\section{Задача непрерывной резки -- Continuous Cutting Problem}

Рассмотрим Эквлидову плоскость
$\mathbb R ^ 2$
и на ней фигуру
$B$
(в большинстве практических случаев -- прямоугольник),
ограниченную замкнутым контуром.
Это -- модель листового материала,
подлежащего резке.
Пусть
$N$
попаркно непересекающихся плоских контуров
$\{C_1, C_2, ... C_N\}$
расположены внутри
$B$,
ограничивая
$n$
деталей
$\{A_1, A_2 ... A_n\}$.
Деталь может быть ограничена
одним или несколькими контурами
(одним внешним и несколькими отверстиями),
так что в общем случае
$n \leqslant N$.

Контуры
$C_i$
могут быть произвольной формы,
но мы будем рассматривать только
состоящие из
(конечного числа)
отрезков прямых линий и дуг окружностей,
так как именно такие геометрические примитивы
поддерживаются программным обеспечением
современных машин термической резки с ЧПУ.
Частный случай,
когда контура состоят только
из отрезков прямых,
сводится к одному из вариантов
задачи обхода прямоугольников
(Touring Polygon Problem, TPP),
см.
\cite{bi13}.

Далее,
внутри
$B$
(как правило, на границе)
выберем две точки и обозначим их
$M_0$, $M_{N + 1}$
(почти всегда $M_0 = M_{N + 1}$),
которые будут использоваться
как начало и конец
маршрута резки.

Задача непрерывной резки
(Continuous Cutting Problem, CCP)
состоит в поиске:
\begin{enumerate}
\item
$N$ штук точек врезки $M_i \in C_i, i \in \overline{1, N}$
\item
Последовательности обхода контуров
$C_i$,
то есть перестановки
$N$
элементов
$I = (i_1, i_2, ... i_N)$
\end{enumerate}

Результатом решения задачи будет являться маршрут
$\{M_0, M_{i_1}, M_{i_2}, \dots M_{i_N}, M_{N + 1}\}$.
Целевая функция в данном случае сильно упрощается
по сравнению с общей задачей маршрутизации резки
и сводится фактически к минимизации длины холостого хода:

\begin{equation}
  \mathcal{L} = \sum_{j=0}^N|M_{i_j}M_{i_{j+1}}|
  \label{air-move-length}
\end{equation}
$$
\mathcal{L} \to \min
$$
где, для простоты записи мы полагаем
$M_{i_0} = M_0$,
$M_{i_{N + 1}} = M_{N + 1}$.

Кроме того,
мы наложим на решение дополнительное ограничение,
известное как
<<ограничение предшествования>>
(``precedence constraint'').
Хотя контуры
$C_i$
по условию не пересекаются,
они могут быть вложены друг в друга:
\( \tilde C_a \subset \tilde C_b \),
где
$\tilde C_a$
обозначает 2-мерную фигуру,
ограниченную контуром
$C_a$
(в более традиционных обозначениях
$C_a = \partial \tilde C_a$).
В общей задаче маршрутизации
режущего инструмента это
соответствует двум разным случаям
(наличие отверстий в деталях с одной стороны
и размещение меньших деталей в отверстиях больших),
но в нашем случае оба этих
варианта обрабатываются одинаково.

Если один контур расположен внутри другого,
то внутренний должен быть вырезан
(посещён)
ранее, чем внешний:
\( \tilde C_a \subset \tilde C_b \Rightarrow i_a < i_b \),
в перестановке
$I = (i_1, i_2, ... i_N)$.
Таким образом,
множество допустимых перестановок ограничено.

\section{Алгоритм решения задачи непрерывной резки}

Предлагаемый алгоритм решения задачи непрерывной резки
(см. \cite{berlin2019})
состоит из нескольких шагов,
что хорошо соответствует самой природе
решаемой задачи.

\subsection{Удаление <<внешних>> контуров}


\subsection{Непрерывная оптимизация}

\subsection{Дискретная оптимизация}

\subsection{Восстановление удалённых контуров}

\section{Условия оптимальности решения задачи непрерывной оптимизации}

\section{Численные эксперименты}

\section{Заключение}

\printbibliography[heading=bibintoc]
% \nocite{*}

\end{document}
