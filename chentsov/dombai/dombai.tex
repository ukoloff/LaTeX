\documentclass[10pt]{SPIIRAS_Proceedings}

\udk{000.00}


\titleRus{
  Маршрутные процессы в задачах последовательного обхода множеств при наличии ограничений
}

\authorsRus{
}

\authorsTitleRus{
  А.А. П\smallcapsfake{етунин},
  А.Г. Ч\smallcapsfake{енцов},
  П.А. Ч\smallcapsfake{енцов}
} % \smallcapfake необходим для имитации малых прописных букв ввиду отсутствия поддержки в используемом шрифте

\abstractRus{
}

\keywordsRus{
}


\begin{document}

\maketitle

\normalsize

\section*{Введение}

Объектом исследования в статье
являются задачи маршрутизации перемещений
с ограничениями различных типов;
среди последних особо выделяем условия
предшествования и ограничения динамического характера,
возникающие по мере развития процесса и проведения тех или иных работ.
При должной формализации возникает постановка,
идейно близкая к дискретным задачам управления большой размерности
(имеется в виду дискретность и по времени и  по фазовому состоянию).
Оптимизируется комплекс, включающий точку старта,
вариант очередности исполнения заданий
(далее – маршрут)
и конкретную траекторию;
сам возникающий при этом комплекс (триплет)
именуем маршрутным процессом.
Возможные применения могут быть,
в частности, связаны с атомной энергетикой
(см. \cite{1,2,3};
задача минимизации дозовой нагрузки работников при демонтаже радиационно опасных объектов)
и машиностроением (см. \cite{4,5,6};
задача управления инструментом при фигурной листовой резке на машинах с ЧПУ);
имеются и другие приложения.
В настоящей статье ориентируемся на применение разрабатываемых методов в машиностроении;
следуем при этом монографии \cite{4}.
Здесь первоначальная задача управления режущим инструментом
с условиями предшествования и динамическими ограничениями
преобразуется к строгой математической постановке
оптимизационной задачи в классе вышеупомянутых маршрутных процессов,
в которой нашей целью является нахождение
глобального экстремума и соответствующего оптимального решения.
Кратко излагаются элементы общей теории и
конструируемый на ее основе оптимальный алгоритм,
реализованный на многоядерной ПЭВМ.
Используются понятия и обозначения
\cite[часть II]{4},
относящиеся к математической постановке,
а также содержательные построения \cite[часть I]{4}.

\section{Сводка общих понятий}

Используем стандартную теоретико множественную символику
(кванторы, связки и др.);
через $\varnothing$ обозначаем пустое множество,
$\stackrel{\triangle}{=}$ -- равенство по определению.
Множество, все элементы которого сами являются множествами,
называем семейством.
Если $x$ и $y$ -- объекты,
то $\{x;y\}$
есть их неупорядоченная пара:
$\{x;y\}$ содержит $x,\;y$
и не содержит никаких других элементов.
Для всякого объекта $z$ в виде $\{z\} \stackrel{\triangle}{=} \{z;z\}$
имеем синглетон,
содержащий $z$.
Множества являются объектами.
Если $a$ и $b$ -- объекты, то
\cite[c.~67]{15}
$(a,b) \stackrel{\triangle}{=} \{\{a\};\{a;b\}\}$
есть упорядоченная пара с первым элементом $a$ и вторым элементом $b$.
Для каждой упорядоченной пары $h$ через
$\mathrm{pr}_1(h)$ и $\mathrm{pr}_2(h)$
обозначаем первый и второй элементы $h$,
однозначно определяемые условием
$h = (\mathrm{pr} _1(h),\mathrm{pr} _1(h))$.
Если же $x,\;y$ и $z$ -- объекты,
то $(x,y,z) \stackrel{\triangle}{=} ((x,y),z)$
есть их упорядоченный триплет.
Соответственно,
$A \times B \times C = (A \times B) \times C$
для любых трех множеств $A,\;B$ и $C;$ см.
\cite[c.17]{16}.

Множеству $H$
сопоставляется семейство $\mathcal{P}(H)$
всех подмножеств (п/м) $H$
и $\mathcal{P}'(H) \stackrel{\triangle}{=}
\mathcal{P}(H) \setminus \{\varnothing\}$ -- семейство всех непустых п/м $H;$
через $\mathrm{Fin}(H)$
обозначаем семейство всех непустых конечных п/м
$H,\;\mathrm{Fin}(H) \subset \mathcal{P}'(H).$
Для непустого конечного множества $H$ имеем равенство
$\mathrm{Fin}(H) = \mathcal{P}'(H).$
Если $A$ и $B$ -- непустые множества,
$f$ -- отображение (функция) из $A$ в $B,$
а $C \in \mathcal{P}(A),$ то
$f^1(C) \stackrel{\triangle}{=} \{f(x):\;x \in C\} \in \mathcal{P}(B)$
есть образ $C$ при действии $f.$

Как обычно,
$\mathbb{R}$ -- вещественная прямая,
$\mathbb{R}_+ \stackrel{\triangle}{=} \{\xi \in \mathbb{R} \vert 0 \le \xi\} = [0,\infty[,\;\mathbb{N} \stackrel{\triangle}{=} \{1;2;...\}$
и $\mathbb{N}_0 \stackrel{\triangle}{=} \{0\} \cup \mathbb{N} = \{0;1;2;...\};$
при $p \in \mathbb{N}_0$ и $q \in \mathbb{N}_0$
$$
\overline{p,q} = \{\;k \in \mathbb{N}_0 \vert (p \le k) \& (k \in q)\}
$$
(при $q < p$ имеем $\overline{p,q} = \varnothing$).
Если $S$ -- непустое множество, то
$\mathcal{R}_+[S]$
есть множество всех неотрицательных вещественнозначных (в/з) функций на $S.$
Каждому непустому конечному множеству $K$
сопоставляем его мощность $|K| \in \mathbb{N}$
и непустое множество $(\mathrm{bi})[K]$
всех биекций
\cite[c.87]{17} <<промежутка>>
$\overline{1,|K|}$ на $K;$
пусть
$|\varnothing| \stackrel{\triangle}{=} 0.$
Ясно, что при
$m \in \mathbb{N}$ в виде $(\mathrm{bi})[\overline{1,m}]$
имеем множество всех перестановок
\cite[c.87]{17} множества
$\overline{1,m};$
если $\alpha \in (\mathrm{bi})[\overline{1,m}],$
то определена перестановка
$\alpha^{-1} \in (\mathrm{bi})[\overline{1,m}],$
обратная к
$\alpha:\;\alpha(\alpha^{-1}(k)) = \alpha^{-1}(\alpha(k)) = k$
при $k \in \overline{1,m}.$
Напомним, что здесь и ниже символика соответствует
\cite[$\S$3.1]{4}.

\end{document}
