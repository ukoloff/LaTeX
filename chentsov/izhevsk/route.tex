%%% Macro style File of Vestnik. Don't change!
\documentclass[12pt,a4paper,twoside]{article}  %MikTeX
% \usepackage[cp1251]{inputenc}                  %Win
\usepackage[utf8]{inputenc}                  % Unicode!!!
\usepackage[T1,T2A]{fontenc}
\usepackage{amsmath,amsfonts,amssymb,amscd,euscript}
\usepackage[russian, english]{babel}

\usepackage{graphicx}
\usepackage{psfrag}
\usepackage{cite}
\usepackage{caption}
\captionsetup[figure]{labelfont={bf},labelformat={default},labelsep=period,name={Fig.}}
\captionsetup[table]{labelfont={bf},labelformat={default},labelsep=period,name={Table}}

\usepackage[textwidth=166mm,top=2cm,textheight=250mm,left=20mm,showframe=false]{geometry}
\tolerance=500
\voffset=0.5cm
%\headsep=5mm
%\textwidth=166mm
%\textheight=250mm
%\oddsidemargin=-4mm
%\evensidemargin=-4mm
%\topmargin=-7mm

\usepackage{ifthen}
\newcommand{\seriesrus}{\ifthenelse{\equal{MATHEMATICS}{\serieseng}}{МАТЕМАТИКА}{\ifthenelse{\equal{MECHANICS}{\serieseng}}{МЕХАНИКА}{\ifthenelse{\equal{COMPUTER SCIENCE}{\serieseng}}{КОМПЬЮТЕРНЫЕ НАУКИ}{UNDEFINED}}}}

\makeatletter

\global\let\@fundingrus\@empty
\def\fundingrus#1{\gdef\@fundingrus{\@ifempty{#1}{\@empty}{\noindent{\bf Финансирование.} #1}}}

\global\let\@fundingeng\@empty
\def\fundingeng#1{\gdef\@fundingeng{\@ifempty{#1}{\@empty}{\noindent{\bf Funding.} #1}}}

%information about the article on english
\def\titleeng{\begin{otherlanguage}{english}\thispagestyle{firstpagestyleeng}\label{paperfirstpage}
\hbox{MSC2020: \MSC}
\vspace{30pt plus 6pt}
\begin{flushleft}
{\bf\copyright~{\textit{\authorseng}}\\[2ex]
{\MakeUppercase{\articletitleeng}}}
\end{flushleft}
\end{otherlanguage}}

\def\annotationandkeywordseng{\begin{otherlanguage}{english}\noindent {\small \referateng \par } \vspace{8pt}
\noindent {\small {\it Keywords}: \keywordseng} \par%
\vspace{8pt}
\noindent {\small {\textrm DOI:} \paperdoi} \par%
\vspace{10pt plus 6pt minus 1pt}\end{otherlanguage}}

%information about the article on russian
\def\titlerus{\clearpage
\thispagestyle{firstpagestylerus}
\begin{otherlanguage}{russian}\noindent\parbox{\textwidth}{\small\noindent\textbf{\textit{\authorsrus}}}\par
\vspace{2pt plus 1pt minus 0pt}\par
\noindent\parbox{\textwidth}{\small\noindent\textbf{\articletitlerus}}\par
\vspace{6pt plus 1pt minus 0pt}\par
\noindent\parbox{\textwidth}{\small\noindent{\it Ключевые слова: }\keywordsrus} \par
\vspace{8pt}
\noindent{\small{УДК} \UDC}\par%
\vspace{7pt}
\noindent{\small{\textrm DOI:} \paperdoi}
\end{otherlanguage}}

\def\referrus{\begin{otherlanguage}{russian}\vspace{20pt}  \par \small  \noindent \referatrus\par\vspace{3ex}\@fundingrus\end{otherlanguage}}

%Received
\def\receivedeng{\begin{otherlanguage}{english}\vspace{3ex} \hfill Received~~\datereceive \par \vspace{5ex} \par\end{otherlanguage}}

\def\receivedrus{\begin{otherlanguage}{russian}\vspace{3ex} \hfill Поступила  в редакцию~~\datereceive \par \vspace{5ex} \par\end{otherlanguage}}

%contact information about authors
\def\contactseng{\begin{otherlanguage}{english}\noindent\parbox{\textwidth}{\small\noindent \contactinformationeng\par\vspace{20pt}\par
\noindent{\bf Citation: }\authorseng.~\articletitleeng, {\it Vestnik Udmurtskogo Universiteta. Matematika. Mekhanika. Komp'yuternye Nauki}, \paperyear, vol.~\papervolume, issue~\papernumber, \mbox{pp.~\pageref{paperfirstpage}--\pageref{paperlastpage}.}}\end{otherlanguage}}

\def\contactsrus{%
\begin{otherlanguage}{russian}\noindent\parbox{\textwidth}{\small\noindent\contactinformationrus\par\vspace{10pt}\par
\noindent{\bf Цитирование: }\authorsrus.~\articletitlerus~// Вестник Удмуртского университета. Математика. Механика. Компьютерные науки. \paperyear. Т.~\papervolume. Вып.~\papernumber. \mbox{С.~\pageref{paperfirstpage}--\pageref{paperlastpage}.}\label{paperlastpage}}\end{otherlanguage}
}

\@addtoreset{equation}{section}
\renewcommand{\section}{\@startsection{section}{1}{0pt}{1.3ex
plus 1ex minus .1ex}{1.3ex plus .1ex}{\bf\,\S\,}}
\newcommand{\point}{\hspace*{-4mm}{\bf.}\;}
\newcommand{\sect}[1]{\begin{flushleft}%
\protect{\section{\point#1}}\end{flushleft}}

\renewcommand{\@begintheorem}[2]{\begin{trivlist}
\item[\hspace{\labelsep}{\bf \mbox{~~~}#1\ #2.}]}
\renewcommand{\@opargbegintheorem}[3]{\begin{trivlist}
\item[\hspace{\labelsep}{\bf \mbox{~~~}#1\ #2 {\rm (#3).}}]}
\renewcommand{\@endtheorem}{\end{trivlist}}

\newtheorem{theo}{Theorem}
\newtheorem{lem}{Lemma}
\newtheorem{prop}{Proposition}
\newtheorem{assert}{Assertion}
\newtheorem{cor}{Corollary}
\newtheorem{hyp}{Hypothesis}
\newtheorem{df}{Definition}
\newtheorem{rem}{Remark}
\newtheorem{examp}{Example}
\newtheorem{assum}{Assumption}
\newtheorem{cond}{Condition}

\newcommand{\doc}{\mbox{P r o o f}}
\headsep=5mm

\renewcommand{\@evenfoot}{}
\renewcommand{\@oddfoot}{}

\let\OLDthebibliography\thebibliography
\renewcommand\thebibliography[1]{
  \OLDthebibliography{#1}
  \setlength{\parskip}{0pt}
  \setlength{\itemsep}{0pt plus 0.3ex}
}
\renewcommand*{\@biblabel}[1]{#1.\hfill}

\newcommand{\sectnn}[1]{%
\begin{center}%
\large\section*{#1}%
\end{center}%
}

\renewcommand{\@evenhead}%
{%
\begin{otherlanguage}{english}%
\raisebox{0pt}%
[\headheight]%
[0pt]%
{%
\vbox{\hbox to\textwidth{\thepage\strut\hfil\authorseng\hfil}\hrule\vspace{8pt}
}}%
\end{otherlanguage}%
}

\renewcommand{\@oddhead}%
{%
\begin{otherlanguage}{english}%
\raisebox{0pt}%
[\headheight]%
[0pt]%
{%
\vbox{\hbox to\textwidth{\strut\hfil\articleshorttitleeng\hfil\thepage}\hrule\vspace{8pt}% \hbox to \textwidth{\series \hfil  \issue}
}}\end{otherlanguage}}


\makeatother

\usepackage{fancyhdr}

\fancypagestyle{firstpagestylerus}{
\fancyhf{}
\headheight=30pt
\fancyhead[C]{\parbox{\textwidth}{{\scriptsize ВЕСТНИК\hfill УДМУРТСКОГО\hfill УНИВЕРСИТЕТА.\hfill МАТЕМАТИКА.\hfill МЕХАНИКА.\hfill КОМПЬЮТЕРНЫЕ\hfill НАУКИ}
\vspace{2pt}
\hrule
\vspace{8pt}
{\small \hbox to \textwidth{\seriesrus \hfill  \paperyear. Т.~\papervolume. Вып.~\papernumber. С.~\pageref{paperfirstpage}--\pageref{paperlastpage}.}}}}
\renewcommand{\headrulewidth}{0.0pt}
}

\fancypagestyle{basestylerus}{
\fancyhf{}
\headheight=15pt
\fancyhead[LE,RO]{\thepage}
\fancyhead[CE]{\articleshorttitlerus}
\fancyhead[CO]{\authorsrus}
\renewcommand{\headrulewidth}{0.4pt}
}

\fancypagestyle{firstpagestyleeng}{
\fancyhf{}
\headheight=30pt
\fancyhead[C]{\parbox{\textwidth}{{\scriptsize VESTNIK \hfill UDMURTSKOGO \hfill UNIVERSITETA. \hfill MATEMATIKA. \hfill MEKHANIKA. \hfill KOMP'UTERNYE \hfill NAUKI}\vspace{2pt}
\hrule
{\small \vspace{8pt} \hbox to \textwidth{\serieseng \hfill  \paperyear. Vol.~\papervolume. Issue~\papernumber. Pp.~\pageref{paperfirstpage}--\pageref{paperlastpage}.}}}}
\renewcommand{\headrulewidth}{0pt}
}

\fancypagestyle{basestyleeng}{
\fancyhf{}
\headheight=15pt
\fancyhead[LE,RO]{\thepage}
\fancyhead[CE]{\articleshorttitleeng}
\fancyhead[CO]{\authorseng}
\renewcommand{\headrulewidth}{0.4pt}
}

\pagestyle{basestyleeng}
 %%% Macro style file. Don't Change!!!
\inputencoding{utf8}

\newcommand{\serieseng}{MATHEMATICS}

\newcommand{\paperyear}{2022}  %%% Year
\newcommand{\papervolume}{?} %%% Volume
\newcommand{\papernumber}{?}  %%% Issue

%%% Names of authors
\newcommand{\authorseng}{A.~Petunin, A.~Chentsov, P.~Chentsov} %%% English
\newcommand{\authorsrus}{А.\,А.~Петунин, А.\,Г.~Ченцов, П.\,А.~Ченцов} %%% Russian


%%% Full English title of article
\newcommand{\articletitleeng}{%
Some applications of optimization routing problems with additional constraints%
}
%%% Short English title of article
\newcommand{\articleshorttitleeng}{Routing optimization with constraints}
%%% Full Russian title of article
\newcommand{\articletitlerus}{%
Некоторые приложения задач оптимизации маршрутизации с дополнительными ограничениями%
}
%%% Short Russian title of article
\newcommand{\articleshorttitlerus}{Приложения задач оптимизации маршрутизации}

%%%  Mathematical Subject Classification (no more than 3) UDC Classification.
\newcommand{\UDC}{517.977} %%% Specified by authors
\newcommand{\MSC}{34D08, 93C15} %%%  Specified by authors

\newcommand{\paperdoi}{10.35634/vmXXXXXX}

\fundingeng{%
This research was funded by the Russian Foundation for Basic Research, grant no. 20-08-00873.%
} %Funding English

\fundingrus{%
Работа выполнена при финансовой поддержке Российского фонда фундаментальных исследований, грант №. 20-08-00873.%
} %Funding Russian

%%% Russian abstract
%%% Don't use references to the literature in the abstract.
\newcommand{\referatrus}{%
В статье рассматривается экстремальная задача маршрутизации с ограничениями.
В общей формулировке
предполагается, что объектами посещения являются любые непустые конечные множества - мегаполисы.
Основной прикладной задачей, рассматриваемой в данном исследовании,
является задача траектории движения инструмента для станков листовой резки с ЧПУ,
известная как задача резки.
Эта проблема возникает на этапе разработки управляющих программ для станков с ЧПУ.
Возможны и другие приложения.
В частности, результаты исследования могут быть использованы в задаче
минимизация дозы облучения при демонтаже системы радиационно-опасных элементов
после аварий на АЭС и в транспортных проблемах.
В качестве ограничений исследуются ограничения предшествования.
Они могут быть использованы для уменьшения вычислительной сложности.
В качестве основного метода исследования использовалось широко понимаемое динамическое программирование.
Предлагаемая реализация метода учитывает ограничения предшествования и
зависимость целевых функций от списка задач.
Последняя относится к классу очень сложных состояний
которые определяют допустимость маршрута на каждом шаге маршрутизации,
в зависимости от уже выполненных или, наоборот,
еще не завершенных задач.
Применительно к задаче резки
зависимость целевой функции от списка задач позволяет
уменьшать термические деформации материала при резке.
В работе ​​математическая формализация
экстремальной задачи маршрутизации с дополнительными ограничениями,
описание метода и полученный с его помощью точный алгоритм.
Оптимизации подлежат
порядок выполнения задач, конкретная траектория процесса,
и его начальная точка.
}
%%% English abstract
%%% not exceeding 250 words
\newcommand{\referateng}{%
The paper deals with an extremal routing problem with constraints.
In the general formulation,
it is assumed that the objects of visiting are any non-empty finite sets -- megalopolises.
The main applied problem considered in this study is the tool path problem for CNC sheet-cutting machines,
known as the Cutting Path Problem.
This problem arises at the stage of developing control programs for CNC machines.
Other applications are also possible.
In particular, the results obtained in the chapter can be used in the problem of
minimizing the radiation dose when dismantling a system of radiation-hazardous elements
after accidents at nuclear power plants and in transport problems.
Among tasks constraints, the precedence constraints are investigated.
These constraints can be used to reduce computational complexity.
As the main method, the study used widely understood dynamic programming.
The offered realization of the method takes into account the precedence constraints
and the dependence of the objective functions on the task list.
This dependence belongs to the class of very complex conditions
that determine the route admissibility at each routing step,
depending on the tasks already completed or, on the contrary,
not yet completed.
As applied to the Cutting Path Problem,
the dependence of the objective function on the task list makes it possible
to reduce thermal deformations of the material during cutting.
The chapter provides a mathematical formalization of an
extremal routing problem with additional constraints,
a description of the method, and the exact algorithm obtained with its help.
The order of tasks execution, the specific trajectory of the process,
and the starting point are optimized.
}
%%% Keywords (no more than 10 words).
\newcommand{\keywordseng}{%
dynamic programming,
additional constraints,
megalopolises,
routing
CNC sheet cutting machines
tool path optimization problem
}
\newcommand{\keywordsrus}{%
динамическое программирование,
дополнительные ограничения,
мегаполисы,
маршрутизация
Станки листовой резки с ЧПУ
проблема оптимизации пути инструмента
}

%%% Own notation
\newcommand{\pX}{{\mathcal X}}
\let\msf=\mathsf
\newcommand{\var}{\mathop{\sf Var}}
%---------------------------------------
\setcounter{page}{1} %
%---------------------------------------

\begin{document}

\titleeng
\annotationandkeywordseng

\begin{flushleft}
  {\bf{Introduction}}
\end{flushleft}

The study considers the problem of movement routing
with various constraints.
Among the latter, we highlight
precedence constraints,
as well as those of a dynamic nature,
arising during the process when certain tasks are performed.
With proper formalization, a problem statement arises,
conceptually close to discrete control problems of large dimension
(meaning discreteness both in time and in phase state).
Several objects are optimized,
including the starting point,
order of task execution
(hereinafter referred to as the route),
and a specific trajectory;
we call this triplet
the routing process.
This approach can be applied
to the task of minimizing the radiation dose
when dismantling radiation-hazardous facilities
(see~\cite {1,3})
and the tool path routing problem
for CNC sheet-cutting machines
in mechanical engineering (see \cite {4,5});
other applications also exist.
We focus on the application of the developed methods in mechanical engineering
in this article,
following the monograph~\cite{4}.
The initial task of controlling the cutting tool
with precedence and dynamic constraints
is converted to a strict mathematical statement of
the optimization problem
in the class of the aforementioned routing processes.
The goal is to find
the global extremum and the corresponding optimal solution.
The elements of the general theory and
the optimal algorithm constructed on its basis,
implemented on a multi-core PC,
are explained.
The method used is based on
broadly understood dynamic programming
(DP)
that takes into account the precedence constraints.
Concepts and notations from
\cite[part II]{4}
are used,
as well as
meaningful constructions
\cite[part I]{4}.

The problem under consideration
has as its prototype the well-known
traveling salesman problem,
TSP;
see~\cite{7,8,9,10,11,12}.
However, essential qualitative features
(the presence of constraints, first of all)
motivate the need for a specialized theory;
see \cite {1,3,4,5,14}.
In this paper,
we single out \cite{4},
where, considering an actual engineering problem,
a number of fundamental provisions of a theoretical nature
are demonstrated including the use of DP
as a general method for solving various applied problems.

The problem of tool routing for CNC sheet-cutting machines,
known as the Cutting Path Problem or Tool Path Problem \cite{bibx:100},
is considered primarily.
It arises during the development of control programs for a CNC machine,
which set the trajectory of the tool and a number of technological commands,
determining the parameters of cutting sheet material to get
the parts of known shapes and sizes.
The input data for modeling the route of the tool for the CNC machine
is the information about the positions of all the parts
that is generated at the appropriate development stage
after solving the ``nesting'' problem~\cite{bibx:101, bibx:102}.
From the point of view of geometric optimization,
this problem belongs to the class of cutting and packing problems
\cite{bibx:103},
for which, as well as for routing optimization problems,
no algorithms of polynomial complexity are known.
The nesting problem is beyond the scope of this paper.

Generally speaking about the problem of tool path optimization
for CNC sheet-cutting equipment, it should be noted that
there is still no common theoretical basis for solving this problem so far.
Almost no works describe exact algorithms,
used to solve tool routing problems.
Separate groups of scientists are known,
who are studying special cases of this problem.
In addition,
several Computer-Aided-Manufacturing (CAM) systems
contain a special module to solve some optimization problems,
e.g. minimizing air motion;
however, this does not ensure compliance
with the technological requirements for CNC cutting machines
and does not allow getting cutting routes,
close to optimal from the point of view of the integrated criterion of the cost of cutting
taking into account the working stroke of the tool,
the cost of piercing, etc.
However, when combined with interactive design methods,
they provide rational and technologically acceptable options
for tool path development for CNC sheet-cutting machines.
It should be emphasized that algorithms implemented in commercial software
are not described in scientific literature.

Probably the first attempt to classify tool path problems was made by Hoeft and Palekar \cite{bibx:306}.


Among modern scientists who conduct similar research, Devil and his colleagues should be singled out \cite{bibx:100, bibx:109, bibx:110, bibx:307}.
These works make an attempt to link the features of laser cutting with routing algorithms.
The work \cite{bibx:109} provides an overview of routing algorithms,
related to curly sheet cutting on CNC machines.

The authors categorize the existing literature on routing
for six classes of problems:
continuous cutting task (CCP),
endpoint cutting problem (ECP),
intermittent cutting task (ICP),
polygon traversal problem (TPP),
traveling salesman problem (TSP),
and the generalized traveling salesman problem (GTSP).
All of the above classes of problems, except for CCP, use discrete mathematical models.
The routing problem in general cutting can be thought of as an ICP.
However, the ICP literature is very scarce,
and most scientific articles are limited to solving problems of other classes (see, in particular, \cite{bibx:301}).

Many Russian scientists have also investigated the cutting path problem. The first papers about the optimization of the sheet cutting route for CNC machines
were published by ~Frolovsky~\cite{bibx:104}
and ~Verkhoturov~\cite{bibx:105}.
The authors used simple graph and combinatorial mathematical models,
reduced to the classical traveling salesman problem without additional constraints.
However, these works were not continued.
In recent years, several publications by ~Panyukov
and ~Makarovskikh on this subject appeared
\cite{bibx:106,bibx:107,bibx:108},
involving the use of a combined cutting technique
for the tool path of a CNC machine.
Note that these works can actually be attributed to the class of works only routing in graphs, although they are announced as works on tool routing for CNC laser machines, since in these works a computational experiment is completely absent
and the issues of technological admissibility of the implementation of the resulting trajectories on CNC sheet-cutting machines are poorly investigated.
The reason for this is that the graph model
cannot take into account all the geometric aspects of the cutting map, which is the initial information for solving practical problems of constructing feasible options for the tool route.


The work \cite{bibx:308,bibx:112},
based on the introduced concepts of a ``cutting segment''
and ``basic cutting segment'',
managed to distinguish a fairly wide subclass of problems in the ICP class,
which boil down to the CCP and GTSP classes.
The cutting segment here is defined as the tool path between the pierce point and the corresponding tool switch-off point, and the base segment is the part of the cutting segment without the initial part of the path between the pierce point and the tool entry point into the contour, and the end part between the contour exit point and the tool switch-off point.
This concept made it possible, in particular,
to solve problems of different classes,
in which it is possible to use different cutting techniques within the same route,
including ``combined cut'', ``multi-contour cutting''
\cite[part I] {4}, etc.
We must immediately make a reservation that within the framework of this article, we do not consider optimization problems of the CCP class using continuous models,
since for them, the issue of guaranteed obtaining the global extremum remains practically unexplored.

In general, we note once again that the research of the cutting path problem, as a rule, concerns the development of separate algorithms for only one of the above classes in \cite{bibx:109}.
At the same time,
these studies often do not take into account the important technological limitations of sheet cutting on CNC machines,
limiting themselves only to the conditions of precedence.
The greatest difficulty is presented by the so-called ``dynamic constraints''\cite{bibx:309} generated by thermal deformations of the material and causing changes
in the formal mathematical conditions of the problem
in the process of constructing the tool path of a CNC machine.
To account for this kind of restrictions, at present, basically,
two approaches are applied:
\begin{enumerate}
  \item
  formalization of heuristic rules,
  developed by experienced technologists for routing the tool in
  interactive mode; and
  \item
  the use of engineering analysis systems for
  modeling temperature fields in the material arising in
  the thermal cutting process.
\end{enumerate}

The first approach includes the rule of
``part hardness'',
which limits the choice of feasible tie-in points
at the part selected for cutting, and the rule of
``sheet hardness'',
which, in turn, when forming a route, imposes restrictions on the choice of the next part to be cut
(see \cite[part I]{4}, \cite{bibx:309}).
The second approach is implemented, for example, in the works \cite{bibx:114,bibx:116}.

From other works, to one degree or another taking into account the thermal deformations of the material when modeling the cutting route, we note
\cite{bibx:113,bibx:115,bibx:302,bibx:307}.
The work \cite{bibx:113} proposes a parallel constraint programming approach
for routing laser cutting with explicit precedence constraints
and implicit consideration of thermal constraints. The authors emphasize the importance of considering all
practical routing considerations already in the nesting phase.
However, no follow-up studies aimed at solving this complex problem
have been published.
The work \cite{bibx:114} developed more accurate and faster thermal estimation methods. While this line of research is encouraging, a more detailed study of the problem of the relationship between material temperature and acceptable route options is required.
Sensor solutions for laser
cutting are rarely used in practice. In \cite{bibx:307}, a coaxial photodiode-based monitoring system was
developed and investigated for 4 kW fiber laser cutting of mild and stainless steel thick plates.

It is important to note that \cite{bibx:100, bibx:116, bibx:307} show a practical possibility of using heuristic approaches of the theoretical model of GTSP /megalopolises for the tool route modeling problem for thermal cutting machines with simultaneous control of material temperature.
At the same time, the results of calculations presented in all works, taking into account the thermal aspects for the CNC machine, look not very convincing in terms of real optimization of the time and cost of the cutting process. The main reason for this is that the proposed
techniques for reducing the problems of thermal distortion of
material when cutting are mainly of a qualitative nature. It is sufficiently reliable to assert that until now, no accurate numerical data has been obtained on the magnitude of geometric distortions of parts when choosing one or another cutting route, depending on the degree of fulfillment of heuristic rules of ``stringency'' or depending on the distribution of temperature fields at thermal cutting. It is clear that the magnitude of thermal deformations is also determined by the brand and thickness of the material, as well as the features of
the equipment. The features of the equipment include, first of all, the technology used for sheet thermal cutting (laser,
plasma, gas). Research in this area has not actually been
carried out.
Therefore, the mathematical formalization of dynamic constraints in
tool routing tasks causes obvious difficulties,
in contrast to tasks related to nuclear power, where these
restrictions are set naturally \cite{1,3}.

On the other hand, when optimizing the tool path for CNC waterjet cutting machines, the dynamic constraints are often not taken into account because it is not necessary. It is only important to take into account the conditions of precedence.

Due to this, efficient algorithms for solving classical routing problems of discrete optimization are of certain interest for tool path generation of CNC sheet-cutting machines.

If we talk about algorithms using the classical model of the generalized traveling salesman problem, then there are two main approaches in their development. The first approach is associated with the development of accurate algorithms for special cases and approximation algorithms with theoretical guarantees of performance, the second is based on the application of various heuristics and meta-heuristics.

As part of the first approach, we note the branch and bound and branch and cut algorithms
(see, for example, \cite{bibx:228}) and polynomial-time
approximation schemes for some special cases
\cite{bibx:214}. We also note the studies of the TSP problem with
the dependence of the cost of movements on time
\cite{bibx:218,bibx:219} and with the ``dependency on
prehistory''
\cite{bibx:221}. The latter dependence in terms of its semantic
content corresponds to the function of
cost ``depending on the list of tasks''
\cite{1,3}, which
can be used to construct an admissible (from the point of view of
thermal deformations of the material) trajectory of a
CNC machine tool for thermal cutting.
The second approach is mainly represented by works in which the GTSP model is used in its most general form without any additional
restrictions. So, Gutin and Karapetyan \cite{Gutin-2010}
proposed an efficient memetic algorithm. The work
\cite{Helsgaun-2015} extended the famous heuristic
Lin-Kernighan-Helsgaun solver to the GTSP,
and \cite{bibx:230} developed
powerful meta-heuristics of GNLS, which today is
the most effective. In the case of a GTSP with precedence conditions
of arbitrary form, algorithmic results still remain
rather few in number. We can note heuristics \cite{SALMAN2016} and the specialized algorithm based on the branch-and-bound method idea \cite{SALMAN2020}.

We also note recent works using both models of classical meta-heuristics and specialized heuristics
\cite{Hajad2020SolvingTL,Li2020AutomaticGO}.


Note that classical meta-heuristics are also actively used
when solving problems of routing the tool of CNC machines for
machining (see, for example, \cite{bibx:305,bibx:300,bibx:304}).

Returning to the optimization problems of sheet cutting, we also note
the work \cite{bibx:117}, which explores an approach based on
the technological method of leaving "jumpers" in the process of thermal
cutting to increase the rigidity of sheet material
and reduce the geometric deformations of parts.

The above-mentioned articles demonstrate
that works for optimal tool routing for
sheet-cutting machines are actively developing,
and this topic needs a more structured scientific approach.
Within the framework of this topic, there are two relevant directions:
\begin{enumerate}
  \item
  development of precise algorithms and algorithms with
  guaranteed estimations;
  \item
  adequate consideration of the dynamic constraints of thermal cutting.
\end{enumerate}
The following is a rigorous mathematical formalization
of the
problems of routing
with constraints and cost functions,
depending on the list of tasks,
the study of which, in particular,
allows getting optimal solutions for a variety of
tool routing problems
for sheet-cutting NC machines,
including tasks with some types of
``dynamic'' constraints.

\sect{Summary of notation}
\label{sec:1}


% \bibliographystyle{abbrv}
\bibliographystyle{ieeetr}
\bibliography{Chentsov.bib}


\end{document}
