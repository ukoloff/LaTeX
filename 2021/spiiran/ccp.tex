\documentclass[10pt]{SPIIRAS_Proceedings}

\graphicspath{{media/}}

\usepackage{subfig}
\usepackage{tikz}
\usetikzlibrary{patterns,intersections}

\udk{000.00}

\titleRus{
  Новый алгоритм построения кратчайшего пути обхода
  конечного множества непересекающихся контуров на плоскости
}

\authorsRus{
  А.А. Петунин,
  Е.Г. Полищук,
  С.С. Уколов.
}

\authorsTitleRus{
  А.А. П\smallcapsfake{етунин},
  Е.Г. П\smallcapsfake{олищук},
  С.С. У\smallcapsfake{колов}
} % \smallcapfake необходим для имитации малых прописных букв ввиду отсутствия поддержки в используемом шрифте

\abstractRus{
  Рассматривается проблема маршрутизации режущего
  инструмента машин термической резки с ЧПУ
  когда точки врезки расположены
  на границах деталей,
  ограниченных отрезками прямых и дугами окружностей
  с использованием техники непрерывной резки
  (CCP),
  то есть каждый контур вырезается целиком.
  Общая задача минимизации длины маршрута
  в этом случае сводится к минимизации длины холостого хода.
  Показано, что она эквивалентна поиску
  кратчайшей ломаной с вершинами,
  расположенными на контурах.
  Представлен алгоритм построения
  такой ломаной для заданного порядка
  обхода контуров,
  доставляющий локальный минимум
  а также предложены достаточные условия
  глобального минимума.
  Предложен эвристический алгоритм
  маршрутизации на основе
  метода переменных окрестностей
  (VNS).
  Рассмотрены результаты численных экспериментов
  в сравнении с точным решением задачи GTSP.
}

\keywordsRus{
  задача резки,
  непрерывная резка,
  оптимизация,
  достаточные условия,
  эвристика,
  GTSP,
  метод переменных окрестностей
}

\begin{document}

\maketitle

\normalsize

\section*{Введение}
\label{sec:intro}

В процессе разработки управляющих программ
для машин термической резки листового материала с ЧПУ
возникает несколько оптимизационных задач.
Одна из них -- минимизация
длины холостого хода инструмента.
Она в некоторых случаях может быть сведена
к задаче поиска кратчайшей ломаной,
вершины которой расположены на заданных
плоских контурах,
являющихся границами деталей.
Расположение этих контуров на плоскости
в свою очередь получено решением другой
оптимизационной задачи -- <<раскроя>>.
Обе упомянутые задачи являются NP-полными.

Задача минимизации длины холостого хода инструмента
сама по себе является частным случаем
общей оптимизационной проблемы
маршрутизации инструмента.
Её полное решение в общем случае
не может быть получено
в разумное время
для задач типичных для современного
промышленного производства
(сотни / тысячи контуров),
поэтому на практике применяются разнообразные эвристики,
дающие решения приемлемого качества.

Для полного решения задачи маршрутизации
режущего инструмента для машин листовой резки
с ЧПУ
для минимизации времени или стоимости резки
требуется решить целый ряд задач.
Их описание и классификация приведены подробно в
\cite{bi01,bi02,bi03},
и схематически изображены на рис.~\ref{CP-classes}.

\begin{itemize}
  \item
  \textbf{Задача непрерывной резки}
  (Continuous Cutting Problem, CCP):
  каждый контур
  (ограничивающий одну из деталей)
  вырезается за один раз,
  одним движением инструмента,
  но резка может начаться в любой точке контура
  (и заканчивается в ней же)

  \item
  \textbf{Обобщённая задача коммивояжера}
  (Generalized Traveling Salesman Problem, GTSP):
  резка может начаться в одной из заранее
  заданных точек на контуре
  (количество таких точек конечно),
  после этого контур вырезается целиком

  \item
  \textbf{Задача резки с остановками}
  (Endpoint Cutting Problem, ECP):
  резка контура может начинаться только в
  заранее заданных точках на нём,
  но контур может вырезаться за несколько раз,
  частями

  \item
  \textbf{Сегментная задача непрерывной резки}
  (Segment Continuous Cutting Problem, SCCP):
  вводится понятие сегмента
  как обобщение понятия контура;
  сегмент может быть частью контура
  или объединением нескольких контуров
  и / или их частей.
  Каждый сегмент вырезается целиком,
  от начала до конца,
  таким образом
  $ CCP \subset SCCP$.

  \item
  \textbf{Обобщённая сегментная задача непрерывной резки}
  (Generalized Segment Continuous Cutting Problem, GSCCP):
  подобна сегментной задаче непрерывной резки
  (SCCP),
  но разбивка на сегменты не задана заранее
  и сама подлежит оптимизации

  \item
  \textbf{Задача прерывистой резки}
  (Intermittent Cutting Problem, ICP):
  наиболее общая формулировка задачи резки,
  встречающаяся в научной литературе,
  контуры могут вырезаться частями,
  в несколько подходов,
  начиная с произвольной точки.
\end{itemize}

\begin{figure}
  \centering
  \def\svgwidth{\columnwidth}
  \input{media/classes.pdf_tex}
  \caption{Классификация задач резки}
  \label{CP-classes}
\end{figure}

На практике задача оптимизации маршрута режущего инструмента
зачастую сводится к дискретной оптимизации
за счёт выбора конечного множества возможных точек врезки
на контурах деталей с некоторым заранее заданным шагом
$\varepsilon$,
то есть сводится к задаче ECP
\cite{bi04,bi05,bi06}
и её частному случаю ---
GTSP
\cite{bi07,bi08,bi09,bi10}.
Задача непрерывной резки CCP
тоже может сводиться к GTSP,
в этом случае суммарная ошибка длины
холостого хода оценивается как
$N \cdot \varepsilon$,
где $N$ -- количество контуров.
Для достижения точности результата
$\delta$,
таким образом необходимо выбирать малое
$\varepsilon \approx \delta / N$,
так что полное количество допустимых точек врезки растёт
(как $O (N)$)
и полный перебор требует экспоненциального времени.
Тем не менее, этот класс задач может успешно решаться,
например,
средствами динамического программирования
(DP)
а для небольших
$N \approx 30$ -- даже точно
(см. в частности \cite{bi15}).

В данной работе рассматриваются вопросы
поиска оптимального маршрута без
применения дискретизации
(то есть задача CCP),
которые слабо освещены в
открытых источниках,
см. например
\cite{bi11,bi12},
где описаны некоторые эвристики.

\subsection*{Технологические ограничения}

\begin{figure}
  \centering
  \subfloat[На пересечении с контуром]{
    \label{pierce-thru}
    \tikz[rotate=27,scale=0.68]{
        \draw[thick,name path=C1]
            (0,5) node[above] {$C_{i-1}$}
            to[bend right] (1,0);
        \draw[thick,name path=C2]
            (3,0) node[below] {$C_i$}
            to[bend right] (3,2);
        \draw[thick,name path=C3]
            (4,3) node[above] {$C_{i+1}$}
            to[bend right] (5,0);
        \path[name path=L0]
            (0,1.5) -- (6,2);
        \path[name path=Lx]
            (0,1) -- (6,1);
        \fill[name intersections={of=C1 and Lx, name=X}]
            (X-1) coordinate(M1) circle(3pt) node[below] {$M_{i-1}$};
        \fill[name intersections={of=C2 and L0, name=X}]
            (X-1) coordinate(M2) circle(3pt) node[below left]{$M_i$};
        \draw[name intersections={of=C2 and Lx, name=X}]
            (X-1) coordinate(M2x) node[below right]{$M'_i$};
        \fill[name intersections={of=C3 and Lx, name=X}]
            (X-1) coordinate(M3) circle(3pt) node[right] {$M_{i+1}$};
        \draw[dashed]
            (M1) -- (M3);
        \draw[thin,-latex]
            (M2)
            to[bend right] (M2x);
    }
  }
  \subfloat[Отражение по \textit{принципу Ферма}]{
    \label{pierce-fermat}
    \tikz[rotate=-12,scale=0.68]{
      \draw[thick]
          (0, 0) coordinate(zero) -- (5, 0) coordinate(future) node[right] {$C_i$};
      \fill[black]
          (1.5, 0) circle(3pt) coordinate(middle) node[below left]  {$M_i$}
          (1, 1) circle(3pt) coordinate(from) node[above right] {$M_{i-1}$} ++(-1.5,0) node[above] {$C_{i-1}$}
          (4.5, 2) circle(3pt) coordinate(to) node[above right] {$M_{i+1}$} ++(1.5,0) node[below] {$C_{i+1}$};
      \begin{scope}
          \clip (from) circle(1);
          \draw[thick] (from) ++(0, 3) circle(3);
      \end{scope};
      \begin{scope}
          \clip (to) circle(1.5);
          \draw[thick] (to) ++(3, 4) circle(5);
      \end{scope};
      % \draw[dashed] (from) -- (middle) -- (to);
      \draw[thin] (4.5, -2) circle(0.062) coordinate(mirror) node[right] {$\hat M_{i+1}$};
      \coordinate (opt) at (intersection of zero--future and mirror--from);
      % \draw[thin] (opt) circle(3pt);
      \draw[dotted]
          (mirror) -- (opt)
          (mirror) -- (to);
      \draw[dashed] (from) -- (opt) -- (to);

      \draw[thin,-latex] (middle) to[bend right] (opt) node[below] {$M'_i$};
    }
  }
  \caption{Выбор положения точки врезки}
  \label{shift-pierce-point}
\end{figure}

\end{document}
