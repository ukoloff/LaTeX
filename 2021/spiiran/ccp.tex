\documentclass[10pt]{SPIIRAS_Proceedings}

\graphicspath{{media/}}

\udk{000.00}

\titleRus{
  Новый алгоритм построения кратчайшего пути обхода
  конечного множества непересекающихся контуров на плоскости
}

\authorsRus{
  А.А. Петунин,
  Е.Г. Полищук,
  С.С. Уколов.
}

\authorsTitleRus{
  А.А. П\smallcapsfake{етунин},
  Е.Г. П\smallcapsfake{олищук},
  С.С. У\smallcapsfake{колов}
} % \smallcapfake необходим для имитации малых прописных букв ввиду отсутствия поддержки в используемом шрифте

\abstractRus{
  Рассматривается проблема маршрутизации режущего
  инструмента машин термической резки с ЧПУ
  когда точки врезки расположены
  на границах деталей,
  ограниченных отрезками прямых и дугами окружностей
  с использованием техники непрерывной резки
  (CCP),
  то есть каждый контур вырезается целиком.
  Общая задача минимизации длины маршрута
  в этом случае сводится к минимизации длины холостого хода.
  Показано, что она эквивалентна поиску
  кратчайшей ломаной с вершинами,
  расположенными на контурах.
  Представлен алгоритм построения
  такой ломаной для заданного порядка
  обхода контуров,
  доставляющий локальный минимум
  а также предложены достаточные условия
  глобального минимума.
  Предложен эвристический алгоритм
  маршрутизации на основе
  метода переменных окрестностей
  (VNS).
  Рассмотрены результаты численных экспериментов
  в сравнении с точным решением задачи GTSP.
}

\keywordsRus{
  задача резки,
  непрерывная резка,
  оптимизация,
  достаточные условия,
  эвристика,
  GTSP,
  метод переменных окрестностей
}

\begin{document}

\maketitle

\normalsize

\section*{Введение}
\label{sec:intro}

В процессе разработки управляющих программ
для машин термической резки листового материала с ЧПУ
возникает несколько оптимизационных задач.
Одна из них -- минимизация
длины холостого хода инструмента.
Она в некоторых случаях может быть сведена
к задаче поиска кратчайшей ломаной,
вершины которой расположены на заданных
плоских контурах,
являющихся границами деталей.
Расположение этих контуров на плоскости
в свою очередь получено решением другой
оптимизационной задачи -- <<раскроя>>.
Обе упомянутые задачи являются NP-полными.

Задача минимизации длины холостого хода инструмента
сама по себе является частным случаем
общей оптимизационной проблемы
маршрутизации инструмента.
Её полное решение в общем случае
не может быть получено
в разумное время
для задач типичных для современного
промышленного производства
(сотни / тысячи контуров),
поэтому на практике применяются разнообразные эвристики,
дающие решения приемлемого качества.

Для полного решения задачи маршрутизации
режущего инструмента для машин листовой резки
с ЧПУ
для минимизации времени или стоимости резки
требуется решить целый ряд задач.
Их описание и классификация описаны в
\cite{bi01,bi02,bi03}.

\end{document}
