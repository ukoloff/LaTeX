\documentclass{article}
    \usepackage[utf8]{inputenc}
    \usepackage[T2A]{fontenc}
    \usepackage[english,russian]{babel}

    % Related to math
    \usepackage{amsmath,amssymb,amsfonts,amsthm}

    \usepackage{graphicx}
    \graphicspath{{media/}}

\begin{document}

\section*{Введение}

В различных технических приложениях
возникают задачи моделирования маршрута и маршрутной оптимизации.
Большая часть таких задач обычно рассматривается современными
исследователями через призму различных комбинаторных моделей дискретной оптимизации.
Вместе с тем, при моделировании маршрута в реальных технических задачах
числовые значения некоторых параметров маршрута могут выбираться
из множества допустимых величин, имеющего континуальную мощность,
что усложняет математические модели оптимальной маршрутизации
в сравнении с классическими маршрутными постановками типа задачи коммивояжера (ЗК).
Кроме того, на множество допустимых решений могут накладываться дополнительные ограничения,
вызванные техническими особенностями задачи, например,
технологическими требованиями к маршруту,
порождаемыми спецификой конкретной предметной области.
В результате возникают новые математические постановки,
не охватываемые существующими методами решения.
К числу такого рода сложных задач относится
проблема оптимальной маршрутизации инструмента
машин фигурной листовой резки с числовым программным управлением (ЧПУ).
Эта проблема возникает на этапе разработки управляющих программ для машины с ЧПУ,
которые задают траекторию перемещения инструмента и ряд технологических команд,
определяющих параметры резки листового материала
для получения из него заготовок известных форм и размеров.
Необходимые данные для моделирования маршрута инструмента
машины с ЧПУ определяет информация о раскройных картах,
которые разрабатываются на этапе проектирования раскроя
и порождают известные задачи оптимизации раскроя листового материала.
С точки зрения геометрической оптимизации задачи раскроя относятся
к классу задач раскроя-упаковки
(Gutting \& Packing),
для которых, также как и для маршрутных оптимизационных проблем,
не известны алгоритмы решения полиномиальной сложности.
В данной работе задачи раскроя не рассматриваются.
Основное направление исследования в настоящей монографии
связано с моделированием маршрута инструмента
машин фигурной листовой резки с ЧПУ и
проблемой его оптимизации по временным и стоимостным параметрам.

В исходной задаче требуется осуществить последовательное
посещение всех контуров с целью осуществления резки по эквидистантам,
представляющим из себя замкнутые кривые
(обсуждаются также и более сложные типы резки);
точки, определяющие начало и окончание реза,
могут при этом назначаться произвольно.
В интересах построения конкретных решений приходится,
однако, использовать дискретизацию эквидистант
и некоторые дополнительные преобразования последних в непустые конечные множества — мегаполисы,
что и делается в настоящей монографии
(см. в этой связи \cite{intro01,intro02}).

Если рассматривать сформулированное научное направление в его полной общности,
то приходится признать, что адекватной математической теории здесь не разработано.
Имеются отдельные направления, среди которых особо отметим проблему полиномиальной разрешимости
для отдельных классов оптимизационных задач,
которые могут использоваться в качестве подзадач рассматриваемой проблемы.
Известные результаты, которые получены в последние годы в предметных областях,
связанных с разработкой алгоритмов дискретной оптимизации
и исследованием проблемы полиномиальной разрешимости,
при всей своей значимости не охватывает проблемы “диапазонных”
(в смысле размерности) задач и особенно задач,
осложнённых ограничениями.
В монографии авторы исследуют вопросы разработки
теоретических и методологических основ решения проблемы
оптимальной маршрутизации инструмента для машин фигурной листовой резки с ЧПУ,
включая разработку адекватных математических моделей
и алгоритмов решения для исследуемой прикладной задачи.
Результаты работы могут быть использованы и для решения
других прикладных задач,
описываемых предложенными в монографии математическими моделями.

\begin{thebibliography}{99}

\bibitem{intro01}
{\bf Ченцов А.Г., Ченцов А.А.}
Дискретно-непрерывная задача маршрутизации с условиями предшествования
//Труды института  математики и механики УрО РАН. 2017. Т. 23. № 1. С. 275-292.

\bibitem{intro02}
{\bf Chentsov A.A., Chentsov A.G.}
Dynamic programming method in the generalized traveling salesman problem: the influence of inexact calculations
// Mathematical and computing modelling. 2001. Vol. 33. P. 801-819.

\bibitem{intro03}
{\bf Hoeft, J., Palekar, U. S.}
(1997). Heuristics for the plate-cutting traveling salesman problem.
IIE Transactions, 29, 719-731.

\bibitem{intro04}
{\bf Lee M.-K., Kwon K.-B.}
Cutting path optimization in CNC cutting processes using a two-step genetic algorithm.
Dec. 2006. International Journal of Production Research 44 (24), P.5307-5326.

\bibitem{intro05}
{\bf Yang, W. B., Zhao, Y. W., Jie, J., Wang, W. L.}
(2010). An Effective Algorithm for Tool-Path Airtime. Optimization during Leather Cutting.
Advanced Materials Research. 102, 373-377.

\bibitem{intro06}
{\bf Jing Y., Zhige C.}
An Optimized Algorithm of Numerical Cutting-Path Control in Garment Manufacturing.
— 2013. — Advanced Materials Research. 796. P.454-457.

\bibitem{intro07}
{\bf Yu W, Lu L}
(2014) A route planning strategy for the automatic garment cutter based on genetic algorithm.
In: IEEE congress on evolutionary computation (CEC), pp. 379–386

\bibitem{intro08}
{\bf Dewil, R., Vansteenwegen, P., Cattrysse, D.}
(2014) Construction heuristics for generating tool paths for laser cutters.
International Journal of Production Research, Mar. 2014, 1-20.

\bibitem{intro09}
{\bf Dewil, R., Vansteenwegen, P., Cattrysse, D., Laguna, M., Vossen, T.}
(2015) An improvement heuristic framework for the laser cutting tool path problem.
International Journal of Production Research., 53 (6) (2015), pp. 1761–1776.

\bibitem{intro10}
{\bf Фроловский В.Д.}
Автоматизация проектирования управляющих программ тепловой резки металла на оборудовании с ЧПУ
// Информационные технологии в проектировании и производстве. 2005. № 4. С. 63-66 .

\bibitem{intro11}
{\bf Ганелина Н.Д., Фроловский В.Д.}
Исследование методов построения кратчайшего пути обхода отрезков на плоскости
// Сибирский журнал вычислительной математики. №3, т. 9. 2006. с. 201-212.

\bibitem{intro12}
{\bf Верхотуров M.A., Тарасенко П.Ю.}
Математическое обеспечение задачи оптимизации пути режущего инструмента при плоском фигурном раскрое на основе цепной резки.
// Вестник УГАТУ. Управление, ВТиИТ. Уфа: Изд-во УГАТУ. 2008. Т.10. №2 (27). с.123-130.

\bibitem{intro13}
{\bf Dewil, R., Vansteenwegen, P., Cattrysse, D.}
(2016) A review of cutting path algorithms for laser cutters.
Int J Adv. Manuf. Techno.l 87:1865–1884.

\bibitem{intro14}
{\bf Sherif S. U., Jawahar N., Balamurali M.}
(2014) Sequential optimization approach for nesting and cutting sequence in laser cutting
//Journal of Manufacturing Systems.  2014, V. 33, №. 4, pp. 624-638.

\bibitem{intro15}
{\bf Меламед И.И., Сергеев С.И., Сигал И.Х.}
Задача коммивояжера. Вопросы теории
// Автоматика и телемеханика. 1989. № 9. С. 3-34.

\bibitem{intro16}
{\bf Меламед И.И., Сергеев С.И., Сигал И.Х.}
Задача коммивояжера. Точные алгоритмы
// Автоматика и телемеханика. 1989. № 10. С. 3-29.

\bibitem{intro17}
{\bf Меламед И.И., Сергеев С.И., Сигал И.Х.}
Задача коммивояжера. Приближенные алгоритмы
// Автоматика и телемеханика. 1989. № 11. С. 3-26.

\bibitem{intro18}
{\bf Веллман Р.}
Применение динамического программирования к задаче о коммивояжере
// Кибернет. сб. М.: Мир, 1964. Т. 9. С. 219-228.

\bibitem{intro19}
{\bf Хелд М., Карп Р.М.}
Применение динамического программирования к задачам упорядочения
// Кибернет. сб. М.: Мир, 1964. Т. 9. С. 202-218.

\bibitem{intro20}
{\bf Gutin G., А.Р. Punnen А.Р.}
(editors) The Traveling Salesman problem and its variations
// Kluwer Academic Publishers, 2002. vol. 12, p. 585-607.

\bibitem{intro21}
{\bf William J. Cook.}
In pursuit of the traveling salesman. Mathematics at the limits of computation.
Princeton University Press, NJ, 2012. P.248.

\bibitem{intro22}
{\bf Сигал И.Х.}
Декомпозиционный подход к решению задачи коммивояжера большой размерности и некоторые его приложения
// Изв. АН СССР. Техн. киберн. 1990. № 6. С. 143-155.

\bibitem{intro23}
{\bf Литл Дж., Мурти К., Суини Д., Кэрел К.}
Алгоритм для решения задачи о коммивояжере
// Экономика и математические методы. 1965. Т. 1 (Вып. 1) С. 94-107.

\bibitem{intro24}
{\bf Ascheuer, N., Jünger, M., Reinelt, G.}
(2000) A Branch \& Cut Algorithm for the Asymmetric Traveling Salesman Problem with Precedence Constraints.
Computational Optimization and Applications. Volume 17, Issue 1, pp 61-84.

\bibitem{intro25}
{\bf Karapetyan, D., Gutin G.}
(2011) Lin-Kernighan Heuristic Adaptations for the Generalized Traveling Salesman Problem.
European J. of Operational Research 208 (3): 221–232.

\bibitem{intro26}
{\bf Karapetyan. D., Gutin, G.}
(2012) Efficient Local Search Algorithms for Known and New Neighborhoods for the Generalized Traveling Salesman Problem.
Eur. J. Oper. Res, 219(2):234-251.

\bibitem{intro27}
Concorde TSP Solver Lin–Kernighan heuristic software,
downloaded in April 2016, http://www.math.uwaterloo.ca/tsp/concorde/downloads/downloads.htm

\bibitem{intro28}
{\bf Xie, S. Q., Gan, J., Wang, G. G., Vn, C.}
(2009). Optimal process planning for compound laser cutting and punch using Genetic Algorithms.
International Journal of Mechatronics and Manufacturing Systems. 2 (1/2), 20-38.

\bibitem{intro29}
{\bf Петунии А.А.}
О некоторых стратегиях формирования маршрута инструмента при разработке управляющих программ для машин термической резки материала
// Вестник УГАТУ. Серия: Управление, вычислительная техника и информатика. 2009. Т. 13, № 2 (35). С. 280-286.

\bibitem{intro30}
{\bf Petunin, A. A., \& Stlios, C.}
(2016). Optimization Models of Tool Path Problem for CNC Sheet Metal Cutting Machines. Ifac papersonline, 49(12), 23-28.

\bibitem{intro31}
{\bf Петунин, А.А.}
Методологические и теоретические основы автоматизации проектирования раскроя листовых материалов на машинах с числовым программным управлением:
дисс. на соиск. уч. ст. докт. техн. наук: 05.13.12/А.А. Петунин -Свердловск, 2009. -348 с.

\bibitem{intro32}
{\bf Петунии А.А., Ченцов А.Г., Ченцов П.А.}
К вопросу о маршрутизации движения инструмента в машинах листовой резки с числовым программным управлением
// Научно-технические ведомости СПбГПУ, № 2 (169), 2013, С. 103-111.

\bibitem{intro33}
{\bf Петунии А.А., Ченцов А.Г., Ченцов П.А.}
Локальные вставки на основе динамического программирования в задаче маршрутизации с ограничениями
// Вестник Удмуртского университета. Математика. Механика. Компьютерные науки. 2014. Вып. 2. С. 56-75

\bibitem{intro34}
Heuristic algorithms for solving of the tool routing problem for CNC cutting machines
/ Chentsov P.A., Petunin A.A., Sesekin A.N., Shipacheva E.N., Sholohov A.E. // AIP Conference Proceedings. - 2015. - V. 1690, 030004-1 – 030004-6.

\bibitem{intro35}
Modeling of tool path for the CNC sheet cutting machines
Petunin, Aleksandr A. // AIP Conference Proceedings. – 2015. V. 1690, 060002-1 – 060002-7.

\bibitem{intro36}
{\bf Petunin, Alexander}
(2019). General Model of Tool Path Problem for the CNC Sheet Cutting Machines.
Ifac papersonline, V.??, l. 12. - P. ????-????

\bibitem{intro37}
Tool Routing Problem for CNC Plate Cutting Machines
/ Chentsov P.A., Petunin A.A. // IFAC-PapersOnLine. - 2016. - V. 49, l. 12. - P. 645-650.

\bibitem{intro38}
{\bf Petunin, A. A., Polishuk, E. G., Chentsov, A. G., Chentsov, P. A., \& Ukolov, S. S.}
(2016). About some types of constraints in problems of routing.
Pasheva, N. Popivanov, \& G. Venkov (Edit.), APPLICATIONS OF MATHEMATICS IN ENGINEERING AND ECONOMICS (AMEE'16) (Vol 1789).
[060002] (AIP Conference Proceedings; Vol.1789). American Institute of Physics Publising LLC.

\bibitem{intro39}
{\bf A.A. Petunin, E.G. Polyshuk, P.A. Chentsov, S.S. Ukolov, V. I. Krotov}
(2019). The termal deformation reducing in sheet metal at manufacturing parts by CNC cutting machines.
IOP Conference Series: Materials Science and Ebgineering (MSE).

\bibitem{intro40}
Routing problems: constraints and optimality
/ Chentsov A.G., Chentsov P.A., Petunin A.A., Sesekin A.N.
// IFAC-PapersOnLine. - 2016. - V. 49, l. 12. - P. 640-644.

\bibitem{intro41}
Элементы динамического программирования в конструкциях локального улучшения эвристических решений задач маршрутизации с ограничениями
/ Петунин А.А., Ченцов А.А., Ченцов А.Г., Ченцов П.А // Автоматика и телемеханика. 2017. № 4. С. 106-125.

\bibitem{intro42}
{\bf Chentsov, A. G., Chentsov, P. A., Petunin, A. A., \& Sesekin, A. N.}
(2018). Model of megalopolises in the tool path optimisation for CNC plate cutting machines.
International Journal of Production Research, 56(14), 4819-4830.

\bibitem{intro43}
{\bf Petunin, A. A., Chentsov, A. G., \& Chentsov, P. A.}
(2019). Optimizing Insertions in a Constraint Routing Problem with Complicated Cost Functions // Journal of Computer and Systems Sciences International, 58(1), 113-125.

\bibitem{intro44}
{\bf Петунин, А. А., \& Таваева, А. Ф.}
(2015). ОБ ОПТИМИЗАЦИИ МАРШРУТА ИНСТРУМЕНТА ДЛЯ МАШИН ФИГУРНОЙ ЛИСТОВОЙ РЕЗКИ С ЧПУ ПРИ УСЛОВИИ НЕПОСТОЯНСТВА СКОРОСТИ РАБОЧЕГО ХОДА. Фундаментальные исследования, (6-1), 56-62.

\bibitem{intro45}
{\bf Tavaeva, A. F., \& Petunin, A. A.}
(2017). Investigation of Cutting Speed Influence on Optimality of the Tool Path Route for CNC Laser Cutting Machines. International Conference on Industrial Engineering, Applications and Manufacturing, ICIEAM 2017 - Proceedings [8076452] Institute of Electrical and Electronics Engineers Inc.

\bibitem{intro46}
{\bf Tavaeva A., Petunin A., Krotov V.}
(2017) About effectiveness of special cutting techniques application during development of automatic methods of tool path optimization applied to CNC thermal cutting machines. Proceedings of the 19th international workshop on computer science and information technologies CSIT’2017, Germany, Baden-Baden, 2017, pp.221-226.

\bibitem{intro47}
{\bf Таваева, А. Ф., \& Петунин, А. А.}
(2018). Точное вычисление стоимости резки заготовок из листового материала на машине лазерной резки с числовым программным управлением в задаче оптимизации маршрута перемещения режущего инструмента //Моделирование, оптимизация и информационные технологии, 6(4 (23)), 298-312.

\bibitem{intro48}
{\bf Ченцов А.Г.}
Экстремальные задачи маршрутизации и распределения заданий: вопросы теории. М.-Ижевск: НИЦ «Регулярная и хаотическая динамика». 2008. 238с.

\bibitem{intro49}
{\bf Ченцов А.А., Ченцов А.Г., Ченцов П.А.}
Экстремальная задача маршрутизации с внутренними потерями // Труды Института математики и механики УрО РАН. 2008. Т. 14, № 3, с. 183-201.

\bibitem{intro50}
{\bf Ченцов А.А., Ченцов А.Г., Ченцов П.А.}
Экстремальная задача маршрутизации перемещений с ограничениями и внутренними потерями // Изв. вузов. Математика. 2010. № 6. С.64-81.

\bibitem{intro51}
{\bf Ченцов А.Г.}
Об оптимальной маршрутизации в условиях ограничений // Доклады Академии Наук, 2008, Т. 423, № 3, с. 303-307.

\bibitem{intro52}
{\bf Ченцов А.Г., Ченцов А.А.}
Задача маршрутизации с ограничениями, зависящими от списка заданий // Доклады Академии Наук, 2015, т.465, № 2, с. 154-158.

\bibitem{intro53}
{\bf Кошелева М.С., Ченцов А.А., Ченцов А.Г.}
О задаче маршрутизации с ограничениями, включающими зависимость от списка заданий// Труды Института математики и механики УрО РАН. 2015. Т. 21. № 4. С. 178-195.

\bibitem{intro54}
{\bf Ченцов А.Г., Ченцов П.А.}
Маршрутизация перемещений с ограничениями и нестационарными функциями стоимости// Научно-технические ведомости СПбГПУ. Информатика. Телекоммуникации. Управление. 4(152)/2012, с. 88-93.

\bibitem{intro55}
{\bf Ченцов А.Г., Ченцов П.А.}
Об одном нестационарном варианте обобщенной задачи курьера с внутренними работами / А.Г.Ченцов, П.А.Ченцов // Вести. ЮУрГУ. 2013. Т.6, № 2. С.88-107.

\bibitem{intro56}
{\bf Ченцов А.А., Ченцов А.Г.}
К вопросу о нахождении значения маршрутной задачи с ограничениями // Проблемы управления и информатики. 2016. № 1. С.41-54

\bibitem{intro57}
{\bf Коробкин В.В., Сесекин А.Н., Ташлыков О.Л., Ченцов А.Г.}
Методы маршрутизации и их приложения в задачах повышения безопасности и эффективности эксплуатации атомных станций / Под общ. ред. член-корр. РАН П.А. Каляева. - М.: Новые технологии, 2012. 234

\end{thebibliography}


\end{document}
