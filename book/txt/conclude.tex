% !TeX root = ..

\chapter*{ЗАКЛЮЧЕНИЕ}
\addcontentsline{toc}{chapter}{ЗАКЛЮЧЕНИЕ}

В монографии приведены постановки и математические модели инженерных задач,
связанных с листовой резкой на машинах с ЧПУ.
В первой части изложены содержательные конструкции,
относящиеся к вопросам оптимизации в задачах маршрутизации,
касающихся управления режущим инструментом.
Подробно обсуждаются общие подходы и соображения
по различным вариантам осуществления листовой резки на машинах с ЧПУ
(авторы не ограничиваются здесь стандартным вариантом резки по замкнутому контуру).
Введенные в первой части понятия определяют широкий взгляд на задачу,
что позволяет с использованием доступного перебора вариантов резки достигать лучшего качества.
Вместе с тем, из конструкций первой части (главы 1 и 2)
естественным образом возникает необходимость в построении адекватной математической теории,
позволяющей
(при соответствующей формализации)
учитывать различные осложняющие факторы и,
прежде всего, различные ограничения,
возникающие из соображений технологического характера.
Эта цель достигается
(возможно, не в полной мере)
во второй части книги.

Задачи маршрутизации перемещений имеют своим прототипом известную труднорешаемую задачу коммивояжера
(ЗК или {\it TSP} в англоязычной литературе).
Вместе с тем, и это видно уже из содержательных построений первой части,
имеются существенные различия исследуемых постановок с ЗК,
причем это различия не только количественного, но и качественного характера.
В этой связи возникает потребность в формализации,
которая существенно отличается от аналогичной формализации для ЗК и ей подобных.
Используется модель мегаполисов с условиями предшествования и функциями стоимости,
допускающими зависимость от списка заданий.
Последнее обстоятельство естественно для задачи минимизации дозовой нагрузки в атомной энергетике
(см. \cite{Cha2`}),
здесь оно связано с использованием штрафов за нарушение ограничений.
Итак, нарушать ограничения
(в том числе ограничения динамического характера)
формально разрешается, но ценой существенного проигрыша в качестве.

Для реальных задач управления инструментом при листовой резке на машинах с ЧПУ
типичным является случай достаточно большой размерности,
что крайне затрудняет реализацию оптимального алгоритма на основе ДП
даже при наличии большого количества условий предшествования.
В этой связи во второй части последовательно развиваются
конструкции локального улучшения эвристических решений
посредством оптимизирующих вставок и итерационных процедур с применением таких вставок.
Здесь речь идет всякий раз об оптимизации <<в окне>>.
Объектом применения оптимизирующих вставок является
эвристическое решение со свойством допустимости
в смысле соблюдения полной системы ограничений.
При построении оптимизирующих вставок <<умеренной>> размерности
активно используется аппарат ДП.
Итак, ДП находит свое применение и в задачах большой размерности.
В самое последнее время упомянутый подход получил дальнейшее развитие
в конструкциях мультивставок
(см. \cite{ChenGrig, ChenChenGrig}).

Другое <<недавнее>> направление исследований,
не затрагиваемое в настоящей книге,
связано с вопросом оптимизации точки старта в задачах последовательного обхода мегаполисов
(см. \cite{StartPoint,StartFinishPoint,ChenChen}).
Здесь также используется аппарат ДП.
В этих постановках объектом оптимизации звляется триплет
$(\alpha,(z_t)_{t\in \overline{0,N}},x)$,
где $\alpha$ -- маршрут (перестановка индексов),
а $(z_t)_{t\in \overline{0,N}}$ -- траектория, стартующая из точки $(x,x)$,
где $x$ -- точка старта, которая может выбираться из заданного множества $X^0$.

Таким образом,
исследования авторов и их коллег, работающих в ИММ УрО РАН и УрФУ,
находятся в постоянном развитии.
Получаемые при этом результаты важны,
как представляется,
не только в теоретическом аспекте,
но и в многочисленных приложениях.
