% !TeX root = ../mat_mod2.tex

\section{\protect\raggedright
  Общая постановка задачи;
  обсуждение на содержательном уровне
}
\label{sect:3.2}

Инженерная задача,
связанная с маршрутизацией движения инструмента при листовой
резке на машинах с ЧПУ,
требует для своего исследования привлечения достаточно
развитых и разнообразных математических конструкций,
относящихся объективно к
теории экстремальных задач и,
в частности, к дискретной оптимизации.
Данные
конструкции ранее разрабатывались в абстрактной постановке \cite{Cha1`},
либо применительно к другим инженерным задачам \cite{Cha2`}.
В содержательных постановках
первой части имеется целый ряд особенностей,
которые требуют отдельного рассмотрения,
но есть и достаточно основательная общая
(cм.~\cite{Cha1`,Cha2`}) часть,
что позволяет
говорить о применении методов
\cite{Cha1`,Cha2`}
(и целого ряда других работ)
для решения задач,
сформулированных на инженерном уровне в первой части книги при
соответствующей модификации этих методов.
В настоящем разделе мы,
не вникая в
подробности математического характера,
обсудим элементы постановки и наметим
некоторые идеи в части построения решений.

Итак, мы обсуждаем сейчас плоскую маршрутную задачу,
что отвечает специфике
листовой резки:
в конкретной нашей постановке будут рассматриваться перемещения на плоскости
$\bbr \times \bbr,$
что правда во многих отношениях не будет существенно,
и мы иногда будем прибегать к обобщениям.
Будем считать, что в
ограниченной области упомянутой плоскости <<размещены>> детали,
подлежащие вырезке
(конечно, речь идет не о готовых деталях,
а лишь о заготовках, из которых должны
получиться нужные детали).
Каждая из упомянутых деталей представляет для нас
плоскую фигуру,
имеющую один внешний и, возможно, один или несколько внутренних
контуров, которые в своей совокупности образуют границу соответствующей фигуры.
Возле каждого из контуров с внешней стороны намечена эквидистанта,
по которой и
будет всякий раз осуществляться резка.
Возле эквидистанты намечаются,
как указано в первой части,
точки врезки и точки выключения инструмента,
объединяемые в соответствующие пары.
Предполагается, что у каждой эквидистанты могут располагаться
несколько упомянутых пар,
что порождает явление многовариантности процесса
последовательного перемещения инструмента.
Упомянутая многовариантность делает
естественной модель с использованием мегаполисов,
располагаемых возле эквидистант
и представляющих из себя непустые конечные множества,
мощность которых определяется
возможностями последующей вычислительной реализации алгоритмов.
В состав мегаполисов
<<попадают>> пары, первые элементы которых --- суть точки врезки,
а вторые --- точки выключения инструмента.
Это обстоятельство мотивирует введение для каждого мегаполиса
соответствующего отношения в виде
непустого подмножества декартова <<квадрата>> данного мегаполиса;
элементами отношения будут вышеупомянутые (упорядоченные) пары.
Впрочем,
применяемая ниже модель будет допускать
и более общие варианты реализации, но сейчас
мы на этом останавливаться не будем.

Как уже отмечалось, детали, подлежащие резке,
имеют каждая внешний и, возможно,
несколько внутренних контуров.
По соображениям технологического характера резка
внутренних контуров должна осуществляться раньше,
чем резка внешнего.
Это обстоятельство порождает условия предшествования,
которые при формализации
будут учитываться посредством задания
соответствующих адресных пар,
элементами которых являются индексы --- номера заданий.

Подобная ситуация возникает в случае,
когда одна деталь располагается внутри
другой (реализация <<матрешки>>).
В этом случае резка внешнего контура внутренней
детали безусловно должна предшествовать
резке внешнего контура объемлющей детали.

Помимо вышеупомянутых условий предшествования
потенциально возможными следует
признать и другие ограничения,
являющиеся зачастую плохо формализуемыми.
Эти ограничения могут учитывать
требования к обеспечению жесткости листа в точках
врезки, влияние тепловых факторов и т. п.

Упомянутые обстоятельства следует учитывать
при построении математической модели,
хотя в ряде случаев это приводит
к существенному усложнению математической
постановки.
В своих последующих построениях
мы не рассматриваем сразу самый общий случай,
а показываем варианты усложнения модели
по мере учета все большего числа ограничений.
В то же время некоторые типы условий
будут все же учтены не в полной мере.
