\chapter*{ВВЕДЕНИЕ}
\addcontentsline{toc}{chapter}{ВВЕДЕНИЕ}
%\section{ВВЕДЕНИЕ}

В различных технических приложениях
возникают задачи моделирования маршрута и маршрутной оптимизации.
Большая часть таких задач обычно рассматривается современными
исследователями через призму различных комбинаторных моделей дискретной оптимизации.
Вместе с тем, при моделировании маршрута в реальных технических задачах
числовые значения некоторых параметров маршрута могут выбираться
из множества допустимых величин, имеющего континуальную мощность,
что усложняет математические модели оптимальной маршрутизации
в сравнении с классическими маршрутными постановками типа задачи коммивояжера (ЗК).
Кроме того, на множество допустимых решений могут накладываться дополнительные ограничения,
вызванные техническими особенностями задачи, например,
технологическими требованиями к маршруту,
порождаемыми спецификой конкретной предметной области.
В результате возникают новые математические постановки,
не охватываемые существующими методами решения.
К числу такого рода сложных задач относится
проблема оптимальной маршрутизации инструмента
машин фигурной листовой резки с числовым программным управлением (ЧПУ).
Эта проблема возникает на этапе разработки управляющих программ для машины с ЧПУ,
которые задают траекторию перемещения инструмента и ряд технологических команд,
определяющих параметры резки листового материала
для получения из него заготовок известных форм и размеров.
Необходимые данные для моделирования маршрута инструмента
машины с ЧПУ определяет информация о раскройных картах,
которые разрабатываются на этапе проектирования раскроя
и порождают известные задачи оптимизации раскроя листового материала.
С точки зрения геометрической оптимизации задачи раскроя относятся
к классу задач раскроя--упаковки
(\textit{Cutting \& Packing}),
для которых, также как и для маршрутных оптимизационных проблем,
не известны алгоритмы решения полиномиальной сложности.
В данной работе задачи раскроя не рассматриваются.
Основное направление исследования в настоящей монографии
связано с моделированием маршрута инструмента
машин фигурной листовой резки с ЧПУ и
проблемой его оптимизации по временным и стоимостным параметрам.

В исходной задаче требуется осуществить последовательное
посещение всех контуров с целью осуществления резки по эквидистантам,
представляющим собой замкнутые кривые
(обсуждаются также и более сложные типы резки);
точки, определяющие начало и окончание реза,
могут при этом назначаться произвольно.
В интересах построения конкретных решений приходится,
однако, использовать дискретизацию эквидистант
и некоторые дополнительные преобразования последних в непустые конечные множества — мегаполисы,
что и делается в настоящей монографии
(см.~в этой связи \cite{intro01,intro02}).

Если рассматривать сформулированное научное направление в его полной общности,
то приходится признать, что адекватной математической теории здесь не разработано.
Имеются отдельные направления, среди которых особо отметим проблему полиномиальной разрешимости
для отдельных классов оптимизационных задач,
которые могут использоваться в качестве подзадач рассматриваемой проблемы.
Известные результаты, которые получены в последние годы в предметных областях,
связанных с разработкой алгоритмов дискретной оптимизации
и исследованием проблемы полиномиальной разрешимости,
при всей своей значимости не охватывают проблемы <<диапазонных>>
(в смысле размерности) задач и особенно задач,
осложненных ограничениями.
В монографии авторы исследуют вопросы разработки
теоретических и методологических основ решения проблемы
оптимальной маршрутизации инструмента для машин фигурной листовой резки с ЧПУ,
включая разработку адекватных математических моделей
и алгоритмов решения для исследуемой прикладной задачи.
Результаты работы могут быть использованы и для решения
других прикладных задач,
описываемых предложенными в монографии математическими моделями.

Монография структурно состоит из двух частей и пяти глав.

В первой главе рассмотрены основные понятия
фигурной листовой резки на машинах с ЧПУ,
формулируется содержательная постановка исследуемой проблемы,
приводятся общие постановки и классификация
возникающих оптимизационных маршрутных задач.
Здесь же приведена <<первичная>> математическая формализация
рассматриваемой проблемы и описана дискретная модель
некоторых сформулированных ранее оптимизационных задач,
основанная на использовании модели мегаполисов.

Во второй главе рассматриваются некоторые
практические аспекты оптимизации траектории
инструмента для машин листовой резки с ЧПУ:
описываются способы уменьшения термических деформаций
материала при оптимальной маршрутизации инструмента,
исследуется проблема точного вычисления целевых функций
на примере машины лазерной резки
\textit{ByStar3015}
и эффективность применения специальных техник резки
в сравнении со стандартной техникой <<резки по контуру>>.

Третья глава содержит описание математических моделей и методов,
используемых при решении задачи последовательного обхода
мегаполисов с условиями предшествования.

В четвертой главе исследованы задачи маршрутизации
с ограничениями и усложненными функциями стоимости.
Рассматриваются вопросы, связанные с
локальным улучшением эвристических решений.

В пятой главе приводятся описание разработанных авторами
алгоритмов для решения задач маршрутизации,
а также результаты вычислительных экспериментов,
содержащих данные решения некоторых практических
задач оптимизации маршрута инструмента для машин
фигурной листовой резки с ЧПУ.

Две первые главы образуют в своей совокупности
первую часть настоящей работы,
непосредственно связанную с решением инженерных задач,
относящихся к листовой резке на машинах с ЧПУ.
Здесь обсуждаются конкретные варианты весьма общей постановки,
указываются характерные
особенности
и обозначаются
на идейном уровне основные элементы этой общей постановки.
Особую значимость приобретает обсуждение
различных вариантов осуществления резки,
включая многие подробности,
важные в инженерном отношении,
а также характерные ограничения.
Последние существенно влияют на математическую постановку;
учет некоторых ограничений оказывается весьма затруднительным.

В первой главе подробно обсуждается
стандартная техника резки (резка по замкнутому контуру),
которая, как представляется,
более близка к известным математическим постановкам задач
о последовательном обходе мегаполисов с условиями предшествования
(данное обстоятельство существенно используется во второй части работы).
Упомянутые условия играют важную роль как на этапе инженерной постановки,
так и на этапе математического исследования.
Их конкретный вариант состоит (в данной задаче)
в необходимости более раннего вырезания внутренних контуров
деталей и <<внутренних>> деталей, то есть деталей,
располагаемых (после раскроя) внутри других (объемлющих) деталей,
что соответствует размещению по схеме <<матрешки>>.
Само решение задачи является многоэтапным,
и упомянутые
условия предшествования касаются всей совокупности упомянутых этапов.
В то же время сам характер этих условий оказывается
до некоторой степени удобным для их последующего учета
на этапе общей постановки;
они касаются выбора очередности достаточно
крупных фрагментов решения и имеют комбинаторный характер.

В первой части монографии обсуждаются также
различные варианты нестандартной техники резки
(цепная резка, резка с перемычками, резка <<змейкой>> и др.).
Вводятся важные понятия сегмента резки и базового сегмента резки,
определяющие общий взгляд на проблему классификации вариантов резки
(резка по замкнутому контуру, мультисегментная и мультиконтурная резки).
Понятия сегмента резки и базового сегмента являются
по сути объединяющими различные варианты резки в естественные классы,
допускающие исследование соответствующих конкретных вариантов
с единых позиций и существенно расширяющие существующую
классификацию задач маршрутизации инструмента
для машин листовой резки с ЧПУ.

Особое внимание уделено в монографии вопросам,
связанным с формализацией и математической постановкой
рассматриваемых инженерных задач.
Частично эти вопросы затрагиваются в первой части,
где проблемы формализации обсуждаются с позиций инженерного исследования;
решения трактуются как маршруты резки,
являющиеся объектами выбора исследователем с целью
по возможности максимального улучшения (совокупного)
результата при соблюдении комплекса разнообразных ограничений.
Такой подход позволяет сформулировать определенные ориентиры,
которые особенно полезны при разработке эффективных эвристических алгоритмов.
Само же применение эвристических методов для решения
практических задач представляется неизбежным.
Здесь же рассматривается задача точного вычисления целевых функций,
в рамках решения которой исследуются практические
вопросы определения зависимости фактической скорости резки
от числа кадров управляющей программы
(на примере машины лазерной резки
\textit{ByStar3015}),
описывается методика определения параметров
для целевой функции стоимости лазерной резки
с вычислением стоимостных параметров этой функции
для различных марок и толщин листовых материалов.

В результате вышеупомянутой и,
по смыслу, <<первичной>> формализации проблемы,
проведенной в первых двух главах монографии,
мы получаем дискретные задачи нелинейного программирования
большой размерности, представляющие в своей исходной постановке
серьезные затруднения как для качественного исследования,
так и для процедур поиска конкретных решений.
Определенные возможности для теоретического исследования
подобных задач открывает,
как представляется, весьма общий подход,
последовательно развиваемый во второй части
(третья, четвертая и пятая главы монографии)
и связанный с применением аппарата широко понимаемого
динамического программирования (ДП),
реализуемого в условиях ограничений исходной задачи.
Данный подход, естественно связываемый с идеями
Р. Беллмана и широко используемый, в частности,
в современной теории управления, требует, однако,
определенного переосмысливания самой постановочной части.
Так, выбор решения (маршрут резки в первой части)
полезно трактовать как выбор пары маршрут -- трасса,
где маршрут связывается уже с перестановкой индексов,
используемых для нумерации контуров вырезаемых деталей,
а трасса имеет смысл, подобный маршруту резки первой части.
При этом возникает определенная иерархия:
маршрут (в виде перестановки индексов)
определяет пучок согласованных с ним и
потому подчиненных ему трасс или траекторий,
которые уже перестановками, вообще говоря, не являются.
Маршрут позволяет занумеровать контуры, подлежащие резке,
а трасса определяет конкретный вариант их посещения
(точнее, посещения эквидистант, соответствующих данным контурам).
Имеется целый ряд обстоятельств, мотивирующих упомянутую иерархию.
Сейчас отметим только одно:
условия предшествования относятся,
строго говоря, к способу нумерации контуров.
Таким образом, эти условия порождают ограничения
именно в выборе перестановки индексов,
то есть в выборе маршрута, понимаемого в традиционном для ЗК смысле.
Это важное обстоятельство позволяет затем
использовать условия предшествования <<в положительном>>
направлении в смысле снижения сложности вычислений
(имеется в виду процедура на основе ДП).

Итак, во второй части монографии само понятие
решения определенным образом структурируется;
выделяются две компоненты:
маршрут
(как перестановка индексов)
и трасса,
или траектория.
Такая логика естественна с точки зрения теории управления,
элементы которой (имеются в виду задачи управления с дискретным временем)
используются в построениях второй части монографии.
При этом реализация трассы осуществляется в пределах пучка,
однозначно определяемого маршрутом.
Критерий качества предполагается аддитивным.
Это означает, что для каждого конкретного решения
значение критерия получается суммированием стоимостей,
характеризующих все этапы перемещений,
связанных с реализацией упомянутого
решения в виде пары маршрут -- трасса.

Для задач, связанных с листовой резкой,
исключительно важным является учет ограничений,
связанных с тепловыми деформациями материала
и порождаемыми этими деформациями эвристическими правилами
(т.~н. <<жесткостью>> листа и деталей),
сформулированными в первой главе монографии.
Характерной особенностью таких ограничений
является то, что все они формируются по мере
развития процесса резки и, по большому счету,
зависят от истории последнего, что определяет
принципиальное отличие рассматриваемых задач от
оптимизационных задач с фиксированными ограничениями.
Здесь опять-таки оказывается уместным двухуровневое представление решения,
поскольку целый ряд данных <<динамических>> ограничений удается представить
в терминах зависимостей от маршрута,
определяемого в виде перестановки индексов.

Учет динамических ограничений
осуществляется в настоящей монографии
посредством введения специальных функций стоимости,
которые объективно играют роль штрафов.
При этом, однако, возникают функции стоимости,
включающие зависимость от списка заданий,
уже выполненных на момент соответствующего перемещения.
Данная особенность существенно осложняет конструкции на основе ДП;
в этой связи сначала в третьей главе рассматривается случай,
когда вышеупомянутая зависимость от списка заданий отсутствует,
что позволяет привлечь для целей качественного исследования
более простую и понятную версию ДП.

Более общий случай,
когда зависимость функций стоимости от
списка заданий уже допускается,
рассматривается в четвертой главе.
С точки зрения применения аппарата ДП
оказывается более удобным использовать
при формализации задачи функции стоимости,
допускающие зависимость от списка еще не выполненных заданий.
Кроме того, по постановке допускаются условия предшествования,
которые в задачах, связанных с листовой резкой,
имеют ясный содержательный смысл:
внутренние контуры деталей должны вырезаться раньше внешних;
в случае расположения одних деталей <<внутри>> других
резка <<внутренних>> деталей должна осуществляться раньше,
чем резка <<внешних>>.

Для вышеупомянутой общей постановки
в рамках концепции двухуровневого решения
(определяемого всякий раз в виде пары маршрут -- трасса)
осуществляется построение специального расширения исходной задачи.
Потребность в данном расширении связана с учетом условий предшествования,
которые порождают <<неудобные>> ограничения на маршрут в целом.
Эти ограничения удается, однако,
эквивалентным образом преобразовать к условиям,
определяемым некоторым естественным правилом вычеркивания заданий из списка.
Итак, допустимость по предшествованию эквивалентным образом
заменяется допустимостью по вычеркиванию.
Последняя более удобна для целей применения аппарата ДП,
поскольку связывается с условиями на отдельные этапы процесса перемещений.
Грубо говоря,
такая допустимость нужным образом локализуется,
что и позволяет затем задействовать конструкции
широко понимаемого ДП и получить затем уравнение Беллмана.

В связи с трудностями вычислительной реализации
на основе этого уравнения конструируется
система преобразования так называемых слоев функции Беллмана.
Речь идет о том, чтобы при условиях предшествования
(а это типичный случай в рассматриваемом классе задач)
ограничиться частичным построением массива функции Беллмана,
а точнее, системы ее слоев.
Последние, в свою очередь, определяются
соответствующими слоями пространства позиций,
в определении которых задействуются
так называемые существенные списки заданий.

Разумеется, даже при использовании усеченного
вышеупомянутым способом массива значений
функции Беллмана практическое использование
(оптимальной) процедуры на основе ДП возможно лишь
в задачах умеренной размерности.
В то же время представляют интерес методы
локального улучшения маршрутных решений
посредством применения оптимизирующих вставок,
при построении которых удается уже задействовать схему на основе ДП.

Важно отметить, что само применение оптимизирующих вставок
в задаче маршрутизации с условиями предшествования и стоимостями,
зависящими от списка заданий,
потребовало серьезного теоретического обоснования,
которое приведено в четвертой главе.

В целях более глубокого воздействия на
исходное эвристическое решение
(имеется в виду решение задачи достаточно большой размерности)
предлагается использовать итерационные процедуры с варьированием начала вставки.
Конкретные варианты построения таких процедур
приведены в пятой главе,
в которой также содержатся соответствующие результаты
вычислительного эксперимента.

В целом использование аппарата ДП
на уровне вставок,
в том числе применение режима итераций,
представляется реальной возможностью включения упомянутого
(теоретического) аппарата в процесс решения маршрутных задач,
представляющих практический интерес.
Здесь особенно важным кажется
разработка методов и алгоритмов решения задач с ограничениями разных типов.
В частности, это касается динамических ограничений,
которые складываются по мере развития процесса.
Данный тип ограничений <<обрабатывается>> в настоящей монографии
(это уже отмечалось ранее)
посредством введения функций стоимости с зависимостью от списка заданий,
что требует конструирования таких функций
и насчитывания соответствующих массивов их значений.
Последнее существенно осложняет вычисления
(особенно при использовании ДП).
Поэтому представляется важной разработка
эффективных эвристических алгоритмов,
для которых предварительное глобальное построение
вышеупомянутых массивов значений функций стоимости не делается;
вместо этого осуществляется построение локальных массивов,
реализующихся по мере развития процесса.
Один из таких алгоритмов приведен в пятой главе.

Оценивая содержание монографии,
можно отметить основательную инженерную и математическую проработку материала.
Обсуждаются различные варианты фигурной резки и
намечены обобщения известных понятий,
позволяющие применять специальные математические методы.
В частности, предлагается при описании процесса резки
использовать естественную модель мегаполисов,
в рамках которой допускается на каждом этапе
возможность выбора точки врезки из заданной
и достаточно представительной совокупности.
Это позволяет, с одной стороны,
свести трудно
решаемую непрерывно-дискретную задачу нелинейного программирования
к задаче дискретной оптимизации, а с другой --
существенно расширить возможности получения оптимальной
(или близкой к ней)
управляющей программы резки в сравнении
с тем случаем,
когда точка врезки фиксирована для каждого контура.

Отдельного обсуждения заслуживает вопрос о применении ДП.
Прежде всего, следует отметить, что ДП в изложении,
принятом в настоящей монографии,
является теоретическим методом.
На его основе, конечно, может быть построен алгоритм,
применимый для построения оптимальных решений
в задачах малой размерности. Но все же это уже следствие.
Роль ДП как общего метода решения экстремальных задач
очень велика.
Но, пожалуй, в наибольшей степени эта роль проявляется
в задачах теории управления, что связано прежде всего
с работами Р. Беллмана.
В настоящей монографии конструкции широко понимаемого ДП
идейно соответствуют взгляду на данный метод,
принятому в теории управления.
В частности, значительное место занимает
получение уравнения Беллмана и следствий этого уравнения,
связанных с использованием условий предшествования
в положительном направлении.
В то же время вывод уравнения Беллмана
опирается на специальную процедуру расширения исходной задачи,
в основе которой находится эквивалентное преобразование системы ограничений.
Итак, широко понимаемое ДП является
(в настоящей монографии)
прежде всего теоретическим методом,
позволяющим изучать структуру очень сложных задач маршрутизации.
Грубо говоря, ДП <<справляется>> с разнообразными ограничениями,
проявляя при этом большую универсальность
(так, например, данный метод без каких-либо
существенных изменений идейного характера
удалось использовать при неаддитивном агрегировании затрат и,
в частности, в маршрутных задачах <<на узкие места>>).

В то же время в дискретной оптимизации ДП
нередко воспринимается только как алгоритм;
здесь имеется в виду прежде всего применение
ДП для решения ЗК
(в англоязычной редакции -- \textit{TSP}).
Вполне естественным является тот факт,
что в такой <<простой>> по постановке задаче,
как ЗК, алгоритм на основе ДП нередко проигрывает
другим алгоритмам
(например, методу ветвей и границ).
Это и неудивительно в силу определенной <<всеядности>> ДП.
Однако вопрос о месте ДП в решении
сложных задач маршрутизации с ограничениями
все же стоит достаточно остро.
В настоящей монографии, наряду с организацией оптимизирующих вставок
с применением ДП
развивается также следующий взгляд
на упомянутую проблему.
Речь идет о тестировании эвристик на
задачах маршрутизации умеренной размерности,
но при тех же ограничениях,
что и реальная исходная постановка
(таким образом, реализуется своеобразная
<<дрессировка>> эвристик; при этом,
конечно, требуется достаточно представительная выборка решенных задач).
Итак, принимая точку зрения о неизбежности эвристик
в маршрутных задачах большой размерности,
мы с помощью ДП стараемся <<наладить>>
сравнение эвристик на выборках задач умеренной размерности.

Сейчас мы совсем кратко коснемся имеющихся источников,
обозначая тем самым сложившиеся направления исследований.

В связи с конкретной задачей оптимизации управления
режущим инструментом машин листовой резки с ЧПУ
отметим работы
\cite{intro03,intro04,intro05,intro06,intro07,intro08,intro09,Cha10`,intro11,intro12}
и обзор \cite{intro13}.
В целом ряде российских и зарубежных исследований предполагалось,
что точка врезки инструмента в листовой материал
выбрана заранее для каждого вырезаемого контура.
Это позволяет использовать модель ЗК,
но снижает практическую ценность,
поскольку уже на постановочном уровне
исключает из рассмотрения основную часть
полезных вариантов решения.
Еще одна группа новых зарубежных публикаций
описывает алгоритмы решения задач,
в которых точки врезки для каждого контура
выбираются из некоторого конечного множества
(что было предложено авторами монографии ранее),
но применяется только стандартная техника резки
(резка по замкнутому контуру – задача
\textit{GTSP}).
В качестве математической модели оптимизационной задачи
в этом случае используется модель обобщенной задачи коммивояжера.
Более общий случай -- задача резки с конечным набором точек врезки:
резка может начаться только в одной из заранее заданных точек на контуре,
однако контур может быть вырезан за несколько подходов,
по частям.
Некоторые алгоритмы для решения частных случаев
этой задачи описаны, например,
в~\cite{intro12,intro14}.
Следует отметить, что для задачи т. н. <<произвольной резки>>,
когда не накладывается никаких ограничений
на выбор точек начала и конца резки,
а также на последовательность резки контуров и их частей,
пока не предложено формальных математических моделей
и каких-либо алгоритмов решения.
Кроме того, во всех современных исследованиях
остаются практически не рассмотренными
вопросы учета технологических требований резки,
связанных с упомянутой выше <<жесткостью>> материала,
порождающей ограничения в выборе точек врезки в материал
и последовательность резки контуров вырезаемых деталей.
На практике эти вопросы часто решаются
с использованием интерактивных методов проектирования,
когда пользователь системы автоматизированного проектирования
управляющих программ для машин листовой резки с ЧПУ
в диалоговом режиме определяет и набор сегментов резки
и точки врезки для каждого сегмента.
Кажущаяся естественной идея использования результатов
моделирования тепловых полей для соблюдения
технологических требований термической резки
пока не получила адекватной для практики реализации.

Другой особенностью публикаций
по рассматриваемой оптимизационной проблеме
является отсутствие разработок точных алгоритмов.

В связи с исследованиями ЗК отметим сейчас обстоятельный обзор
\cite{intro15,intro16,intro17},
работы \cite{Cha4`,Cha16`},
связанные с применением ДП для решения ЗК,
а также более поздние монографии \cite{intro20,intro21}.
Отметим, что в обзоре
\cite{intro15,intro16,intro17}
обсуждаются также задачи типа ЗК
(то есть варианты ЗК с теми или иными особенностями);
в этой связи см. также \cite{intro22}.
Имеется и много других работ, идейно ориентированных на подходы,
сложившиеся в связи с решением ЗК.
Это касается, например,
использования метода ветвей и границ
\cite{Cha17`},
который находит широкое применение и в других задачах дискретной оптимизации,
в частности, в задачах с условиями предшествования
\cite{intro24}.
Традиционно много публикаций
появляется в последнее время в связи с разработкой
различных вариантов метаэвристик \cite{intro25,intro26,intro27,intro28},
однако они ориентированы в основном
на решение ЗК без дополнительных ограничений.

Несколько слов о работах авторского коллектива монографии
и его соавторов по теме работы.
Решение задач оптимизации управления
инструментом для машин листовой резки с ЧПУ,
помимо уже упомянутых публикаций,
рассматривалось авторами, в частности,
в~\cite{intro29,intro30,intro31,intro32,Cha15`,intro35,intro36,intro37}.
В~\cite{intro29,intro30}
были сформулированы эвристические правила
(правила <<жесткости>>) резки фигурных заготовок
на машинах для термической резки листовых материалов.
В~\cite{intro31}
для формализации задачи оптимизации маршрута
для случая стандартной техники резки
предложено при программировании в CAM
(системе
управляющих программ резки)
использовать
математическую модель обобщенной задачи коммивояжера
с дополнительными ограничениями.
В~\cite{intro32} применена модель ДП
для решения задачи о последовательном
обходе мегаполисов А. Г. Ченцова,
позволяющая разрабатывать точные алгоритмы
решения маршрутной задачи со сложными видами ограничений.
Для задач большой размерности
был разработан ряд приближенных алгоритмов
(см., например, \cite{Cha15`,intro34}).
Впоследствии на основе введенных понятий <<сегмента резки>>
и <<базового сегмента резки>> \cite{intro35,intro36}
проведено обобщение полученных результатов
для случая задач с заранее определенным набором сегментов резки,
а в \cite{intro37} реализован алгоритм,
учитывающий динамические ограничения жесткости
детали при выборе точек врезки.
В работах \cite{intro38,intro39}
было показано, что этот выбор
может быть сделан на основе моделирования температурных полей
при термической резке материала.
Вопросы оптимальности разрабатываемых алгоритмов
при применении метода ДП были рассмотрены в
\cite{intro40,intro41,intro42,intro43}.
В \cite{intro44,intro45,intro46,intro47}
исследованы вопросы точного вычисления
целевых функций и эффективность применения
специальных техник резки при решении
практических оптимизационных задач
лазерной резки деталей на машинах с ЧПУ.

Построения, связанные с используемым в монографии
вариантом метода ДП, восходят к
\cite{Cha1`} и последующей большой серии
журнальных статей,
среди которых сейчас отметим лишь некоторые
(см.~\cite{Cha8`,intro51,intro52,intro53,intro54,intro55,intro56}).
Упомянутые работы
\cite{Cha8`,intro51,intro52,intro53,intro54,intro55,intro56}
в основном посвящены решению абстрактных задач маршрутизации,
но математический аппарат, разработанный в этих работах,
оказался полезным и для решения различных прикладных задач.
В числе последних
необходимо
отметить практические задачи первой части монографии,
связанные с разработкой УП для машин с ЧПУ.
С другой стороны, развиваемые в этих работах подходы
нашли применение в некоторых задачах атомной энергетики,
связанных с проблемой снижения облучаемости работников АЭС
при выполнении комплекса работ.
Одна из постановок такого рода связана с
актуальной проблемой демонтажа энергоблока АЭС,
выведенного из эксплуатации.
Возможно также применение к решению задач,
возникающих при аварийных ситуациях,
подобных Чернобылю и Фукусиме.
В этой связи отметим монографию \cite{Cha2`}.

Полезно отметить, что существует много других прикладных задач
с элементами маршрутизации и ограничениями,
подобными рассмотренным в монографии.
Отметим задачи о морских и авиационных перевозках,
где также могут возникать условия предшествования,
определяющие, в частности, порядок перевозки грузов
между промежуточными пунктами (портами, аэродромами).
Элементы маршрутизации при-сутствуют в задаче
авиапожарного патрулирования лесных массивов.

Монография подготовлена при финансовой поддержке РФФИ
(проект 20-08-00873).

% Вступление от Буслаевой 17.07.2016
%^^^^^^^^^^^^^^^^^^^^^^^^^^^^^^^^^^^^^^^^^^^^^^^^^^^^^^^^^^^^^^^^^^^^^^^^^^^^^^^
