% !TeX root = ..

\section{
  Математическая постановка задачи.
  Объект~исследования и~некоторые характерные ограничения
}
\label{sect:3.3}
\setcounter{equation}{0}

Рассмотрим непустое множество $X$,
которое в конкретной постановке будет обычно
отождествляться с плоскостью либо с прямоугольником на плоскости.
Фиксируем натуральное число
$N\in \bbn, N \geqslant 2,$
а также $N$
(непустых конечных)
множеств
\bfn
  \label{3.3.1}
  M_1\in \mathrm{Fin}(X),\ldots,M_N\in \mathrm{Fin}(X),
\efn
именуемых в дальнейшем мегаполисами.
Кроме того, фиксируем (непустые) отношения
\bfn
  \label{3.3.2}
  \bbm_1\in \cp^\prime(M_1 \times M_1),\ldots,\bbm_N\in
  \cp^\prime(M_N\times M_N),
\efn
характеризующие возможности исполнителя в части выполнения работ в каждом
из упомянутых мегаполисов.
Итак,
$\bbm_1,\,\ldots,\bbm_N$ --- непустые множества,
и при этом
$$
  \bbm_1\subset M_1\times M_1,\ldots,\bbm_N\subset M_N\times M_N.
$$

С каждым из отношений (\ref{3.3.2})
связываем множество вторых компонент
упорядоченных пар --- элементов данного отношения:
\bfn
  \label{3.3.3}
  \mathbf{M}_j \df \{\mathrm{pr}_2(z):\, z\in \bbm_j\}\in
  \cp^\prime(M_j)\ \ \fa j\in \ov{1,N}.
\efn

Пусть
$x^o\in X$
есть точка, именуемая базой и определяющая
начало процесса маршрутизации.
Будем предполагать, что
\bfn
  \label{3.3.4}
  (x^o\notin M_j\ \ \fa j\in \ov{1,N})\,\&\,(M_p \cap\,M_q =
  \emp\ \ \fa p\in \ov{1,N}\ \ \fa q\in \ov{1,N}\setminus \{p\})
\efn
В (\ref{3.3.4}) имеем условия, типичные для задач маршрутизации.
Введем также
\bfn
  \label{3.3.5}
  \mathbb{X} \df \{x^o\} \cup \Bigl(\bigcup\limits_{i=1}^N
  M_i\Bigl)\in \mathrm{Fin}(X), \ \mathbf{X}\df \{x^o\} \cup
  \Bigl(\bigcup\limits_{i=1}^N \mathbf{M}_i\Bigl)\in \mathrm{Fin}(\bbx).
\efn

\begin{zam}
\label{z3.3.1}
Обсудим нужный в дальнейшем вариант $(\ref{3.3.1})$---$(\ref{3.3.5})$,
где
$X = \bbr \times \bbr$ (плоскость).
Полагаем, что в $X$ размещены заготовки деталей
(для краткости будем называть их деталями)
$\mathbf{D}_1,\,
\ldots,\mathbf{D}_n,$
где
$n\in \mathbb{N}, 2\leqslant n.$
При этом
$$
  \mathbf{D}_1\su \bbr \times \bbr,
  \,\ldots,
  \mathbf{D}_n\su \bbr \times \bbr.
$$
Каждое из множеств
$\mathbf{D}_1,\,\ldots,\mathbf{D}_n$
считаем непустым,
ограниченным, замкнутым и телесным
(имеющим внутренние точки).
Если
$j\in \ov{1,n},$
то сопоставляем детали $\mathbf{D}_j$ ее границу
$\Gamma_j,$
полагая, что она получается объединением непересекающихся непустых,
ограниченных и замкнутых множеств
$\Gamma_{j,1},\,\ldots,\Gamma_{j,\bmn_j},$
где $\bmn_j\in \bbn$.
Каждое из множеств
$\Gamma_{j,k}, k\in \ov{1,\bmn_j}$
является в данном построении замкнутой кривой,
соответствующей либо внешнему,
либо одному из внутренних контуров,
ограничивающих деталь
$\mathbf{D}_j.$ Итак,
$$
  \Gamma_j = \bigcup\limits_{k=1}^{\bmn_j} \Gamma_{j,k}\ \ \fa j\in \ov{1,n}.
$$

Внешний контур каждой детали полагаем выделенным,
он соответствует завершению резки детали.

Примем, что возле каждого контура
$\Gamma_{j,k}, j\in \ov{1,n},\ k\in \ov{1,\bmn_j}$
имеется замкнутая кривая --- эквидистанта
$\Om_{j,k},$ расположенная с внешней
(по отношению к контуру)
стороны и не пересекающаяся с $\mathbf{D}_j$.
Предположим, что
$\Gamma_{j,k}$ и $\Om_{j,k}$ близки.
Эквидистанта
(по большому счету)
копирует контур и создается
(см. первую часть)
для осуществления определенного  запаса на резку по
данному контуру. Разумеется,
$$
  \Om_{j,k} \cap \mathbf{D}_j = \emp\ \
  \fa j\in \ov{1,n}\ \
  \fa k\in \ov{1,\bmn_j}.
$$

Условимся,
что каждое из множеств
$\Om_{j,k}, j\in\ov{1,n},\ k\in\ov{1,\bmn_j}$
непусто, ограничено и замкнуто.
Пусть
(в данном рассуждении)
\bfn
  \label{3.3.6}
  N \df \sum\limits_{j=1}^n\bmn_j.
\efn

Теперь мы нумеруем все эквидистанты подряд, используя в качестве индексов числа из
$\ov{1,N}$
(более подробно это излагается далее).
Тогда
$N \hm \in \bbn, N \geqslant 2$,
число $N$ определяет фактическую размерность задачи.
С каждым значением
$j\in \ov{1,N}$
у нас теперь связана эквидистанта контура некоторой детали.
Мы рассматриваем далее всю совокупность эквидистант
\bfn
  \label{3.3.7}
  \Om_{1,1},\,\ldots,\Om_{1,\bmn_1},
  \Om_{2,1},\,\ldots,\Om_{2,\bmn_2},
  \,\ldots,
  \Om_{n,1},\,\ldots,\Om_{n,\bmn_n}.
\efn

Их теперь целесообразно, как уже отмечалось, занумеровать подряд, получая систему
\bfn
  \label{3.3.8}
  \Om^{(1)},\,\ldots,\Om^{(N)}
  ,
\efn
учитывая
$(\ref{3.3.6})$ и $(\ref{3.3.7})$.
Теперь возле каждого из множеств
(замкнутых кривых)
$\Om^{(j)}$ в $(\ref{3.3.8})$
размещаем мегаполис $M_j,$
точки которого --- точки врезки и выключения инструмента --- сгруппированы в пары
(при этом точки врезки и выключения инструмента в совокупности,
составляющие $M_j,$
располагаются с внешней по отношению к $\Om^{(j)}$ стороны в следующем порядке:
для индексов
$s\in \ov{1,n}$ и $t\in \ov{1,\bmn_s}$
со свойством $\Om^{(j)} = \Om_{s,t}$
упомянутые точки располагаются с внешней
стороны детали
$\mathbf{D}_s$ по
отношению к $\Om_{s,t})$.
С этих пор мы, в принципе, можем <<забыть>> о самих деталях
и рассматривать эквидистанты и <<привязанные>> к ним мегаполисы
(последние также играют
роль дискретизированных эквидистант,
а точнее -- эквидистант второго уровня).

Следует, однако, учесть,
что при любом варианте маршрутизации длины проходимых
инструментом эквидистант $(\ref{3.3.8})$
не изменяются, остаются одними и теми же.
Если иметь в виду оптимизацию времени исполнения всех операций,
то суммарное время
резки эквидистант является общей константой,
а потому может не учитываться  при
построении аддитивного критерия задачи на быстродействие.

Сумма стоимостей внешних перемещений составляет время холостого хода,
которое пропорционально пройденному расстоянию.
По этой причине мы можем рассматривать
здесь метрический
(точнее, евклидов) вариант задачи.

В то же время продвижения от точки врезки к эквидистанте и далее
по окончанию резки по данной эквидистанте
к точке выключения инструмента, осуществляемые в металле,
могут вносить ощутимый вклад в значение совокупного критерия,
так как они осуществляются <<в медленном времени>>.

Заметим, что замена (при $j\in \ov{1,N})$ эквидистант $\Om^{(j)}$
мегаполисами
$M_j$
мотивируется соображениями, связанными с <<безопасным>> осуществлением врезки
и касающимися вычислительной реализации
(имеется в виду построение локальных задач переборного типа).
Разделяя эти два обстоятельства,
можно было бы на идейном уровне говорить и о второй,
более удаленной от контура, эквидистанте,
дискретизацию которой реализует соответствующий мегаполис.
\hfill $\Box$
\end{zam}

Возвращаясь к общим построениям,
связанным с (\ref{3.3.5}),
будем полагать, что
у нас имеются ограничения, связанные с вопросом достижимости извне городов
мегаполиса
(городами в нашей интерпретации являются точки множеств (\ref{3.3.1})).

Конечно,  на макроуровне нашей постановки речь идет о перемещениях вида
\bfn
  \label{3.3.8`}
  \begin{array}{c}
    (x_o = x^o) \rightarrow (x_{1,1}\in M_{\al(1)} \rightsquigarrow
    x_{1,2}\in M_{\al(1)})\rightarrow \ldots \\  \rightarrow(x_{N,1}\in M_{\al(N)}
    \rightsquigarrow x_{N,2}\in M_{\al(N)})
  \end{array}
\efn
при условиях
\bfn
  \label{3.3.8``}
  (x_{1,1},x_{1,2})\in \bbm_{\al(1)},\,\ldots,(x_{N,1},x_{N,2})\in \bbm_{\al(N)},
\efn
где
$\al$ --- перестановка  индексов из $\ov{1,N},$
выбор которой может быть стеснен
условиями предшествования.
Перестановка $\al,$ а также <<города>>
$x_{1,1}$, $x_{1,2}$,
\ldots,$x_{N,1}$, $x_{N,2}$
выбираются исследователем с целью оптимизации критерия качества,
который будет определен ниже.
Условия
(\ref{3.3.8`}), (\ref{3.3.8``})
дополняются и некоторыми
другими ограничениями, связанными, в частности, с внешними перемещениями
$$
  x^o\longrightarrow x_{1,1},\, x_{1,2} \longrightarrow x_{2,1},\,\ldots,x_{N-1,2}
  \longrightarrow x_{N,1}
$$
(имеется в виду случай $N>3)$.
В этой связи, следуя \cite{Cha3`},
фиксируем отображения
\bfn
  \label{3.3.9}
  A_1:\,\mathbf{X}\longrightarrow \cp^\prime(M_1),\ldots,A_N:\,
  \mathbf{X}\longrightarrow \cp^\prime(M_N)
  ,
\efn
именуемые далее мультифункциями.
Согласно (\ref{3.3.9})
при
$x\in \mathbf{X}$ и $t\in \ov{1,N}$
в виде $A_t(x)$
имеем непустое п/м $M_t$
(то есть $A_t(x)\su M_t),$
интерпретируемое как своеобразная область достижимости
точек $M_t$
(то есть городов в $M_t)$
из состояния $x.$
Полагаем при этом, что точки
множества $A_t(x)$
и только они достижимы в мегаполисе
$M_t$ из состояния $x.$

\begin{zam}
  \label{z3.3.2}
Обсудим одну естественную конкретизацию $(\ref{3.3.9})$:
будем рассматривать
требование, состоящее в том, чтобы новая точка врезки отстояла <<достаточно далеко>>
от предыдущей.
Сделаем это сначала на идейном уровне,
полагая в условиях замечания~$\ref{z3.3.1}$,
что при $s\in \ov{1,N}$ и $x\in \mathbf{X}\sm M_s$
множество $A_s(x)$ определяется правилом
\bfn
  \label{3.3.10}
  A_s(x) \df \{\mathrm{pr}_1(z):\,z\in \bbm_s\
  d_s -\eps_s\leqslant\rho\bigl(x,\mathrm{pr}_1(z)\bigl)\},
\efn
где $\eps_s\in [0,\infty[$ --- параметр точности, а
$$
  d_s = \max\limits_{h\in \bbm_s}\rho\bigl(x,\mathrm{pr}_1(h)\bigl).
$$

Здесь
$\rho= \rho(\cdot,\cdot)$ --- обычное евклидово расстояние на
плоскости.
Заметим, что с практической точки зрения $(\ref{3.3.10})$
представляет интерес в~случае, когда
$x = \mathrm{pr}_2(z),$
где $z\in \bbm_j$ при
$j\in \ov{1,N}\sm \{s\}.$
Поскольку в этом случае $\mathrm{pr}_1(z)\approx \mathrm{pr}_2(z),$
конструкция $(\ref{3.3.10})$
реализует (в некоторой степени) идею достаточного
<<удаления>> новой точки врезки от предыдущей.
По~самому смыслу задачи
рассматриваемый случай $j$ отвечает той ситуации,
когда на соответствующем маршруте
индекс $j$ непосредственно предшествует $s.$

При практической реализации возможны различные модификации правила $(\ref{3.3.10})$.
Эти модификации будут использоваться и отдельно оговариваться при компьютерном
моделировании.
В общих построениях каких-либо дополнительных существенных ограничений
на $(\ref{3.3.9})$ накладываться не будет,
что собственно и позволяет конструировать
данные мультифункции <<под задачу>>.
\hfill $\Box$
\end{zam}

Примем
в дальнейшем выполненными некоторые естественные условия,
позволяющие <<встраивать>> множества --- значения отображений (\ref{3.3.9}) ---
в схемы перемещений между:
1) базой и мегаполисами,
2) самими мегаполисами.
Итак, предроложим далее, что
\bfn
  \label{3.3.11}
  \fa j\in\ov{1,N}\ \ \fa x \in \mathbf{X}\ \exists\, z\in
  \bbm_j:\,\mathrm{pr}_1(z)\in A_j(x).
\efn

В (\ref{3.3.11}), на первый взгляд, имеется некоторая избыточность:
при $j \hm \in\ov{1,N}$
допускается рассмотрение перемещений из
$x\in \mathbf{M}_j$ в точки
(то есть в <<города>>)
из множества $\{\mathrm{pr}_1(z);\, z\in \bbm_j\},$
что никогда не будет возникать при реализации процесса последовательного обхода мегаполисов.
Допустим,
что значения операторов (\ref{3.3.9})
содержательны лишь в
ситуациях, отвечающих перемещению из $x^o$ в какой-либо мегаполис и перемещениям
из одного мегаполиса в другой.
Нереализуемым в <<маршрутных>> решениях ситуациям,
когда при
$j\in \ov{1,N}$ и $x\in \mathbf{M}_j$
рассматривается перемещение снова
в точки мегаполиса $M_j,$ можно сопоставить в качестве значения
$A_j(x)$ оператора
$A_j$ все множество $M_j$ с тем,
чтобы формально удовлетворить условию (\ref{3.3.11}).

Таким образом, определяя значения операторов (\ref{3.3.9})
<<по настоящему>> для
ситуации перемещения из $x^o$ в мегаполисы и при перемещении из одного мегаполиса
в другой, во всех прочих (нереализуемых) случаях можно просто отождествить значение
соответствующего оператора с мегаполисом, имеющим тот же номер.
В итоге, если условия
(\ref{3.3.11}) будут выполнены в случае потенциально возможных перемещений, то
(при упомянутом способе доопределения)
они будут выполняться и для всевозможных
$j\in \ov{1,N}$ и $x\in \mathbf{X}.$

Всюду в дальнейшем полагаем (\ref{3.3.11}) выполненным,
имея в виду, что для
содержательных ситуаций, отвечающих
$j\in \ov{1,N}$ и $x\in \mathbf{X},$
значения
$A_j(x)$ <<правильно>> определены, а для всех прочих ситуаций соответствующие
значения <<достроены>> (доопределены) вышеупомянутым способом:
имеется в виду
вариант $A_j(x) = M_j$ при $x\in \mathbf{M}_j.$


Теперь, используя (\ref{3.3.11}),
можно ввести модификации мультифункций
(\ref{3.3.9}), отвечающие перемещениям из $\mathbf{X}$ в множества
$\bbm_1,\,\ldots,\bbm_N$ (\ref{3.3.2}).
Итак, введем в рассмотрение
мультифункции
\bfn
  \label{3.3.12}
  \bba_1:\,\mathbf{X}\longrightarrow \cp^\prime(\bbm_1),
  \,\ldots,\bba_N:\,\mathbf{X}\longrightarrow \cp^\prime(\bbm_N),
\efn
а именно:
при $j\in \ov{1,N}$ полагаем, что отображение
$$
  \bba_j:\,\mathbf{X}\longrightarrow \cp^\prime(\bbm_j)
$$
определяется условиями
\bfn
  \label{3.3.13}
  \bba_j(x) \df \{z\in \bbm_j |\,\mathrm{pr}_1(z) \in
  A_j(x)\}\ \ \fa x\in \mathbf{X}.
\efn

Из (\ref{3.3.11}) следует, что определения (\ref{3.3.12}), (\ref{3.3.13})
корректны:
мы действительно получаем отображения, значениями которых
являются непустые множества.

Условимся через $\bbp$ обозначать множество всех перестановок индексов из
$\ov{1,N}:\,\bbp \df (\mathrm{bi})[\ov{1,N}],\, \bbp \neq \emp.$
Элементы множества $\bbp$ именуем (полными) маршрутами.
В частности,
при $\al\in \bbp$ имеем
$$
  \al(1)\in \ov{1,N},\ldots,\al(N) \in \ov{1,N}.
$$

{\bf Трассы, согласованные с маршрутами.}
С учетом (\ref{3.3.8`}) и
(\ref{3.3.8``}) логично связывать с каждым маршрутом из $\bbp$ соответствующий
кортеж упорядоченных пар, имеющий <<длину>> $N.$
Сделать это можно разными способами,
а потому будем говорить о множестве таких кортежей.
Более того,
в~определении данного множества будут учтены дополнительные ограничения,
связанные с мультифункциями (\ref{3.3.12}), (\ref{3.3.13}).

Условимся сначала о следующем соглашении:
через $\bbz$ обозначим в дальнейшем
множество всех кортежей
\bfn
  \label{3.3.14}
  (z_i)_{i\in\ov{0,N}}:\,\ov{0,N}
  \longrightarrow \bbx\times \mathbf{X}
  .
\efn

Кортежи (\ref{3.3.14}) являются,
строго говоря,
отображениями из
$\ov{0,N} \hm = \{l\in \bbn_o |\,l\leqslant N\}$
в декартово произведение $\bbx\times \mathbf{X}$
множеств $\bbx$ и $\mathbf{X};$
при $j\in \ov{0,N}$ упорядоченная пара
$z_j\in \bbx\times \mathbf{X}$
есть всякий раз значение соответствующего отображения в
точке $j.$

Если
$\al\in \bbp,$
то полагаем в дальнейшем, что
\bfn
  \label{3.3.15}
  \mathbf{Z}_\al \df \{(z_i)_{i\in\ov{0,N}}\in \bbz |\,\bigl(z_o =
  (x^o,x^o)\bigl)\,\&\,(z_t\in \bbm_{\al(t)}\ \ \fa t\in \ov{1,N})\}
  ,
\efn
получая некоторое предваряющее определение
(впрочем, в целом ряде частных случаев
оно будет достаточным для наших построений).
Заметим в связи с (\ref{3.3.15}),
что при
$\al\in \bbp, (z_i)_{i\in\ov{0,N}}\in \mathbf{Z}_\al$ и $j\in \ov{1,N}$
имеем, в частности, согласно (\ref{3.3.3}) и (\ref{3.3.5}),
что
$\mathrm{pr}_2(z_{j-1})\in \mathbf{X},$
а потому (см. (\ref{3.3.9})) определено непустое множество
$A_{\al(j)} \bigl(\mathrm{pr}_2(z_{j-1})\bigl)$:
$$
  A_{\al(j)}\bigl(\mathrm{pr}_2(z_{j-1})\bigl)\su M_{\al(j)}.
$$

С учетом этого полагаем теперь, что
\bfn
  \label{3.3.16}
  \cz_\al \df \{(z_i)_{i\in\ov{0,N}}\in \mathbf{Z}_\al |\,
  \mathrm{pr}_1(z_s)\in  A_{\al(s)}\bigl(\mathrm{pr}_2(z_{s-1})\bigl)\ \ \fa
  s\in \ov{1,N}\}\ \ \fa \al\in \bbp.
\efn

Заметим, что из \cite[(3.21)]{Cha3`}
легко извлекается свойство
\bfn
  \label{3.3.17}
  \cz_\al\neq \emp\ \ \fa \al\in \bbp.
\efn

В этой связи в силу \cite[(3.5)--(3.7)]{Cha3`}
определение
$\cz_\al,$ где
$\al\in \bbp,$
данное в \cite[64]{Cha3`}, сводится к
(\ref{3.3.16}),
а тогда (\ref{3.3.17}) извлекается из
\cite[(3.21)]{Cha3`}.
С~учетом (\ref{3.3.5}) имеем, что множество
$\bbx \times \mathbf{X}$ конечно,
а тогда конечно и множество $\bbz$
(декартова степень непустого конечного
множества).
Стало быть (см. (\ref{3.3.15})),
при $\al\in \bbp$ множество
(\ref{3.3.15}) также конечно, что приводит в силу (\ref{3.3.16})
к конечности каждого из множеств
$\cz_\al, \al\in \bbp.$
С учетом (\ref{3.3.17}) имеем, что
\bfn
  \label{3.3.18}
  \cz_\al\in \mathrm{Fin}(\mathbf{Z}_\al)\ \ \fa \al\in \bbp.
\efn

Из (\ref{3.3.15}) и (\ref{3.3.18}) вытекает в свою очередь, что
\bfn
  \label{3.3.19}
  \cz_\al\in \mathrm{Fin}(\bbz)\ \ \fa \al\in \bbp.
\efn

В силу (\ref{3.3.15}) и (\ref{3.3.17})
имеем свойство
$\mathbf{Z}_\al\neq \emp\ \ \fa \al\in \bbp.$
Это означает, что
\bfn
  \label{3.3.20}
  \mathbf{Z}_\al\in \mathrm{Fin}(\bbz)\ \ \fa \al\in \bbp
\efn
(конечность множеств (\ref{3.3.15}) уже отмечалась ранее).
Свойства (\ref{3.3.19}) и~(\ref{3.3.20})
следуют на самом деле из \cite[(3.21)]{Cha3`}.
Итак, с каждым маршрутом $\al\in\bbp$
связывается непустое конечное множество
кортежей, определяемое в~(\ref{3.3.16});
упомянутые кортежи называем трассами
или траекториями, согласованными с маршрутом $\al.$

{\bf Условия предшествования.}
В настоящем пункте мы подробно рассмотрим вопрос,
связанный с формализацией условий предшествования.
Речь идет об очень важном
варианте ограничений, который в задачах,
обсуждаемых в первой части, связан с
обеспечением следующих технологических условий:
для каждой детали резка внутренних
контуров должна предшествовать резке внешнего контура.
Аналогичное требование
предъявляется к резке <<вложенных>> деталей.

Заметим, что в других прикладных задачах условия предшествования могут иметь
иной содержательный смысл.
Так, в задачах о морских и авиационных перевозках
могут задаваться директивные условия перемещения грузов из одного пункта
(отправителя) в другой (получатель).
Соответственно, такие упорядоченные пары
пунктов иногда называют адресными.
У каждой такой пары выделяют первый элемент
в качестве отправителя (груза, сообщения)
и второй элемент в качестве получателя.
Таким образом, в упомянутых задачах о перевозках речь идет о перемещении грузов в
направлении от отправителя к получателю,
причем допускаются и промежуточные пункты
(мы не ограничиваемся случаем немедленного перемещения в упомянутом направлении).

Поскольку в нашей общей постановке мегаполисы занумерованы биективно,
мы можем отождествлять отправителей и получателей с индексами из <<промежутка>>
$\ov{1,N}.$
Тогда адресные пары могут быть представлены в виде $(i,j),$
где $i\in \ov{1,N}$
и $j\in \ov{1,N}.$
В данной редакции условий предшествования должно быть введено
множество всех таких упорядоченных адресных пар~$(i,j).$

Итак, фиксируем множество
$\mathbf{K}\in \cp(\ov{1,N}\times \ov{1,N}),$
(иными словами,
$\mathbf{K}$ есть множество, для которого
$\mathbf{K}\su \ov{1,N} \times \ov{1,N},$
то есть $\mathbf{K}$ состоит из упорядоченных пар вышеупомянутого типа).
Будем предполагать в дальнейшем, что
\bfn
  \label{3.3.21}
  \fa \mathbf{K}_o\in \cp^\prime(\mathbf{K})\ \ \exists\, z_o\in \mathbf{K}_o:\,
  \mathrm{pr}_1(z_o)\neq \mathrm{pr}_2(z)\ \ \fa z\in \mathbf{K}_o.
\efn

Свойство (\ref{3.3.21}) означает, что при всяком выборе непустого множества
$\mathbf{K}_o,\, \mathbf{K}_o\su \mathbf{K}$
в этом множестве всегда найдется
упорядоченная пара $z_o = (i_o,j_o),$
для которой при любом выборе упорядоченной пары
$z = (i,j)\in \mathbf{K}_o$
непременно $i_o \neq j.$
Условие (\ref{3.3.21})
(подробное обсуждение см. в \cite[часть~2]{Cha1`})
исключает зацикливание
$\mathbf{K}$-допустимых
(по предшествованию) маршрутов.
При этом сама
$\mathbf{K}$-допустимость маршрута
$\al\in \bbp$ состоит в следующем:
$\fa z \hm \in \mathbf{K}\ \ \fa t_1\in \ov{1,N}\ \ \fa t_2\in \ov{1,N}$
\bfn
  \label{3.3.22}
  \Bigl(\bigl(\al(t_1) = \mathrm{pr}_1(z)\bigl)\,\&\,
  \bigl(\al(t_2) = \mathrm{pr}_2(z)\bigl)\Bigl) \Longrightarrow (t_1 < t_2).
\efn

В \cite[часть~2]{Cha1`}
указано эквивалентное представление свойства
(\ref{3.3.22}),
использующее понятие обратной перестановки.
Итак, маршрут
$\al\in\bbp$ $\mathbf{K}$-допустим тогда и только тогда, когда
\bfn
  \label{3.3.23}
  \al^{-1}\bigl(\mathrm{pr}_1(z)\bigl) < \al^{-1}\bigl(\mathrm{pr}_2(z)\bigl)\ \
  \fa z\in \mathbf{K}.
\efn

В связи с (\ref{3.3.23}) отметим,
что при $\al\in \bbp$ и $z\in \mathbf{K}$
индексы
$t_1= \al^{-1}\bigl(\mathrm{pr}_1(z)\bigl)\in \ov{1,N}$ и
$t_2= \al^{-1}\bigl(\mathrm{pr}_2(z)\bigl)\in \ov{1,N}$
играют роль моментов времени, соответствующих посещению мегаполисов
$M_{\mathrm{pr}_1(z)}$ и $M_{\mathrm{pr}_2(z)}$ соответственно.
С~учетом
упомянутой отождествимости условий,
связанных с (\ref{3.3.22}) и (\ref{3.3.23}),
имеем, что
\bfn
  \label{3.3.24}
  \mathbf{A} \df \{\al\in \bbp|\,\al^{-1}\bigl(\mathrm{pr}_1(z)\bigl) <
  \al^{-1}\bigl(\mathrm{pr}_2(z)\bigl)\ \ \fa z\in \mathbf{K}\}
\efn
есть множество всех $\mathbf{K}$-допустимых маршрутов.
В \cite[часть~2]{Cha1`}
показано, что
(при условии (\ref{3.3.21}))
$\mathbf{A}\neq \emp.$
Поэтому
\bfn
  \label{3.3.25}
  \mathbf{A}\in \cp^\prime(\bbp),
\efn
то есть $\mathbf{A}$ есть непустое п/м $\bbp$,
в котором элементами являются
$\mathbf{K}$-допустимые
(по предшествованию)
маршруты и только они.

{\bf Допустимые решения.}
Упорядоченные пары
$$
  \bigl(\al,(z_i)_{i\in\ov{0,N}}\bigl), \al\in \mathbf{A},\,
  (z_i)_{i\in\ov{0,N}}\in \cz_\al
$$
рассматриваем в дальнейшем в качестве допустимых решений (ДР)
задачи о построении системы перемещений (\ref{3.3.8`}), (\ref{3.3.8``})
с нужными свойствами.
С~учетом (\ref{3.3.19}) и (\ref{3.3.25})
имеем важное свойство
непустоты множества всех ДР.
Итак, получили, что
\bfn
  \label{3.3.26}
  \mathbf{D}\df \{\bigl(\al,(z_i)_{i\in\ov{0,N}}\bigl)\in
  \mathbf{A}\times \bbz\,\bigl| \,(z_i)_{i\in\ov{0,N}}\in \cz_\al\}\in
  \cp^\prime(\mathbf{A}\times \bbz).
\efn

В связи с (\ref{3.3.26})
необходимо отметить, что
\bfn
  \label{3.3.27}
  \widetilde{\mathbf{D}} \df \{\bigl(\al,(z_i)_{i\in\ov{0,N}}
  \bigl)\in \mathbf{A}\times \bbz\,\bigl|\,(z_i)_{i\in\ov{0,N}}\in
  \mathbf{Z}_\al\}\in \cp^\prime(\mathbf{A}\times \bbz).
\efn

В (\ref{3.3.26}) мы имеем <<настоящее>> множество всех ДР,
причем здесь в понятие допустимости включается набор условий,
определяемых посредством мультифункций (\ref{3.3.9})
(в конкретизированной постановке
(см. замечание~\ref{z3.3.2}),
это соответствует технологическому требованию,
связанному с достаточным удалением <<новой>> точки врезки от предыдущей).
В (\ref{3.3.27}) мы отказываемся от условий, связанных с (\ref{3.3.9}),
и создаем возможности для <<более свободного>> режима.
Сравнение результатов,
достигаемых в классах ДР из множеств (\ref{3.3.26}), (\ref{3.3.27}) может
оказаться полезным с точки зрения оценки влияния условий, касающихся
(\ref{3.3.9}), на окончательный результат.

{\bf Функции стоимости, агрегирование затрат.}
Мы полагаем, что внешние
перемещения и работы, связанные с посещением мегаполисов, сопровождаются
потерями, которые затем агрегируются.
В настоящей книге рассматривается
только аддитивный вариант агрегирования:
затраты суммируются.
В конкретной постановке, связанной с листовой резкой на машинах с ЧПУ, оценивается общее
время холостого хода и некоторые операции, связанные с началом и
завершением резки.

Однако сейчас мы рассмотрим абстрактный вариант постановки,
а стало быть,
и функции стоимости здесь могут быть достаточно произвольными.

По соображениям вычислительного характера мы <<сократим>>
исходное множество
состояний $X$ до конечного,
но достаточного для всех наших построений его п/м:
введем в рассмотрение (см. (\ref{3.3.5}))
\bfn
  \label{3.3.28}
  \bbx = \{x^o\} \cup\Bigl(\bigcup\limits_{i=1}^NM_i\Bigl) \in \mathrm{Fin}(X).
\efn

Из (\ref{3.3.8`}) и (\ref{3.3.28}) видно, что все состояния
(определяемые как элементы $X),$
которые могут быть получены на реализациях исследуемого
процесса, <<укладываются>> в $\bbx$ (\ref{3.3.28}).
С этих пор мы просто
<<заменяем>> $X$ на $\bbx,$
что будет учитываться и при определении функций стоимости.

\noindent Итак, на уровне общей постановки мы фиксируем
(произвольные) функции
\bfn
  \label{3.3.29}
  \mathbf{c}\in \mathcal{R}_+[\bbx\times \bbx],\ c_1\in
  \mathcal{R}_+[\bbx\times \bbx],
  \,\ldots,c_N\in \mathcal{R}_+[\bbx\times
  \bbx],\ f\in \mathcal{R}_+[\bbx].
\efn

Допустим, что значения этих функций содержательны на соответствующих п/м
областей определения, а для прочих аргументов они доопределяются тем или иным
способом.
Это особенно важно сделать в конкретной постановке.
В общем же случае
(\ref{3.3.29})  полагаем только, что функция $\mathbf{c}$
используется для оценивания внешних перемещений,
функции $c_1,\,\ldots,c_N$ --- для оценивания работ,
связанных с посещением мегаполисов,
а $f$ --- для оценивания терминального состояния (элемент
$x_{N,2}$
в (\ref{3.3.8`}), (\ref{3.3.8``})).

\begin{zam}
  \label{z3.3.3}
Из общих соображений, связанных с $(\ref{3.3.8`})$ и
$(\ref{3.3.8``})$, вытекает,
что значения $\mathbf{c}(x,y)$ функции $\mathbf{c}$
содержательны
$($и, по сути дела, достаточны для оценивания перемещений в
$(\ref{3.3.8`}), (\ref{3.3.8``}))$
в следующих двух случаях:

$1)\ x = x^o$ и $y\in A_j(x^o),$ где $j\in \ov{1,N};$

$2)\ x\in \mathbf{M}_k,$ где $k\in \ov{1,N},$ и $y\in A_l(x),$ где
$l\in \ov{1,N}\sm\{k\}.$

Для всех упорядоченных пар $(x,y), x\in \bbx, y\in \bbx,$ не удовлетворяющих
случаям
ни $1)$, ни $2)$,
значения $\mathbf{c}(x,y)$ могут задаваться произвольно и,~в~частности, могут полагаться нулевыми.

Если $j\in\ov{1,N},$
то значения $c_j(x,y)$  функции $c_j$ содержательны
(и существенны с точки зрения (\ref{3.3.8`}, (\ref{3.3.8``}))
только при $(x,y) \in \bbm_j;$
для $(x,y)\in (\bbx\times \bbx)\sm \bbm_j$ можно определять
$c_j(x,y)$ произвольным образом и,~в~частности, можно полагать
для таких $(x,y)$ эти значения нулевыми.

Наконец,
$f(x)$ содержательно при $x\neq x^o,$
а точнее, при
$x\in \bbx\sm \{x^o\}$,
значение $f(x^o)$  может быть произвольным и, в частности, нулевым.
\hfill $\Box$
\end{zam}

В связи с естественным вопросом о том, а нужно ли вообще определять
$\mathbf{c},c_1,\,\ldots,c_N$
на всем $\bbx \times \bbx$ и $f$ --- на $\bbx,$
заметим только, что это связано всего лишь с некоторыми соображениями методического
характера,
касающимися известной <<логической однородности>> используемых конструкций.
Это доставляет и некоторые удобства в вопросах введения последующих определений.
Отметим, в частности, что при
$(z_i)_{i\in\ov{0,N}}\in \bbz$ и $t\in \ov{1,N}$
определены значения
$$
  \mathbf{c}\bigl(\mathrm{pr}_2(z_{t-1}),\mathrm{pr}_1(z_t)\bigl)\in
  [0,\infty[, \ c_1(z_t)\in [0,\infty[,\ldots,c_N(z_t)\in [0,\infty[
  ,
$$
если при этом $t = N,$
то имеем
$f\bigl(\mathrm{pr}_2(z_t)\bigl) = f\bigl(\mathrm{pr}_2(z_N)\bigl)\in [0,\infty[.$
С учетом этого получаем,
что при
$\al\in \bbp$ и $(z_i)_{i\in\ov{0,N}}\in \bbz$
\bfn
  \label{3.3.30}
  \mathfrak{C}_\al[(z_i)_{i\in\ov{0,N}}] \df
  \sum\limits_{t=1}^N\bigl[\mathbf{c}\bigl(\mathrm{pr}_2(z_{t-1}),
  \mathrm{pr}_1(z_t)\bigl) + c_{\al(t)}(z_t)\bigl] + f\bigl(\mathrm{pr}_2(z_N)
  \bigl) \in [0,\infty[.
\efn

Для наших последующих целей $(\ref{3.3.30})$ важно в случаях,
когда
$\al\in \mathbf{A}$
(маршрут $\al$ допустим по предшествованию),
а
$(z_i)_{i\in\ov{0,N}} \in \cz_\al$.
Полезно также рассмотреть случай
$(z_i)_{i\in\ov{0,N}}\in \mathbf{Z}_\al$.

{\bf Основная задача}
формулируется следующим образом:
\bfn
  \label{3.3.31}
  \mathfrak{C}_\al[(z_i)_{i\in\ov{0,N}}]\longrightarrow
  \min,\ \al\in \mathbf{A}, (z_i)_{i\in\ov{0,N}}\in \cz_\al.
\efn

С учетом (\ref{3.3.26}) получаем,
что  в задаче (\ref{3.3.31}) существуют ДР,
составляющие непустое конечное множество.
Поэтому определено значение задачи
(экстремум)
\bfn
  \label{3.3.32}
  V \df \min\limits_{\al\in\mathbf{A}}\
  \min\limits_{(z_i)_{i\in\ov{0,N}}\in \cz_\al} \mathfrak{C}_\al[(z_i)_{i\in\ov{0,N}}]
  \in [0,\infty[
  .
\efn

Минимумы в $(\ref{3.3.32})$ действительно  достигаются.
С учетом этого называем ДР
$\bigl(\al^o,(z_i^o)_{i\in\ov{0,N}}\bigl)\in \mathbf{D},$
где
$\al^o\in \mathbf{A}$ и
$(z_i^o)_{i\in\ov{0,N}} \in \cz_{\al^o},$
оптимальным, если
\bfn
  \label{3.3.33}
  \mathfrak{C}_{\al^o}[(z_i^o)_{i\in\ov{0,N}}] = V.
\efn

Такие ДР (то есть ДР со свойством (\ref{3.3.33})) существуют.
Решение
задачи (\ref{3.3.32}) состоит в определении
$V$ (\ref{3.3.32})
и какого-либо ДР
$\bigl(\al^o,(z_i^o)_{i\in\ov{0,N}}\bigl) \hm \in \mathbf{D}$
со свойством (\ref{3.3.33}).

Наряду с (\ref{3.3.31}) отметим следующую задачу:
\bfn
  \label{3.3.34}
  \mathfrak{C}_\al[(z_i)_{i\in\ov{0,N}}]\longrightarrow \min,\
  \al\in \mathbf{A}, (z_i)_{i\in\ov{0,N}}\in \mathbf{Z}_\al.
\efn

С учетом (\ref{3.3.27}) имеем,
что в задаче (\ref{3.3.34}) существуют ДР,
которые составляют непустое конечное множество.
Тогда определено значение
\bfn
  \label{3.3.35}
  \widetilde{V} \df \min\limits_{\al\in\mathbf{A}}\
  \min\limits_{(z_i)_{i\in\ov{0,N}}\in \mathbf{Z}_\al} \mathfrak{C}_\al[
  (z_i)_{i\in\ov{0,N}}]
  .
\efn

При этом существуют ДР
$\bigl(\tilde{\al}^o,(\tilde{z}_i^o)_{i\in\ov{0,N}}
\bigl)\in \widetilde{\mathbf{D}}$
со свойством
\bfn
  \label{3.3.36}
  \mathfrak{C}_{\tilde{\al}^o}
  [(\tilde{z}_i^o)_{i\in\ov{0,N}}] = \widetilde{V}.
\efn

Напомним, что
$\cz_\al \su \mathbf{Z}_\al$ при
$\al\in \mathbf{A}$,
см. (\ref{3.3.18}).
Поэтому согласно (\ref{3.3.26}) и (\ref{3.3.27})
\bfn
  \label{3.3.37}
  \mathbf{D}\su \widetilde{\mathbf{D}}.
\efn

Из $(\ref{3.3.26}), (\ref{3.3.27}), (\ref{3.3.32})$ и $(\ref{3.3.35})$
вытекает неравенство
\bfn
  \label{3.3.38}
  \widetilde{V} \leqslant V.
\efn

Разность $V -\widetilde{V}\in [0,\infty[$
характеризует плату
(по результату)
за соблюдение ограничений, определяемых посредством мультифункций (\ref{3.3.9}),
в этой связи см. (\ref{3.3.16}).

Возвращаясь к (\ref{3.3.31}),
отметим, что данная задача является сложной
оптимизационной задачей с зависимыми (связанными) переменными:
имеются в виду
маршрут и трасса (траектория).
Для ее решения будем использовать метод ДП, что,
в свою очередь,
потребует построения специального расширения задачи (\ref{3.3.31}).
При этом сама эта задача будет погружаться в семейство специальным образом
построенных частичных или укороченных задач.
