% !TeX root = ..

\section*{
  Предварительные замечания по математическим методам
  маршрутизации инструмента машин листовой резки с ЧПУ
}
\setcounter{equation}{0}

Как видно из построений двух предыдущих глав,
образующих первую часть монографии,
инженерные задачи, связанные с управлением инструментом
при листовой резке на машинах с ЧПУ,
являются очень сложными и
в ряде случаев плохо формализуемыми.
С точки зрения идей,
используемых в дискретной оптимизации,
они существенно отличаются не только в количественном,
но и в качественном отношениях от своего естественного прототипа --
задачи коммивояжера (ЗК) или
{\it TSP}
(в англоязычной литературе).
Прежде всего эти отличия связаны с наличием
большого числа разнообразных ограничений,
обусловленных технологическими требованиями.
Исследование вышеупомянутых инженерных задач
требует построения специализированной математической теории.
Данная теория, как представляется,
может рассматриваться как ветвь теории управления
с дискретным временем.
Поэтому вполне естественным представляется
широкое использование аппарата динамического программирования
(ДП)
при должной формализации
основных элементов содержательной постановки первой части
монографии.

Прежде всего отметим,
что в процессе последовательного управления
режущим инструментом объектами посещения
являются не <<города>>
(как в ЗК),
а мегаполисы, получающиеся,
в свою очередь,
дискретизацией эквидистант контуров,
подлежащих резке.
Сама эта дискретизация нужна для последующего
использования компьютеров и соответствующих
вычислительных методов решения <<больших>> переборных задач.
В принципе следовало бы рассмотреть
дискретно-непрерывную экстремальную задачу
о посещении системы эквидистант.
В связи с подменой упомянутой
дискретно-непрерывной задачи маршрутизации
дискретной задачей о последовательном обходе
мегаполисов отметим работу
\cite{intro01},
в которой обосновывается свойство,
имеющее смысл устойчивости по результату.

Итак,
мы можем рассматривать некоторый усложненный аналог
обобщенной задачи коммивояжера
({\it GTSP} в англоязычной литературе),
а точнее,
задачу о последовательном посещении
непустых конечных множеств -- мегаполисов.
Точками последних,
т. е. <<городами>>,
логично полагать возможные точки врезки
и~выключения инструмента,
разукрупняя таким образом соответствующие точки начала реза,
лежащие на эквидистантах.
Эта процедура связана со следующей
естественной возможностью переформулировки исходной задачи, именно 
из математической постановки
можно исключить сами процессы резки по эквидистантам,
поскольку для всех допустимых решений
эти процессы вносят один и тот же вклад
в критерий в виде соответствующего слагаемого.

В таком случае можно считать,
что при резке по замкнутому контуру,
рассмотрением которой мы здесь и ограничимся,
этап перехода от внешних перемещений к внутренним работам,
связанным с посещением мегаполиса,
имеет следующую структуру:
инструмент прибывает в одну из точек врезки
в режиме холостого хода,
осуществляет врезку,
перемещается в режиме рабочего хода
(в металле)
к точке начала реза,
а от нее -- также в режиме рабочего хода
перемещается к точке выключения инструмента,
связанной с исходной точкой врезки.
Поэтому в математической модели
мы можем
говорить о перемещениях упорядоченных пар,
у каждой из которых первый элемент
(точка прибытия в мегаполис)
есть точка врезки, а второй элемент --
отвечающая этой точке врезки
точка выключения инструмента.
Сама же очередность посещения мегаполисов
также выбирается исследователем
в~виде перестановки индексов,
что соответствует традиции ЗК.
Всюду в дальнейшем маршрут понимается
только в этом смысле,
т. е. как перестановка индексов.
Варианты движения по занумерованным мегаполисам,
а точнее,
по подмножествам их декартовых квадратов
(т. е. по отношениям, связанным с мегаполисами),
образуют траектории или трассы.
Само же решение, выбираемое исследователем,
является парой маршрут -- трасса
(трасса не является, вообще говоря, перестановкой)
и имеет иерархическую структуру:
трасса проходит по занумерованным
(с помощью маршрута) мегаполисам
и таким образом подчинена маршруту.

В итоге каждому маршруту
(перестановке индексов)
сопоставляется непустой пучок трасс
(траекторий);
точки на трассах определяются
в виде упорядоченных пар,
составленных из точек врезки и выключения инструмента.
По сути мы имеем задачу управления с дискретным временем
и применение ДП для построения решения представляется
вполне обоснованным.
Одним словом,
представление задач первой части
(имеется в виду стандартная резка по замкнутому контуру)
на этапе их решения существенно меняется.
Это обстоятельство вполне естественно
и соответствует применению математической модели,
которую условимся именовать моделью мегаполисов.

Отметим одно важное обстоятельство.
Применяемая ниже модель мегаполисов
излагается в весьма общей форме и может быть
использована для решения других содержательных задач.
Так, в частности, вариант этой модели был использован
в~\cite{Cha2`}
для исследования задачи о демонтаже энергоблока АЭС,
выведенного из эксплуатации.

Итак,
рассматриваемая ниже математическая модель
применима к исследованию многих прикладных задач,
включающих элементы маршрутизации.
В частности,
эта модель применима для задачи управления
инструментом при листовой резке
в случае использования специальных техник резки
(см. главы 1, 2).
При этом, конечно,
надо должным образом определить
мегаполисы и функции стоимости,
переопределив параметры решаемой задачи.
