% !TeX root = ../mat_mod2.tex

\section{
  Общие подходы
  к решению задач маршрутизации
}
\label{sect:5.1}

В данной главе рассматриваются точные и эвристические алгоритмы решения задач
маршрутизации с ограничениями,
а также различные комбинации таких алгоритмов.

Сначала приводятся точные алгоритмы на основе ДП,
позволяющие получать оптимальный результат.
Размерность решаемых задач ограничена несколькими десятками контуров,
что, в принципе, соответствует некоторым типам раскроя.
Проблема с размерностью заключается в том,
что данный алгоритм имеет экспоненциальный рост времени счета
и размера требуемой оперативной памяти.
Кроме того, в задачах оптимизации движения режущего инструмента важную роль
играет точное соблюдение всех ограничений.

Важный момент, который следует особо отметить, состоит в том,
что время счета и память сильно зависят от количества вложенных контуров,
приводящих к условиям предшествования.
Здесь речь идет о деталях с внутренними отверстиями
и областями, и о других, меньших делалях, размещенных внутри этих областей.
В случае простых деталей, не имеющих врутренних вырезаемых областей или отверстий,
размерность решаемой задачи снижается,
как показывает вычислительный эксперимент,
на несколько контуров за счет снижения количества условий предшествования.

Принципиально иной подход к решению рассматриваемых задач состоит в применении
эвристических алгоритмов.
Данные алгоритмы строятся на основе эмпирических правил и не подтверждаются оценками.
Следовательно, говорить об оптимальности здесь невозможно.
Более того, нельзя сказать,
насколько тот или иной результат отличается от оптимального значения.
Можно лишь производить сравнение с результатами
для точных алгоритмов на примерах задач, имеющих сравнительно малую размерность.
Конечно, результаты работы эвристических алгоримов на примерах <<малых>>
задач могут сильно отличаться от аналогичных результатов на <<больших>> примерах
(сотни и тысячи контуров).
Вследствие этого упомянутое сравнение может дать лишь
общее представление о работе того или иного эвристического алгоритма.
Сильными сторонами эвристических алгоритмов являются их высокая производительность
и во многих случаях полиномиальная зависимость времени счета и объема
требуемой памяти от размерности задачи.
Кроме того, эвристические алгоритмы позволяют в полной мере учитывать
детально проработанные тепловые и геометрические ограничения.

Еще один подход к решению задач реализации раскроя состоит в использовании
вставок на основе ДП.
Сначала производятся вычисления с использованием эвристического алгоритма,
после чего участки маршрута реорганизуются с использованием ДП
(данные вставки охватывают участки маршрута с размером, приемлемым для ДП).
Такой подход уже не может претендовать на оптимальность,
точных оценок здесь также нет,
что автоматически причисляет его к разряду эвристических.
Основная идея состоит в том, что мы применяем
к участкам маршрута и трассы гарантированно неухудшающие преобразования.
Здесь опять возникают проблемы, присущие методу ДП, ---
малая размерность вставок и большое время счета.

Для учета тепловых и геометрических ограничений могут быть применены два подхода
(в обоих используется зависимость от списка вырезанных к данному моменту контуров).
В первом реализуется дополнительное жесткое ограничение,
запрещающее перемещение на тот или иной контур,
не согласующийся с тепловыми или геометрическими ограничениями.
Во втором подходе такое перемещение разрешено,
но к функции стоимости перемещения добавляется дополнительный штраф
(большое число).
В настоящей монографии используется второй подход.
Рассматриваемая в предыдущих главах схема решения на основе ДП
допускает применение функций стоимости такого типа и вполне реализуема для
задач малой размерности.
При решении же практических задач,
связанных с листовой резкой и имеющих ощутимую размерность,
серьезную проблему представляет само построение функций стоимости с упомянутой особенностью
(фактическая размерность резко возрастает при учете зависимости от списков заданий).
В настоящей главе приведен эффективный эвристический алгоритм,
не предусматривающий априорного построения упомянутых усложненных функций
стоимости и реализующий построение их фрагментов по мере развития процесса.
Данный алгоритм пригоден для решения маршрутных задач с ограничениями
различных типов,
имеющих достаточно большую размерность.

Все вычисления производились на ПЭВМ с процессором
{\it Intel i7-2630QM}
с 8 Гб оперативной памяти, работающей под управлением
{\it Windows 7 (64-bit)}.
Для разработки программы была использована среда
{\it Microsoft Visual C++ 2013}.
