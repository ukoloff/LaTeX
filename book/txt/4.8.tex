% !TeX root = ../mat_mod2.tex

\section{
  Итерационные методы с использованием оптимизирующих вставок
  (общие соображения)
}
\label{sect:4.8}
\setcounter{equation}{-1}

В предыдущих разделах 4 главы подробно рассматривалась конструкция,
реализующая локальное улучшение
(точнее, неухудшение)
ДР <<большой>> задачи посредством однократного воздействия в виде вставки,
организованной по способу \ref{sect:4.3}.
В случае, когда $\nn$ достаточно велико
(исходная задача имеет большую размерность),
такое воздействие, скорее всего,
приведет к незначительному выигрышу в качестве,
поскольку число $N$ --- <<длина>> окна ---
выбиралось небольшим с тем, чтобы использовать вариант ДП \ref{sect:4.3}
для непосредственного проведения вычислений.
В связи с этим имеет смысл  осуществлять вставки многократно,
то есть использовать итерационные режимы с применением
упомянутых оптимизирующих вставок.
Напомним, что в \ref{sect:4.4}
предполагалось заданным ДР
\bfn
  \label{4.8.0}
  \delta\df \bigl(\la,(\mathbf{h}_i)_{i\in\ov{0,\nn}}\bigl)\in \mathrm{SOL}
\efn
(см. (\ref{4.4.14}), (\ref{4.4.15})),
которое после применения вставки было преобразовано в
$$
  \hat{\delta}\df \bigl(\eta,(\hat{\mathbf{h}}_t)_{t\in\ov{0,\nn}}\bigl)\in \mathrm{SOL}
  .
$$

Теорема~\ref{t4.6.1} и
(\ref{4.7.54}) наглядно демонстрируют выигрыш в качестве решения при замене
\bfn
  \label{4.8.1}
  \delta \longrightarrow \hat{\delta}
  .
\efn

Теперь мы можем вновь осуществить вставку,
считая уже $\hat{\delta}$
фиксированным решением <<большой>> задачи.
Иными словами, считая замену (\ref{4.8.1}) сделанной,
предполагается снова реализовать эффект,
определяемый в теореме~\ref{t4.6.1} и в (\ref{4.7.54}).
Для этого, конечно,
следует разумным образом выбрать начало вставки
(индекс $\nu$ в построениях предыдущих разделов)
и, возможно, <<длину>> (или размер) окна $N.$
В конкретных реализациях итерационных процедур были
использованы три варианта (см. \cite{Cha18`}):

1) Итерационный метод с комбинацией вставок разной <<длины>>
(вставка малого размера используется
для поиска приемлемого <<начала>> $\nu$
основной вставки большего размера);

2) итерации, осуществляемые  по принципу максимизации в <<окне>>
количества адресных пар <<большой>> задачи;

3) итерации со случайным выбором <<начала>> вставки.

Сейчас обсудим упомянутые варианты на содержательном уровне.

\subsubsection*{
  Итерационный метод с комбинацией вставок разной <<длины>>
}

Ограничимся обсуждением этапа,
связанного с реализацией системы вставок малого размера и одной рабочей вставки.
Итак, полагаем,
что ДР
$\delta$ (\ref{4.8.0}) так или иначе выбрано,
то есть,
как и в \ref{sect:4.4},
мы имеем маршрут $\la\in\ca$
и согласованную с ним трассу
$(\mathbf{h}_i)_{i\in\ov{0,\nn}}\in \mathfrak{Z}_\la.$

В части использования
$N\in \bbn,\,N\geqslant 2,$
будем допускать две возможности:
$N=N_1$ и $N=N_2,$
где
$$N_1 < N_2.$$

С целью поиска <<момента>> начала основной
(рабочей) вставки будем рассматривать
серию всевозможных вставок <<длины>>
$N_1$
или, короче,
$N_1$-вставок, отвечающих всевозможным значениям
\bfn
  \label{4.8.2}
  \nu\in \ov{0,\nn-N_2}
  .
\efn

Каждая из упомянутых
$N_1$-вставок реализуется по схеме
\ref{sect:4.4} -- \ref{sect:4.6}
с одной существенной оговоркой:
для любого значения $\nu$ (\ref{4.8.2})
при решении соответствующей локальной задачи с $N=N_1$
мы ограничиваемся поиском экстремума
$\bbv$ (\ref{4.3.10})
упомянутой задачи и не занимаемся построением
оптимального решения в виде пары маршрут-трасса.

При этом, конечно, упомянутый экстремум $\bbv$
зависит от $\nu,$
то есть на самом деле
$\bbv = \bbv(\nu),$
где $\nu$ удовлетворяет (\ref{4.8.2}).
Как следствие, используемая в теореме~\ref{t4.6.1} и в (\ref{4.7.54})
оценочная константа $\kappa$
также зависит от $\nu,$
то есть
$\kappa = \kappa_1(\nu)$.
При этом $\kappa_1(\nu)$
однозначно определяется значением
$\bbv(\nu)$ соответствующей локальной задачи.

Таким образом, мы в режиме
$N_1$-вставок реализуем зависимость
$$
  \tilde{\nu}\longmapsto \kappa_1(\tilde{\nu}):\,\ov{0,\nn-N_2}\longrightarrow [0,\infty[
  ,
$$
которую предлагается максимизировать с целью определения точки
$\nu,$ наиболее чувствительной
(по нашим представлениям)
к действию рабочей $N_2$-вставки
(вставки при $N=N_2).$

Итак, решаем задачу
\bfn
  \label{4.8.3}
  \kappa_1(\tilde{\nu}) \longrightarrow \max,\ \ \tilde{\nu}\in\ov{0,\nn-N_2}
  .
\efn

Выбираем индекс $\nu$ (\ref{4.8.2})
в виде точки максимума задачи (\ref{4.8.3})
и фиксируем в качестве <<начала>> рабочей $N_2$-вставки,
действие которой нам представляется достаточно ощутимым.
Более трудоемкая $N_2$-вставка реализуется в <<точке>>,
выбранной в виде точки максимума задачи (\ref{4.8.3}).
Далее реализуется замена (\ref{4.8.1}),
осуществляемая по рецептам \ref{sect:4.4} -- \ref{sect:4.7}
при $N=N_2$.
Мы получаем новое ДР
$\hat{\delta}$ большой задачи,
в терминах которого снова реализуем
$N_1$-вставки с целью последующего решения задачи,
подобной (\ref{4.8.3}),
и выбора <<начала>> $N_2$-вставки.

\subsubsection*{
  Итерации, осуществляемые по принципу максимизации
  (в~<<окне>>) количества адресных пар
  <<большой>> задачи
}

Известно \cite[\S\,4.9]{Cha1`},
что условия предшествования могут использоваться
<<в положительном направлении>>
(см. также \cite[гл.~4]{Cha2`})
в смысле снижения вычислительной сложности:
в условиях рациональной организации вычислительного процесса при большем количестве адресных пар
достигается соответствующий положительный эффект.
В этой связи А. А. Ченцовым был предложен подход
к организации итерационной процедуры,
в рамках которого предлагается выбирать
<<окно>> вставки по
принципу максимизации количества адресных пар из $\mathfrak{K},$
полностью <<укладывающихся>>
(и в смысле отправителя, и в смысле получателя)
в упомянутое <<окно>>.


Возвращаясь к (\ref{4.8.0}),
имеем (см. (\ref{4.4.25})),
что множество $Q$ зависит от $\nu,$
то есть $Q = Q(\nu).$
Поэтому рассматрим зависимость
\bfn
  \label{4.8.4}
  \tilde{\nu}\longmapsto |Q(\tilde{\nu})|:\,\ov{0,\nn-N} \longrightarrow \bbn_o
  ,
\efn
учитывая, что $N_1 = N_2 =N,$
а <<окно>> может быть размещено в любой точке из $\ov{0,\nn-N}$.
Располагая зависимостью (\ref{4.8.4}),
легко воспроизводимой по $\delta$
(\ref{4.8.0}),
решаем задачу
\bfn
  \label{4.8.5}
  |Q(\tilde{\nu})| \longrightarrow \max,\ \ \tilde{\nu}\in \ov{0,\nn-N}
  .
\efn

В качестве $\nu$ в (\ref{4.4.23})
выбираем точку максимума задачи (\ref{4.8.5}),
полагая, что большое число условий предшествования позволит
(см. \ref{sect:4.3})
более эффективно осуществить
локальную оптимизацию в смысле построения
$\hat{\delta}$ по рецептам \ref{sect:4.3}.

\subsubsection*{
  Итерации со случайным выбором <<начала>> вставки
}

Предполагается, что выбор $\nu$ (\ref{4.8.2})
осуществляется случайно.
В настоящем варианте использовалось равномерное распределение.
Итак, располагая ДР $\delta$ (\ref{4.8.0}),
осуществляем случайное испытание по выбору $\nu$
(возможный вариант --- использовать не весь <<промежуток>> $\ov{0,\nn-N},$
а какое-то его п/м),
после чего оптимизирующая вставка конструируется по рецептам \ref{sect:4.5} -- \ref{sect:4.7},
в результате чего получаем ДР $\hat{\delta},$
которое принимается за новое эвристическое решение <<большой>> задачи,
после чего следует новое случайное испытание для <<начала>> новой вставки.
