% !TeX root = ..

\subsection{
  Вычисление стоимости резки заготовок
  на машине с~ЧПУ в режиме моделирования процесса резки
}
\label{sect:2.1.2}

Другая проблема точного вычисления целевой функции
при оптимизации маршрута резки связана
с поиском адекватных значений стоимостных параметров
в формуле (\ref{cutting-cost}):
$$
F_{cost}=
L_{on} \cdot C_{on} +
L_{off} \cdot C_{off} +
N_{pt} \cdot C_{pt}
.
$$

Напомним:
$L_{on}$ – длина реза с включенным режущим инструментом;
$C_{on}$ – стоимость единицы пути с включенным режущим инструментом;
$L_{off}$~– длина переходов с выключенным режущим инструментом (холостой ход);
$C_{off}$ – стоимость единицы пути с выключенным режущим инструментом;
$N_{pt}$~– количество точек врезки,
$C_{pt}$ – стоимость одной точки врезки.

Рассмотрим вопрос точного вычисления
стоимости лазерной резки в задаче
оптимизации маршрута режущего инструмента
применительно к машине лазерной резки (тип лазера: СО$_2$)
с ЧПУ на примере машины
{\it ByStar 3015}.

Как нетрудно видеть,
проблема точного вычисления целевой функции
связана с
определением значений параметров
$C_{on}, C_{off}, C_{pt}$.

Для расчета
$C_{on}$
введем следующие обозначения для стоимостных параметров,
вычисляемых на 1 м рабочего хода инструмента:
$C_\text{расх}$ -- стоимость расходных материалов (например, сопло, защитное стекло, газовые трубки);
$C_\text{тех}$ -- стоимость технологического газа (азот или кислород в зависимости от типа обрабатываемого материала);
$C_\text{лаз}$ -- стоимость лазерного газа (при работе на машине с ЧПУ на проточном газовом лазере),
$C_\text{э/э}^{on}$ -- стоимость электроэнергии;
$C_\text{зп}^{on}$ -- затраты, связанные с заработной платой сопровождающего персонала;
$C_\text{А}^{on}$ -- амортизация оборудования.
Тогда в общем виде
$C_{on}$
будем вычислять по следующей формуле:
\begin{equation}
  C_{on} =
  C_\text{э/э}^{on} +
  C_\text{тех} +
  C_\text{лаз} +
  C_\text{расх} +
  C_\text{зп}^{on} +
  C_\text{А}^{on}
  .
  \label{c-on}
\end{equation}

Для вычисления значений
$C_{on}, C_\text{э/э}^{on}, C_\text{тех},
C_\text{лаз}, C_\text{расх}, C_\text{зп}^{on}, C_\text{А}^{on}$
введем дополнительные обозначения:
$t_{on}$ -- время, затрачиваемое на единицу длины рабочего хода инструмента, ч;
$P_{on}$ -- затраты электроэнергии за один час работы лазерного комплекса на рабочем ходу, кВт/ч;
$V_\text{тех}$ -- расход технологического газа, м$^3$/ч;
$V_\text{лаз}$ -- расход лазерного газа, м$^3$/ч;
$C_\text{э/э}$ -- стоимость электроэнергии за 1 кВт;
$C_{\text{лазМ}^3}$ -- стоимость 1м$^3$ лазерного газа;
$C_{\text{техМ}^3}$ -- стоимость 1м$^3$ технологического газа;
$C_\text{расхЕд}$ -- стоимость единицы расходных материалов;
$t_\text{расхСрок}$ -- срок службы расходных материалов;
$C_\text{зп}$ -- стоимость 1~ч работы обслуживающего персонала;
$A$ -- амортизация за 1~ч работы лазерного комплекса, руб;
$N$ -- срок полезного использования оборудования, год;
$C_\text{оборуд}$ -- первоначальная стоимость лазерного комплекса.
Тогда
$C_{on}, C_\text{э/э}^{on}, C_\text{тех},
C_\text{лаз}, C_\text{расх}, C_\text{зп}^{on}, C_\text{А}^{on}$
вычислим по следующим формулам:
\begin{equation}
  C_\text{э/э}^{on} =
  P_{on} t_{on}   C_\text{э/э}
  ,
  \label{c-on-ee}
\end{equation}
\begin{equation}
  C_\text{тех} =
  V_\text{тех} C_{\text{техМ}^3} t_{on}
  ,
  \label{c-on-teh}
\end{equation}
\begin{equation}
  C_\text{лаз} =
  V_\text{лаз} C_{\text{лазМ}^3} t_{on}
  ,
  \label{c-on-laz}
\end{equation}
\begin{equation}
  C_\text{расх} =
  \frac{C_\text{расхЕд}}{t_\text{расхСрок}}
  ,
  \label{c-on-rasx}
\end{equation}
\begin{equation}
  C_\text{зп}^{on} =
  C_\text{зп} t_{on}
  ,
  \label{c-on-zp}
\end{equation}
\begin{equation}
  C_\text{А}^{on} =
  \frac{1}N \frac{C_\text{оборуд}}{1920} t_{on}
  .
  \label{c-on-A}
\end{equation}

Параметр
$C_\text{тех}$
необходимо учитывать при расчете стоимости резки
только в тех случаях,
когда применяется вспомогательный рабочий газ
(кислород, азот в зависимости от типа обрабатываемого материала)
для увеличения скорости резки,
возможности обработки материалов более высоких толщин
и для сокращения затрат электроэнергии.
Расход газа зависит от диаметра используемого сопла и давления газа.

Для расчета
$C_{off}$
введем следующие обозначения параметров,
вычисляемых на 1~м холостого хода режущего инструмента:
$P_{off}$ -- затраты электроэнергии за 1~ч работы лазерного комплекса на холостом ходу, кВт/ч;
$t_{off}$~-- время, затрачиваемое на один метр холостого хода инструмента, ч.
Тогда
\begin{equation}
  C_{off} =
  P_{off} t_{off} C_\text{э/э}
  + C_\text{зп} t_{off}
  + \frac{1}N \frac{C_\text{оборуд}}{1920} t_{off}
  .
  \label{c-off}
\end{equation}

Аналогично для расчета
$C_{pt}$
введем следующие обозначения для стоимостных параметров,
вычисляемых на одну точку врезки:
$C_\text{э/э}^{pt}$ -- стоимость электроэнергии;
$C_\text{расх}^{pt}$ -- стоимость расходных материалов;
$C_\text{лаз}^{pt}$ -- стоимость лазерного газа;
$C_\text{тех}^{pt}$ -- стоимость технологического газа,
$C_\text{зп}^{pt}$ -- затраты, связанные с заработной платой сопровождающего персонала;
$C_\text{А}^{pt}$ -- амортизация оборудования.
Тогда
\begin{equation}
  C_{pt} =
  C_\text{э/э}^{pt} +
  C_\text{расх}^{pt} +
  C_\text{лаз}^{pt} +
  C_\text{тех}^{pt} +
  C_\text{зп}^{pt} +
  C_\text{А}^{pt}
  .
  \label{c-pt}
\end{equation}

Для вычисления значений
$C_\text{э/э}^{pt}, C_\text{расх}^{pt}, C_\text{лаз}^{pt}, C_\text{тех}^{pt}$
введем дополнительные параметры:
$P_{pt}$ -- затраты электроэнергии на одну точку врезки, кВт/ч;
$t_{pt}$ -- время, затрачиваемое на одну точку врезки, ч.
Тогда
\begin{equation}
  C_\text{э/э}^{pt} =
  P_{pt} t_{pt}   C_\text{э/э}
  ,
  \label{c-pt-ee}
\end{equation}
\begin{equation}
  C_\text{тех}^{pt} =
  V_\text{тех} C_{\text{техМ}^3} t_{pt}
  ,
  \label{c-pt-teh}
\end{equation}
\begin{equation}
  C_\text{лаз}^{pt} =
  V_\text{лаз} C_{\text{лазМ}^3} t_{pt}
  ,
  \label{c-pt-laz}
\end{equation}
\begin{equation}
  C_\text{расх}^{pt} =
  \frac{C_\text{расхЕд}}{t_\text{расхСрок}}
  ,
  \label{c-pt-rasx}
\end{equation}
\begin{equation}
  C_\text{зп}^{pt} =
  C_\text{зп} t_{pt}
  ,
  \label{c-pt-zp}
\end{equation}
\begin{equation}
  C_\text{А}^{pt} =
  \frac{1}N \frac{C_\text{оборуд}}{1920} t_{pt}
  .
  \label{c-pt-A}
\end{equation}

При расчете стоимости одной точки врезки параметр
$C_\text{лаз}^{pt}$
необходимо учитывать только при обработке материала
на проточном газовом лазере.
Параметр
$C_\text{тех}^{pt}$
необходимо учитывать при расчете себестоимости резки только в тех случаях,
когда применяется вспомогательный рабочий газ.

Тогда целевую функцию стоимости резки (\ref{cutting-cost})
можно записать в следующем виде:
\begin{multline}
  F_{cost} =
  L_{on} \left(
    C_\text{э/э}^{on} +
    C_\text{тех} +
    C_\text{лаз} +
    C_\text{расх} +
    C_\text{зп}^{on} +
    C_\text{А}^{on}
      \right) +
  L_{off} C_{off} +
  \\
  + N_{pt} \left(
    C_\text{э/э}^{pt} +
    C_\text{расх}^{pt} +
    C_\text{лаз}^{pt} +
    C_\text{тех}^{pt} +
    C_\text{зп}^{pt} +
    C_\text{А}^{pt}
      \right)
  .
  \label{c-full}
\end{multline}

К основным расходным материалам и запчастям
для газового лазера можно отнести:
поворотные зеркала, фокусирующие линзы,
защитные стекла, сопла, юстировочные узлы,
газовые трубки.
К основным расходным материалам для
волоконного лазера можно отнести:
сопла, защитные стекла, фокусирующие линзы.
А для случая применения твердотельных лазеров
выделяют следующие основные расходные материалы и запчасти:
лампы оптической накачки, защитные стекла, зеркала,
квантрон, активный элемент.
Следует отметить, что стоимость расходных материалов
может изменяться в зависимости от фактических сроков
службы расходных материалов,
которые зависят от качества используемого газа,
опыта персонала, эксплуатирующего лазерный станок.
Следует отметить, что
$C_\text{расхЕд}$
зависит от ценообразования, курса доллара
({\it USD}) и евро
({\it EUR}),
а параметры
$C_\text{э/э}$,
$C_{\text{лазМ}^3}$ и
$C_{\text{техМ}^3}$
зависят от цен, которые устанавливает поставщик услуг,
поэтому при расчете
$F_{cost}$
для конкретных производственных задач,
изменения цен целесообразно учитывать,
используя изменяющиеся в зависимости от перечисленных
факторов таблицы стоимостных параметров в
{\it MS Excel}.
В частности, была создана сводная таблица в
{\it MS Excel}
для расчета себестоимости лазерной резки по разработанной
выше методике для газового СО$_2$
лазерного комплекса
{\it ByStar 3015}
для следующих материалов:

\begin{itemize}
\item
нержавеющая сталь (на примере 12Х18Н10Т) толщиной $\Delta$ = 1--10 мм;
\item
углеродистая сталь (на примере 10кп) толщиной $\Delta$ = 1--15 мм;
\item
алюминий и его сплавы (на примере АМг3М) толщиной $\Delta$ = 1--5 мм.
\end{itemize}

Были определены значения основных стоимостных характеристик
$C_{on}$, $C_{off}$, $C_{pt}$
с учетом всех перечисленных параметров, приведенных в
(\ref{c-on})--(\ref{c-pt-A}).
В табл. \ref{c-table} приведены значения стоимости
одного погонного метра лазерного реза при максимальной
$C_{on}^{max}$
и минимальной
$C_{on}^{min}$
возможной рабочей скорости перемещения режущего инструмента
$V_{on}$
в зависимости от требуемого качества изготовления деталей.

Изложенная выше методика является универсальной
для такого класса лазерного оборудования с ЧПУ и,
следовательно, может применяться для вычисления значений
целевой функции стоимости резки
$F_{cost}$,
а также для создания таблиц стоимостных параметров по формуле (\ref{cutting-cost})
для других марок стали и толщин материала.
Аналогичный подход следует использовать и при создания
стоимостных параметров целевой функции стоимости резки
для другого технологического оборудования термической
резки листового материала с ЧПУ.

Как отмечалось выше,
на предприятиях реального сектора экономики
фактически не применяют научно-обоснованные методы расчета стоимости резки.
Например, предприятия, оказывающие услуги по листовой резке заготовок из листового металла,
могут использовать для расчета стоимости только периметр вырезаемых деталей,
толщину и тип материала
(конструкционная сталь, нержавеющая сталь, цветные металлы и т. д.),
а также тип используемой технологии резки
(лазерная, плазменная, газовая, гидроабразивная).

\begin{table}[p]
  \caption{
    Значения основных стоимостных параметров
    при вычислении целевой функции для
    CO$_2$ лазерного комплекса
    {\it ByStar 3015}
  }
  \label{c-table}
  \centering
  \begin{tabular}{c*{5}{|r}}
    \hline
    Материал & Толщина, мм & $C_{on}^{max}$, руб. & $C_{on}^{min}$, руб. & $C_{off}$, руб. & $C_{pt}$, руб. \\
    \hline
    10кп	& 1	& 5,3	& 7,5	& 0,42	& 0,7 \\
    10кп	& 1,2	& 6,6	& 9,5	& 0,42	& 1,0 \\
    10кп	& 1,5	&  6,6	& 9,5	& 0,42	& 1,1 \\
    10кп	& 2	& 8,1	& 11,7	& 0,42	& 1,3 \\
    10кп	& 2,5	&	9,7	& 14,0	& 0,42	& 1,5 \\
    10кп	& 3	& 12,0	& 17,4	& 0,42	& 1,6 \\
    10кп	& 3,5	&	13,3	& 19,0	& 0,42	& 1,6 \\
    10кп	& 3,9	&	13,3	& 19,0	& 0,42	& 1,9 \\
    10кп	& 4	&	14,8	& 21,0	& 0,42	& 2,2 \\
    10кп	& 5	&	17,9	& 26,1	& 0,42	& 2,7 \\
    10кп	& 8	&	26,1	& 38,2	& 0,42	& 3,4 \\
    10кп	& 10	&	31,8	& 44,1	& 0,42	& 5,1 \\
    10кп	& 15	&	52,1	& 71,7	& 0,42	& 6,0 \\
    АМг3М	& 1	& 11,1	& 18,6	& 0,42	& 3,7 \\
    АМг3М	& 2	& 18,0	& 30,0	& 0,42	& 5,6 \\
    АМг3М	& 3	& 56,8	& 92,8	& 0,42	& 14,2 \\
    АМг3М	& 5	& 193,0	& 328,2	& 0,42	& 32,2 \\
    12Х18Н10Т	& 1	& 14,9	& 24,9	& 0,42	& 2,5 \\
    12Х18Н10Т	& 1,5	& 18,7	& 31,4	& 0,42	& 3,8 \\
    12Х18Н10Т	& 2	& 25,3	& 42,4	& 0,42	& 4,5 \\
    12Х18Н10Т	& 2,5	& 38,1	& 63,5	& 0,42	& 6,8 \\
    12Х18Н10Т	& 3	& 46,4	& 76,1	& 0,42	& 8,6 \\
    12Х18Н10Т	& 4	& 87,2	& 143,7	& 0,42	& 13,1 \\
    12Х18Н10Т	& 5	& 122,6	& 198,1	& 0,42	& 18,9 \\
    12Х18Н10Т	& 6	& 241,5	& 386,5	& 0,42	& 31,7 \\
    12Х18Н10Т	& 8	& 475,5	& 856,0	& 0,42	& 42,2 \\
    12Х18Н10Т	& 10	& 1038,7	& 2077,3	& 0,42	& 72,0 \\
    \hline
  \end{tabular}
\end{table}

\clearpage

Описанный в этом параграфе подход позволяет
исследовать новый класс оптимизационных задач,
предусматривающих интегрированный критерий целевой функции
для совместно рассматриваемой задачи оптимального фигурного
{\it 2D}-раскроя
({\it Nesting problem})
и задачи оптимальной маршрутизации инструмента для машин фигурной листовой резки с ЧПУ.
В качестве целевой функции интегрированной проблемы раскроя
(т. н. {\it Integrated Nesting and Routing Problem, INRP})
рассматривается функция стоимости раскроя
$C_{INRP}$,
являющаяся суммой величины стоимости материала
$C_{NEST}$,
использованного для раскроя и величины стоимости резки
$F_{cost}$:
\begin{equation}
  \label{eq:inrp}
  C_{INRP} = C_{NEST} + F_{cost}
  .
\end{equation}


При этом сама оптимизационная задача
{\it INRP}
формулируется предельно просто:
\begin{equation}
  \label{eq:inrp-min}
  C_{INRP} \to \min
  .
\end{equation}

Вместе с тем,
в научной литературе практически отсутствует описание
методов и алгоритмов решения задачи (\ref{eq:inrp-min}),
что как раз и объясняется сложностью вычисления значений для конкретных задач
{\it 2D}-раскроя
и оптимизации стоимости резки при получении раскроенных деталей из листового материала.
Разработка методов решения этой задачи является предметом дальнейших исследований авторов.
