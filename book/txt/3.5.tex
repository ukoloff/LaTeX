% !TeX root = ..

\section{
  Экономичная версия метода динамического программирования
}
\label{sect:3.5}
\setcounter{equation}{0}

В настоящем разделе мы стремимся к организации вычислений значений функции $v$
(\ref{3.4.26}) на меньшем,
в сравнении с $\mathbf{X}\times \cp(\ov{1,N}),$
множестве позиций.
Напомним, что позицией мы называем всякую упорядоченную пару $(x,K),$
где
$x\in \mathbf{X}$ и $K$ --- список заданий.
Итак, приведем схему \cite[\S\,4.9]{Cha1`},
полагая, что
\bfn
  \label{3.5.1}
  \cg\df \{K\in \mathfrak{N} |\,\fa z\in \mathbf{K}\ \ \bigl(\mathrm{pr}_1(z)\in K\bigl)
  \Longrightarrow \bigl(\mathrm{pr}_2(z) \in K\bigl)\}
  .
\efn

В (\ref{3.5.1}) определено семейство непустых п/м $\ov{1,N},$
обладающих одним специальным свойством,
которое на содержательном уровне имеет следующий смысл:
если $K\in \mathfrak{N},$
то
(в соответствии с (\ref{3.5.1}))
имеем, что
$K\in \cg$
тогда и только  тогда,
когда при всяком выборе адресной пары
$(m,n)\in \mathbf{K}$
из того, что $m\in K,$ следует включение $n\in K$
(если отправитель адресной пары содержится в списке $K,$
то и получатель также <<обязан>> быть элементом $K).$

{\bf Пример.}
Пусть $N=3,\, \mathbf{K}= \{(1,3)\}$
(имеется единственная адресная пара с отправителем 1 и получателем 3).
Тогда
(см. (\ref{3.5.1}))
$\{1;3\}\in \cg,$
а $\{1;2\}\notin \cg.$\hfill $\Box$

Множества --- элементы семейства $\cg$ --- ранжируем по мощности.
В этой связи полагаем, что
\bfn
  \label{3.5.2}
  \cg_s \df \{K\in \cg |\,s= |K|\}\ \ \fa s\in \ov{1,N}
  .
\efn

При этом согласно (\ref{3.5.1})
$|K| \in \ov{1,N}\ \ \fa K\in \cg.$
Поэтому
(см. (\ref{3.5.1}), (\ref{3.5.2}))
\bfn
  \label{3.5.3}
  \cg =
  \bigcup\limits_{s=1}^N\cg_s
  .
\efn

Кроме того, из (\ref{3.5.2}) имеем,
что
$\cg_{s_1}\cap\,\cg_{s_2} =
\emp\ \ \fa s_1\in\ov{1,N}\ \ \fa s_2\in \ov{1,N}\sm \{s_1\}.$
Поэтому
(см. (\ref{3.5.3}))
в виде
$$
  (\cg_s)_{s\in\ov{1,N}}:\ \ov{1,N} \longrightarrow \cp(\cg)
$$
мы имеем разбиение $\cg$ в сумму $N$ п/м.

Элементы $\cg$ называем в дальнейшем существенными списками (заданий).
Соответственно, при $s\in \ov{1,N}$
элементы $\cg_s$ --- суть $s$-элементные
существенные списки.
Отметим, что
каждое из семейств
(\ref{3.5.2}) непусто,
см. \cite[\S\,4.9]{Cha1`}.
Ясно, что
\bfn
  \label{3.5.4}
  \cg_N = \{\ov{1,N}\}
\efn
($N$-элементный существенный список является единственным и совпадает с $\ov{1,N}).$
Так же просто определяются одноэлементные существенные списки:
при
\bfn
  \label{3.5.5}
  \mathbf{K}_1\df \{\mathrm{pr}_1(z):\,z\in \mathbf{K}\}
\efn
очевидно, что
\bfn
  \label{3.5.6}
  \cg_1 = \bigl\{\{t\}:\,t\in \ov{1,N}\sm \mathbf{K}_1\bigl\}
\efn
(в связи с (\ref{3.5.6}) см. (\ref{3.3.21})).
В итоге имеем, что
(см. \cite{Cha3`})
\bfn
  \label{3.5.7}
  \cg_{s-1} = \{K\sm\{t\}:\,K\in \cg_s,\ t\in \mathbf{I}(K)\}\ \ \fa s\in \ov{2,N}.
\efn

Итак, семейства (\ref{3.5.2})
реализуются посредством рекуррентной процедуры
\bfn
  \label{3.5.8}
  \cg_N \longrightarrow\cg_{N-1}\longrightarrow \ldots \longrightarrow\cg_1
\efn
(в (\ref{3.5.8}) подразумевается, что $N >2),$
использующей (\ref{3.5.7}):
$\cg_N$ известно,
а далее <<включается>> рекуррентная процедура на основе (\ref{3.5.7});
финалом является семейство
$\cg_1,$
определяемое посредством (\ref{3.5.6}).

С каждым из семейств $\cg_s, s\in\ov{1,N},$
связывается множество в пространстве позиций,
называемое далее слоем;
кроме того, определяется особый нулевой слой
\bfn
  \label{3.5.9}
  D_o \df \bigl\{(x,\emp):\,x\in \bigcup\limits_{i\in \ov{1,N}\sm\mathbf{K}_1}\mathbf{M}_i\bigl\}
  .
\efn

Очень просто определяется слой
$D_N \df \{(x^o,\ov{1,N})\},$
содержащий ровно одну позицию
$(x^o,\ov{1,N}).$
Тем самым
определены два крайних слоя пространства позиций,
см. (\ref{3.5.9}).

Для определения промежуточных слоев
при $s\in \ov{1,N-1}$ и $K\in \cg_s$
введем множество
\bfn
  \label{3.5.10}
  J_s(K) \df \{t\in \ov{1,N}\sm K\,|\,\{t\} \cup\,K\in \cg_{s+1}\}
  .
\efn

В \cite[предложение~2]{Cha5`} установлено,
что $J_s(K) \neq \emp,$ тогда
\bfn
  \label{3.5.11}
  \clm_s[K] \df \bigcup\limits_{j\in J_s(K)}\mathbf{M}_j\in \cp^\prime(\mathbf{X})
  ,
\efn
и как следствие, получаем, что
\bfn
  \label{3.5.12}
  \bbd_s[K] \df \{(x,K):\,x\in \clm_s[K]\}\in \cp^\prime(\mathbf{X}\times \mathfrak{N})
  .
\efn

Заметим, что (\ref{3.5.12})
получается несущественным преобразованием множества (\ref{3.5.11}):
данное преобразование, однако, весьма полезно, поскольку посредством (\ref{3.5.12})
реализуется соответствующая клетка пространства позиций
(это обстоятельство рассмотрим далее).
Теперь при
$s \in\ov{1,N-1}$ слой $D_s$
определяем следующим правилом:
\bfn
  \label{3.5.13}
  D_s \df \bigcup\limits_{K\in \cg_s}\bbd_s[K]\in \cp^\prime(\mathbf{X}\times \mathfrak{N})
  .
\efn

Отметим, что семейство
\bfn
  \label{3.5.14}
  \mathfrak{D}_s \df \{\bbd_s[K]:\,K\in \cg_s\}\in
  \cp^\prime\bigl(\cp^\prime(\mathbf{X}\times \mathfrak{N})\bigl)
\efn
является
разбиением слоя $D_s:$
\bfn
  \label{3.5.14}
  \bigl(D_s = \bigcup\limits_{H\in \mathfrak{D}_s}H\bigl)\,\&\,
  \bigl(\fa H_1\in \mathfrak{D}_s\ \ \fa H_2\in \mathfrak{D}_s\ \ (H_1 \cap H_2\neq \emp)
  \Longrightarrow (H_1 = H_2)\bigl)
  .
\efn

Посредством (\ref{3.5.14})
реализуется клеточная структура промежуточных слоев:
каждый такой слой
разбивается в сумму клеток  вида (\ref{3.5.12}).
В свою очередь, сами слои
$D_o, D_1,\,\ldots,D_N$
являются в силу (\ref{3.5.13}) непустыми множествами в пространстве позиций.
Отметим одно важное свойство \cite[(6.9)]{Cha3`}
\bfn
  \label{3.5.15}
  (y,K\sm\{k\})\in D_{s-1}\ \ \fa s\in \ov{1,N}\ \ \fa K\in \cg_s\ \ \fa k\in \mathbf{I}(K)\ \
  \fa y\in \mathbf{M}_k
  .
\efn

Свойство (\ref{3.5.15})
дополняется следующим простым фактом,
непосредственно вытекающим из
(\ref{3.5.12}):
если $s\in \ov{1,N}$ и $(x,K) \in D_s,$
то непременно $K\in \cg_s.$
Поэтому из
(\ref{3.5.15}) получаем, что
\bfn
  \label{3.5.16}
  \bigl(\mathrm{pr}_2(z),K\sm\{j\}\bigl)\in D_{s-1}\ \ \fa s\in \ov{1,N}\ \ \fa (x,K)\in D_s\ \
  \fa j\in \mathbf{I}(K)\ \ \fa z\in \bba_j(x)
  ;
\efn
см. \cite[(6.11)]{Cha3`}.
Свойство (\ref{3.5.16}) означает, что,
придерживаясь правила вычеркивания,
при произвольном выборе элементов множеств (\ref{3.3.13})
мы можем сохраняться в системе слоев
вплоть до исчерпывания всех индексов из
$\ov{1,N}.$

Используя (\ref{3.5.13}),
легкопроверяемое свойство
$D_o\neq \emp$ и (\ref{3.4.26}),
введем в рассмотрение слои функции Беллмана:
если $s\in \ov{0,N},$
то полагаем, что функция
\bfn
  \label{3.5.17}
  v_s \in \clr_+[D_s]
\efn
определяется следующим условием:
\bfn
  \label{3.5.18}
  v_s(x,K) \df v(x,K)\ \ \fa (x,K)\in D_s
  .
\efn

Иными словами,
слои (\ref{3.5.17}), (\ref{3.5.18}) отождествляются с сужениями  функции
Беллмана на соответствующие слои пространства позиций.
В итоге мы получим следующие функции:
\bfn
  \label{3.5.19}
  v_o\in \clr_+[D_o], v_1\in \clr_+[D_1],\ldots,v_N\in \clr_+[D_N]
  ,
\efn
где каждая из функций неотрицательна и определена на непустом множестве.
В связи с (\ref{3.5.19}) отметим,
что функция $v_N$ определена на одноточечном множестве,
а потому полностью определяется одним своим значением
$$
  v_N(x^o,\ov{1,N})= v(x^o,\ov{1,N}) = V\in [0,\infty[
  .
$$

С учетом (\ref{3.5.9}) и (\ref{3.5.18})
определяем значения функции $v_o$:
$$
  v_o(x,\emp) = v(x,\emp)\ \ \fa x\in \bigcup\limits_{i\in \ov{1,N}\sm \mathbf{K}_1}\mathbf{M}_i
  .
$$

Поэтому согласно (\ref{3.4.25}) получаем, что
\bfn
  \label{3.5.20}
  v_o(x,\emp) = f(x)\ \ \fa x\in \bigcup\limits_{i\in \ov{1,N}\sm
  \mathbf{K}_1}\mathbf{M}_i
  .
\efn

Итак
$v_o$ является  по сути дела <<частью>> терминальной функции $f$,
см. (\ref{3.5.20}).

Возвращаясь к (\ref{3.5.16}),
отметим, что определены значения
$$
  v_{s-1}\bigl(\mathrm{pr}_2(z),K\sm\{j\}\bigl)\in [0,\infty[\ \ \fa s \in \ov{1,N}\ \ \fa
  (x,K)\in D_s\ \ \fa j\in \mathbf{I}(K)\ \ \fa z\in \bba_j(x)
  .
$$

Как следствие,
при $s\in \ov{1,N}$ и $(x,K)\in D_s$
определено значение
$$
  \min\limits_{j\in\mathbf{I}(K)}\ \min\limits_{z\in \bba_j(x)}\bigl[\mathbf{c}\bigl(x,
  \mathrm{pr}_1(z)\bigl) + c_j(z) + v_{s-1}\bigl(\mathrm{pr}_2(z),K\sm\{j\}\bigl)\bigl]\in [0,\infty[
  .
$$

\begin{pred}
  \label{p3.5.1}{\TL}
Если $s\in \ov{1,N},$ то преобразование функции
$v_{s-1}$ в $v_s$
имеет следующий вид:
\bfn
  \label{3.5.21}
  v_s(x,K) = \min\limits_{j\in \mathbf{I}(K)}\ \min\limits_{z\in \bba_j(x)}
  \bigl[\mathbf{c}\bigl(x,\mathrm{pr}_1(z)\bigl) + c_j(z) + v_{s-1}\bigl(\mathrm{pr}_2(z),
  K\sm\{j\}\bigl)\bigl]\ \ \fa (x,K) \in D_s
  .
\efn
\end{pred}

Доказательство очевидным образом следует из теоремы~\ref{t3.4.1}
с учетом
(\ref{3.5.16}) и (\ref{3.5.18}).

Таким образом, мы располагаем рекуррентной процедурой
\bfn
  \label{3.5.22}
  v_o \longrightarrow v_1 \longrightarrow \ldots \longrightarrow v_N
  ;
\efn
здесь функция $v_o$ определена в (\ref{3.5.20})
и преобразование $v_{s-1}$ в $v_s,$
(где $s\in \ov{1,N}$)
указано в предложении~\ref{p3.5.1} (см. (\ref{3.5.21})).
После выполнения всех преобразований (\ref{3.5.22})
мы получим, в частности, что (см. (\ref{3.4.27}))
\begin{eqnarray}
  &V = v_N(x^o,\ov{1,N}) = v(x^o,\ov{1,N}) =
  \min\limits_{j\in \mathbf{I}(\overline{1,N})}\
  \min\limits_{z\in \bba_j(x^o)} \bigl[\mathbf{c}\bigl(x^o,\mathrm{pr}_1(z)\bigl) +
  &\nonumber\\
  &+c_j(z) +
  v_{N-1}\bigl(\mathrm{pr}_2(z),\ov{1,N}\sm\{j\}\bigl)\bigl]
  .
  \label{3.5.23}
\end{eqnarray}

\begin{zam}
\label{z3.5.1}{\TL}
Располагая значением $V,$ определяемым в $(\ref{3.5.23}),$
мы можем оценить свои возможности в части затрат на реализацию процедур $(\ref{3.3.8`}),
(\ref{3.3.8``}),$
не занимаясь еще непосредственным построением маршрута и трассы.
Если при
этом мы располагаем некоторым допуском $\mathbf{d}, \mathbf{d}\geqslant 0$
на величину
совокупных затрат, то после построения функций-слоев
$(\ref{3.5.22})$ мы можем сравнить $V$
с упомянутым допуском.
Если
\bfn
  \label{3.5.24}
  V \leqslant \mathbf{d}
  ,
\efn
то имеет смысл приступать к построению маршрута и трассы,
см. $(\ref{3.3.8`})$, $(\ref{3.3.8``})$.
Если же $V > \mathbf{d},$
то, по всей вероятности, такое построение
окажется бессмысленным и следует изменить что-то в условиях задачи с целью последующего
осуществления соотношения,
подобного $(\ref{3.5.24})$.
\hfill $\Box$
\end{zam}

{\bf Построение оптимального решения.}
Рассмотрим теперь процедуру построения оптимального ДР,
используя для этого слои функции Беллмана,
конструируемые посредством (\ref{3.5.21}), (\ref{3.5.22}).

Полагаем при этом
$\mathbf{z}^{(o)} \df (x^o,x^o).$
Учитываем, что согласно (\ref{3.5.16})
и определению $D_N$ при
$j\in \mathbf{I}(\ov{1,N})$ и $z\in \bba_j(x^o)$
$$
  \bigl(\mathrm{pr}_2(z),\ov{1,N}\sm \{j\}\bigl) \in D_{N-1}
  ,
$$
что позволяет определять значение
$$
  v_{N-1}\bigl(\mathrm{pr}_2(z),\ov{1,N}\sm \{j\}\bigl)\in [0,\infty[
    .
$$

Используя теперь (\ref{3.5.23}),
мы выбираем
$\mathbf{j}_1\in \mathbf{I}(\ov{1,N})$ и $\mathbf{z}^{(1)}\in \bba_{\mathbf{j}_1}(x^o)$
так, что при этом
\bfn
  \label{3.5.25}
  V = \mathbf{c}\bigl(x^o,\mathrm{pr}_1(\mathbf{z}^{(1)})\bigl) +
  c_{\mathbf{j}_1}(\mathbf{z}^{(1)}) + v_{N-1}\bigl(\mathrm{pr}_2(\mathbf{z}^{(1)}),
  \ov{1,N}\sm \{\mathbf{j}_1\}\bigl)
  .
\efn

Иными словами, мы решаем задачу
$$
  \mathbf{c}\bigl(x^o,\mathrm{pr}_1(z)\bigl) + c_j(z) + v_{N-1}\bigl(\mathrm{pr}_2(z),
  \ov{1,N}\sm\{j\}\bigl) \longrightarrow \min, j\in \mathbf{I}(\ov{1,N}), z\in \bba_j(x^o)
  ,
$$
оптимальное решение которой соответствует (\ref{3.5.25}).
Тогда
\bfn
  \label{3.5.26}
  \bigl(\mathrm{pr}_2(\mathbf{z}^{(1)}),\ov{1,N}\sm \{\mathbf{j}_1\}\bigl) \in D_{N-1}
\efn
и мы располагаем величиной
$v_{N-1}\bigl(\mathrm{pr}_2(\mathbf{z}^{(1)}),\ov{1,N}\sm
\{\mathbf{j}_1\}\bigl) \in [0,\infty[,$
для представления которой можно воспользоваться
соответствующим вариантом (\ref{3.5.21}).
Действительно, из (\ref{3.5.16}) и (\ref{3.5.26})
при $j\in \mathbf{I}(\ov{1,N}\sm \{\mathbf{j}_1\})$
и $z\in \bba_j\bigl(\mathrm{pr}_2 (\mathbf{z}^{(1)})\bigl)$
имеем
\bfn
  \label{3.5.26`}
  \bigl(\mathrm{pr}_2(z),\ov{1,N}\sm
  \{\mathbf{j}_1;j\}\bigl) = \bigl(\mathrm{pr}_2(z),(\ov{1,N}\sm\{\mathbf{j}_1\})\sm\{j\}
  \bigl)\in D_{N-2}
  ,
\efn
а потому определена величина
$$
  v_{N-2}\bigl(\mathrm{pr}_2(z),\ov{1,N}\sm\{\mathbf{j}_1;j\}\bigl)\in [0,\infty[
  .
$$

Более того, из предложения~\ref{p3.5.1} следует
(см. (\ref{3.5.21}), (\ref{3.5.26})),
что
\begin{eqnarray}
  &v_{N-1}\bigl(\mathrm{pr}_2(\mathbf{z}^{(1)}),\ov{1,N}\sm\{\mathbf{j}_1\}\bigl) =
  \min\limits_{j\in \mathbf{I}(\ov{1,N}\sm\{\mathbf{j}_1\})}\ \min\limits_{z\in
  \bba_j\bigl(\mathrm{pr}_2(\mathbf{z}^{(1)})\bigl)}\bigl[\mathbf{c}
  \bigl(\mathrm{pr}_2(\mathbf{z}^{(1)}),
  \mathrm{pr}_1(z)\bigl) +
  &\nonumber\\
  &+ c_j(z) + v_{N-2}\bigl(\mathrm{pr}_2(z),\ov{1,N}\sm\{\mathbf{j}_1;j\}\bigl)\bigl].
  &
  \label{3.5.27}
\end{eqnarray}

С учетом (\ref{3.5.27})
выбираем
$\mathbf{j}_2\in \mathbf{I}(\ov{1,N}\sm\{\mathbf{j}_1\})$
и $\mathbf{z}^{(2)}\in \bba_{\mathbf{j}_2}\bigl(\mathrm{pr}_2(\mathbf{z}^{(1)})\bigl)$
так, что при этом
\begin{eqnarray}
  &v_{N-1}\bigl(\mathrm{pr}_2(\mathbf{z}^{(1)}),\ov{1,N}\sm\{\mathbf{j}_1\}\bigl) =
  \mathbf{c}\bigl(\mathrm{pr}_2(\mathbf{z}^{(1)}),\mathrm{pr}_1(\mathbf{z}^{(2)})) +
  &\nonumber\\
  &+c_{\mathbf{j}_2}(\mathbf{z}^{(2)}) + v_{N-2}\bigl(\mathrm{pr}_2(\mathbf{z}^{(2)}),
  \ov{1,N}\sm\{\mathbf{j}_1;\mathbf{j}_2\}\bigl)
  .
  \label{3.5.28}
\end{eqnarray}

При этом согласно (\ref{3.5.26`}) получаем включение
\bfn
  \label{3.5.28}
  \bigl(\mathrm{pr}_2(\mathbf{z}^{(2)}),\ov{1,N}\sm\{\mathbf{j}_1;\mathbf{j}_2\}\bigl)\in D_{N-2}
  .
\efn

С другой стороны,
подставляя (\ref{3.5.28}) в (\ref{3.5.25}),
получаем равенство
\begin{eqnarray}
  &V = \mathbf{c}\bigl(x^o,\mathrm{pr}_1(\mathbf{z}^{(1)})\bigl) +
  c_{\mathbf{j}_1}(\mathbf{z}^{(1)}) + \mathbf{c}\bigl(\mathrm{pr}_2(\mathbf{z}^{(1)}),
  \mathrm{pr}_1(\mathbf{z}^{(2)})\bigl) + c_{\mathbf{j}_2}(\mathbf{z}^{(2)}) +
  &\nonumber\\
  &+v_{N-2}\bigl(\mathrm{pr}_2(\mathbf{z}^{(2)},\ov{1,N}\sm\{\mathbf{j}_1;\mathbf{j}_2\}\bigl) =
  \mathbf{c}\bigl(\mathrm{pr}_2(\mathbf{z}^{(o)}),\mathrm{pr}_1(\mathbf{z}^{(1)})\bigl) +
  &\nonumber\\
  &+ \mathbf{c}\bigl(\mathrm{pr}_2(\mathbf{z}^{(1)}),\mathrm{pr}_1(\mathbf{z}^{(2)})\bigl) +
  c_{\mathbf{j}_1}(\mathbf{z}^{(1)}) + c_{\mathbf{j}_2}(\mathbf{z}^{(2)}) +
  v_{N-2}\bigl(\mathrm{pr}_2(\mathbf{z}^{(2)}),\ov{1,N}\sm\{\mathbf{j}_1;\mathbf{j}_2\}\bigl) =
  &\nonumber\\
  &= \sum\limits_{t=1}^2 \bigl[\mathbf{c}\bigl(\mathrm{pr}_2(\mathbf{z}^{(t-1)}),
  \mathrm{pr}_1(\mathbf{z}^{(t)})\bigl) + c_{\mathbf{j}_t}(\mathbf{z}^{(t)})\bigl] +
  &\nonumber\\
  &+v_{N-2}\bigl(\mathrm{pr}_2(\mathbf{z}^{(2)}),\ov{1,N}\sm\{\mathbf{j}_1;\mathbf{j}_2\}\bigl).
  &
  \label{3.5.30}
\end{eqnarray}

По выбору $\mathbf{j}_2$
имеем, что
\bfn
  \label{3.5.31`}
  \mathbf{j}_2 \in \ov{1,N}\sm\{\mathbf{j}_1\}
  ,
\efn
то есть
$\mathbf{j}_1\in \ov{1,N}, \mathbf{j}_2\in \ov{1,N}$ и
$\mathbf{j}_1 \neq \mathbf{j}_2$,
см. (\ref{3.4.2}).

\begin{zam}
\label{z3.5.2}{\TL}
В связи с $(\ref{3.5.30})$
допустим
(в пределах настоящего замечания),
что $N = 2.$
Тогда согласно $(\ref{3.5.31`})$
мы в виде
$$
  (\mathbf{j}_s)_{s\in\ov{1,2}} = (\mathbf{j}_s)_{s\in\ov{1,N}}
$$
имеем маршрут:
$(\mathbf{j}_s)_{s\in\ov{1,2}} \in \bbp.$
Отметим, что
$\mathbf{j}_1\in \mathbf{I}(\ov{1,N}),$ где $\mathbf{I}(\ov{1,N}) = \mathbf{I}(\ov{1,2});\,
\mathbf{j}_2\in \mathbf{I}(\ov{1,N}\sm\{\mathbf{j}_1\}),$
где
$\ov{1,N} \sm\{\mathbf{j}_1\} =
\ov{1,N}\sm\{\mathbf{j}_t:\,t\in \ov{1,1}\}.$
Поскольку
$\ov{2,N} = \ov{2,2} = \{2\},$
получаем, что
$$
  \mathbf{j}_s \in \mathbf{I} \left(\ov{1,N}\sm\{\mathbf{j}_t:\,t\in \ov{1,s-1}\}\right)\ \ \fa
  s\in \ov{2,N}
  .
$$

Из $(\ref{3.4.8})$
выводим теперь следующее свойство допустимости:
\bfn
  \label{3.5.31}
  \cj \df (\mathbf{j}_s)_{s\in \ov{1,2}} = (\mathbf{j}_s)_{s\in \ov{1,N}}\in
  \mathbf{A}
  .
\efn

Далее, отметим, что
$\bba_{\mathbf{j}_1}(x^o) = \bba_{\mathbf{j}_1}\bigl(\mathrm{pr}_2(
\mathbf{z}^{(o)})\bigl) \subset \bbm_{\mathbf{j}_1}$
(учитываем, что $x^o\in \mathbf{X}),$ а потому
\bfn
  \label{3.5.32}
  \mathbf{z}^{(1)} \in \bbm_{\mathbf{j}_1}
  .
\efn

Тогда $($см. $(\ref{3.3.3}))$,
в частности,
$\mathrm{pr}_2(\mathbf{z}^{(1)})\in \mathbf{X}$
$($см. $(\ref{3.3.5}))$.
Поэтому, согласно $(\ref{3.3.12})$,
\bfn
  \label{3.5.33}
  \bba_{\mathbf{j}_2}\bigl(\mathrm{pr}_2(\mathbf{z}^{(1)})\bigl)\subset
  \bbm_{\mathbf{j}_2}
  .
\efn
Из $(\ref{3.5.33})$
имеем по выбору
$\mathbf{z}^{(2)},$ что
\bfn
  \label{3.5.34}
  \mathbf{z}^{(2)} \in \bbm_{\mathbf{j}_2}
  ,
\efn
а тогда, в частности,
$\mathrm{pr}_2(\mathbf{z}^{(2)})\in \mathbf{M}_{\mathbf{j}_2}$
и, как следствие,
$\mathrm{pr}_2(\mathbf{z}^{(2)})\in \mathbf{X}.$

Теперь с учетом $(\ref{3.3.2}), (\ref{3.3.3}), (\ref{3.5.32}), (\ref{3.5.34})$ и
определения $\mathbf{z}^{(o)}$ получаем, что $($см. $(\ref{3.3.14}))$
\bfn
  \label{3.5.35}
  (\mathbf{z}^{(s)})_{s\in\ov{0,2}} =
  (\mathbf{z}^{(s)})_{s\in\ov{0,N}}\in \bbz
  .
\efn

Из $(\ref{3.3.15}), (\ref{3.5.32}), (\ref{3.5.34})$ и $(\ref{3.5.35})$ следует, что
\bfn
  \label{3.5.36}
  (\mathbf{z}^{(s)})_{s\in\ov{0,2}} = (\mathbf{z}^{(s)})_{s\in\ov{0,N}}
 \in \mathbf{Z}_\cj
 .
\efn

Далее, из $(\ref{3.5.31})$ имеем, что
$\cj(1) = \mathbf{j}_1$ и $\cj(2) = \mathbf{j}_2,$
а потому по выбору
$\mathbf{z}^{(1)}$ и $\mathbf{z}^{(2)}$
получаем
\bfn
  \label{3.5.37}
  \mathbf{z}^{(1)} \in \bba_{\cj(1)} \bigl(\mathrm{pr}_2(\mathbf{z}^{(o)})\bigl),
  \ \mathbf{z}^{(2)} \in \bba_{\cj(2)} \bigl(\mathrm{pr}_2(\mathbf{z}^{(1)})\bigl)
  .
\efn

С учетом $(\ref{3.3.16}), (\ref{3.5.36})$ и $(\ref{3.5.37})$
находим
(в рассматриваемом случае $N = 2$),
что
$$
  (\mathbf{z}^{(s)})_{s\in\ov{0,2}} = (\mathbf{z}^{(s)})_{s\in\ov{0,N}} \in \cz_\cj
  ,
$$
а тогда, согласно
$(\ref{3.3.26})$ и $(\ref{3.5.31})$
\bfn
  \label{3.5.38}
  \bigl(\cj,(\mathbf{z}^{(s)})_{s\in\ov{0,2}} \bigl) =
  \bigl((\mathbf{j}_s)_{s\in\ov{1,2}},(\mathbf{z}^{(s)})_{s\in\ov{0,2}} \bigl) \in \mathbf{D}
  .
\efn

Итак, в $(\ref{3.5.38})$
имеем упорядоченную пару, являющуюся ДР основной задачи.
При этом,
согласно
$(\ref{3.3.30})$ и $(\ref{3.5.30})$,
получаем, что в нашем случае
\begin{eqnarray}
  &V = \mathfrak{C}_\cj[(\mathbf{z}^{(s)})_{s\in\ov{0,N}}] +
  v_{N-2}\bigl(\mathrm{pr}_2(\mathbf{z}^{(2)}),\ov{1,N}\sm\{\mathbf{j}_1;\mathbf{j}_2\}\bigl) -
  f\bigl(\mathrm{pr}_2(\mathbf{z}^{(N)})\bigl) =
  &\nonumber\\
  &= \mathfrak{C}_\cj[(\mathbf{z}^{(s)})_{s\in\ov{0,2}}] + v_o\bigl(\mathrm{pr}_2(\mathbf{z}^{(2)}),
  \ov{1,2}\sm \{\mathbf{j}_1;\mathbf{j}_2\}\bigl) - f\bigl(\mathrm{pr}_2(\mathbf{z}^{(2)})\bigl)
  ,
  &
  \label{3.5.39}
\end{eqnarray}
где $($см. $(\ref{3.5.9}))$
имеет место включение
\bfn
  \label{3.5.40}
  \bigl(\mathrm{pr}_2(\mathbf{z}^{(2)}),\ov{1,N}\sm\{\mathbf{j}_1;\mathbf{j}_2\}\bigl) =
  \bigl(\mathrm{pr}_2(\mathbf{z}^{(2)}),\emp\bigl) \in D_o
  .
\efn
Напомним, что
$\mathbf{j}_1\neq \mathbf{j}_2.$
Из $(\ref{3.5.9})$
и
$(\ref{3.5.40})$ вытекает, что
$$
  \mathrm{pr}_2(\mathbf{z}^{(2)})\in \bigcup\limits_{i\in\ov{1,N}\sm \mathbf{K}_1}\mathbf{M}_i
  ,
$$
а тогда из $(\ref{3.5.20})$ получаем
при помощи
$(\ref{3.5.40})$
$$
  v_o\bigl(\mathrm{pr}_2(\mathbf{z}^{(2)}),\ov{1,2}\sm\{\mathbf{j}_1;\mathbf{j}_2\}\bigl) =
  f\bigl(\mathrm{pr}_2(\mathbf{z}^{(2)})\bigl)
  .
$$

В итоге из $(\ref{3.5.39})$
получаем, что
\bfn
  \label{3.5.40`}
  \mathfrak{C}_\cj[(\mathbf{z}^{(s)})_{s\in\ov{0,2}}]=
  \mathfrak{C}_\cj[(\mathbf{z}^{(s)})_{s\in\ov{0,N}}] = V
  .
\efn

Итак, ДР $(\ref{3.5.38})$
оптимально в основной задаче маршрутизации перемещений.
\hfill $\Box$
\end{zam}

Возвращаясь к общему случаю $N\geqslant 2,$
отметим, что процедуру, намеченную
в (\ref{3.5.25})--(\ref{3.5.30}),
следует
(при $N \neq 2)$
продолжать далее,
вплоть до исчерпывания всего индексного множества $\ov{1,N}$
(см. в этой связи
\cite[раздел~7]{Cha3`}).
В итоге будет построено оптимальное ДР
\bfn
  \label{3.5.41}
  \bigl(\eta,(\mathbf{z}^{(k)})_{k\in\ov{0,N}}\bigl)\in \mathbf{D}
  ,
\efn
где
$\eta = (\mathbf{j}_k)_{k\in\ov{1,N}}\in \mathbf{A}$
и
$(\mathbf{z}^{(k)})_{k\in\ov{0,N}}\in \cz_\eta;\
\mathfrak{C}_\eta[(\mathbf{z}^{(k)})_{k\in\ov{0,N}}] = V.$
Данное построение, приводящее к ДР (\ref{3.5.41})
со свойством
оптимальности, на идейном уровне соответствует рассуждению в
замечании~\ref{z3.5.2},
где рассмотрен случай $N = 2.$
