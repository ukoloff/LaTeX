% !TeX root = ../mat_mod2.tex

\section{
  Алгоритм на функциональном уровне;
  вставка в начало
}
\label{sect:4.5}
\setcounter{equation}{0}

Сейчас, располагая  значением (\ref{4.4.70}),
отвечающим улучшению ДР в локальной задаче,
мы распространим данное улучшение на случай <<большой>> задачи
(используем конструкции \cite{Cha13`}).
Для этой цели будем использовать
локальное ДР с компонентами (\ref{4.4.65}), (\ref{4.4.66}).
Напомним, что данное ДР
$$
  \bigl(\al^o,(z_i^o)_{i\in\ov{0,N}}\bigl)\in \widetilde{\mathbf{D}}
$$
реализует равенство (\ref{4.4.67}),
что уже было учтено в (\ref{4.4.70}).

Отметим, что согласно предложению~\ref{p4.4.2}
\bfn
  \label{4.5.1}
  \eta \df (\al^o - \mathrm{sew})[\la;\nu] \in\ca
\efn
есть
$\mathfrak{K}$-допустимый маршрут <<большой>> задачи.
Тогда, в частности,
\bfn
  \label{4.5.1`}
  \eta: \ov{1,\mathbf{n}}{\stackrel{\mbox{\footnotesize{на}}}{\longrightarrow}}\,\ov{1,\mathbf{n}}
  ,
\efn
при этом (см. определение~\ref{p4.4.1})
\bfn
  \label{4.5.2}
  \bigl(\eta(t) = \la(t)\ \ \fa t\in \ov{1,\mathbf{n}}\setminus
  \ov{\nu+1,\nu+N}\bigl)\,\&\,\bigl(\eta(t) = (\La\circ \al^o)(t- \nu)\ \ \fa t\in \ov{\nu+1,\nu+N}\bigl)
  .
\efn

Ясно, что в силу (\ref{4.5.2})
\bfn
  \label{4.5.3}
  \bigl(\eta(t) = \la(t)\ \ \fa t\in \ov{1,\nu}\bigl)\,\&\,\bigl(\eta(t) =
  \la(t)\ \ \fa t\in \ov{\nu+N+1,\mathbf{n}}\bigl)
  .
\efn

Кроме того, из (\ref{4.5.2}) вытекает, что
\bfn
  \label{4.5.4}
  \eta(t) = \la\bigl(\nu + \al^o(t-\nu)\bigl)\ \ \fa t\in \ov{\nu+1,\nu+N}
  .
\efn

Соотношения (\ref{4.5.2}) -- (\ref{4.5.4})
исчерпывающим образом характеризуют
допустимый маршрут $\eta$ (\ref{4.5.1})
(имеется в виду допустимость в <<большой>> задаче).
Назовем маршрут (\ref{4.5.1}) склеенным,
что вполне соответствует
(\ref{4.5.2}) -- (\ref{4.5.4}).
Отметим некоторые свойства данного маршрута.

Из (\ref{4.5.2}) непосредственно следует, что
\bfn
  \label{4.5.5}
  \eta^1(\ov{t,\mathbf{n}}) = \la^1(\ov{t,\mathbf{n}})\ \ \fa t\in \ov{\nu+N+1, \mathbf{n}}
  .
\efn

Свойство (\ref{4.5.5}) дополняется следующим положением.

\begin{pred}
\label{p4.5.1}{\TL}
Если
$t\in \ov{1,\nu},$
то
$\eta^1(\ov{t,\mathbf{n}}) = \la^1(\ov{t,\mathbf{n}}).$
\end{pred}

Д о к а з а т е л ь с т в о.
Фиксируем индекс $t\in \ov{1,\nu}.$
Тогда $t\leqslant \nu.$ При этом
\bfn
  \label{4.5.6}
  (t=1) \vee (t\in \ov{2,\nu}).
\efn

Заметим теперь, что
$\la^1(\ov{1,\mathbf{n}}) = \ov{1,\mathbf{n}}$
по выбору $\la.$
С другой стороны, из (\ref{4.5.1`}) имеем, что
$\eta^1(\ov{1,\mathbf{n}}) =\ov{1,\mathbf{n}}.$
Поэтому
\bfn
  \label{4.5.6`}
  \la^1(\ov{1,\mathbf{n}})  =\ov{1,\mathbf{n}} =\eta^1(\ov{1,\mathbf{n}})
  .
\efn

Как следствие получаем импликацию
\bfn
  \label{4.5.7}
  (t=1)\Longrightarrow \bigl(\la^1(\ov{t,\mathbf{n}}) = \eta^1(\ov{t,\mathbf{n}})\bigl)
  .
\efn

Рассмотрим теперь из (\ref{4.5.6})) второй случай
$t\in \ov{2,\nu}.$
Тогда
$$
  t-1 \in \ov{1,\nu-1}
$$
и в частности
$t-1\in \ov{1,\mathbf{n}}.$
По свойствам операции взятия образа имеем
\bfn
  \label{4.5.8}
  \la^1(\ov{1,\mathbf{n}}) = \la^1(\ov{1,t-1} \cup \ov{t,\mathbf{n}}) = \la^1(\ov{1,t-1})
  \cup \la^1(\ov{t,\mathbf{n}})
  ,
\efn
\bfn
  \label{4.5.9}
  \eta^1(\ov{1,\mathbf{n}}) = \eta^1(\ov{1,t-1} \cup \ov{t,\mathbf{n}}) = \eta^1(\ov{1,t-1})
  \cup \eta^1(\ov{t,\mathbf{n}})
  .
\efn

Далее в силу инъективности $\la$ и $\eta$ получаем, что
\bfn
  \label{4.5.10}
  \la^1(\ov{1,t-1}) \cap  \la^1(\ov{t,\mathbf{n}}) = \emp
  ,
\efn
\bfn
  \label{4.5.11}
  \eta^1(\ov{1,t-1}) \cap  \eta^1(\ov{t,\mathbf{n}}) = \emp
  .
\efn

Из (\ref{4.5.8}), (\ref{4.5.10})
следует очевидное равенство
$$
  \la^1(\ov{t,\mathbf{n}}) = \la^1(\ov{1,\mathbf{n}})\setminus \la^1(\ov{1,t-1})
  ,
$$
а из (\ref{4.5.9}) и (\ref{4.5.11})
имеем, в свою очередь, что
$$
  \eta^1(\ov{t,\mathbf{n}}) = \eta^1(\ov{1,\mathbf{n}})\setminus \eta^1(\ov{1,t-1})
  .
$$

Исходя из (\ref{4.5.3})
$\eta^1(\ov{1,t-1}) = \la^1(\ov{1,t-1}),$
теперь и в случае $t\in \ov{2,\nu}$
(см. (\ref{4.5.6`}))
имеем требуемое равенство
\bfn
  \label{4.5.12}
  \la^1(\ov{t,\mathbf{n}}) = \eta^1(\ov{t,\mathbf{n}})
  .
\efn

Итак, установлена импликация
$$
  (t\in \ov{2,\nu}) \Longrightarrow \bigl(\la^1(\ov{t,\mathbf{n}}) = \eta^1(\ov{t,\mathbf{n}})\bigl)
  .
$$

С учетом (\ref{4.5.6}) и (\ref{4.5.7})
получаем, что (\ref{4.5.12}) имеет место во всех
возможных случаях.
\hfill $\Box$

\begin{cor}
\label{c4.5.1}
Если
$t\in \ov{1,\mathbf{n}}\setminus \ov{\nu+1,\nu+N},$
то
$\eta^1(\ov{t,\mathbf{n}}) = \la^1(\ov{t,\mathbf{n}}).$
\end{cor}

Действительно, по выбору $\nu$ имеем, что
$$
  \ov{1,\mathbf{n}}\setminus \ov{\nu+1,\nu+N} = \ov{1,\nu} \cup \ov{\nu+N+1,\mathbf{n}}
  ,
$$
а потому, комбинируя (\ref{4.5.5})
и предложение~\ref{p4.5.1},
получаем нужное утверждение.

\begin{pred}
\label{p4.5.2}
{\TL}
Если $t\in \ov{\nu+1,\nu+N},$
то справедливо равенство
\bfn
  \label{4.5.12`}
  \eta^1(\ov{t,\mathbf{n}}) = \la^1\bigl(\{\nu + \al^o(s):\,s\in \ov{t-\nu,N}\} \cup
  \ov{\nu+N+1,\mathbf{n}}\bigl)
  .
\efn
\end{pred}

Д о к а з а т е л ь с т в о.
Введем в рассмотрение множество
\bfn
  \label{4.5.13}
  A \df \{\nu +\al^o(s):\,s\in \ov{t-\nu,N}\}\in \cp^\prime(\ov{\nu+1,\nu+N})
  ,
\efn
где
$t\in \ov{\nu+1,\nu+N}$ фиксировано и
$t-\nu\in \ov{1,N}.$
Из (\ref{4.5.13}) имеем, что
$A$ --- непустое множество и при этом
$$
  A \subset \ov{\nu+1,\nu+N}
  .
$$

В связи с (\ref{4.5.12`})
отметим, что требуется установить равенство
\bfn
  \label{4.5.14}
  \eta^1(\ov{t,\mathbf{n}}) = \la^1(A \cup \ov{\nu+N+1,\mathbf{n}})
  .
\efn

Выберем произвольный индекс
\bfn
  \label{4.5.15}
  p\in \eta^1(\ov{t,\mathbf{n}})
  .
\efn

Тогда
$p\in \ov{1,\mathbf{n}}$
и при этом $p=\eta(\tau),$ где
$\tau\in \ov{t,\mathbf{n}}$ таково, что
\bfn
  \label{4.5.16}
  (\tau\in \ov{\nu+1,\nu+N}) \vee (\tau\in \ov{\nu+N+1,\mathbf{n}})
  .
\efn

Оба случая в (\ref{4.5.16})
рассмотрим отдельно.

1) Пусть $\tau\in \ov{\nu+1,\nu+N}.$
Получаем тогда, что $\tau- \nu\in \ov{1,N}$
и в силу (\ref{4.5.2})
$$
  p = (\La \circ \al^o)(\tau-\nu) = \La\bigl(\al^o(\tau-\nu)\bigl)
  ,
$$
откуда по определению $\La$ следует, что
\bfn
  \label{4.5.17}
  p = \la\bigl(\nu + \al^o(\tau-\nu)\bigl)
  .
\efn

Из (\ref{4.5.13}) следует, что
$\nu + \al^o(\tau-\nu) \in A,$
так как по выбору
$\tau$ имеем в рассматриваемом случае
$$
  \tau-\nu \in \ov{t-\nu,N}
  .
$$

Поэтому $p\in \la^1(A),$ где по свойствам операции взятия образа
$$
  \la^1(A) \subset \la^1(A \cup \ov{\nu+N+1,\mathbf{n}})
  .
$$

В итоге
$p\in \la^1(A \cup \ov{\nu+N+1,\mathbf{n}})$
в случае 1),
чем завершается проверка импликации
\bfn
  \label{4.5.18}
  (\tau\in \ov{\nu+1,\nu+N}) \Longrightarrow \bigl(p\in \la^1(A \cup
  \ov{\nu+N+1,\mathbf{n}})\bigl)
  .
\efn

2) Пусть $\tau \in \ov{\nu+N+1,\mathbf{n}}.$
Тогда в силу (\ref{4.5.2})
$$
  p= \eta(\tau) \in \eta^1(\ov{\nu+N+1,\mathbf{n}})
  ,
$$
где согласно (\ref{4.5.5})
$\eta^1(\ov{\nu+N+1,\mathbf{n}}) = \la^1(\ov{\nu+N+1,\mathbf{n}}),$
а потому
\bfn
  \label{4.5.19}
  p\in \la^1(\ov{\nu+N+1,\mathbf{n}})
  .
\efn

Вновь, используя простейшие свойства  операции взятия образа, имеем, что
\bfn
  \label{4.5.20}
  p\in \la^1(A \cup \ov{\nu+N+1,\mathbf{n}})
\efn
и в случае 2). Получаем импликацию
$$
  (\tau\in \ov{\nu+N+1,\mathbf{n}})\Longrightarrow \bigl(p\in \la^1(A \cup
  \ov{\nu+N+1,\mathbf{n}})\bigl)
  .
$$

С учетом (\ref{4.5.16}) и (\ref{4.5.18})
получаем, что (\ref{4.5.20})
справедливо во всех возможных случаях.
Поскольку выбор (\ref{4.5.15}) был
произвольным, установлено, что
\bfn
  \label{4.5.21}
  \eta^1(\ov{t,\mathbf{n}}) \subset \la^1(A \cup \ov{\nu+N+1,\mathbf{n}})
  .
\efn

Выберем теперь произвольный индекс
\bfn
  \label{4.5.22}
  q\in  \la^1(A \cup \ov{\nu+N+1,\mathbf{n}})
  .
\efn

Из (\ref{4.5.22}) следует, что
$q\in \ov{1,\mathbf{n}},$ причем для некоторого
\bfn
  \label{4.5.23}r \in A \cup \ov{\nu+N+1,\mathbf{n}}
\efn
справедливо следующее равенство:
\bfn
  \label{4.5.23`}q = \la(r)
  .
\efn

Из (\ref{4.5.23})
с очевидностью вытекает, что
\bfn
  \label{4.5.24}
  (r\in A) \vee (r\in  \ov{\nu+N+1,\mathbf{n}})
  .
\efn

Обе возможности в (\ref{4.5.24}) рассмотрим отдельно.

$1^\prime)$
Пусть $r\in A.$
Тогда (см. (\ref{4.5.13}))
для некоторого
\bfn
  \label{4.5.25}
  \theta\in \ov{t-\nu,N}
\efn
справедливо равенство
$r = \nu + \al^o(\theta),$
а потому (см. (\ref{4.5.23`}))
\bfn
  \label{4.5.26}
  q = \la\bigl(\nu + \al^o(\theta)\bigl)
  .
\efn

Поскольку $\theta\in\ov{1,N},$
то для индекса
  $$\widetilde{\theta} \df \nu +\theta \in \ov{t,\nu+N}
$$
имеем, в частности,
$\widetilde{\theta}\in \ov{\nu+1,\nu +N},$
а тогда, согласно (\ref{4.5.2}),
\bfn
  \label{4.5.27}
  \eta(\widetilde{\theta}) =
  \La\bigl(\al^o(\widetilde{\theta}-\nu)\bigl) = \la\bigl(\nu+ \al^o(\theta)\bigl)
  ,
\efn
где $\eta(\widetilde{\theta}) \in \eta^1(\ov{t,\nu+N})$
и в частности
$\eta(\widetilde{\theta}) \in \eta^1(\ov{t,\mathbf{n}}),$
поскольку
$$
  \eta^1(\ov{t,\nu+N})\subset \eta^1(\ov{t,\mathbf{n}})
  .
$$

С учетом (\ref{4.5.26}) и (\ref{4.5.27}) получаем, что
$$
  q = \eta(\widetilde{\theta}) \in \eta^1(\ov{t,\mathbf{n}})
$$
в рассматриваемом случае $1^\prime).$
Итак,
\bfn
  \label{4.5.28}
  (r\in A) \Longrightarrow \bigl(q \in \eta^1(\ov{t,\mathbf{n}})\bigl)
  .
\efn

$2^\prime)$  Пусть теперь
$r\in \ov{\nu+N +1,\mathbf{n}}.$
Тогда с учетом
(\ref{4.5.5}) и (\ref{4.5.23`})
\bfn
  \label{4.5.29}
  q\in \eta^1(\ov{\nu+N+1,\mathbf{n}})
  ,
\efn
поскольку
$\la^1(\ov{\nu+N+1,\mathbf{n}})= \eta^1(\ov{\nu+N+1,\mathbf{n}})$
и
$q = \la(r)\in \la^1(\ov{\nu+N+1,\mathbf{n}}).$

Из (\ref{4.5.29}) вытекает, что
$q\in \eta^1(\ov{t,\mathbf{n}})$
и в случае $2^\prime).$
Итак,
$$
  (r\in \ov{\nu+N+1,\mathbf{n}})\Longrightarrow \bigl(q\in \eta^1(\ov{t,\mathbf{n}})\bigl)
  .
$$

С учетом (\ref{4.5.24}) и (\ref{4.5.28}) имеем теперь, что
$q\in \eta^1(\ov{t,\mathbf{n}})$
во всех возможных случаях.
Поскольку выбор
(\ref{4.5.22}) был произвольным,
установлено, что
$$
  \la^1(A \cup \ov{\nu+N+1,\mathbf{n}})\subset \eta^1(\ov{t,\mathbf{n}})
  ,
$$
откуда с учетом (\ref{4.5.21})
получаем требуемое равенство (\ref{4.5.14}).
\hfill $\Box$

Из предложения~\ref{p4.5.2} вытекает
(с использованием простейших свойств операции взятия образа),
что
\begin{eqnarray}
  &\eta^1(\ov{t,\mathbf{n}}) = \la^1\bigl(\{\nu + \al^o(s):\,s\in \ov{t-\nu,N}\}\bigl)
  \cup \la^1 (\ov{\nu+N+1,\mathbf{n}}) =
  &\nonumber\\
  &=\{\la\bigl(\nu+\al^o(s)\bigl):\,s\in \ov{t-\nu,N}\}) \cup \la^1(\ov{\nu+N+1,\mathbf{n}})=
  &\nonumber\\
  &= \{(\La \circ \al^o)(s):\,s\in\ov{t-\nu,N}\} \cup \la^1(\ov{\nu+N+1,\mathbf{n}}) =
  &\nonumber\\
  &= (\La\circ \al^o)^1(\ov{t-\nu,N}) \cup \la^1(\ov{\nu+N+1,\mathbf{n}})\ \ \fa t\in \ov{\nu+1,\nu+N}
  .
  &
  \label{4.5.30}
\end{eqnarray}

Из (\ref{4.5.5}),
учитывая следствия~\ref{c4.5.1} и (\ref{4.5.30}),
получаем, что
\begin{eqnarray}
  &\bigl(\eta^1(\ov{t,\mathbf{n}}) = \la^1(\ov{t,\mathbf{n}})\ \ \fa t\in
  \ov{1,\mathbf{n}}\setminus \ov{\nu+1,\nu+N}\bigl)\,\&\,\bigl(\eta^1(\ov{t,\mathbf{n}}) =
  &\nonumber\\
  &=(\La\circ \al^o)^1(\ov{t-\nu,N}) \cup \la^1(\ov{\nu+N+1,\mathbf{n}})\ \ \fa t\in \ov{\nu+1,\nu+N}\bigl)
  .
  &
  \label{4.5.31}
\end{eqnarray}

В (\ref{4.5.31})
явным образом указаны списки заданий, складывающиеся вдоль склеенного маршрута.
Введем теперь в рассмотрение склеенную трассу.
Точнее, речь идет о вклеивании в
$(\mathbf{h}_t)_{t\in \ov{0,\mathbf{n}}}$
фрагмента, определяемого посредством (\ref{4.4.66}).

Итак, введем в рассмотрение кортеж
\bfn
  \label{4.5.31`}
  (\hat{\mathbf{h}}_t)_{t\in \ov{0,\mathbf{n}}}\in \widetilde{\mathfrak{Z}}
  ,
\efn
определяемый с помощью следующих условий:
\bfn
  \label{4.5.32}
  (\hat{\mathbf{h}}_t \df z_{t-\nu}^o\ \ \fa t\in \ov{\nu+1,\nu+N})\,
  \&\,(\hat{\mathbf{h}}_t \df \mathbf{h}_t\ \ \fa t\in \ov{0,\mathbf{n}}\setminus \ov{\nu+1,\nu+N})
  .
\efn

\begin{pred}
\label{p4.5.3}
{\TL}
Кортеж $(\ref{4.5.31`})$ является трассой,
согласованной с маршрутом $\eta:$
\bfn
  \label{4.5.33}
  (\hat{\mathbf{h}}_t)_{t\in \ov{0,\mathbf{n}}}\in \mathfrak{Z}_\eta
  .
\efn
\end{pred}

Д о к а з а т е л ь с т в о.
Из (\ref{4.5.32}) вытекает, что
\bfn
  \label{4.5.34}
  \hat{\mathbf{h}}_o = \mathbf{h}_o = (\mathbf{x}_o,\mathbf{x}_o)
  .
\efn

Если $t\in \ov{\nu+1,\nu+N},$
то
$t-\nu \in \ov{1,N}$
и в силу (\ref{4.5.32}) имеем включение
$$
  \hat{\mathbf{h}}_t = z_{t-\nu}^o \in \mathbb{M}_{\al^o(t-\nu)}
  ,
$$
а потому с учетом (\ref{4.4.59})
$\hat{\mathbf{h}}_t \in  \mathbb{L}_{(\La\circ \al^o)(t-\nu)}$.

Как следствие получаем (см. (\ref{4.5.2})), что
$$
  \hat{\mathbf{h}}_t \in  \mathbb{L}_{\eta(t)}
  .
$$

Наконец, при
$t\in \ov{1,\mathbf{n}}\setminus \ov{\nu+1,\nu+N}$
имеем, согласно (\ref{4.5.2}),
что $\eta(t) = \la(t),$
а потому (см. (\ref{4.4.18}), (\ref{4.5.32}))
$$
  \hat{\mathbf{h}}_t = \mathbf{h}_t\in \mathbb{L}_{\eta(t)}
  .
$$

Таким образом (см. (\ref{4.5.34}))
установлено, что кортеж (\ref{4.5.31`}) обладает свойствами
$$
  \bigl(\hat{\mathbf{h}}_o = (\mathbf{x}_o,\mathbf{x}_o)\bigl)\,\&\,(\hat{\mathbf{h}}_t  \in
  \mathbb{L}_{\eta(t)}\ \ \fa t\in \ov{1,\mathbf{n}})
  .
$$

Это означает справедливость требуемого включения (\ref{4.5.33}).
\hfill $\Box$

Из (\ref{4.5.1}) и предложения~\ref{p4.5.3} вытекает, что
$\bigl(\eta,(\hat{\mathbf{h}}_t)_{t\in\ov{0,\mathbf{n}}}\bigl)$
есть ДР <<большой>> задачи, а потому определено
$$
  \widehat{\mathfrak{C}}_\eta[(\hat{\mathbf{h}}_t)_{t\in\ov{0,\mathbf{n}}}]\in [0,\infty[
  ,
$$
при этом
$\bigl(\eta,(\hat{\mathbf{h}}_t)_{t\in\ov{0,\mathbf{n}}}\bigl)\in \mathrm{SOL}.$

Обсудим некоторые свойства данного ДР,
выделяя для отдельного рассмотрения следующие два случая:
$\nu = 0$, $\nu \neq 0.$
В настоящем разделе обсудим более простой первый случай.
Рассмотрение второго будет проведено в следующем разделе.

Итак, пусть до конца  настоящего раздела
\bfn
  \label{4.5.35}
  \nu = 0
  .
\efn

В этом случае имеем, следующие два равенства
\bfn
  \label{4.5.36}
  (\ov{\nu+1,\nu+N} = \ov{1,N})\,\&\,(\ov{1,\mathbf{n}}\setminus
  \ov{\nu+1,\nu+N}= \ov{N+1,\mathbf{n}})
  .
\efn

Рассмотрим коррекцию начальной <<части>> исходного ДР <<большой>> задачи.
В этом случае согласно (\ref{4.5.2}), (\ref{4.5.3}) и (\ref{4.5.36})
\bfn
  \label{4.5.37}
  \bigl(\eta(t) = (\La \circ \al^o)(t) \ \ \fa t\in \ov{1,N}\bigl)\,\&\,\bigl(\eta(t) =\la(t) \ \
  \fa t\in \ov{N+1,\mathbf{n}}\bigl)
  .
\efn

При этом
$\La(s) = \la(s)\ \ \fa s\in \ov{1,N}.$
Как следствие
$$
  \Gamma = \la^1(\ov{1,N})
  .
$$

С учетом первого выражения в (\ref{4.5.37}) имеем, что
\bfn
  \label{4.5.38}
  \eta(t) = \La\bigl(\al^o(t)\bigl) = \la\bigl(\al^o(t)\bigl) \ \ \fa t\in \ov{1,N}
  .
\efn

Кроме того, отметим, что (в случае (\ref{4.5.35}))
\bfn
  \label{4.5.39}
  x^o = \mathrm{pr}_2(\mathbf{h}_\nu) = \mathrm{pr}_2(\mathbf{h}_o)= \mathbf{x}_o
  .
\efn

Из общих построений с учетом (\ref{4.5.35}) вытекает, что
$\fa s\in \ov{1,N}$
\bfn
  \label{4.5.40}
  (M_s = \mathbf{L}_{\la(s)})\,\&\,(\mathbb{M}_s = \mathbb{L}_{\la(s)})
  .
\efn

Таким образом, в рассматриваемом случае имеем
$$
  \mathbf{M}_s = \{\mathrm{pr}_2(z):\,z\in \bbl_{\la(s)}\}\in \cp^\prime(\mathbf{L}_{\la(s)})
  .
$$
Кроме того, очевидно равенство
$$
  \bbx = \{\mathbf{x}_o\} \cup \Bigl(\bigcup\limits_{s=1}^N\mathbf{L}_{\la(s)}\Bigl)
  .
$$

Если
$K\in \mathfrak{N},$
то
$\La^1(K) \cup \la^1(\ov{\nu+N+1,\mathbf{n}}) = \la^1(K \cup \ov{N+1,\mathbf{n}});$
поэтому
$$
  \mathbf{c}(z,K) = \mathbf{c}^\natural\bigl(z,\la^1(K \cup \ov{N+1,\mathbf{n}})\bigl)\ \
  \fa z\in \bbx \times \bbx\ \ \fa K\in \mathfrak{N}
  .
$$

Из общих определений легко следует, что
$$
  c_j(z,K) = c_{\la(j)}^\natural\bigl(z,\la^1(K \cup \ov{N+1,\mathbf{n}})\bigl)\ \ \fa j\in
  \ov{1,N}\ \ \fa z\in \bbx \times \bbx\ \ \fa K\in \mathfrak{N}
  .
$$

Наконец имеем
(в рассматриваемом случае (\ref{4.5.35})), что
$$
  f(x) = \mathbf{c}^\natural\bigl(x,\mathrm{pr}_1(\mathbf{h}_{N+1}),\la^1(\ov{N+1,
  \mathbf{n}})\bigl)\ \ \fa x\in \bbx
  .
$$

С учетом упомянутых конкретизаций рассмотрим нужный вариант представления значения
$$
  \mathfrak{B}_{\al^o}[(z_i^o)_{i\in\ov{0,N}}] \in [0,\infty[
    .
$$

В самом деле из (\ref{4.2.4}) имеем следующее равенство
\begin{eqnarray}
  &\mathfrak{B}_{\al^o}[(z_i^o)_{i\in\ov{0,N}}] =
  \sum\limits_{t=1}^N\bigl[\mathbf{c}^\natural\Bigl(\mathrm{pr}_2(z_{t-1}^o),\mathrm{pr}_1(z_t^o),\la^1\bigl(
  \{\al^o(s):\,s\in\ov{t,N}\} \cup \ov{N+1,\mathbf{n}})\bigl)+&\nonumber\\
  &+ c_{\la(\al^o(t))}^\natural\Bigl(z_t^o,\la^1\bigl(\{\al^o(s):\,s\in \ov{t,N}\}\cup\ov{N+1,\mathbf{n}})\bigl)\bigl]+
  &\nonumber\\
  &+\mathbf{c}^\natural\bigl(\mathrm{pr}_2(z_N^o),\mathrm{pr}_1(\mathbf{h}_{N+1}),\la^1(\ov{N+1,\mathbf{n}})\bigl)
  .
  &
  \label{4.5.41}
\end{eqnarray}

Напомним, что в рассматриваемом сейчас случае
согласно (\ref{4.5.32}) и (\ref{4.5.39})
\bfn
  \label{4.5.42}
  (\hat{\mathbf{h}}_t = z_t^o\ \ \fa t\in \ov{1,N})\,\&\,\bigl(\hat{\mathbf{h}}_o = \mathbf{h}_o =
  (\mathbf{x}_o,\mathbf{x}_o)\bigl)\,\&\,(\hat{\mathbf{h}}_t = \mathbf{h}_t\ \ \fa t\in \ov{N+1,\mathbf{n}})
  .
\efn

Из (\ref{4.5.32}) и (\ref{4.5.42}), в частности, следует, что
\bfn
  \label{4.5.43}
  \mathbf{c}^\natural\bigl(\mathrm{pr}_2(z_N^o),\mathrm{pr}_1(\mathbf{h}_{N+1}),
  \la^1(\ov{N+1,\mathbf{n}})\bigl) = \mathbf{c}^\natural\bigl(\mathrm{pr}_2(\hat{\mathbf{h}}_N),
  \mathrm{pr}_1(\hat{\mathbf{h}}_{N+1}),
  \eta^1(\ov{N+1,\mathbf{n}})\bigl)
  .
\efn

В связи с конкретизацией слагаемых в первой сумме в (\ref{4.5.41})
заметим, что
$$
  \mathbf{c}^\natural\Bigl(\mathrm{pr}_2(z_o^o),\mathrm{pr}_1(z_1^o), \la^1\bigl(\{\al^o(s):\,s\in
  \ov{1,N}\}\cup \ov{N+1,\mathbf{n}}\bigl)\Bigl) =
  \mathbf{c}^\natural\bigl(x^o,\mathrm{pr}_1(\hat{\mathbf{h}}_1),\ov{1,\mathbf{n}}\bigl)
  ,
$$
откуда с учетом (\ref{4.5.32}) и (\ref{4.5.39}) следует равенство
\begin{eqnarray}
  &\mathbf{c}^\natural\Bigl(\mathrm{pr}_2(z_o^o),\mathrm{pr}_1(z_1^o),
  \la^1\bigl(\{\al^o(s):\,s\in \ov{1,N}\}\cup \ov{N+1,\mathbf{n}}\bigl)\Bigl) =
  &\nonumber\\
  &= \mathbf{c}^\natural\bigl(\mathrm{pr}_2(\hat{\mathbf{h}}_o),\mathrm{pr}_1(\hat{\mathbf{h}}_1),
  \eta^1(\ov{1,\mathbf{n}})\bigl)
  .
  &
  \label{4.5.43`}
\end{eqnarray}

Если же
$t\in \ov{2,N},$ то $t-1\in \ov{1,N-1}$
и
$\mathrm{pr}_2(z_{t-1}^o) = \mathrm{pr}_2(\hat{\mathbf{h}}_{t-1})$.
Кроме того,
$\mathrm{pr}_1(z_t^o) = \mathrm{pr}_1(\hat{\mathbf{h}}_t)$
и
$$
  \mathbf{c}^\natural\Bigl(\mathrm{pr}_2(z_{t-1}^o),\mathrm{pr}_1(z_t^o),
  \la^1\bigl(\{\al^o(s):\,s\in \ov{t,N}\}\cup \ov{N+1,\mathbf{n}}\bigl)\Bigl) =
$$
$$
  = \mathbf{c}^\natural\bigl(\mathrm{pr}_2(\hat{\mathbf{h}}_{t-1}),
  \mathrm{pr}_1(\hat{\mathbf{h}}_t),
  \la^1\bigl(\{\al^o(s):\,s\in\ov{t,N}\}\bigl)\cup \la^1(\ov{N+1,\mathbf{n}})\bigl)=
$$
$$
  = \mathbf{c}^\natural\bigl(\mathrm{pr}_2(\hat{\mathbf{h}}_{t-1}),
  \mathrm{pr}_1(\hat{\mathbf{h}}_t),
  \{\la\bigl(\al^o(s)\bigl):\,s\in\ov{t,N}\}\cup\la^1(\ov{N+1,\mathbf{n}})\bigl)=
$$
$$
  =\mathbf{c}^\natural\bigl(\mathrm{pr}_2(\hat{\mathbf{h}}_{t-1}),
  \mathrm{pr}_1(\hat{\mathbf{h}}_t),
  \{\eta(s):\,s\in\ov{t,N}\}\cup \eta^1(\ov{N+1,\mathbf{n}})\bigl)=
$$
$$
  = \mathbf{c}^\natural\bigl(\mathrm{pr}_2(\hat{\mathbf{h}}_{t-1}),
  \mathrm{pr}_1(\hat{\mathbf{h}}_t),
  \eta^1(\ov{t,N})\cup \eta^1(\ov{N+1,\mathbf{n}})\bigl)=
$$
$$
  =\mathbf{c}^\natural\bigl(\mathrm{pr}_2(\hat{\mathbf{h}}_{t-1}),
  \mathrm{pr}_1(\hat{\mathbf{h}}_t),
  \eta^1(\ov{t,\mathbf{n}})\bigl).
$$

С учетом (\ref{4.5.43`}) получаем, что
\begin{eqnarray}
  &\mathbf{c}^\natural\Bigl(\mathrm{pr}_2(z_{t-1}^o),\mathrm{pr}_1(z_t^o),
  \la^1\bigl(\{\al^o(s):\,s\in \ov{t,N}\}
  \cup \ov{N+1,\mathbf{n}}\bigl)\Bigl)=
  &\nonumber\\
  &= \mathbf{c}^\natural\bigl(\mathrm{pr}_2(\hat{\mathbf{h}}_{t-1}),
  \mathrm{pr}_1(\hat{\mathbf{h}}_t),
  \eta^1(\ov{t,\mathbf{n}})\bigl)\ \ \fa t\in \ov{1,N}
  .
  &
  \label{4.5.44}
\end{eqnarray}

Аналогичным образом устанавливается, что
$$
  c_{\la(\al^o(t))}^\natural\Bigl(z_t^o,\la^1\bigl(\{\al^o(s):\,s\in
  \ov{t,N}\} \cup \ov{N+1,\mathbf{n}}\bigl)\Bigl) =
  c_{\eta(t)}^\natural\bigl(\hat{\mathbf{h}}_t,\eta^1(\ov{t,\mathbf{n}})\bigl) \ \
  \fa t\in\ov{1,N}
  .
$$

С учетом (\ref{4.5.41}), (\ref{4.5.43}) и (\ref{4.5.44})
имеем теперь следующее равенство:
\begin{eqnarray}
  &\mathfrak{B}_{\al^o}[(z_i^o)_{i\in\ov{0,N}}] =
  \sum\limits_{t=1}^N\bigl[\mathbf{c}^\natural\bigl(\mathrm{pr}_2(\hat{\mathbf{h}}_{t-1}),
  \mathrm{pr}_1(\hat{\mathbf{h}}_t),\eta^1(\ov{t,\mathbf{n}})\bigl) +
  &\nonumber\\
  &+c_{\eta(t)}^\natural\bigl(\hat{\mathbf{h}}_t,\eta^1(\ov{t,\mathbf{n}})\bigl)
  \bigl] + \mathbf{c}^\natural\bigl(\mathrm{pr}_2(\hat{\mathbf{h}}_N),
  \mathrm{pr}_1(\hat{\mathbf{h}}_{N+1}),
  \eta^1(\ov{N+1,\mathbf{n}})\bigl)
  .
  &
  \label{4.5.45}
\end{eqnarray}

С другой стороны,
$\bigl(\eta,(\hat{\mathbf{h}}_t)_{t\in\ov{0, \mathbf{n}}}\bigl)$
есть ДР <<большой>> задачи, а потому
$$
  \begin{array}{c}
    \widehat{\mathfrak{C}}_\eta[(\hat{\mathbf{h}}_t)_{t\in\ov{0,\mathbf{n}}}] =
    \sum\limits_{t=1}^\mathbf{n}\bigl[\mathbf{c}^\natural\bigl(\mathrm{pr}_2(\hat{\mathbf{h}}_{t-1}),
    \mathrm{pr}_1(\hat{\mathbf{h}}_t),\eta^1(\ov{t,\mathbf{n}})\bigl) +
    \\
    +c_{\eta(t)}^\natural\bigl(\hat{\mathbf{h}}_t,\eta^1(\ov{t,\mathbf{n}})\bigl)\bigl]+
    f^\natural\bigl(\mathrm{pr}_2(\hat{\mathbf{h}}_\mathbf{n})\bigl)
    .
  \end{array}
$$

Теперь с учетом (\ref{4.5.45}) получаем из последнего выражения, что
\bfn
  \label{4.5.46}
  \widehat{\mathfrak{C}}_\eta[(\hat{\mathbf{h}}_t)_{t\in\ov{0,\mathbf{n}}}] =
  \mathfrak{B}_{\al^o}[(z_i^o)_{i\in\ov{0,N}}] + c_{\eta(N+1)}^\natural\bigl(\hat{\mathbf{h}}_{N+1},
  \eta^1(\ov{N+1,\mathbf{n}})\bigl)  + \xi
  ,
\efn
где $\xi$ -- есть некоторое неотрицательное число,
описываемое разными выражениями в зависимости от того, что
$N+1 <\mathbf{n}$  или $N+1 = \mathbf{n}.$
Первый случай, конечно, более
интересен, но мы рассмотрим все варианты.

Итак, у нас
\bfn
  \label{4.5.47}
  (N+1 <\mathbf{n}) \vee (N+1 = \mathbf{n})
  .
\efn

Рассмотрим отдельно обе возможности в (\ref{4.5.47}).

а) Пусть $N+1 <\mathbf{n}.$
Иными словами, $N+2\leqslant \mathbf{n}$ и, как следствие,
$$
  \ov{N+2,\mathbf{n}}\neq \emp
  .
$$

Тогда значение $\xi$ определяется выражением
$$
  \xi = \sum\limits_{t=N+2}^\mathbf{n}\bigl[\mathbf{c}^\natural\bigl(\mathrm{pr}_2(\hat{\mathbf{h}}_{t-1}),
  \mathrm{pr}_1(\hat{\mathbf{h}}_t),\eta^1(\ov{t,\mathbf{n}})\bigl) +
  c_{\eta(t)}^\natural\bigl(\hat{\mathbf{h}}_t,\eta^1(\ov{t,\mathbf{n}})\bigl)\bigl] +
  f^\natural\bigl(\mathrm{pr}_2(\hat{\mathbf{h}}_\mathbf{n})\bigl)
  .
$$

С учетом (\ref{4.5.42}) и следствия~\ref{c4.5.1} в данном случае имеем, что
$$
  \xi = \sum\limits_{t=N+2}^\mathbf{n}\bigl[\mathbf{c}^\natural\bigl(\mathrm{pr}_2(\mathbf{h}_{t-1}),
  \mathrm{pr}_1(\mathbf{h}_t),\la^1(\ov{t,\mathbf{n}})\bigl) +
  c_{\la(t)}^\natural\bigl(\mathbf{h}_t,\la^1(\ov{t,\mathbf{n}})\bigl)\bigl] +
  f^\natural\bigl(\mathrm{pr}_2(\mathbf{h}_\mathbf{n})\bigl)
  .
$$

Тогда из (\ref{4.5.46})
с учетом (\ref{4.5.42}) и следствия~\ref{c4.5.1}
получаем (при $N+1 < \mathbf{n}$), что
\begin{eqnarray}
  &\widehat{\mathfrak{C}}_\eta[(\hat{\mathbf{h}}_t)_{t\in\ov{0,\mathbf{n}}}] =
  \mathfrak{B}_{\al^o}[(z_i^o)_{i\in\ov{0,N}}] +
  c_{\la(N+1)}^\natural\bigl(\mathbf{h}_{N+1},\la^1(\ov{N+1,\mathbf{n}})\bigl)  +
  &\nonumber\\
  &+ \sum\limits_{t=N+2}^\mathbf{n}\bigl[\mathbf{c}^\natural\bigl(\mathrm{pr}_2(\mathbf{h}_{t-1}),
  \mathrm{pr}_1(\mathbf{h}_t),\la^1(\ov{t,\mathbf{n}})\bigl) +
  &\nonumber\\
  &+ c_{\la(t)}^\natural\bigl(\mathbf{h}_t,\la^1(\ov{t,\mathbf{n}})\bigl)\bigl] +
  f^\natural\bigl(\mathrm{pr}_2(\mathbf{h}_\mathbf{n})\bigl)
  .
  &
  \label{4.5.48}
\end{eqnarray}

Используя (\ref{4.4.70}),
получаем из (\ref{4.5.48}), что
$$
  \widehat{\mathfrak{C}}_\eta[(\hat{\mathbf{h}}_t)_{t\in\ov{0,\mathbf{n}}}] =
  \mathfrak{B}_\mathbf{e}[(\tilde{\mathbf{h}}_i)_{i\in\ov{0,N}}] -\kappa +
  c_{\la(N+1)}^\natural\bigl(\mathbf{h}_{N+1},\la^1(\ov{N+1,\mathbf{n}})\bigl) +
$$
$$
  + \sum\limits_{t=N+2}^\mathbf{n}\bigl[\mathbf{c}^\natural\bigl(\mathrm{pr}_2(\mathbf{h}_{t-1}),
  \mathrm{pr}_1(\mathbf{h}_t),\la^1(\ov{t,\mathbf{n}})\bigl) +
  c_{\la(t)}^\natural\bigl(\mathbf{h}_t,\la^1(\ov{t,\mathbf{n}})\bigl)\bigl] +
  f^\natural\bigl(\mathrm{pr}_2(\mathbf{h}_\mathbf{n})\bigl)
  .
$$

Теперь воспользуемся  представлением (\ref{4.4.72}). Тогда
$$
  \widehat{\mathfrak{C}}_\eta[(\hat{\mathbf{h}}_t)_{t\in\ov{0,\mathbf{n}}}] =
  \sum\limits_{t=1}^N\bigl[\mathbf{c}^\natural\bigl(\mathrm{pr}_2(\mathbf{h}_{t-1}),
  \mathrm{pr}_1(\mathbf{h}_t),\la^1(\ov{t,\mathbf{n}})\bigl) +
  c_{\la(t)}^\natural\bigl(\mathbf{h}_t,\la^1(\ov{t,\mathbf{n}})\bigl)\bigl] +
$$
$$
  + c^\natural\bigl(\mathrm{pr}_2(\mathbf{h}_N),\mathrm{pr}_1\bigl(\mathbf{h}_{N+1}),
  \la^1(\ov{N+1,\mathbf{n}})\bigl) - \kappa +
  c_{\la(N+1)}^\natural\bigl(\mathbf{h}_{N+1},\la^1(\ov{N+1,\nn})\bigl)+ $$
$$
  + \sum\limits_{t=N+2}^\mathbf{n}\bigl[\mathbf{c}^\natural\bigl(\mathrm{pr}_2(\mathbf{h}_{t-1}),
  \mathrm{pr}_1(\mathbf{h}_t),\la^1(\ov{t,\mathbf{n}})\bigl) + c_{\la(t)}^\natural\bigl(\mathbf{h}_t,
  \la^1(\ov{t,\mathbf{n}})\bigl)\bigl] + f^\natural\bigl(\mathrm{pr}_2(\mathbf{h}_\mathbf{n})\bigl) =
$$
$$
  = \sum\limits_{t=1}^\mathbf{n}\bigl[\mathbf{c}^\natural\bigl(\mathrm{pr}_2(\mathbf{h}_{t-1}),
  \mathrm{pr}_1(\mathbf{h}_t),\la^1(\ov{t,\mathbf{n}})\bigl) +
  c_{\la(t)}^\natural\bigl(\mathbf{h}_t,\la^1(\ov{t,\mathbf{n}})\bigl)\bigl] +
  f^\natural\bigl(\mathrm{pr}_2(\mathbf{h}_\mathbf{n})\bigl) -\kappa =
$$
$$
  = \widehat{\mathfrak{C}}_\la[(\mathbf{h}_i)_{i\in\ov{0,\mathbf{n}}}] - \kappa
  .
$$

Итак,
при $\nu =0$ установлена следующая импликация
\bfn
  \label{4.5.49}
  (N+1 < \mathbf{n}) \Longrightarrow \bigl(\widehat{\mathfrak{C}}_\eta[(\hat{\mathbf{h}}_t)_{t\in \ov{0,\mathbf{n}}}] =
  \widehat{\mathfrak{C}}_\la[(\mathbf{h}_i)_{i\in \ov{0,\mathbf{n}}}] - \kappa\bigl)
  .
\efn

б) Пусть теперь $N+1 = \mathbf{n}$ (при  условии (\ref{4.5.35})).
Тогда
(см. (\ref{4.5.42}))
имеем цепочку равенств
\bfn
  \label{4.5.50}
  \xi = f^\natural\bigl(\mathrm{pr}_2(\hat{\mathbf{h}}_\mathbf{n})\bigl) =
  f^\natural\bigl(\mathrm{pr}_2(\mathbf{h}_\mathbf{n})\bigl)
  .
\efn

Из (\ref{4.5.46}) в рассматриваемом случае следует равенство
$$
  \widehat{\mathfrak{C}}_\eta[(\hat{\mathbf{h}}_t)_{t\in\ov{0,\mathbf{n}}}] =
  \mathfrak{B}_{\al^o}[(z_i^o)_{i\in\ov{0,N}}] +
  c_{\eta(\mathbf{n})}^\natural\bigl(\mathbf{h}_\mathbf{n},\la^1(\{\mathbf{n}\})\bigl) +
  f^\natural\bigl(\mathrm{pr}_2(\mathbf{h}_\mathbf{n})\bigl)
  .
$$

Тогда, учитывая определение $\kappa,$ получаем, что (см. (\ref{4.4.72}))
$$
  \widehat{\mathfrak{C}}_\eta[(\hat{\mathbf{h}}_t)_{t\in\ov{0,\mathbf{n}}}] =
  \mathfrak{B}_\mathbf{e}[(\tilde{\mathbf{h}}_i)_{i\in\ov{0,N}}] - \kappa +
  c_{\la(\mathbf{n})}^\natural\bigl(\mathbf{h}_\mathbf{n},\la^1(\{\mathbf{n}\})\bigl) +
  f^\natural\bigl(\mathrm{pr}_2(\mathbf{h}_\mathbf{n})\bigl) =
$$
$$
  = \sum\limits_{t=1}^N\bigl[\mathbf{c}^\natural\bigl(\mathrm{pr}_2(\mathbf{h}_{t-1}),\mathrm{pr}_1
  (\mathbf{h}_t), \la^1(\ov{t,\mathbf{n}})\bigl) + c_{\la(t)}^\natural\bigl(\mathbf{h}_t,
  \la^1(\ov{t,\mathbf{n}})\bigl)\bigl] +
$$
$$
  + \mathbf{c}^\natural\bigl(\mathrm{pr}_2(\mathbf{h}_N),
  \mathrm{pr}_1(\mathbf{h}_{N+1}),
  \la^1(\ov{N+1,\mathbf{n}})\bigl) +
  c_{\la(\mathbf{n})}^\natural\bigl(\mathbf{h}_\mathbf{n},\la^1 (\{\mathbf{n}\})\bigl) +
  f^\natural\bigl(\mathrm{pr}_2(\mathbf{h}_\mathbf{n})\bigl) -\kappa =
$$
$$
  =\sum\limits_{t=1}^{\mathbf{n}-1}\bigl[\mathbf{c}^\natural\bigl(\mathrm{pr}_2(\mathbf{h}_{t-1}),
  \mathrm{pr}_1(\mathbf{h}_t), \la^1(\ov{t,\mathbf{n}})\bigl) +
  c_{\la(t)}^\natural\bigl(\mathbf{h}_t,\la^1(\ov{t,\mathbf{n}})\bigl)\bigl] +
$$
$$
  +\bigl[\mathbf{c}^\natural\bigl(\mathrm{pr}_2(\mathbf{h}_{\mathbf{n}-1}),\mathrm{pr}_1
  (\mathbf{h}_\mathbf{n}),\la^1(\ov{\mathbf{n},\mathbf{n}})\bigl) +
  c_{\la(\mathbf{n})}^\natural\bigl(\mathbf{h}_\mathbf{n},\la^1(\ov{\mathbf{n},\mathbf{n}})\bigl)\bigl] +
  f^\natural\bigl(\mathrm{pr}_2(\mathbf{h}_\mathbf{n})\bigl) - \kappa =
$$
$$
  = \sum\limits_{t=1}^\mathbf{n}\bigl[\mathbf{c}^\natural\bigl(\mathrm{pr}_2(\mathbf{h}_{t-1}),
  \mathrm{pr}_1(\mathbf{h}_t),\la^1(\ov{t,\mathbf{n}})\bigl) +
  c_{\la(t)}^\natural\bigl(\mathbf{h}_t,\la^1(\ov{t,\mathbf{n}})\bigl)\bigl]
  +f^\natural\bigl(\mathrm{pr}_2(\mathbf{h}_\mathbf{n})\bigl) - \kappa
  .
$$

С учетом (\ref{4.4.11})
и при $N+1 = \mathbf{n}$
составляем нужное равенство
$$
  \widehat{\mathfrak{C}}_\eta[(\hat{\mathbf{h}}_t)_{t\in\ov{0,\mathbf{n}}}] =
  \widehat{\mathfrak{C}}_\la[(\mathbf{h}_i)_{i\in\ov{0,\mathbf{n}}}] - \kappa
  .
$$

Итак, имеем (при $\nu = 0)$ следующую импликацию
$$
  (N+1 = \mathbf{n}) \Longrightarrow \bigl(\widehat{\mathfrak{C}}_\eta[(\hat{\mathbf{h}}_t)_{t\in
  \ov{0,\mathbf{n}}}] = \widehat{\mathfrak{C}}_\la[(\mathbf{h}_i)_{i\in\ov{0,\mathbf{n}}}] - \kappa\bigl)
  .
$$

С учетом (\ref{4.5.47}), (\ref{4.5.49}) и (\ref{4.5.50})
получаем, что
при $\nu=0$
во всех возможных  случаях
\bfn
\label{4.5.51}
  \widehat{\mathfrak{C}}_\eta[(\hat{\mathbf{h}}_t)_{t\in\ov{0,\mathbf{n}}}] =
  \widehat{\mathfrak{C}}_\la[(\mathbf{h}_i)_{i\in\ov{0,\mathbf{n}}}] - \kappa
  .
\efn

Итак (см. (\ref{4.5.35}), (\ref{4.5.51})),
окончательно имеем, что
\bfn
  \label{4.5.52}
  (\nu=0) \Longrightarrow \bigl(\widehat{\mathfrak{C}}_\eta[(\hat{\mathbf{h}}_t)_{t\in
  \ov{0,\mathbf{n}}}] = \widehat{\mathfrak{C}}_\la[(\mathbf{h}_i)_{i\in\ov{0,\mathbf{n}}}] - \kappa\bigl)
  .
\efn

Таким образом,
в случае применения <<начальной>> оптимизирующей вставки реализуется улучшение
качества на величину $\kappa$.
