% !TeX root = ..

\section{
  Алгоритм на функциональном уровне 
  (вставка~в~середину)
}
\label{sect:4.6}
\setcounter{equation}{0}

Обсудим случай построения оптимизирующей вставки при
$\nu \neq 0$.
Итак, ниже предполагается, что
\bfn
  \label{4.6.1}
  \nu\in \ov{1,\mathbf{n}-(N+1)}
  .
\efn

Из (\ref{4.6.1}) вытекают очевидные свойства
\bfn
  \label{4.6.2}
  (\ov{1,\nu}\neq \emp)\,\&\,(\ov{\nu+N+1,\mathbf{n}}\neq \emp)
  .
\efn

По построению вставка, рассматриваемая в настоящем разделе, является <<средней>>.
Напомним, что $\eta\in \ca$ имеет следующий вид:
\begin{eqnarray}
  &\bigl(\eta(t) = \la(t)\ \ \fa t\in \ov{1,\nu}\bigl)\,\&\,\bigl(\eta(t) =
  \la(\nu+ \al^o(t-\nu)\bigl)\ \ \fa t\in \ov{\nu+1,\nu+N}\bigl)\,\&
  &\nonumber\\
  &\&\,\bigl(\eta(t)= \la(t)\ \ \fa t\in \ov{\nu+N+1,\mathbf{n}}\bigl)
  .
  &
  \label{4.6.3}
\end{eqnarray}

Из (\ref{4.6.2}) и (\ref{4.6.3}) видно,
что склеенный маршрут $\eta$
имеет три характерных невырожденных фрагмента.

Напомним предложения~\ref{p4.5.1}, \ref{p4.5.2} и следствие~\ref{c4.5.1}.
В частности,
из предложения~\ref{p4.5.2}
по свойствам операции взятия образа следует, что
\begin{eqnarray}
  &\eta^1(\ov{t,\mathbf{n}}) = \la^1\bigl(\{\nu+\al^o(s):\,s\in \ov{t-\nu,N}\}\,\cup\,
  \ov{\nu+N+1,\nn}\bigl) =
  &\nonumber\\
  &= (\La \circ\al^o)^1(\ov{t-\nu,N})\,\cup\, \la^1(\ov{\nu+N+1,\nn})\ \ \fa t\in \ov{\nu+1,\nu+N}
  .
  &
  \label{4.6.3`}
\end{eqnarray}

Кроме того,
необходимо учесть,
что при
$s\in \ov{1,N}$
$$
  \bigl(M_{\al^o(s)} = \mathbf{L}_{(\La\circ \al^o)(s)}\bigl)\,\&\,\bigl(\bbm_{\al^o(s)} =
  \bbl_{(\La\circ \al^o)(s)}\bigl)
  .
$$

Полагаем теперь, что \cite[(5.10), (5.11)]{Cha13`}
\begin{eqnarray}
  &\mathbf{\Gamma}\df \sum\limits_{t=\nu+1}^{\nu+N}
  \bigl[\zc^\natural\bigl(\mathrm{pr}_2(\hat{\mathbf{h}}_{t-1}),\mathrm{pr}_1(\hat{\mathbf{h}}_t),
  \eta^1(\ov{t,\nn})\bigl) + c_{\eta(t)}^\natural\bigl(\hat{\mathbf{h}}_t,\eta^1(\ov{t,\nn})\bigl)\bigl] +
  &\nonumber\\
  &+\zc^\natural\bigl(\mathrm{pr}_2(\hat{\mathbf{h}}_{\nu+N}),\mathrm{pr}_1
  (\hat{\mathbf{h}}_{\nu+N+1}),\eta^1(\ov{\nu+N+1,\nn})\bigl)
  ,
  &
  \label{4.6.4}
\end{eqnarray}
\begin{eqnarray}
  &\Omega\df \sum\limits_{t=\nu+N+1}^\nn \bigl[\zc^\natural\bigl(\mathrm{pr}_2(\hat{\mathbf{h}}_{t-1}),
  \mathrm{pr}_1(\hat{\mathbf{h}}_t),
  \eta^1(\ov{t,\nn})\bigl) + c_{\eta(t)}^\natural\bigl(\hat{\mathbf{h}}_t,\eta^1(\ov{t,\nn})\bigl)\bigl] -
  &\nonumber\\
  &- \zc^\natural\bigl(\mathrm{pr}_2(\hat{\mathbf{h}}_{\nu+N}),\mathrm{pr}_1
  (\hat{\mathbf{h}}_{\nu+N+1}),\eta^1(\ov{\nu+N+1,\nn})\bigl)
  .
  &
  \label{4.6.5`}
\end{eqnarray}

Легко видеть
(см. (\ref{4.4.11})),
что согласно (\ref{4.6.4}), (\ref{4.6.5`}) 
справедливо равенство
$$
  \begin{array}{c}
  \widehat{\mathfrak{C}}_\eta^\natural[(\hat{\mathbf{h}}_t)_{t\in\ov{0,\nn}}]=
  \sum\limits_{t=1}^\nu \bigl[\zc^\natural\bigl(\mathrm{pr}_2(\hat{\mathbf{h}}_{t-1}),
  \mathrm{pr}_1(\hat{\mathbf{h}}_t),
  \eta^1(\ov{t,\nn})\bigl) +
  \\
  +c_{\eta(t)}^\natural\bigl(\hat{\mathbf{h}}_t,\eta^1(\ov{t,\nn})\bigl)\bigl]
  +\mathbf{\Gamma} + \Omega +f^\natural\bigl(\mathrm{pr}_2(\hat{\mathbf{h}}_\nn)\bigl)
  .
\end{array}
$$

Ясно, что в силу (\ref{4.5.32}), (\ref{4.6.3}) и (\ref{4.6.3`})
\begin{eqnarray}
  &\widehat{\mathfrak{C}}_\eta^\natural[(\hat{\mathbf{h}}_t)_{t\in\ov{0,\nn}}]=
  \sum\limits_{t=1}^\nu \bigl[\zc^\natural\bigl(\mathrm{pr}_2(\mathbf{h}_{t-1}),\mathrm{pr}_1(\mathbf{h}_t),
  \la^1(\ov{t,\nn})\bigl) +
  &\nonumber\\
  &+c_{\la(t)}^\natural\bigl(\mathbf{h}_t,\la^1(\ov{t,\nn})\bigl)\bigl] +\mathbf{\Gamma} +
  \Omega + f^\natural\bigl(\mathrm{pr}_2(\mathbf{h}_\nn)\bigl)
  .
  &
  \label{4.6.5``}
\end{eqnarray}

Отметим, что согласно (\ref{4.5.32}) справедливо равенство
$$
  \hat{\mathbf{h}}_\nu = \mathbf{h}_\nu
  .
$$

Как следствие,
$\mathrm{pr}_2(\hat{\mathbf{h}}_\nu) = \mathrm{pr}_2(\mathbf{h}_\nu) = x^o = \mathrm{pr}_2(z_o^o).$
Кроме того,
$$
  \hat{\mathbf{h}}_{\nu+1} = z_1^o
  .
$$

Поэтому имеем следующее очевидное равенство
\begin{eqnarray}
  &\zc^\natural\bigl(\mathrm{pr}_2(\hat{\mathbf{h}}_\nu),\mathrm{pr}_1
  (\hat{\mathbf{h}}_{\nu+1}),\eta^1(\ov{\nu+1,\nn})\bigl)= \zc^\natural\bigl(\mathrm{pr}_2(z_o^o),
  &\nonumber\\
  &\mathrm{pr}_1 (z_1^o),\eta^1(\ov{\nu+1,\nn})\bigl)= \zc^\natural\bigl(x^o,\mathrm{pr}_1(z_1^o),
  \eta^1(\ov{\nu+1,\nn})\bigl)
  .
  &
  \label{4.6.5}
\end{eqnarray}

С другой стороны,
из (\ref{4.5.32}) при
$t\in\ov{\nu+2,\nu+N}$
получаем, что
$t-1 \hm \in \ov{\nu+1,(\nu+N)- 1}$ и
$$
  \zc^\natural\bigl(\mathrm{pr}_2(\hat{\mathbf{h}}_{t-1}),\mathrm{pr}_1(\hat{\mathbf{h}}_t),
  \eta^1(\ov{t,\nn})\bigl) = \zc^\natural\bigl(\mathrm{pr}_2(z_{(t-1)-\nu}^o),\mathrm{pr}_1(z_{t-\nu}^o),
  \eta^1(\ov{t,\nn})\bigl)
  .
$$

Комбинируя последнее свойство с (\ref{4.6.5}), имеем, что
\begin{eqnarray}
  &\zc^\natural\bigl(\mathrm{pr}_2(\hat{\mathbf{h}}_{s-1}),\mathrm{pr}_1
  (\hat{\mathbf{h}}_s), \eta^1(\ov{s,\nn})\bigl)=
  &\nonumber\\
  &=\zc^\natural\bigl(\mathrm{pr}_2(z_{(s-1)-\nu}^o),\mathrm{pr}_1(z_{s-\nu}^o),
  \eta^1(\ov{s,\nn})\bigl)\ \ \fa s\in \ov{\nu+1,\nu+N}
  .
  \label{4.6.6}
\end{eqnarray}

В свою очередь, из (\ref{4.6.3`}) и (\ref{4.6.6}) вытекает, что при
$t\in\ov{\nu+1,\nu+N}$
\begin{eqnarray}
  &\zc^\natural\bigl(\mathrm{pr}_2(\hat{\mathbf{h}}_{t-1}),\mathrm{pr}_1
  (\hat{\mathbf{h}}_t), \eta^1(\ov{t,\nn})\bigl)=
  \zc^\natural\bigl(\mathrm{pr}_2(z_{(t-1)-\nu}^o),\mathrm{pr}_1(z_{t-\nu}^o),
  &\nonumber\\
  &(\La \circ \al^o)^1(\ov{t-\nu,N})\,\cup
  \la^1(\ov{\nu+N+1,\nn})\bigl)=
  \zc^\natural\bigl(\mathrm{pr}_2(z_{(t-1)-\nu}^o),\mathrm{pr}_1(z_{t-\nu}^o),
  &\nonumber\\
  &\La^1\bigl(\{\al^o(j):\,j\in\ov{t-\nu,N}\})\,\cup
  \la^1(\ov{\nu+N+1,\nn})\bigl)=
  &\nonumber\\
  &=\zc\bigl(\mathrm{pr}_2(z_{(t-1)-\nu}^o),\mathrm{pr}_1(z_{t-\nu}^o),
  \{\al^o(j):\,j\in\ov{t-\nu,N}\}\bigl)
  .
  &
  \label{4.6.7}
\end{eqnarray}

Аналогичным образом получаем, что при
$t\in \ov{\nu+1,\nu+N}$
\begin{eqnarray}
  &c_{\eta(t)}^\natural\bigl(\hat{\mathbf{h}}_t,\eta^1(\ov{t,\nn})\bigl) =
  c_{(\La\circ \al^o)(t-\nu)}^\natural\bigl(z_{t-\nu}^o,(\La\circ
  \al^o)^1(\ov{t-\nu,N})\,\cup\,\la^1(\ov{\nu+N+1,\nn})\bigl) =
  &\nonumber\\
  &= c_{\La(\al^o(t-\nu))}^\natural\Bigl(
  z_{t-\nu}^o,\La^1\bigl(\{\al^o(j):\,j\in\ov{t-\nu,N}\}\bigl)
  \cup\,\la^1(\ov{\nu+N+1,\nn})\Bigl) =
  &\nonumber\\
  &= c_{\al^o(t-\nu)}\bigl(z_{t-\nu}^o,\{\al^o(j):\,j\in\ov{t-\nu,N}\}\bigl)
  ,
  &
  \label{4.6.8}
\end{eqnarray}
где
$t -\nu\in\ov{1,N}.$
Рассмотрим теперь первую сумму в (\ref{4.6.4}).
Тогда с учетом   (\ref{4.6.7}), (\ref{4.6.8})
получаем, что
\begin{eqnarray}
  &\sum\limits_{t=\nu+1}^{\nu+N}\bigl[\zc^\natural\bigl(\mathrm{pr}_2
  (\hat{\mathbf{h}}_{t-1}),\mathrm{pr}_1(\hat{\mathbf{h}}_t),\eta^1(\ov{t,\nn})\bigl) +
  c_{\eta(t)}^\natural\bigl(\hat{\mathbf{h}}_t,\eta^1(\ov{t,\nn})\bigl)\bigl] =
  &\nonumber\\
  &=\sum\limits_{t=\nu+1}^{\nu+N}\bigl[\zc\bigl(\mathrm{pr}_2
  (z_{(t-1)-\nu}^o),\mathrm{pr}_1(z_{t-\nu}^o),\{\al^o(j):\,j\in\ov{t-\nu,N}\}\bigl) +
  &\nonumber\\
  &+ c_{\al^o(t-\nu)}\bigl(z_{t-\nu}^o,\{\al^o(j):\,j\in\ov{t-\nu,N}\}\bigl)\bigl] =
  &\nonumber\\
  &\hspace{-0.3cm}=\sum\limits_{s=1}^N\bigl[\zc\bigl(\mathrm{pr}_2
  (z_{s-1}^o),\mathrm{pr}_1(z_s^o),\{\al^o(j):j\in\ov{s,N}\}\bigl) +
  &\nonumber\\
  &+c_{\al^o(s)}\bigl(z_s^o,\{\al^o(j):j\in\ov{s,N}\}\bigl)\bigl]
  .
  &
  \label{4.6.9}
\end{eqnarray}

Заметим, наконец, что с учетом (\ref{4.5.32})
\begin{eqnarray}
  &\zc^\natural\bigl(\mathrm{pr}_2
  (\hat{\mathbf{h}}_{\nu+N}),\mathrm{pr}_1(\hat{\mathbf{h}}_{\nu+N+1}),\eta^1(\ov{\nu+N+1,\nn})\bigl)=
  &\nonumber\\
  &=\zc^\natural\bigl(\mathrm{pr}_2(z_N^o),\mathrm{pr}_1(\mathbf{h}_{\nu+N+1}),\la^1(\ov{\nu+N+1,\nn})
  \bigl) = f\bigl(\mathrm{pr}_2(z_N^o)\bigl)
  .
  &
  \label{4.6.10}
\end{eqnarray}

В итоге из (\ref{4.6.4}), (\ref{4.6.9}) и (\ref{4.6.10})
выводим равенство
$$
  \begin{array}{c}
    \mathbf{\Gamma} = \sum\limits_{s=1}^N\bigl[\zc\bigl(\mathrm{pr}_2(z_{s-1}^o),
    \mathrm{pr}_1(z_s^o),\{\al^o(j):
    \,j\in\ov{s,N}\}\bigl)+
    \\
    +c_{\al^o(s)}\bigl(z_s^o,\{\al^o(j):\,j\in\ov{s,N}\}\bigl)\bigl]
    +f\bigl(\mathrm{pr}_2(z_N^o)\bigl)
    .
  \end{array}
$$

С учетом (\ref{4.2.4}) получаем следующее представление
\bfn
  \label{4.6.11}
  \mathbf{\Gamma}= \mathfrak{B}_{\al^o}[(z_i^o)_{i\in\ov{0,N}}] =\mathbb{V}=
  \mathfrak{B}_\mathbf{e}[(\tilde{\mathbf{h}}_i)_{i\in\ov{0,N}}] - \kappa
  .
\efn

Из (\ref{4.6.5``}) и (\ref{4.6.11}) вытекает, что
\begin{eqnarray}
  &\widehat{\mathfrak{C}}_\eta^\natural[(\hat{\mathbf{h}}_t)_{t\in\ov{0,\nn}}]=
  \sum\limits_{t=1}^\nu\bigl[\zc^\natural\bigl(\mathrm{pr}_2(\mathbf{h}_{t-1}),
  \mathrm{pr}_1(\mathbf{h}_t),
  \la^1(\ov{t,\nn})\bigl) + c_{\la(t)}^\natural\bigl(\mathbf{h}_t,\la^1(\ov{t,\nn})\bigl)\bigl] +
  &\nonumber\\
  &+ \mathfrak{B}_\mathbf{e}
  [(\tilde{\mathbf{h}}_i)_{i\in\ov{0,N}}] + \Om +
  f^\natural\bigl(\mathrm{pr}_2(\mathbf{h}_\nn)\bigl) -\kappa
  .
  &
  \label{4.6.12}
\end{eqnarray}

Учтем теперь (\ref{4.4.72}).
Тогда из (\ref{4.6.12}) следует, что
\begin{eqnarray}
  &\widehat{\mathfrak{C}}_\eta^\natural[(\hat{\mathbf{h}}_t)_{t\in\ov{0,\nn}}]=
  \sum\limits_{t=1}^\nu\bigl[\zc^\natural\bigl(\mathrm{pr}_2(\mathbf{h}_{t-1}),\mathrm{pr}_1(\mathbf{h}_t),
  \la^1(\ov{t,\nn})\bigl) + c_{\la(t)}^\natural\bigl(\mathbf{h}_t,\la^1(\ov{t,\nn})\bigl)\bigl] +
  &\nonumber\\
  &+\sum\limits_{t=\nu+1}^{\nu+N}\bigl[\zc^\natural\bigl(\mathrm{pr}_2(\mathbf{h}_{t-1}),\mathrm{pr}_1
  (\mathbf{h}_t),\la^1(\ov{t,\nn})\bigl)+ c_{\la(t)}^\natural\bigl(\mathbf{h}_t,\la^1(\ov{t,\nn})\bigl)\bigl] +
  &\nonumber\\
  &+ \zc^\natural\bigl(\mathrm{pr}_2(\mathbf{h}_{\nu+N}),\mathrm{pr}_1(\mathbf{h}_{\nu+N+1}),
  \la^1(\ov{\nu+N+1,\nn})\bigl)+ \Om + f^\natural\bigl(\mathrm{pr}_2(\mathbf{h}_\nn)\bigl) -\kappa =
  &\nonumber\\
  &= \sum\limits_{t=1}^{\nu+N}\bigl[\zc^\natural\bigl(\mathrm{pr}_2(\mathbf{h}_{t-1}),\mathrm{pr}_1
  (\mathbf{h}_t),\la^1(\ov{t,\nn})\bigl)+ c_{\la(t)}^\natural\bigl(\mathbf{h}_t,\la^1(\ov{t,\nn})\bigl)\bigl] +
  &\nonumber\\
  &+\zc^\natural\bigl(\mathrm{pr}_2(\mathbf{h}_{\nu+N}),\mathrm{pr}_1(\mathbf{h}_{\nu+N+1}),
  \la^1(\ov{\nu+N+1,\nn})\bigl)+
  &\nonumber\\
  &+\Om + f^\natural\bigl(\mathrm{pr}_2(\mathbf{h}_\nn)\bigl) -\kappa
  .
  &
  \label{4.6.13}
\end{eqnarray}

Рассмотрим теперь два возможных варианта представления $\Om$ (\ref{4.6.5`}),
имея в виду случаи
\bfn
  \label{4.6.14}
  (\nu+N+1 = \nn)\,\vee\,(\nu+N+ 1< \nn)
  .
\efn

По выбору $\nu$ имеем, что один из случаев (\ref{4.6.14})
непременно имеет \mbox{место}.

$1^o.$
Пусть $\nu+N+1 = \nn.$
Тогда из (\ref{4.6.5`}) вытекает, что
\begin{eqnarray}
  &\Om = \zc^\natural\bigl(\mathrm{pr}_2(\hat{\mathbf{h}}_{\nn-1}),\mathrm{pr}_1(\hat{\mathbf{h}}_\nn),
  \eta^1(\{\nn\})\bigl) + c_{\eta(\nn)}^\natural\bigl(\hat{\mathbf{h}}_\nn,\eta^1(\{\nn\})\bigl) -
  &\nonumber\\
  &- \zc^\natural\bigl(\mathrm{pr}_2(\hat{\mathbf{h}}_{\nn-1}),\mathrm{pr}_1(\hat{\mathbf{h}}_\nn),
  \eta^1(\{\nn\})\bigl)=
  c_{\eta(\nn)}^\natural\bigl(\hat{\mathbf{h}}_\nn,\eta^1(\{\nn\})\bigl)
  .
  &
  \label{4.6.15}
\end{eqnarray}

При этом
$\nn\in \ov{0,\nn}\setminus \ov{\nu+1,\nu+N},$
так как $\nu+N < \nn.$
Поэтому, согласно (\ref{4.5.32}),
$\hat{\mathbf{h}}_\nn =\mathbf{h}_\nn$
и в силу (\ref{4.6.3})
$\eta(\nn) = \la(\nn).$
Учитывая (\ref{4.6.15}), имеем, что
(при $\nu+N+1 = \nn$)
$$
  \Om = c_{\la(\nn)}^\natural\bigl(\mathbf{h}_\nn,\la^1(\{\nn\})\bigl)
  ,
$$
где предусмотрено, что
$\eta^1(\{\nn\}) = \{\eta(\nn)\} = \{\la(\nn)\} = \la^1(\{\nn\}).$
Теперь из
(\ref{4.6.13}) имеем в рассматриваемом сейчас случае, что (см.(\ref{4.4.22}))
$$
  \widehat{\mathfrak{C}}_\eta^\natural[(\hat{\mathbf{h}}_t)_{t\in\ov{0,\nn}}]=
  \sum\limits_{t=1}^{\nn-1}\bigl[\zc^\natural\bigl(\mathrm{pr}_2(\mathbf{h}_{t-1}),\mathrm{pr}_1
  (\mathbf{h}_t),\la^1(\ov{t,\nn})\bigl)+ c_{\la(t)}^\natural\bigl(\mathbf{h}_t,\la^1(\ov{t,\nn})\bigl)\bigl] +
$$
$$
  + \zc^\natural\bigl(\mathrm{pr}_2(\mathbf{h}_{\nn-1}),\mathrm{pr}_1
  (\mathbf{h}_\nn),\la^1(\{\nn\})\bigl)+ c_{\la(\nn)}^\natural\bigl(\mathbf{h}_\nn,\la^1(\{\nn\})\bigl) +
  f^\natural\bigl(\mathrm{pr}_2(\mathbf{h}_\nn)\bigl) -\kappa = $$
$$
  = \sum\limits_{t=1}^{\nn-1}\bigl[\zc^\natural\bigl(\mathrm{pr}_2(\mathbf{h}_{t-1}),\mathrm{pr}_1
  (\mathbf{h}_t),\la^1(\ov{t,\nn})\bigl)+ c_{\la(t)}^\natural\bigl(\mathbf{h}_t,\la^1(\ov{t,\nn})\bigl)\bigl] +
$$
$$
  +\bigl[\zc^\natural\bigl(\mathrm{pr}_2(\mathbf{h}_{\nn-1}),\mathrm{pr}_1(\mathbf{h}_\nn),\la^1(\ov{\nn,\nn})
  \bigl)+c_{\la(\nn)}^\natural\bigl(\mathbf{h}_\nn,\la^1(\ov{\nn,\nn})\bigl)\bigl] +
$$
$$
  +f^\natural\bigl(\mathrm{pr}_2(\mathbf{h}_\nn)\bigl) -\kappa =
  \sum\limits_{t=1}^\nn\bigl[\zc^\natural\bigl(\mathrm{pr}_2(\mathbf{h}_{t-1}),\mathrm{pr}_1
  (\mathbf{h}_t),\la^1(\ov{t,\nn})\bigl)+
$$
$$
  + c_{\la(t)}^\natural\bigl(\mathbf{h}_t,\la^1(\ov{t,\nn})\bigl)\bigl] +
  f^\natural\bigl(\mathrm{pr}_2(\mathbf{h}_\nn)\bigl) -\kappa =
  \widehat{\mathfrak{C}}_\lambda^\natural[(\mathbf{h}_i)_{i\in\ov{0,\nn}}]-\kappa
  .
$$

Итак, установлена следующая импликация:
\bfn
  \label{4.6.16}
  (\nu+N+1 =\nn)\Longrightarrow \bigl(\widehat{\mathfrak{C}}_\eta^\natural[(\hat{\mathbf{h}}_t)_{t\in\ov{0,\nn}}]=
  \widehat{\mathfrak{C}}_\la^\natural[(\mathbf{h}_i)_{i\in\ov{0,\nn}}]-\kappa\bigl)
  .
\efn

$2^o.$
Пусть теперь выполнено неравенство
\bfn
  \label{4.6.17}\nu+N+1 < \nn
  .
\efn

Из (\ref{4.6.17}) в частности вытекает, что
$\nu+N+2 \leqslant \nn.$
Возвращаясь к~(\ref{4.6.5`}), отметим, что
\begin{eqnarray}
  &\Om = \bigl[\zc^\natural\bigl(\mathrm{pr}_2(\hat{\mathbf{h}}_{\nu+N}),\mathrm{pr}_1(\hat{\mathbf{h}}_{\nu+N+1}),
  \eta^1(\ov{\nu+N+1,\nn})\bigl) +
  &\nonumber\\
  & + c_{\eta(\nu+N+1)}^\natural\bigl(\hat{\mathbf{h}}_{\nu+N+1},
  \eta^1(\ov{\nu+N+1,\nn})\bigl)\bigl] +
  &\nonumber\\
  & + \sum\limits_{t=\nu+N+2}^\nn\bigl[
  \zc^\natural\bigl(\mathrm{pr}_2(\hat{\mathbf{h}}_{t-1}),\mathrm{pr}_1(\hat{\mathbf{h}}_t),
  \eta^1(\ov{t,\nn})\bigl)
  + c_{\eta(t)}^\natural\bigl(\hat{\mathbf{h}}_t,\eta^1(\ov{t,\nn})\bigl)\bigl] -
  &\nonumber\\
  & -
  \zc^\natural\bigl(\mathrm{pr}_2(\hat{\mathbf{h}}_{\nu+N}),\mathrm{pr}_1(\hat{\mathbf{h}}_{\nu+N+1}),
  \eta^1(\ov{\nu+N+1,\nn})\bigl) =
  &\nonumber\\
  &= \sum\limits_{t=\nu+N+2}^\nn    \zc^\natural\bigl(\mathrm{pr}_2(\hat{\mathbf{h}}_{t-1}),
  \mathrm{pr}_1(\hat{\mathbf{h}}_t),
  \eta^1(\ov{t,\nn})\bigl) + \sum\limits_{t=\nu+N+1}^\nn c_{\eta(t)}^\natural\bigl(\hat{\mathbf{h}}_t,
  \eta^1(\ov{t,\nn})\bigl) =
  &\nonumber\\
  %%
  &=  \sum\limits_{t=\nu+N+2}^\nn    \zc^\natural\bigl(\mathrm{pr}_2(\mathbf{h}_{t-1}),\mathrm{pr}_1(\mathbf{h}_t),
  \la^1(\ov{t,\nn})\bigl) +
  &\nonumber\\
  &+\sum\limits_{t=\nu+N+1}^\nn c_{\la(t)}^\natural\bigl(\mathbf{h}_t, \la^1(\ov{t,\nn})\bigl)
  .
  &
  \label{4.6.18}
\end{eqnarray}

Рассмотрим теперь естественную комбинацию (\ref{4.6.13}) и (\ref{4.6.18}):
\begin{eqnarray}
  &\widehat{\mathfrak{C}}_\eta^\natural[(\hat{\mathbf{h}}_t)_{t\in\ov{0,\nn}}]=
  \sum\limits_{t=1}^{\nu+N}\bigl[\zc^\natural\bigl(\mathrm{pr}_2(\mathbf{h}_{t-1}),\mathrm{pr}_1
  (\mathbf{h}_t), \la^1(\ov{t,\nn})\bigl) + c_{\la(t)}^\natural\bigl(\mathbf{h}_t,\la^1(\ov{t,\nn})\bigl)\bigl] +
  &\nonumber\\
  &\zc^\natural\bigl(\mathrm{pr}_2(\mathbf{h}_{\nu+N}),
  \mathrm{pr}_1(\mathbf{h}_{\nu+N+1}), \la^1(\ov{\nu+N+1,\nn})\bigl)+
  &\nonumber\\
  & +
  \sum\limits_{t=\nu+N+2}^\nn\zc^\natural\bigl(\mathrm{pr}_2(\mathbf{h}_{t-1}),
  \mathrm{pr}_1
  (\mathbf{h}_t), \la^1(\ov{t,\nn})\bigl) +
  &\nonumber\\
  & +
  \sum\limits_{t=\nu+N+1}^\nn
  c_{\la(t)}^\natural\bigl(\mathbf{h}_t,\la^1(\ov{t,\nn})\bigl)
  +f^\natural\bigl(\mathrm{pr}_2(\mathbf{h}_\nn)\bigl) - \kappa =
  &\nonumber\\
  &=\sum\limits_{t=1}^{\nu+N}\bigl[\zc^\natural
  (\mathrm{pr}_2(\mathbf{h}_{t-1}),\mathrm{pr}_1(\mathbf{h}_t), \la^1(\ov{t,\nn})\bigl)
  +c_{\la(t)}^{\natural}\bigl(\mathbf{h}_t,\la^1(\ov{t,\nn})\bigl)\bigl] +
  &\nonumber\\
  &+
  \sum\limits_{t=\nu+N+1}^\nn\zc^\natural\bigl(\mathrm{pr}_2(\mathbf{h}_{t-1}),\mathrm{pr}_1
  (\mathbf{h}_t),\la^1(\ov{t,\nn})\bigl) +
  &\nonumber\\
  &+\sum\limits_{t=\nu+N+1}^\nn c_{\la(t)}^\natural\bigl(\mathbf{h}_t,\la^1(\ov{t,\nn})\bigl)
  +f^\natural\bigl(\mathrm{pr}_2(\mathbf{h}_\nn)\bigl) - \kappa  =
  &\nonumber\\
  &=\sum\limits_{t=1}^\nn \bigl[\zc^\natural\bigl(\mathrm{pr}_2(\mathbf{h}_{t-1}),\mathrm{pr}_1(\mathbf{h}_t),
  \la^1(\ov{t,\nn})\bigl) + c_{\la(t)}^\natural(\mathbf{h}_t,\la^1(\ov{t,\nn})\bigl)\bigl] +
  f^\natural\bigl(\mathrm{pr}_2(\mathbf{h}_\nn)\bigl) - \kappa=
  &\nonumber\\
  &= \widehat{\mathfrak{C}}_\la^\natural[(\mathbf{h}_i)_{i\in\ov{0,\nn}}]
  - \kappa
  .
  &
  \label{4.6.19}
\end{eqnarray}

Итак,
(см. (\ref{4.6.18}), (\ref{4.6.19})),
установлена импликация
\bfn
  \label{4.6.20}
  (\nu+N+1 < \nn)\Longrightarrow \bigl(\widehat{\mathfrak{C}}_\eta^\natural[(\hat{\mathbf{h}}_t)_{t\in\ov{0,\nn}}] =
  \widehat{\mathfrak{C}}_\la^\natural[(\mathbf{h}_i)_{i\in\ov{0,\nn}}]  - \kappa\bigl)
  .
\efn

Из (\ref{4.6.14}), (\ref{4.6.16}) и (\ref{4.6.20})
с учетом того, что
(\ref{4.6.16}) и (\ref{4.6.19}) установлены при условии $\nu \neq 0$,
возникает, в свою очередь, следующая импликация:
\bfn
  \label{4.6.21}
  (\nu\neq 0) \Longrightarrow
  \bigl(\widehat{\mathfrak{C}}_\eta^\natural[(\hat{\mathbf{h}}_t)_{t\in\ov{0,\nn}}] =
  \widehat{\mathfrak{C}}_\la^\natural[(\mathbf{h}_i)_{i\in\ov{0,\nn}}]  - \kappa\bigl)
  .
\efn

Теперь из (\ref{4.5.52}) и (\ref{4.6.21})
получаем весьма важное общее положение.

\begin{theo}
\label{t4.6.1}
Значение $\kappa, \kappa \geqslant 0,$
характеризует изменение критерия, реализуемое оптимизирующей вставкой,
посредством следующего соотношения
$$
  \widehat{\mathfrak{C}}_\eta^\natural[(\hat{\mathbf{h}}_t)_{t\in\ov{0,\nn}}] =
  \widehat{\mathfrak{C}}_\la^\natural[(\mathbf{h}_i)_{i\in\ov{0,\nn}}]  - \kappa \leqslant
  \widehat{\mathfrak{C}}_\la^\natural[(\mathbf{h}_i)_{i\in\ov{0,\nn}}]
  .
$$
\end{theo}

Подчеркнем, что данная теорема справедлива при любом выборе
$\nu \hm \in \ov{0,\nn-(N+1)}.$
В следующем разделе рассмотрим случай $\nu= \nn-N,$
для которого соответствующая вставка
будет конструироваться особым,
хотя и~более простым способом.
