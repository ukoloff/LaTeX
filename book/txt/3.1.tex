% !TeX root = ..

\section{
  Используемые соглашения
  и~обозначения
}
\label{sect:3.1}

В двух последующих главах излагается общий подход к решению задач маршрутизации,
возникающих при построении математических моделей для инженерных задач, рассматриваемых
в первой части монографии.
В основе данного подхода находится широко понимаемый вариант
динамического программирования.
Конструкции на основе ДП широко используются в самых
различных математических дисциплинах, включая теорию оптимального управления и теорию
дифференциальных игр.
Мы ограничиваемся применением ДП для решения задач маршрутизации,
которые составляют важный раздел дискретной оптимизации.
Здесь прежде всего следует отметить
использование аппарата ДП  для решения задачи коммивояжера (ЗК);
В этой связи
см.~\cite{Cha4`,Cha16`}.
Разумеется, учитывая известную труднорешаемость ЗК,
заметим, что
метод ДП является в большей степени теоретическим инструментом,
нежели эффективным
алгоритмом
(упомянем в этой связи метод ветвей и границ \cite{Cha17`},
применение методов решения на основе линейного программирования,
многочисленные эвристические методы и алгоритмы).
Однако определенная <<всеядность>> ДП делает полезным его применение
хотя бы на идейном уровне в задачах типа ЗК,
содержащих различные  осложняющие элементы
(в первую очередь -- ограничения).
Говоря о самой ЗК, можно отметить серьезные трудности,
возникающие в <<неметрических>> вариантах этой задачи,
где наибольшее распространение имеют на сегодняшний день
эвристические методы и алгоритмы.

Необходимо учесть,
что при построении решения задачи по методу ДП
мы вообще не рассматриваем на ключевых этапах маршруты,
то есть перестановки  индексов.
Это очень упрощает локальные экстремальные задачи,
используемые при построении функции Беллмана,
к сожалению самих этих задач возникает
достаточно много, что приводит к трудностям
вычислительной реализации,
особенно в части использования ресурсов памяти.

Тем не менее с теоретической точки зрения широко понимаемое ДП важно для изучения
структуры и свойств оптимальных решений.
В рассматриваемом круге задач имеется, однако,
много осложняющих факторов
(ограничения, достаточно сложные функциональные зависимости
и т. п.),
а потому построение надлежащих вариантов ДП
(включая вывод уравнения Беллмана)
представляет самостоятельный интерес.
Получается, что ограничения и условия прикладного
характера при их отображении на математическую постановку
делают последнюю более интересной
для качественного исследования.
Сам же круг возможных приложений может быть,
на наш взгляд,
существенно расширен,
как это обычно и бывает при построении достаточно мотивированной
математической теории.

В настоящей главе основное внимание уделяется
как раз теоретическим вопросам,
так или иначе связанным с ДП,
включая формализацию задачи,
построение расширения последней,
вывод уравнения Беллмана,
разработку экономичных вариантов реализации процедуры на основе ДП
(вопросам сочетания ДП с эвристическими алгоритмами посвящена следующая глава).
Конечно, в математической постановке
удалось отразить далеко не все ограничения и условия,
реально присутствующие в исходных инженерных задачах;
учет целого ряда таких
ограничений --- дело будущего.
Тем не менее, ряд важных обстоятельств, отмеченных ранее
(в I части)
удалось учесть в рамках соответствующей формализации,
и на этой основе построить,
как представляется, весьма содержательную теорию.

\subsubsection*{Обозначения и определения общего характера}

Применяемые ниже методы решения требуют основательной формализации.
В этой связи рассмотрим в пределах настоящей сводки
нужные общематематические понятия и введем
необходимые обозначения.

Ниже используется стандартная
теоретико-множественная символика
(кванторы, связки и др.);
через $\df$
обозначается равенство по определению;
семейством называем множество,
все элементы которого сами являются множествами.
Итак, семейства --- суть множества,
составленные из множеств.
Через $\emp$ обозначаем пустое множество.

Как обычно,
множество $B$ называем подмножеством (п/м)
множества $A$,
если каждый
элемент $b\in B$ содержится в $A,$
то есть $b\in A$;
при этом $B\su A.$

Если $x$ и $y$ --- объекты,
то через $\{x;y\}$
обозначаем единственное множество, содержащее
(в качестве своих элементов)
$x,\,y$  и не содержащее никаких других
элементов.
Иными словами, $\{x;y\}$
есть неупорядоченная пара объектов $x, y.$
Если
$z$ -- какой-либо объект,
то $\{z\} \df \{z;z\}$
есть одноэлементное множество,
содержащее $z$ в качестве своего элемента;
будем говорить также, что $\{z\}$ ---
синглетон, содержащий $z.$

Для любых двух объектов $\al$ и $\beta$ полагаем,
что
$(\al,\beta) \df \bigl\{\{\al\};\{\al;\beta\}\bigl\}$
(см. \cite[67]{Cha6`}),
получая
упорядоченную пару с первым элементом $\al$ и вторым элементом $\beta.$
Отметим известное свойство упорядоченных пар:
если $x, y, u$ и $v$ --
объекты,
то \cite[67]{Cha6`}
\bfn\label{3.1.1}\bigl((x,y) = (u,v)\bigl) \Longleftrightarrow
\bigl((x = u)\,\&\,(y = v)\bigl).
\efn
С учетом (\ref{3.1.1}) легко проверяется корректность следующего определения:
если $z$ есть упорядоченная пара, то единственным образом  определяются объекты
$\mathrm{pr}_1(z)$ и $\mathrm{pr}_2(z),$
для которых
\bfn\label{3.1.2}z = \bigl(\mathrm{pr}_1(z),\mathrm{pr}_2(z)\bigl),
\efn
где
$\mathrm{pr}_1(z)$ есть первый элемент $z,$
а $\mathrm{pr}_2(z)$ -- второй элемент
$z.$
Заметим, что
упорядоченная пара является семейством,
однако,
при практическом применении данного понятия
указанное обстоятельство не используется,
существенно лишь свойство (\ref{3.1.1})
и следующее из него представление (\ref{3.1.2}).

Всюду в дальнейшем используем соглашение:
если $S$ -- множество,
то через $\cp(S)$
(через $\cp^\prime(S))$
обозначаем семейство всех (всех непустых)
п/м $S;$
через
$\mathrm{Fin}(S)$
обозначаем семейство всех конечных множеств из $\cp^\prime(S).$

Итак, $\cp^\prime(S) = \cp(S) \sm \{\emp\},$
а $\mathrm{Fin}(S)$
есть семейство всех
непустых конечных п/м $S.$
Если само $S$ --- непустое конечное множество, то
$\mathrm{Fin}(S) = \cp^\prime(S).$
В этой связи условимся, что
$\bbn  \df \{1;2;\ldots\}$
и
$\bbn_o \df \{0\} \cup \bbn = \{0;1;2;\ldots\} \in \cp^\prime(\bbr),$
где $\bbr$ ---
вещественная прямая;
если $p\in \bbn_o$ и $q\in \bbn_o,$
то
$$\ov{p,q} \df \{j\in \bbn_o |\,(p\leqslant j)\,\&\,(j\leqslant q)\}
$$
(при $q < p$ имеем равенство
$\ov{p,q} = \emp;$
при $p\in \bbn$ и $q\in \bbn$
непременно
$\ov{p,q}\subset \bbn);$
если $n\in \bbn,$ то
$$
\ov{1,n} = \{j\in \bbn |\,j\leqslant n\}\in \cp^\prime(\bbn).
$$

Для всякого непустого множества $S$
через $\car_+[S]$
обозначаем множество всех функций,
действующих из $S$ в полупрямую
$$
[0,\infty[ \df \{\xi\in\bbr |\,0\leqslant \xi\}\in \cp^\prime(\bbr),
$$
то есть множество всех функций
$\vp:\,S \rightarrow [0,\infty[.$

Среди всевозможных отображений одного множества в другое
для нас особенно важны биекции,
то есть взаимно однозначные отображения
первого множества на второе.
Итак, если $P$ и
$Q$ --- непустые множества,
то через
$(\mathrm{Bi})[P;Q]$
обозначаем множество всех
биекций множества $P$ на множество
$Q$:
$(\mathrm{Bi})[P;Q]$
есть множество всех отображений
\bfn
  \label{3.1.3}
  \vp:\ P{\stackrel{\mbox{\footnotesize{на}}}{\longrightarrow}}\,Q,
\efn
для каждого из которых
$\fa p_1\in P\ \ \fa p_2\in P$

\bfn
  \label{3.1.3`}
  \bigl(\vp(p_1) = \vp(p_2)\bigl) \Longrightarrow (p_1 =  p_2).
\efn

Отметим в этой связи, что (\ref{3.1.3})
включает требование
$Q = \{\vp(p):\,p\in P\}.$
Напомним также, что \cite[67]{Cha7`}
перестановка непустого множества $A$ есть биекция
$A$ на себя,
поэтому
$(\mathrm{Bi})[A;A]$
есть множество всех перестановок $A$.

Для произвольных непустых множеств $P$ и $Q,$
а также биекции $\vp\in (\mathrm{Bi})[P;Q]$
определена биекция
$\vp^{-1}\in (\mathrm{Bi})[Q;P],$
обратная по отношению к $\vp,$ для которой
\bfn
  \label{3.1.4}
  \Bigl(\vp\bigl(\vp^{-1}(q)\bigl) = q\ \ \fa q\in Q\Bigl)\,\&\,
  \Bigl(\vp^{-1}\bigl(\vp(p)\bigl) = p\ \ \fa p\in P\Bigl).
\efn

Заметим, что (\ref{3.1.4})
применимо к перестановкам.

Каждому непустому конечному множеству $K$
сопоставляется его мощность $|K|\in \bbn$
(количество элементов),
причем
\bfn
  \label{3.1.5}
  (\mathrm{bi})[K] \df (\mathrm{Bi})
  [\ov{1,|K|};K]\neq \emp.
\efn

В этой связи отметим,
что
$(\mathrm{bi})[\ov{1,n}] = (\mathrm{Bi})[\ov{1,n};
\ov{1,n}]\ \ \fa n\in \bbn.$
Полагаем в дальнейшем, что
$|\emp| \df 0.$
Теперь
каждому конечному множеству $K$
сопоставлена его мощность $|K| \in \bbn_o.$

Обратим внимание,
что для всякого множества $S$ имеем, что
$|K|\in \bbn\ \ \fa
K\in \mathrm{Fin}(S).$
Если $S$ --- непустое конечное множество, то
$|T|\in
\bbn\ \ \fa T\in \cp^\prime(S).$

Мы используем (\ref{3.1.5})
при определении так называемых частичных маршрутов,
а биекции из
$(\mathrm{bi})[\ov{1,n}]$ (где $n\in \bbn$)
при определении полных маршрутов.
Разумеется,
$(\mathrm{bi})[\ov{1,n}] \neq \emp$,
см. (\ref{3.1.5}).
