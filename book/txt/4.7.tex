% !TeX root = ..

\section{
  Финальная оптимизирующая вставка
}
\label{sect:4.7}
\setcounter{equation}{0}

В настоящем разделе рассматривается построение финальной вставки
и устанавливается оценка ее эффективности.
Речь идет о ситуации, когда в отличие от построений предыдущих
разделов реализуется случай
\bfn
  \label{4.7.1}
  \nu =\nn - N
  .
\efn

При этом мы отказываемся от (\ref{4.4.60}),
но по-прежнему полагаем фиксированным ДР
$$
  \bigl(\la,(\mathbf{h}_i)_{i\in\ov{0,\nn}}\bigl)\in \mathrm{SOL}
$$
<<большой>> задачи.
Заметим, что случай (\ref{4.7.1})
<<укладывается>> в (\ref{4.4.24}),
а потому в соответствии с прежней схемой \ref{sect:4.4}
конструируются склеенные маршруты:
используем определение~\ref{o4.4.1}.
При этом остаются в силе предложения~\ref{p4.4.1} и \ref{p4.4.2}.
Кроме того, при условии (\ref{4.7.1})
справедливы предложения~\ref{p4.4.3} и \ref{p4.4.4}.
В связи с предложениями~\ref{4.4.2} и \ref{p4.4.3}
заметим, что
(в случае (\ref{4.7.1}))
для множества $\mathbf{K}$ (\ref{4.4.26})
выполняется условие (\ref{3.3.23})
и в полной мере сохраняется (\ref{4.4.30}).

При построении финальной вставки мы опираемся на правило (\ref{4.4.58})
выбора базы локальной задачи.
Сохраняем прежними обозначения (\ref{4.4.59}),
а также
$x^o,\mathbf{M}_1,\,\ldots,\mathbf{M}_N,\mathbf{X}$ и
$\mathbb{X}.$

Заметим теперь, что в нашем случае
$\nn=\nu+N.$
В этой связи определения функций
$$
  \mathbf{c},c_1,\,\ldots,c_N,f
$$
изменяются.
Так, в пределах настоящего раздела функция
$$
  \mathbf{c}\in \car_+[\bbx \times \bbx\times \mathfrak{N}]
$$
определяется следующим условием
\bfn
  \label{4.7.2}
  \mathbf{c}(z,K) \df \mathbf{c}^\natural\bigl(z,\La^1(K)\bigl)\ \
  \fa z\in \bbx\times \bbx\ \ \fa K\in \mathfrak{N}
  ,
\efn
где определение $\La$ соответствует \ref{sect:4.4}.
Аналогичным образом, при $j\in\ov{1,N}$
полагаем, что
$$
  c_j \in \car_+[\bbx\times \bbx\times \mathfrak{N}]
$$
определяется
(в настоящем разделе)
условиями
\bfn
  \label{4.7.3}
  c_j(z,K) \df c_{\La(j)}^\natural\bigl(z,\La^1(K)\bigl)\ \ \fa z\in
  \bbx\times \bbx\ \ \fa K\in \mathfrak{N}
  .
\efn

Наконец, мы полагаем здесь, что
$$
  f\in \car_+[\bbx]
$$
есть сужение $f^\natural$ на $\bbx,$
то есть
\bfn
  \label{4.7.4}
  f(x) \df f^\natural(x)\ \ \fa x\in \bbx
  .
\efn

Итак, посредством (\ref{4.7.2}) -- (\ref{4.7.4})
определен новый вариант набора (\ref{4.2.1}).
С учетом этого рассмотрим аддитивный критерий (\ref{4.2.4}):
каждому ДР
$\bigl(\al,(z_i)_{i\in\ov{0,N}}\bigl)\in \widetilde{\mathbf{D}}$
сопоставлено число (\ref{4.2.4}).
В результате мы получаем (новую)
экстремальную задачу (\ref{4.2.5}).
Как и прежде, полагаем, что размерность данной локальной задачи умеренна,
а потому используя, например,  вариант процедуры в заключительной части \ref{sect:4.3},
мы можем определить значение
(локального)
экстремума $\bbv$ и оптимальное ДР
$\bigl(\al^o,(z_i^o)_{i\in\ov{0,N}}\bigl)\in \widetilde{\mathbf{D}}.$

Итак, предположим, что найдены маршрут
\bfn
  \label{4.7.5}
  \al^o\in \mathbf{A}
\efn
и трасса (траектория)
\bfn
  \label{4.7.6}
  (z_i^o)_{i\in\ov{0,N}}\in \mathbf{Z}_{\al^o}
  ,
\efn
для которых справедливо равенство
\bfn
  \label{4.7.7}
  \mathfrak{B}_{\al^o}[(z_i^o)_{i\in\ov{0,N}}]= \bbv
  .
\efn

Теперь, действуя по правилу (\ref{4.5.1}),
мы определяем склеенный маршрут $\eta$
глобальной задачи:
$\eta\in\ca$ -- есть маршрут (\ref{4.5.1}),
соответствующий
(см. определение~\ref{o4.4.1}) представлению (\ref{4.5.2}),
которое в данном случае
(\ref{4.7.1})
сводится к следующему
\bfn
  \label{4.7.8}
  \bigl(\eta(t) = \la(t)\ \ \fa t\in\ov{1,\nn-N}\bigl)\,\&\,\bigl(\eta(t) =
  (\La \circ \al^o)(t-\nn+N)\ \ \fa t\in \ov{\nn-N+1,\nn}\bigl)
  .
\efn

Далее, вводим новый вариант кортежа (\ref{4.5.31`}).
Полагаем, что
$(\hat{\mathbf{h}}_t)_{t\in\ov{0,\nn}}\in \widetilde{\mathfrak{Z}}$
в настоящем разделе имеет
(см. (\ref{4.7.1}))
следующий вид:
\bfn
  \label{4.7.9}
  \bigl(\hat{\mathbf{h}}_t \df \mathbf{h}_t\ \ \fa t\in\ov{0,\nn-N}\bigl)\,
  \&\,\bigl(\hat{\mathbf{h}}_t \df z_{t-\nn+N}^o\ \ \fa t\in\ov{\nn-N+1,\nn}\bigl)
  .
\efn

\begin{pred}
\label{p4.7.1}
Посредством $(\ref{4.7.9})$ определена трасса,
согласованная с маршрутом $\eta:$
\bfn
  \label{4.7.10}
  (\hat{\mathbf{h}}_t)_{t\in \ov{0,\mathbf{n}}}\in \mathfrak{Z}_\eta
  .
\efn
\end{pred}

Д о к а з а т е л ь с т в о.
В силу (\ref{4.4.23})
$0\in\ov{0,\nn-N},$
а потому
(см. (\ref{4.7.9}))
\bfn
  \label{4.7.11}
  \hat{\mathbf{h}}_o = \mathbf{h}_o = (\mathbf{x}_o,\mathbf{x}_o)
  ,
\efn
учитывая
(\ref{4.4.17}).
Выберем произвольно $\tau\in\ov{1,\nn}.$
Тогда
\bfn
  \label{4.7.12}
  (\tau\leqslant \nn-N)\vee (\tau\in \ov{\nn-N+1,\nn})
  .
\efn

Два случая, упомянутых в (\ref{4.7.12}),
рассмотрим отдельно.

1) Пусть
$\tau\leqslant \nn-N.$
Тогда имеем, что
\bfn
  \label{4.7.13}
  \tau\in \ov{1,\nn-N}
  .
\efn

Поэтому
(см. (\ref{4.7.9}))
$\hat{\mathbf{h}}_\tau = \mathbf{h}_\tau.$
С учетом
(\ref{4.4.17})
получаем, что
$\hat{\mathbf{h}}_\tau\in \bbl_{\la(\tau)}.$
Из
(\ref{4.7.8}) и (\ref{4.7.13}) вытекает, что
$$
  \eta(\tau) = \la(\tau)
  ,
$$
а потому
$\hat{\mathbf{h}}_\tau\in\bbl_{\eta(\tau)}.$
Итак (см. (\ref{4.7.13})),
истинна импликация
\bfn
  \label{4.7.14}
  (\tau\leqslant \nn-N) \Longrightarrow (\hat{\mathbf{h}}_\tau\in \bbl_{\eta(\tau)})
  .
\efn

2) Пусть
$\tau\in\ov{\nn-N+1,\nn}.$
Тогда имеем, что
$\tau-\nn+N\in\ov{1,N},$
определена упорядоченная пара
\bfn
  \label{4.7.15}
  z_{\tau-\nn+N}^o\in\bbm_{\al^o(\tau-\nn+N)}
  .
\efn

При этом,
согласно (\ref{4.4.59}),
имеем
(в терминах $\al^o(\tau-\nn+N)\in\ov{1,N}$)
следующее равенство
\bfn
  \label{4.7.16}
  \bbm_{\al^o(\tau-\nn+N)} = \bbl_{\La(\al^o(\tau-\nn+N))}
  ,
\efn
где
$\La\bigl(\al^o(\tau-\nn+N)\bigl) = \la\bigl(\nu + \al^o(\tau-\nn+N)\bigl) =
\la\bigl(\nn-N +\al^o(\tau-\nn+N)\bigl) \in \Gamma.$
С другой стороны, из (\ref{4.7.8})
вытекает, что в рассматриваемом сейчас случае
$$
  \eta(\tau) = (\La\circ \al^o)(\tau-\nn+N) = \La\bigl(\al^o(\tau-\nn+N)\bigl)
  .
$$

Из (\ref{4.7.16}) следует, что
$$
  \bbm_{\al^o(\tau-\nn+N)} = \bbl_{\eta(\tau)}
  ,
$$
а из (\ref{4.7.15})
тогда вытекает, что
$$
  z^o_{\tau-\nn+N} \in \bbl_{\eta(\tau)}
  .
$$

Отметим, что согласно (\ref{4.7.9})
$$
  \hat{\mathbf{h}}_\tau = z^o_{\tau-\nn+N}
  .
$$

Поэтому
$\hat{\mathbf{h}}_\tau \in\bbl_{\eta(\tau)}$ и при $\tau\in \ov{\nn-N+1,\nn}.$
Итак,
\bfn
  \label{4.7.17}
  (\tau\in\ov{\nn-N+1,\nn}) \Longrightarrow (\hat{\mathbf{h}}_\tau\in \bbl_{\eta(\tau)})
  .
\efn

Из (\ref{4.7.12}), (\ref{4.7.14}) и (\ref{4.7.17})
получаем, что
$\hat{\mathbf{h}}_\tau\in \bbl_{\eta(\tau)}$
во всех возможных случаях.
Коль скоро выбор $\tau$ был произвольным, установлено, что
\bfn
  \label{4.7.18}
  \hat{\mathbf{h}}_t\in \bbl_{\eta(t)} \ \ \fa t\in \ov{1,\nn}
  .
\efn

Получили, что кортеж
$(\hat{\mathbf{h}}_t)_{t\in\ov{0,\nn}}$
является
(см. (\ref{4.4.8}), (\ref{4.7.11}), (\ref{4.7.18}))
трассой, согласованной с $\eta,$ то есть
верно (\ref{4.7.10}).
\hfill$\Box$

Итак,
$\eta\in\ca$ и $(\hat{\mathbf{h}}_t)_{t\in\ov{0,\nn}}\in \mathfrak{Z}_\eta.$
Поэтому
(см. (\ref{4.4.9}))
в рассматриваемом сейчас случае (\ref{4.7.1})
$$
  \bigl(\eta,(\hat{\mathbf{h}}_t)_{t\in\ov{0,\nn}}\bigl)\in \mathrm{SOL}
  .
$$

Мы получили склеенное ДР <<большой>> задачи.
Напомним, что (см. (\ref{4.4.11}))
\begin{eqnarray}
  &\widehat{\mathfrak{C}}_\eta[(\hat{\mathbf{h}}_t)_{t\in\ov{0,\nn}}] =
  \sum\limits_{t=1}^\nn\bigl[
  \zc^\natural\bigl(\mathrm{pr}_2(\hat{\mathbf{h}}_{t-1}),\mathrm{pr}_1(\hat{\mathbf{h}}_t),
  \eta^1(\ov{t,\nn})\bigl)+
  &\nonumber\\
  &+c_{\eta(t)}^\natural\bigl(\hat{\mathbf{h}}_t,
  \eta^1(\ov{t,\nn})\bigl)\bigl] +f^\natural\bigl(\mathrm{pr}_2(\hat{\mathbf{h}}_\nn)\bigl)\in [0,\infty[
  .
  \label{4.7.19}
\end{eqnarray}

В связи с данным представлением напомним (\ref{4.4.19}).
Для последующего построения
вставки условимся о том, что (в дальнейшем)
\bfn
  \label{4.7.20}
  N < \nn
  ,
\efn
ограничиваясь рассмотрением практически интересного случая,
когда размер <<окна>> меньше $\nn$
(мы усилили требование к выбору $N$ в сравнении с (\ref{4.4.23}),
но сделали это,
сохраняя в полной мере смысловую сторону конструкций на основе оптимизирующих вставок).
Заметим, что в силу (\ref{4.7.20}) и (\ref{4.7.1})
\bfn
  \label{4.7.21}
  \nu = \nn-N \geqslant 1
  .
\efn

Теперь в отношении (\ref{4.7.19}) отметим, что
\begin{eqnarray}
  &\widehat{\mathfrak{C}}_\eta^\natural[(\hat{\mathbf{h}}_t)_{t\in\ov{0,\nn}}] =
  \sum\limits_{t=1}^{\nn-N}\bigl[
  \zc^\natural\bigl(\mathrm{pr}_2(\hat{\mathbf{h}}_{t-1}),\mathrm{pr}_1(\hat{\mathbf{h}}_t),
  \eta^1(\ov{t,\nn})\bigl)+ c_{\eta(t)}^\natural\bigl(\hat{\mathbf{h}}_t,\eta^1(\ov{t,\nn})\bigl)\bigl] +
  &\nonumber\\
  &+ \sum\limits_{t=\nn-N+1}^\nn\bigl[
  c^\natural\bigl(\mathrm{pr}_2(\hat{\mathbf{h}}_{t-1}),\mathrm{pr}_1(\hat{\mathbf{h}}_t),
  \eta^1(\ov{t,\nn})\bigl)+
  &\nonumber\\
  &+c_{\eta(t)}^\natural\bigl(\hat{\mathbf{h}}_t,\eta^1(\ov{t,\nn})\bigl)\bigl]
  +f^\natural\bigl(\mathrm{pr}_2(\hat{\mathbf{h}}_\nn)\bigl)
  .
  &
  \label{4.7.22}
\end{eqnarray}

Учитываем (\ref{4.7.21}) и то, что согласно (\ref{4.7.20})
$$
  2 \leqslant \nn -N + 1 < \nn
  ,
$$
где используется условие
$N \geqslant 2$, см. (\ref{4.4.23}).
Исходя из
(\ref{4.7.1}) и (\ref{4.7.22}) имеем, что
\begin{eqnarray}
  &\widehat{\mathfrak{C}}_\eta^\natural[(\hat{\mathbf{h}}_t)_{t\in\ov{0,\nn}}] = \sum\limits_{t=1}^\nu\bigl[
  \zc^\natural\bigl(\mathrm{pr}_2(\hat{\mathbf{h}}_{t-1}),\mathrm{pr}_1(\hat{\mathbf{h}}_t),
  \eta^1(\ov{t,\nn})\bigl)+ c_{\eta(t)}^\natural\bigl(\hat{\mathbf{h}}_t,\eta^1(\ov{t,\nn})\bigl)\bigl] +
  &\nonumber\\
  &+\sum\limits_{t=\nu+1}^\nn\bigl[
  \zc^\natural\bigl(\mathrm{pr}_2(\hat{\mathbf{h}}_{t-1}),\mathrm{pr}_1(\hat{\mathbf{h}}_t),
  \eta^1(\ov{t,\nn})\bigl)+
  &\nonumber\\
  &+c_{\eta(t)}^\natural\bigl(\hat{\mathbf{h}}_t,\eta^1(\ov{t,\nn})\bigl)\bigl] +
  f^\natural\bigl(\mathrm{pr}_2(\hat{\mathbf{h}}_\nn)\bigl)
  .
  &
  \label{4.7.23}
\end{eqnarray}

С другой стороны, из (\ref{4.4.19}) вытекает, что
\begin{eqnarray}
  &\widehat{\mathfrak{C}}_\la[(\mathbf{h}_i)_{i\in\ov{0,\nn}}] = \sum\limits_{t=1}^{\nn-N}\bigl[
  \zc^\natural\bigl(\mathrm{pr}_2(\mathbf{h}_{t-1}),\mathrm{pr}_1(\mathbf{h}_t),
  \la^1(\ov{t,\nn})\bigl)+ c_{\la(t)}^\natural\bigl(\mathbf{h}_t,\la^1(\ov{t,\nn})\bigl)\bigl] +
  &\nonumber\\
  &+\sum\limits_{t=\nn-N+1}^\nn\bigl[
  \zc^\natural\bigl(\mathrm{pr}_2(\mathbf{h}_{t-1}),\mathrm{pr}_1(\mathbf{h}_t),
  \la^1(\ov{t,\nn})\bigl)+ c_{\la(t)}^\natural\bigl(\mathbf{h}_t,\la^1(\ov{t,\nn})\bigl)\bigl] +
  f^\natural\bigl(\mathrm{pr}_2(\mathbf{h}_\nn)\bigl)=
  &\nonumber\\
  &=\sum\limits_{t=1}^\nu\bigl[
  \zc^\natural\bigl(\mathrm{pr}_2(\mathbf{h}_{t-1}),\mathrm{pr}_1(\mathbf{h}_t),
  \la^1(\ov{t,\nn})\bigl)+ c_{\la(t)}^\natural\bigl(\mathbf{h}_t,\la^1(\ov{t,\nn})\bigl)\bigl] +
  &\nonumber\\
  &+\sum\limits_{t=\nu+1}^\nn\bigl[
  \zc^\natural\bigl(\mathrm{pr}_2(\mathbf{h}_{t-1}),\mathrm{pr}_1(\mathbf{h}_t),
  \la^1(\ov{t,\nn})\bigl)+
  &\nonumber\\
  &+c_{\la(t)}^\natural\bigl(\mathbf{h}_t,\la^1(\ov{t,\nn})\bigl)\bigl] +
  f^\natural\bigl(\mathrm{pr}_2(\mathbf{h}_\nn)\bigl)
  .
  &
  \label{4.7.24}
\end{eqnarray}

Напомним теперь предложение~\ref{p4.5.1}.
Правда, оно было установлено в предположении о том, что
$\nu \leqslant \nn-(N+1),$
то есть, в частности, при условии
$\nu <\nn - N,$
что не соответствует
(\ref{4.7.1}).
Тем не менее и в нашем случае (\ref{4.7.1})
\bfn
  \label{4.7.25}
  \eta^1(\ov{t,\nn}) = \la^1(\ov{t,\nn})\ \ \fa t\in \ov{1,\nn-N}
  .
\efn

\begin{zam}
\label{z4.7.1}
Проверка $(\ref{4.7.25})$ подобна доказательству предложения~$\ref{p4.5.1}$,
но, учитывая $(\ref{4.7.1})$,
рассмотрим все же соответствующее рассуждение,
фиксируя $t\in\ov{1,\nn-N}.$
Иными словами, $t\in\ov{1,\nu}.$
Поскольку $\la$ и
$\eta$ --- суть перестановки $\ov{1,\nn}$,
получаем, что
$$
  \la^1(\ov{1,\nn}) = \{\la(s):\,s\in\ov{1,\nn}\} = \ov{1,\nn} = \{\eta(s):\,s\in\ov{1,\nn}\} = \eta^1(\ov{1,\nn})
$$
$($имеем аналог $(\ref{4.5.8}))$,
а потому
$(t=1) \Rightarrow \bigl(\la^1(\ov{t,\nn}) = \eta^1(\ov{t,\nn})\bigl).$
Осталось рассмотреть случай $t\in\ov{2,\nn-N}.$
Иными словами, пусть $t\in\ov{2,\nu}$,
тогда в силу $(\ref{4.7.8})$
имеем $($см. $(\ref{4.7.1}))$, что
$$
 \eta^1(\ov{1,t-1}) = \la^1(\ov{1,t-1})
 ,
$$
$$
  \eta^1(\ov{1,\nn}) = \eta^1(\ov{1,t-1} \cup\,\ov{t,\nn}) = \eta^1(\ov{1,t-1}) \cup\,\eta^1(\ov{t,\nn})
  ,
$$
$$
  \la^1(\ov{1,\nn}) = \la^1(\ov{1,t-1} \cup\,\ov{t,\nn}) = \la^1(\ov{1,t-1}) \cup\,\la^1(\ov{t,\nn})
  ,
$$
$$
  \eta^1(\ov{1,t-1}) \cap\,\eta^1(\ov{t,\nn}) = \emp
  ,
$$
$$
  \la^1(\ov{1,t-1}) \cap\,\la^1(\ov{t,\nn}) = \emp
$$
(учитываем инъективность $\eta$ и $\la$).
Тогда
$$
  \eta^1(\ov{t,\nn})= \eta^1(\ov{1,\nn})\setminus \eta^1(\ov{1,t-1}) = \la^1(\ov{1,\nn})\setminus
  \la^1(\ov{1,t-1})= \la^1(\ov{t,\nn})
  ,
$$
то есть
$\la^1(\ov{t,\nn}) = \eta^1(\ov{t,\nn})$
и при $t\in\ov{2,\nn-N}.$
Получили импликацию
$$
  (t\in\ov{2,\nn-N}) \Longrightarrow \bigl(\la^1(\ov{t,\nn}) = \eta^1(\ov{t,\nn})\bigl)
  .
$$

Поскольку $(t=1)\,\vee\,(t\in\ov{2,\nn-N}),$
во всех возможных случаях имеем равенство
$\la^1(\ov{t,\nn}) = \eta^1(\ov{t,\nn})$,
чем и завершается проверка $(\ref{4.7.23})$
в случае, когда
$\nu$ определяется посредством $(\ref{4.7.1})$.
\hfill $\Box$
\end{zam}

Заметим теперь, что согласно
(\ref{4.7.8}), (\ref{4.7.9}) и (\ref{4.7.25})
\begin{eqnarray}
  &\sum\limits_{t=1}^{\nn-N}\bigl[
  \zc^\natural\bigl(\mathrm{pr}_2(\hat{\mathbf{h}}_{t-1}),\mathrm{pr}_1(\hat{\mathbf{h}}_t),
  \eta^1(\ov{t,\nn})\bigl)+ c_{\eta(t)}^\natural\bigl(\hat{\mathbf{h}}_t,\eta^1(\ov{t,\nn})\bigl)\bigl] =
  &\nonumber\\
  &= \sum\limits_{t=1}^{\nn-N}\bigl[
  \zc^\natural\bigl(\mathrm{pr}_2(\mathbf{h}_{t-1}),\mathrm{pr}_1(\mathbf{h}_t),
  \la^1(\ov{t,\nn})\bigl)+ c_{\la(t)}^\natural\bigl(\mathbf{h}_t,\la^1(\ov{t,\nn})\bigl)\bigl]
  .
  &
  \label{4.7.26}
\end{eqnarray}

Напомним, что
(см. предложения~\ref{p4.4.2} и \ref{p4.4.4})
$\mathbf{e}\in \mathbf{A}:$
$$
  (\mathbf{e}-\mathrm{sew})[\la;\nu] = (\mathbf{e}-\mathrm{sew})[\la;\nn-N] = \la
  .
$$

Будем также использовать
(при условии (\ref{4.7.1})) кортеж
$(\tilde{\mathbf{h}}_i)_{i\in\ov{0,N}}\in \bbz$.
В рассматриваемом случае получим
\begin{eqnarray}
  &\Bigl(\tilde{\mathbf{h}}_o = (x^o,x^o) = \bigl(\mathrm{pr}_2(\mathbf{h}_\nu),
  \mathrm{pr}_2(\mathbf{h}_\nu)\bigl) = \bigl(\mathrm{pr}_2(\mathbf{h}_{\nn-N}),
  \mathrm{pr}_2(\mathbf{h}_{\nn-N})\bigl)\Bigl)\,\&
  &\nonumber\\
  &\&\,(\tilde{\mathbf{h}}_t = \mathbf{h}_{\nu+t} = \mathbf{h}_{t+\nn-N}\ \ \fa t\in \ov{1,N})
  .
  &
  \label{4.7.27}
\end{eqnarray}

Из (\ref{4.7.27}) вытекает,
что подобно предложению~\ref{4.4.5}
\bfn
  \label{4.7.28}
  (\tilde{\mathbf{h}}_i)_{i\in\ov{0,N}}\in \mathbf{Z}_\mathbf{e}
  .
\efn

\begin{zam}
\label{z4.7.1-too}
Проверим $(\ref{4.7.28})$,
имея в виду рассматриваемый сейчас случай $(\ref{4.7.1}).$
Пусть $\tau\in \ov{1,N}$,
тогда
$$
  \nu+\tau = \tau+\nn-N\in \ov{\nn-N+1,\nn}
  ,
$$
при котором справедливо равенство
\bfn
  \label{4.7.29}
  \tilde{\mathbf{h}}_\tau = \mathbf{h}_{\nu+\tau} = \mathbf{h}_{\tau+\nn-N}
  .
\efn

При этом
$(\La\circ \mathbf{e})(\tau) = \La(\tau) = \la(\nu+\tau) = \la(\tau+\nn-N).$
Однако из $(\ref{4.7.29})$ вытекает, что $($см. $(\ref{4.4.18}))$
$$
  \tilde{\mathbf{h}}_\tau\in\bbl_{\la(\tau+\nn-N)}
  ,
$$
то есть
$\tilde{\mathbf{h}}_\tau\in\bbl_{\la(\nu+\tau)}$.
В итоге
$\tilde{\mathbf{h}}_\tau\in\bbl_{\La(\tau)}$
и, согласно $(\ref{4.4.59})$,
$$
  \tilde{\mathbf{h}}_\tau\in\bbm_\tau
  ,
$$
то есть
$\tilde{\mathbf{h}}_\tau\in\bbm_{\mathbf{e}(\tau)}.$
Поскольку выбор $\tau$
был произвольным, установлено, что
$$
  \tilde{\mathbf{h}}_t\in\bbm_{\mathbf{e}(t)}\ \ \fa t\in\ov{1,\nn}
  .
$$

С учетом $(\ref{3.3.15})$ и $(\ref{4.7.27})$
получаем требуемое свойство $(\ref{4.7.28}).$
\hfill $\Box$
\end{zam}

Итак,
$\mathbf{e}\in \mathbf{A}$ и
$(\tilde{\mathbf{h}}_i)_{i\in\ov{0,N}}\in \mathbf{Z}_\mathbf{e}.$
Тогда
$\bigl(\mathbf{e},(\tilde{\mathbf{h}}_i)_{i\in\ov{0,N}}\bigl)\in
\widetilde{\mathbf{D}}$
(см. (\ref{3.3.27})). При этом
$\mathfrak{B}_\mathbf{e}[(\tilde{\mathbf{h}}_i)_{i\in\ov{0,N}}]\in [0,\infty[$
и согласно (\ref{4.2.6})
\bfn
  \label{4.7.30}
  \bbv\leqslant \mathfrak{B}_\mathbf{e}[(\tilde{\mathbf{h}}_i)_{i\in\ov{0,N}}]
  .
\efn

С учетом (\ref{4.7.7}) и (\ref{4.4.29})
имеем очевидное неравенство
\bfn
  \label{4.7.31}
  \mathfrak{B}_{\al^o}[(z_i^o)_{i\in\ov{0,N}}] \leqslant
  \mathfrak{B}_\mathbf{e}[(\tilde{\mathbf{h}}_i)_{i\in\ov{0,N}}]
  .
\efn

Используя (\ref{4.7.30}) и (\ref{4.7.31}),
введем $\kappa$ (\ref{4.4.70})
в качестве оценки локального выигрыша за счет замены
$$
  \bigl(\mathbf{e},(\tilde{\mathbf{h}}_i)_{i\in\ov{0,N}}\bigl) \longrightarrow \bigl(\al^o,
  (z_i^o)_{i\in\ov{0,N}}\bigl)
  .
$$

Итак, у нас снова
$\kappa \df \mathfrak{B}_\mathbf{e}[(\tilde{\mathbf{h}}_i)_{i\in\ov{0,N}}]-
\bbv \in [0,\infty[$
и при этом
\bfn
  \label{4.7.32}
  \mathfrak{B}_{\al^o}[(z_i^o)_{i\in\ov{0,N}}] =
  \mathfrak{B}_\mathbf{e}[(\tilde{\mathbf{h}}_i)_{i\in\ov{0,N}}]- \kappa
  .
\efn

Из (\ref{4.7.22}), (\ref{4.2.6})
%(26?)
вытекает, что
\begin{eqnarray}
  &\widehat{\mathfrak{C}}_\eta^\natural[(\hat{\mathbf{h}}_t)_{t\in\ov{0,\nn}}]=
  \sum\limits_{t=1}^{\nn-N}\bigl[\zc^\sharp\bigl(\mathrm{pr}_2(\mathbf{h}_{t-1}),\mathrm{pr}_1
  (\mathbf{h}_t),\la^1(\ov{t,\nn})\bigl) +c_{\la(t)}^\natural\bigl(\mathbf{h}_t,\la^1(\ov{t,\nn})\bigl)\bigl]+
  &\nonumber\\
  &+\sum\limits_{t=\nn-N+1}^\nn\bigl[\zc^\natural\bigl(\mathrm{pr}_2(\hat{\mathbf{h}}_{t-1}),
  \mathrm{pr}_1(\hat{\mathbf{h}}_t),\eta^1(\ov{t,\nn})\bigl) +
  &\nonumber\\
  &+c_{\eta(t)}^\natural\bigl(\hat{\mathbf{h}}_t,\eta^1(\ov{t,\nn})\bigl)\bigl] +
  f^\natural\bigl(\mathrm{pr}_2(\hat{\mathbf{h}}_\nn)\bigl)
  .
  &
  \label{4.7.33}
\end{eqnarray}

Выберем произвольно
$\theta\in\ov{\nn-N+1,\nn}.$
Тогда возможен один из двух случаев:
\bfn
  \label{4.7.38}
  (\theta =\nn-N+1)\,\vee\,(\nn-N+1< \theta)
  .
\efn

Оба случая рассмотрим отдельно.

$1^*)$
Пусть
$\theta =\nn-N+1.$
Тогда имеем согласно (\ref{4.4.58}), что
$$
  \mathrm{pr}_2(\hat{\mathbf{h}}_{\theta-1}) = \mathrm{pr}_2(\hat{\mathbf{h}}_{\nn-N}) =
  \mathrm{pr}_2(\mathbf{h}_{\nn-N})= \mathrm{pr}_2(\mathbf{h}_\nu) =x^o
  .
$$

С другой стороны,
из (\ref{3.3.15}), (\ref{4.7.6}) вытекает, что
$z_o^o = (x^o,x^o),$ а потому
$$
  \mathrm{pr}_2(z_o^o) = x^o
  .
$$

Таким образом, получаем следующее равенство
\bfn
  \label{4.7.39}
  \mathrm{pr}_2(\hat{\mathbf{h}}_{\theta-1}) = \mathrm{pr}_2(z_o^o)
  .
\efn

Далее из (\ref{4.7.9}) вытекает, что
\bfn
  \label{4.7.40}
  \mathrm{pr}_1(\hat{\mathbf{h}}_\theta)= \mathrm{pr}_1(\hat{\mathbf{h}}_{\nn-N+1}) =
  \mathrm{pr}_1(z_1^o)
  .
\efn

Кроме того, имеем следующее очевидное представление,
вытекающее из (\ref{4.7.8}):
$$
  \eta^1(\ov{\theta,\nn}) = \eta^1(\ov{\nn-N+1,\nn}) = \{\eta(j):\,j\in\ov{\nn-N+1,\nn}\} =
  \{(\La\circ\, \al^o)(j-\nn+N):
$$
$$
  \,j\in \ov{\nn-N+1,\nn}\} = \{(\La\circ\, \al^o)(s):\,s\in\ov{1,N}\} = (\La\circ\, \al^o)^1(\ov{1,N}) =
$$
$$
  =\La^1\bigl(\{\al^o(s):\,s\in \ov{1,N}\}\bigl)
  .
$$

Тогда (см. (\ref{4.7.2}), (\ref{4.7.39}), (\ref{4.7.40}))
получаем, что в нашем случае
\bfn
  \label{4.7.41}
  \zc^\natural\bigl(\mathrm{pr}_2(\hat{\mathbf{h}}_{\theta-1}),\mathrm{pr}_1
  (\hat{\mathbf{h}}_\theta),\eta^1(\ov{\theta,\nn})\bigl) = \zc^\natural\Bigl(\mathrm{pr}_2(z_o^o),
  \mathrm{pr}_1(z_1^o),\La^1\bigl(\{\al^o(s):\,s\in\ov{1,N}\}
  \bigl)\Bigl)
  .
\efn

Вместе с тем в рассматриваемом случае имеем
$$
  (\theta-\nn + N = 1)\,\&\,\bigl((\theta-\nn + N) - 1 =0\bigl)
  .
$$

Поэтому из (\ref{4.7.1}) следует, в частности, что
$$
  \zc^\natural\bigl(\mathrm{pr}_2(\hat{\mathbf{h}}_{\theta-1}),\mathrm{pr}_1(\hat{\mathbf{h}}_\theta),
  \eta^1(\ov{\theta,\nn})\bigl) =
$$
$$
  =\zc\bigl(\mathrm{pr}_2(z_{(\theta-\nn+N)-1}^o),
  \mathrm{pr}_1(z_{\theta-\nn+N}^o),\{\al^o(s):\,s\in \ov{\theta-\nn+N,N}\}\bigl)
  .
$$

Итак, истинна импликация
\begin{eqnarray}
  &(\theta = \nn-N+1) \Longrightarrow \Bigl(\zc^\natural\bigl(\mathrm{pr}_2(\hat{\mathbf{h}}_{\theta-1}),\mathrm{pr}_1
  (\hat{\mathbf{h}}_\theta),\eta^1(\ov{\theta,\nn})\bigl) =
  &\nonumber\\
  &= \zc\bigl(\mathrm{pr}_2(z_{(\theta-1) -(\nn-N)}^o),\mathrm{pr}_1(z_{\theta-(\nn-N)}^o),
  &\nonumber\\
  &\{\al^o(s):\,s\in \ov{\theta-(\nn-N),N}\}\bigl)\Bigl)
  .
  &
  \label{4.7.42}
\end{eqnarray}

$2^*)$
Рассмотрим случай
$\nn-N+1 <\theta$
(иными словами, $\nu+1 <\theta).$
Тогда
$\nu+1 \leqslant \theta - 1$
и с учетом (\ref{4.7.9}) получаем, что
\bfn
  \label{4.7.43}
  \hat{\mathbf{h}}_{\theta-1}= z_{(\theta-1)-\nu}^o = z_{(\theta-1) -\nn+N}^o
  ,
\efn
$$
  \hat{\mathbf{h}}_\theta= z_{\theta-\nu}^o = z_{\theta-\nn+N}^o
  .
$$

Из (\ref{4.7.8}) следует также, что
$$
  \eta^1(\ov{\theta,\nn}) = \{\eta(s):\,s\in \ov{\theta,\nn}\}= \{(\La\circ\,\al^o)(s-\nn+N):\,
  s\in \ov{\theta,\nn}\}=
$$
$$
  =\{\La\bigl(\al^o(s-\nn+N)\bigl):\,s\in \ov{\theta,\nn}\}=
  \La^1(\{\al^o(s-\nn+N):\,s\in \ov{\theta,\nn}\})=
$$
$$
  =\La^1\bigl(\{\al^o(l):\,l\in\ov{\theta-\nn+N,N}\}\bigl)
  .
$$

Тогда извлекаем следующую цепочку равенств:
$$
  \zc^\natural\bigl(\mathrm{pr}_2(\hat{\mathbf{h}}_{\theta-1}),\mathrm{pr}_1
  (\hat{\mathbf{h}}_\theta),\eta^1(\ov{\theta,\nn})\bigl) =
  \zc^\natural\bigl(\mathrm{pr}_2(z_{(\theta-1) -\nu}^o),\mathrm{pr}_1(z_{\theta-\nu}^o),
  \eta^1(\ov{\theta,\nn})\bigl)=
$$
$$
  =
  \zc^\natural\bigl(\mathrm{pr}_2(z_{(\theta-1) -\nn+N}^o),\mathrm{pr}_1(z_{\theta-\nn+N}^o),
  \La^1\bigl(\{\al^o(l):\,l\in\ov{\theta-\nn+N,N}\})\bigl) =$$
  $$=\zc\bigl(\mathrm{pr}_2(z_{(\theta-1) -\nn+N}^o),\mathrm{pr}_1(z_{\theta-\nn+N}^o),
  \{\al^o(l):\,l\in\ov{\theta-\nn+N,N}\}\bigl)
  .
$$

Итак, установлена импликация
\begin{eqnarray}
  &(\nn-N+1 < \theta) \Longrightarrow \Bigl(\zc^\natural\bigl(\mathrm{pr}_2(\hat{\mathbf{h}}_{\theta-1}),
  \mathrm{pr}_1
  (\hat{\mathbf{h}}_\theta),\eta^1(\ov{\theta,\nn})\bigl) =
  &\nonumber\\
  &=\zc\bigl(\mathrm{pr}_2(z_{(\theta-1) -\nn+N}^o),\mathrm{pr}_1(z_{\theta-\nn+N}^o),
  \{\al^o(s):
  &\nonumber\\
  &\,s\in\ov{\theta-(\nn-N),N}\}\bigl)\Bigl)
  .
  &
  \label{4.7.44}
\end{eqnarray}

С учетом
(\ref{4.7.38}), (\ref{4.7.42}) и (\ref{4.7.44})
получаем, что во всех возможных случаях
$$
  \zc^\natural\bigl(\mathrm{pr}_2(\hat{\mathbf{h}}_{\theta-1}),\mathrm{pr}_1
  (\hat{\mathbf{h}}_\theta),\eta^1(\ov{\theta,\nn})\bigl) =
$$
$$
  =
  \zc\bigl(\mathrm{pr}_2(z_{(\theta-1) -(\nn-N)}^o),\mathrm{pr}_1(z_{\theta-(\nn-N)}^o),
  \{\al^o(s):\,s\in\ov{\theta-(\nn-N),N}\}\bigl)
  .
$$

Поскольку выбор $\theta$ был произвольным, установлено, что
\begin{eqnarray}
  &\zc^\natural\bigl(\mathrm{pr}_2(\hat{\mathbf{h}}_{t-1}),\mathrm{pr}_1
  (\hat{\mathbf{h}}_t),\eta^1(\ov{t,\nn})\bigl) = \zc\bigl(\mathrm{pr}_2(z_{(t-1) -(\nn-N)}^o),
  \mathrm{pr}_1(z_{t-(\nn-N)}^o),
  &\nonumber\\
  &\{\al^o(s):\,s\in\ov{t-(\nn-N),N}\}\bigl)\ \ \fa t\in \ov{\nn-N+1,\nn}
  .
  &
  \label{4.7.46}
\end{eqnarray}

Пусть теперь
$\xi \in \ov{\nn-N+1,\nn}.$
Рассмотрим представление значения
$$
  c_{\eta(\xi)}^\natural\bigl(\hat{\mathbf{h}}_\xi,\eta^1(\ov{\xi,\nn})\bigl)\in [0,\infty[
    .
$$

Тогда в силу (\ref{4.7.9}) приходим к равенству
$$
  \hat{\mathbf{h}}_\xi = z_{\xi-\nn+N}^o
  .
$$

Кроме того,  из (\ref{4.7.8}) следует, в частности, что
$$
  \eta(\xi) = (\La\circ\,\al^o)(\xi-\nn+N)
  .
$$

Более того, из (\ref{4.7.8}) имеем, что
$$
  \eta^1(\ov{\xi,\nn}) = \{\eta(s):\,s\in \ov{\xi,\nn}\} = \{(\La\circ\,\al^o)(s-\nn+N):\, s\in \ov{\xi,\nn}\} =
$$
$$
  =\La^1\bigl(\{\al^o(s-\nn+N):\,s\in \ov{\xi,\nn}\}\bigl) = \La^1\bigl(\{\al^o(t):\,t\in\ov{\xi-\nn+N,N}\}\bigl)
  .
$$

С учетом трех последних соотношений
\begin{eqnarray}
  &c_{\eta(\xi)}^\natural\bigl(\hat{\mathbf{h}}_\xi,\eta^1(\ov{\xi,\nn})\bigl) =
  &\nonumber\\
  &=c_{(\La\circ\,\al^o)(\xi-\nn+N)}^\natural\Bigl(z_{\xi-\nn+N}^o,\La^1\bigl(\{\al^o(t):
  \,t\in\ov{\xi-\nn+N,N}\}\bigl)\Bigl) =
  &\nonumber\\
  &= c_{\al^o(\xi-\nn+N)} \bigl(z_{\xi-\nn+N}^o,\{\al^o(t):\,t\in \ov{\xi-\nn+N,N}\}\bigl)
  ,
  &
  \label{4.7.47}
\end{eqnarray}
где учтено (\ref{4.7.3}) и то, что
$$
  \xi-\nn+N \in \ov{1,N}
  .
$$

Поскольку выбор $\xi$ был произвольным,
из (\ref{4.7.47}) следует, что
\begin{eqnarray}
  &c_{\eta(j)}^\natural\bigl(\hat{\mathbf{h}}_j,\eta^1(\ov{j,\nn})\bigl) =
  c_{\al^o(j-\nn+N)} \bigl(z_{j-\nn+N}^o,\{\al^o(t):\,t\in \ov{j-\nn+N,N}\}\bigl)\ \
  &\nonumber\\
  &\fa j\in \ov{\nn-N+1,\nn}
  .
  \label{4.7.48}
\end{eqnarray}

Заметим теперь, что согласно (\ref{4.7.46}) и (\ref{4.7.48})
\begin{eqnarray}
  &\sum\limits_{t=\nn-N+1}^\nn\bigl[\zc^\natural\bigl(\mathrm{pr}_2(\hat{\mathbf{h}}_{t-1}),\mathrm{pr}_1
  (\hat{\mathbf{h}}_t),\eta^1(\ov{t,\nn})\bigl) + c_{\eta(t)}^\natural\bigl(\hat{\mathbf{h}}_t,\eta^1(\ov{t,\nn})\bigl)\bigl]=
  &\nonumber\\
  &=\sum\limits_{t=\nn-N+1}^\nn \bigl[\zc\bigl(\mathrm{pr}_2(z_{(t-1) -(\nn-N)}^o),
  \mathrm{pr}_1(z_{t-(\nn-N)}^o),\{\al^o(s):\,s\in \ov{t-(\nn-N),N}\}\bigl) +
  &\nonumber\\
  &+ c_{\al^o(t-\nn+N)}\bigl(z_{t-\nn+N}^o,\{\al^o(s):\,s\in\ov{t-\nn+N,N}\}\bigl)\bigl]=
  &\nonumber\\
  &\hspace{-0.4cm}= \sum\limits_{j=1}^N\bigl[\zc\bigl(\mathrm{pr}_2(z_{j-1}^o),
  \mathrm{pr}_1(z_j^o),\{\al^o(s):\,s\in \ov{j,N}\}\bigl) +
  &\nonumber\\
  &+c_{\al^o(j)}\bigl(z_j^o,\{\al^o(s):\,s\in \ov{j,N}\}\bigl)\bigl]
  .
  &
  \label{4.7.49}
\end{eqnarray}

Наконец, отметим, что согласно (\ref{4.7.9})
$$
  \hat{\mathbf{h}}_\nn = z_N^o\in \mathbb{M}_{\al^o(N)}
  .
$$

При этом, согласно (\ref{4.7.4}),
справедливо следующее равенство
\bfn
  \label{4.7.50}
  f^\natural\bigl(\mathrm{pr}_2(\hat{\mathbf{h}}_\nn)\bigl) =
  f^\natural\bigl(\mathrm{pr}_2(z_N^o)\bigl) = f\bigl(\mathrm{pr}_2(z_N^o)\bigl)
  .
\efn

Следовательно, из
(\ref{4.7.33}), (\ref{4.7.49}) и (\ref{4.7.50})
вытекает, что
(см. (\ref{4.2.4}))
$$
  \widehat{\mathfrak{C}}_\eta^\natural[(\hat{\mathbf{h}}_t)_{t\in\ov{0,\nn}}] =
  \sum\limits_{t=1}^{\nn-N}\bigl[\zc^\natural\bigl(\mathrm{pr}_2(\mathbf{h}_{t-1}),\mathrm{pr}_1
  (\mathbf{h}_t),\la^1(\ov{t,\nn})\bigl) + c_{\la(t)}^\natural\bigl(\mathbf{h}_t,
  \la^1(\ov{t,\nn})\bigl)\bigl]+
$$
$$
  +\sum\limits_{t=1}^N \bigl[\zc\bigl(\mathrm{pr}_2(z_{t-1}^o),
  \mathrm{pr}_1(z_t^o),\{\al^o(s):\,s\in \ov{t,N}\}\bigl) +
$$
$$
  +c_{\al^o(t)}\bigl(z_t^o,\{\al^o(s):\,
  s\in \ov{t,N}\}\bigl)\bigl]+ f\bigl(\mathrm{pr}_2(z_N^o)\bigl) =
$$
$$
  =\sum\limits_{t=1}^{\nn-N}
  \bigl[\zc^\natural\bigl(\mathrm{pr}_2(\mathbf{h}_{t-1}),\mathrm{pr}_1
  (\mathbf{h}_t),\la^1(\ov{t,\nn})\bigl)+ c_{\la(t)}^\natural\bigl(\mathbf{h}_t,\la^1(\ov{t,\nn})
  \bigl)\bigl] + \mathfrak{B}_{\al^o}[(z_i^o)_{i\in\ov{0,N}}]
  .
$$

С учетом (\ref{4.7.32}) получаем теперь, что
\begin{eqnarray}
  &\widehat{\mathfrak{C}}_\eta^\natural[(\hat{\mathbf{h}}_t)_{t\in\ov{0,\nn}}] =
  \sum\limits_{t=1}^{\nn-N}\bigl[\zc^\natural\bigl(\mathrm{pr}_2(\mathbf{h}_{t-1}),\mathrm{pr}_1
  (\mathbf{h}_t),\la^1(\ov{t,\nn})\bigl) +
  &\nonumber\\
  &+ c_{\la(t)}^\natural\bigl(\mathbf{h}_t,\la^1(\ov{t,\nn})\bigl)\bigl]+
  \mathfrak{B}_\mathbf{e}[(\tilde{\mathbf{h}}_i)_{i\in\ov{0,N}}]-\kappa
  .
  &
  \label{4.7.51}
\end{eqnarray}

Напомним в этой связи,
учитывая равенство $x^o= \mathrm{pr}_2(\mathbf{h}_{\nn-N}),$
вытекающее из (\ref{4.4.58}) и (\ref{4.7.1}),
что
\begin{eqnarray}
  &\mathfrak{B}_\mathbf{e}[(\tilde{\mathbf{h}}_i)_{i\in\ov{0,N}}] =
  \sum\limits_{t=1}^N\bigl[\zc\bigl(\mathrm{pr}_2(\tilde{\mathbf{h}}_{t-1}),\mathrm{pr}_1
  (\tilde{\mathbf{h}}_t),\ov{t,N}) + c_t(\tilde{\mathbf{h}}_t,\ov{t,N})\bigl]+
  &\nonumber\\
  &+f\bigl(\mathrm{pr}_2(\tilde{\mathbf{h}}_N)\bigl) =
  \zc\bigl(\mathrm{pr}_2(\tilde{\mathbf{h}}_o),\mathrm{pr}_1
  (\tilde{\mathbf{h}}_1),\ov{1,N}) +c_1(\tilde{\mathbf{h}}_1,\ov{1,N}) +
  &\nonumber\\
  &+ \sum\limits_{t=2}^N\bigl[\zc\bigl(\mathrm{pr}_2(\tilde{\mathbf{h}}_{t-1}),\mathrm{pr}_1
  (\tilde{\mathbf{h}}_t),\ov{t,N}\bigl) +
  c_t(\tilde{\mathbf{h}}_t,\ov{t,N})\bigl] + f\bigl(\mathrm{pr}_2(\tilde{\mathbf{h}}_N)\bigl)=
  &\nonumber\\
  &=\zc\bigl(\mathrm{pr}_2(\mathbf{h}_{\nn-N}),\mathrm{pr}_1(\mathbf{h}_{1+\nn-N}),\ov{1,N}\bigl) +
  c_1(\mathbf{h}_{1+\nn-N},\ov{1,N}) +
  &\nonumber\\
  &+\sum\limits_{t=2}^N\bigl[\zc\bigl(\mathrm{pr}_2(\mathbf{h}_{(t-1)+\nn-N}),\mathrm{pr}_1
  (\mathbf{h}_{t+\nn-N}),\ov{t,N})\bigl) + c_t(\mathbf{h}_{t+\nn-N},\ov{t,N})\bigl]+
  &\nonumber\\
  &+f\bigl(\mathrm{pr}_2(\mathbf{h}_\nn)\bigl)=
  \sum\limits_{t=1}^N\bigl[\zc\bigl(\mathrm{pr}_2(\mathbf{h}_{(t-1)+(\nn-N)}),\mathrm{pr}_1
  (\mathbf{h}_{t+\nn-N}),\ov{t,N})\bigl) +
  &\nonumber\\
  &+c_t(\mathbf{h}_{t+\nn-N},\ov{t,N})\bigl]+
  f\bigl(\mathrm{pr}_2(\mathbf{h}_\nn)\bigl)
  .
  &
  \label{4.7.52}
\end{eqnarray}

Теперь учтем (\ref{4.7.2}) -- (\ref{4.7.4}).
Тогда из (\ref{4.7.52}) вытекает, что
$$
  \mathfrak{B}_\mathbf{e}[(\tilde{\mathbf{h}}_i)_{i\in\ov{0,N}}] =
  \sum\limits_{t=1}^N\bigl[\zc^\natural\bigl(\mathrm{pr}_2(\mathbf{h}_{(t-1)+(\nn-N)}),\mathrm{pr}_1
  (\mathbf{h}_{t+\nn-N}),\La^1(\ov{t,N})\bigl) +$$ $$+ c_{\La(t)}^{\natural}\bigl(\mathbf{h}_{t+\nn-N},
  \La^1(\ov{t,N})\bigl)\bigl]+ f^\natural\bigl(\mathrm{pr}_2(\mathbf{h}_\nn)\bigl)
  .
$$

С учетом определения $\La$
(см. \ref{sect:4.4}) и
(\ref{4.7.1}) получаем, что
\begin{eqnarray}
  &\mathfrak{B}_\mathbf{e}[(\tilde{\mathbf{h}}_i)_{i\in\ov{0,N}}] =
  \sum\limits_{t=1}^N\bigl[\zc^\natural\bigl(\mathrm{pr}_2(\mathbf{h}_{(t-1)+\nn-N}),\mathrm{pr}_1
  (\mathbf{h}_{t+\nn-N}),
  &\nonumber\\
  &\{\la(s+\nn-N):\,s\in\ov{t,N}\}\bigl) + c_{\la(t+\nn-N)}\bigl(\mathbf{h}_{t+\nn-N},\{\la(s+\nn-N):
  &\nonumber\\
  &\,s\in\ov{t,N}\}\bigl)\bigl]+
  f^\natural\bigl(\mathrm{pr}_2(\mathbf{h}_\nn)\bigl)=
  &\nonumber\\
  &= \sum\limits_{t=1}^N\bigl[\zc^\natural\bigl(\mathrm{pr}_2(\mathbf{h}_{(t+\nn-N)-1}),\mathrm{pr}_1
  (\mathbf{h}_{t+\nn-N}),\{\la(l):\,l\in\ov{t+\nn-N,\nn}\}\bigl)+
  &\nonumber\\
  &+c_{\la(t+\nn-N)}\bigl(\mathbf{h}_{t+\nn-N},\{\la(l):\,l\in \ov{t+\nn-N,\nn}\}\bigl)\bigl]+
  f^\natural\bigl(\mathrm{pr}_2(\mathbf{h}_\nn)\bigl)=
  &\nonumber\\
  &= \sum\limits_{t=1}^N\bigl[\zc^\natural\bigl(\mathrm{pr}_2(\mathbf{h}_{(t+\nn-N)-1}),\mathrm{pr}_1
  (\mathbf{h}_{t+\nn-N}),\la^1(\ov{t+\nn-N,\nn})\bigl)+
  &\nonumber\\
  &+c_{\la(t+\nn-N)}\bigl(\mathbf{h}_{t+\nn-N},\la^1(\ov{t+\nn-N,\nn})\bigl)\bigl]+
  f^\natural\bigl(\mathrm{pr}_2(\mathbf{h}_\nn)\bigl)=
  &\nonumber\\
  &= \sum\limits_{j=\nn-N+1}^\nn\bigl[\zc^\natural\bigl(\mathrm{pr}_2(\mathbf{h}_{j-1}),\mathrm{pr}_1
  \mathbf{h}_j),\la^1(\ov{j,\nn})\bigl) +
  &\nonumber\\
  &+c_{\la(j)}\bigl(\mathbf{h}_j,\la^1(\ov{j,\nn})\bigl)\bigl]+
  f^\natural\bigl(\mathrm{pr}_2(\mathbf{h}_\nn)\bigl)
  .
  &
  \label{4.7.53}
\end{eqnarray}

Теперь из (\ref{4.7.51}) и (\ref{4.7.53}) имеемм следующую цепочку равенств
\begin{eqnarray}
  &\widehat{\mathfrak{C}}_\eta[(\hat{\mathbf{h}}_t)_{t\in\ov{0,\nn}}] =
  \sum\limits_{t=1}^\nn\bigl[\zc^\natural\bigl(\mathrm{pr}_2(\mathbf{h}_{t-1}),\mathrm{pr}_1
  (\mathbf{h}_t),\la^1(\ov{t,\nn})\bigl) +
  &\nonumber\\
  &+c_{\la(t)}^\natural\bigl(\mathbf{h}_t,
  \la^1(\ov{t,\nn})\bigl)\bigl]+ f^\natural\bigl(\mathrm{pr}_2(\mathbf{h}_\nn)\bigl)- \kappa =
  &\nonumber\\
  &=\widehat{\mathfrak{C}}_\la[(\mathbf{h}_i)_{i\in\ov{0,\nn}}]-\kappa
  &
  \label{4.7.54}
\end{eqnarray}
(в (\ref{4.7.54}) учтено (\ref{4.4.19})).
Итак,
для финальной вставки, определяемой <<началом>>
(\ref{4.7.1}),
получен аналог теоремы~\ref{t4.6.1}.
