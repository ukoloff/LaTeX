% !TeX root = ..

\thispagestyle{empty}
{
\setlength{\parindent}{0pt}
\setlength{\parskip}{1em}
\footnotesize

\noindent
УДК  621.9:519.6(035)
\\
ББК  34.638в6+32.965в6

\vspace{2em}

{\bf
А. А. Петунин,
А. Г. Ченцов,
П. А. Ченцов.
Оптимальная маршрутизация инструмента машин фигурной листовой резки
с числовым программным управлением.
Математические модели и алгоритмы.
}
--
Екатеринбург:
Изд-во УрФУ,
2020~г.
--
241 с.

\vspace{2em}

\begin{center}

Научный редактор:
заведующий кафедрой прикладной математики и механики УрФУ
\\
проф., д.т.н. А. Н. Сесекин

Рецензент:
заведующий отделом
Института математики и механики им.~Н.~Н.~Красовского УрО РАН
\\
проф. РАН, д.ф.-м.н. М. Ю. Хачай

Рецензент:
заведующий лабораторией Института Машиноведения УрО РАН
\\
проф., д.т.н. А. В. Коновалов
\end{center}

\vspace{1em}

В монографии\footnote{\it
  Результаты исследований получены при выполнении
  проекта создания и развития научной лаборатории
  <<Лаборатория оптимального раскроя промышленных материалов
  и оптимальных маршрутных технологий>>
  в рамках Программы повышения конкурентоспособности
  Уральского федерального университета
  5-100-2020
  и при поддержке Российского Фонда Фундаментальных Исследований
  (гранты №17-08-01385, №20-08-00873)
}
описаны постановки и методы исследования оптимизационных задач
маршрутизации инструмента для машин листовой резки
с числовым программным управлением.
Эти задачи возникают при проектировании технологических процессов
раскроя листового материала.
Особое внимание в работе уделено разработанным авторами
новым математическим моделям и вычислительным алгоритмам маршрутной оптимизации.
В основе теоретических конструкций находятся идеи
широко понимаемого динамического программирования.

Монография может быть полезна ученым, преподавателям и работникам промышленности,
специализирующимся в области прикладной математики,
исследования операций и систем автоматизации проектирования,
а также аспирантам, магистрам и студентам старших курсов вузов,
обучающимся по соответствующим направлениям подготовки.

\vspace{3em}
{
\setlength{\parindent}{0.5\linewidth}
\setlength{\parskip}{0em}
\scriptsize

\copyright
А.А. Петунин, А.Г. Ченцов, П.А. Ченцов, 2020

\copyright
ИПЦ УрФУ, 2020
}}
