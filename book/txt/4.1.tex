% !TeX root = ../mat_mod2.tex

\section{
  Введение
}
\label{sect:4.1}
\setcounter{equation}{0}

В предыдущих главах в достаточной степени были прояснены особенности задач
маршрутизации, связанных с ЧПУ; в какой-то мере обсуждались и некоторые
обстоятельства, касающиеся применения математических методов к решению других
прикладных задач (в частности, это касается задачи о демонтаже энергоблока
АЭС, выведенного из эксплуатации). Упомянутые особенности связаны, конечно,
с определенной спецификой соответствующих прикладных задач; они вполне
заслуживают специального теоретического исследования, что в достаточной
мере отражено в конструкциях предыдущей главы. Есть, однако, также
известные трудности вычислительного характера, которые, по-видимому,
являются общими для всех задач  маршрутизации, включая хорошо известную
ЗК. Данные переборные задачи являются труднорешаемыми в традиционном
понимании (упомянутая ЗК является одной из классических NP-полных задач).
Эта сторона дела конечно же в полной мере проявляется и в вышеупомянутых
прикладных задачах при соответствующей  их формализации. По этой причине
(и это вполне естественно) основные усилия специалистов направлены здесь
на построение эффективных приближенных алгоритмов. Конструкции на основе
ДП (даже с учетом применения экономии вычислений за счет <<неполного>>
построения массива значений функции Беллмана) не могут быть применены для
непосредственного решения маршрутных задач большой размерности, хотя и
доставляют полезные представления, связанные с глобальной оптимальностью.
Эти представления, в частности, могут помочь в некоторых вопросах,
связанных с локальным улучшением эвристических решений. Данные улучшения
имеет смысл реализовывать в итерационном режиме, то есть неоднократно,
что в ряде случаев приводит, как показывает вычислительный эксперимент,
уже к ощутимым результатам.

Для определенности будем рассматривать сейчас проблему, связанную с
листовой резкой на машинах с ЧПУ (см. раздел~3.3). При использовании
математической модели на основе построения мегаполисов приходится учитывать,
что последние связываются не столько с самими деталями, сколько с контурами,
подлежащими резке. Этих контуров у одной детали может быть несколько.
Общее же количество таких контуров может быть достаточно большим
(см. (\ref{3.3.6})). Таким образом, приходится обращаться к эвристическим
алгоритам; их изучение важно прежде всего с практической точки зрения.
Вместе с тем учет ограничений в классе таких алгоритмов представляется
затруднительным; в особенности это касается требований к жесткости листа
и деталей, выбора направления резки и т.п. На этапе применения ДП некоторые
из этих ограничений удается учитывать (см. задачу (\ref{3.3.31})).  Уже это
соображение делает полезным осуществление <<вмешательств>>, использующих ДП
с целью коррекции эвристических решений. Для последних могут быть при этом
выделены (после их построения) <<неудачные>> фрагменты; на замену этих
фрагментов можно направить соответствующие  усилия, задействуя, в частности,
аппарат ДП. Именно такая стратегия будет использоваться ниже. Отметим,
однако, что в рассматриваемом здесь весьма общем случае задачи с ограничениями
потребуется специальное теоретическое исследование ряда вопросов, касающихся
самой организации вставок.

Один из таких вопросов  связан с условиями предшествования: оптимизирующая
вставка должна иметь (локальные) условия предшествования, согласованные
определенным образом с глобальными (данный вопрос подробно рассматривался
в \cite{Cha14`}). Иными словами, возникает проблема локализации глобальных
условий предшествования.

Второй весьма важный вопрос касается проблемы согласования глобальных и
локальных зависимостей от списка заданий. Данные зависимости естественным
образом возникают в задаче о демонтаже оборудования энергоблока АЭС,
выведенного из эксплуатации (см. \cite{Cha2`}): исполнитель подвергается
радиационному воздействию источников излучения, которые еще не демонтированы
на текущий момент. В задаче, связанной с планированием процедуры листовой
резки на машинах с ЧПУ, таких зависимостей в непосредственном виде нет;
однако, их оказывается полезным ввести с целью учета технологических
ограничений, связанными с обеспечением жесткости листа. Опуская сейчас
обсуждение данного приема, подобного введению штрафов в задачах оптимизации,
отметим, что данная особенность также осложняет изучение конструкций,
связанных со вставками. Следуем здесь процедуре, предложенной и обоснованной
в \cite{Cha13`}. Достаточно подробное ее рассмотрение здесь целесообразно
еще и потому, что упомянутая процедура будет встраиваться в итерационные
алгоритмы с целью реализации более ощутимого влияния на результат,
достигаемый посредством эвристических методов.
