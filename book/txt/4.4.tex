% !TeX root = ..


\section{
  Локальное улучшение допустимых решений
}
\label{sect:4.4}
\setcounter{equation}{0}

В настоящем разделе рассматривается конструкция, связанная с применением вставок,
организованных на основе ДП
(в принципе можно было бы задействовать при построении
вставок тот или иной вариант метода ветвей и границ;
возникают, однако, затруднения,
связанные с учетом ограничений и использованием функций стоимости, включающих
зависимость от списка заданий).
Упомянутая конструкция соответствует \cite{Cha13`},
где рассматривается общий случай, соответствующий постановке задачи (\ref{4.2.5});
в этой связи отметим также работы
\cite{Cha14`,Cha15`}.

Итак, рассмотрим случай,
когда размерность исходной маршрутной задачи (\ref{4.2.5})
оказывается достаточно большой,
что не позволяет организовать вычисления на основе ДП.
В то же время, располагая эффективными эвристическими алгоритмами,
мы можем задействовать схему на основе ДП для целей локального улучшения ДР;
данное улучшение достигается посредством  вставок.
Следует иметь в виду условия предшествования и то,
что функции стоимости,
см. (\ref{4.2.1}),
определяющие критерий задачи (\ref{4.2.5}),
допускают зависимость от списка заданий,
не выполненных на момент перемещения.
Последние два обстоятельства требуют учета при соответствующем построении вставок.
В полной мере это было сделано в \cite{Cha13`, 15`}.
В настоящем разделе мы рассмотрим
вопрос об <<однократном>> построении оптимизирующей (беллмановской) вставки,
опуская соответствующие доказательства,
которые приведены в \cite{Cha13`, 15`}.
При этом нам потребуются новые обозначения.
При построении вставки будем применять
символику, используемую~в~\ref{sect:4.2}~и~\ref{sect:4.3}.

Рассмотрим в общих построениях непустое множество $X$
произвольной природы.
В конкретных построениях, связанных с задачей
управления инструментом при листовой резке на станках с ЧПУ,
в качестве $X$
естественно использовать прямоугольник на плоскости.

Фиксируем в качестве базы процесса точку
$\mathbf{x}_o\in X$ и число
$\mathbf{n}\in \bbn, \mathbf{n}\geqslant 2,$
определяющее количество мегаполисов.
По самому смыслу
предлагаемой далее конструкции логично полагать
$\mathbf{n}$ достаточно
большим;
во всяком случае естественно ориентироваться на такой случай.
Фиксируем $\mathbf{n}$ непустых конечных множеств
$\mathbf{L}_1,\,\cdots,\mathbf{L}_\mathbf{n}:$
\bfn
  \label{4.4.1}
  \mathbf{L}_1\in \mathrm{Fin}(X),\,\cdots,\mathbf{L}_\mathbf{n}\in \mathrm{Fin}(X)
  ,
\efn
именуемые далее мегаполисами,
а точнее --
мегаполисами <<большой>> задачи.
Кроме того, фиксируем отношения
\bfn
  \label{4.4.2}
  \bbl_1\in \cp^\prime(\mathbf{L}_1 \times \mathbf{L}_1).\cdots,
  \bbl_\mathbf{n}\in \cp^\prime(\mathbf{L}_\mathbf{n}\times \mathbf{L}_\mathbf{n})
  .
\efn

Если $j\in \ov{1,\mathbf{n}},$
то отношение $\bbl_j,$
$$
  \bbl_j\subset \mathbf{L}_j \times \mathbf{L}_j
$$
характеризует возможности исполнителя в части выполнения работ,
связанных с посещением $\mathbf{L}_j$.
В случае листовой резки $\bbl_j$
является множеством всех упорядоченных пар,
у каждой из которых первый элемент является точкой врезки,
а второй --- точкой выключения инструмента.
По аналогии с (\ref{3.3.4}) полагаем, что
\bfn
  \label{4.4.3}
  (\mathbf{x}_o\notin \mathbf{L}_j\ \fa j\in \ov{1,\mathbf{n}})\,\&\,(\mathbf{L}_p
  \cap \mathbf{L}_q = \emp\ \ \fa p\in \ov{1,\mathbf{n}}\ \ \fa q\in \ov{1,\mathbf{n}}\sm \{p\})
  .
\efn

Далее занимаемся организацией перемещений следующего вида:
\bfn
  \label{4.4.4}
  \mathbf{x}_o\longrightarrow (z_1\in \bbl_{\beta(1)}) \longrightarrow\ldots
  \longrightarrow (z_\mathbf{n}\in \bbl_{\beta(\mathbf{n})})
  ,
\efn
где
$\beta\in \mathbf{P},$
здесь и далее
$\mathbf{P}\df (\mathrm{bi})[\ov{1,\mathbf{n}}].$
Выбор  $\beta,z_1,\,\ldots,z_\mathbf{n}$
находится в нашем распоряжении.
Правда, на выбор $\beta$ могут накладываться
условия предшествования, подобные рассматриваемым в предыдущей главе.
Итак,
$\mathbf{P}$ --- множество всех полных маршрутов исходной <<большой>> задачи.

Введем в рассмотрение множество
$\mathfrak{K}\in \cp(\ov{1,\mathbf{n}}\times \ov{1,\mathbf{n}}),$
элементами которого являются упорядоченные пары индексов, выбираемых из
$\ov{1,\mathbf{n}}$
(случай $\mathfrak{K}= \emp$
из рассмотрения не исключается и отвечает случаю
отсутствия условий предшествования).
Как и в предыдущей главе, именуем элементы $\mathfrak{K}$ адресными парами.
При $z\in \mathfrak{K}$
индекс $\mathrm{pr}_1(z)\in \ov{1,\mathbf{n}}$
играет роль отправителя
(сообщения, груза и т. д.),
а индекс $\mathrm{pr}_2(z)\in \ov{1,\mathbf{n}}$ ---
роль соответствующего получателя.
Разумеется, в задаче,
связанной с листовой резкой на машинах с ЧПУ, содержательный смысл
$\mathrm{pr}_1(z)$ и $\mathrm{pr}_2(z)$
будет иным,
но мы иногда будем все же использовать вышеупомянутую общую интерпретацию,
поскольку это позволяет в ряде случаев
сокращать рассуждения,
сохраняя требуемый смысл.

Мы постулируем, что
\bfn
  \label{4.4.5}
  \fa  \mathfrak{K}_o
  \in \cp^\prime(\mathfrak{K})\ \exists\,z_o\in \mathfrak{K}_o:\,\mathrm{pr}_1(z_o)
  \neq \mathrm{pr}_2(z)\ \ \fa z\in \mathfrak{K}_o
  ,
\efn
учитывая, что (\ref{4.4.5}) --- аналог (\ref{3.3.21}) --- условие,
исключающее зацикливание маршрутов.
Тогда при условии (\ref{4.4.5})
имеем в виде
\bfn
  \label{4.4.6}
  \ca \df \{\al\in \mathbf{P}|\,\al^{-1}\bigl(\mathrm{pr}_1(z)\bigl) <
  \al^{-1}\bigl(\mathrm{pr}_2(z)\bigl) \ \ \fa z\in \mathfrak{K}\}\in \cp^\prime(\mathbf{P})
\efn
множество всех $\mathfrak{K}$-допустимых (по предшествованию) маршрутов
формулируемой ниже <<большой>> задачи.
В дальнейшем предполагаем, что в (\ref{4.4.4})
допустимо использовать только маршруты из
$\ca$ (\ref{4.4.6}).

Сократим
$X$  до естественного и достаточного для всех наших целей
конечного п/м,
полагая
\bfn
  \label{4.4.7}
  \mathfrak{X}\df \{\mathbf{x}_o\} \cup \Bigl(\bigcup\limits_{i=1}^\mathbf{n}
  \mathbf{L}_i\Bigl) \in \mathrm{Fin}(X)
  .
\efn

Видно, что $\mathbf{x}_o$
и элементы упорядоченных пар
$z_1,\,\ldots,z_\mathbf{n}$
в (\ref{4.4.4}) --- суть точки $\mathfrak{X}$ (\ref{4.4.7}).
Иными словами, (\ref{4.4.7})
можно рассматривать как фазовое пространство процессов типа (\ref{4.4.4}),
которые пока были введены нестрого.
Уточним соответствующие понятия.

Полагаем, что $\widetilde{\mathfrak{Z}}$
есть по определению множество всех кортежей
$$
  (z_i)_{i\in\ov{0,\mathbf{n}}}:\,\ov{0,\mathbf{n}} \longrightarrow \mathfrak{X}\times \mathfrak{X}
  .
$$

Среди таких кортежей выделяем трассы
(траектории),
согласованные с~тем или иным маршрутом.
Итак, если $\beta\in \mathbf{P},$
то полагаем, что
\bfn
  \label{4.4.8}
  \mathfrak{Z}_\beta \df \{(z_i)_{i\in\ov{0,\mathbf{n}}}\in
  \widetilde{\mathfrak{Z}}\bigl|\,\bigl(z_o = (\mathbf{x}_o,\mathbf{x}_o)\bigl)
  \&\,(z_t\in \bbl_{\beta(t)}\ \ \fa t \in \ov{1,\mathbf{n}})\}\in \mathrm{Fin}
  (\widetilde{\mathfrak{Z}})
  ,
\efn
получая множество всех трасс,
согласованных с маршрутом $\beta$.
Уточним, что имеются в виду маршруты и трассы <<большой>> задачи.
Каждую упорядоченную пару
$$
  \bigl(\beta,(z_i)_{i\in\ov{0,\mathbf{n}}}\bigl),\beta\in \ca,
  (z_i)_{i\in\ov{0,\mathbf{n}}}\in \mathfrak{Z}_\beta
$$
рассматриваем как ДР <<большой>> задачи.
Тогда
\bfn
  \label{4.4.9}
  \mathrm{SOL}\df \{(\beta,\mathbf{z})\in \ca \times
  \widetilde{\mathfrak{Z}}|\,\mathbf{z}\in \mathfrak{Z}_\beta\}\in
  \mathrm{Fin}(\ca \times \widetilde{\mathfrak{Z}})
\efn
есть
(непустое конечное)
множество всех ДР <<большой>> задачи.
Как и в \ref{sect:4.2}
для формулировки последней потребуется
ввести соответствующие функции стоимости,
роль которых вполне аналогична функциям  (\ref{4.2.1}).
Предположим далее, что
$$
  \mathbf{N}\df \cp^\prime(\ov{1,\mathbf{n}})
$$
(семейство всех непустых
(и неупорядоченных) списков заданий),
после чего введем
  \bfn
  \label{4.4.10}
  \mathbf{c}^\natural\in \car_+[\mathfrak{X}\times \mathfrak{X}\times \mathbf{N}],
  c_1^\natural\in \car_+[\mathfrak{X}\times \mathfrak{X}\times \mathbf{N}],\ldots,c_\mathbf{n}
  ^\natural\in \car_+[\mathfrak{X}\times \mathfrak{X}\times \mathbf{N}],
  f^\natural\in \car_+[\mathfrak{X}]
  .
\efn

Как и в (\ref{4.2.1}),
мы полагаем функции стоимости <<максимально>> продолженными.
Отметим, что значения $\mathbf{c}^\natural$
будут использоваться для оценивания
внешних перемещений.
Итак,
$\mathbf{c}^\natural$ играет ту же роль, что и $\mathbf{c}$
в \ref{sect:4.2}.
Аналогичным образом при
$j\in \ov{1,\mathbf{n}}$ значения $c_j^\natural$
будут использоваться при оценивании работ,
связанных с посещением мегаполиса $\mathbf{L}_j.$
Наконец, значения $f^\natural$
используются для оценивания терминального состояния
(точка $\mathrm{pr}_2(z_\mathbf{n}$)
в (\ref{4.4.4})).
В связи с существенностью тех или иных значений функций (\ref{4.4.10})
можно практически полностью повторить рассуждения относительно (\ref{4.2.1})
при очевидных изменениях  обозначений.

\begin{zam}
\label{z4.4.1}
Поясним, каким именно образом функции $(\ref{4.4.10})$
будут участвовать в формировании аддитивного критерия.
Пусть
$\beta\in \mathbf{P}$,
$t\in \ov{1,\mathbf{n}}$
и
$(z_i)_{i\in\ov{0,\mathbf{n}}} \in \mathfrak{Z}_\beta$.
Тогда перемещение
$$
  \mathrm{pr}_2(z_{t-1}) \longrightarrow \mathrm{pr}_1(z_t)
  ,
$$
где $\mathrm{pr}_2(z_{t-1}) = \mathbf{x}_o$
при $t=1$ и
$\mathrm{pr}_2(z_{t-1})\in \mathbf{L}_{\beta(t-1)}$
в случае $t> 1,$ оценивается следующим значением потерь (затрат)
$$
  \mathbf{c}^\natural\bigl(\mathrm{pr}_2(z_{t-1}),\mathrm{pr}_1(z_t),
  \{\beta(l):\,l\in \ov{t,\mathbf{n}} \}\bigl)\in [0,\infty[
  .
$$

Соответственно, проведение работ,
связанных  с посещением мегаполиса $\mathbf{L}_{\beta(t)},$
оценивается значением потерь
$$
  c_{\beta(t)}^\natural\bigl(z_t,\{\beta(l):\,l\in \ov{t,\mathbf{n}}\}\bigl)\in [0,\infty[
    .
$$

Наконец, при $t = \mathbf{n}$
определено значение
$f^\natural\bigl(\mathrm{pr}_2(z_t)\bigl)\in [0,\infty[,$
характеризующее  терминальное состояние с точки
зрения качества последнего.
\hfill $\Box$
\end{zam}

С учетом замечания~4.4.1
естественным представляется следующее определение критерия:
если
$\beta\in \mathbf{P}$ и $(z_i)_{i\in\ov{0,\mathbf{n}}}\in \mathfrak{Z}_\beta,$
то полагаем
\begin{eqnarray}
  &\widehat{\mathfrak{C}}_\beta[(z_i)_{i\in\ov{0,\mathbf{n}}}]\df
  \sum\limits_{t=1}^\mathbf{n}
  \bigl[\mathbf{c}^\natural\bigl(\mathrm{pr}_2(z_{t-1}),\mathrm{pr}_1(z_t),\{\beta(l):\,l\in
  \ov{t,\mathbf{n}}\}\bigl) +
  &\nonumber\\
  &+ c_{\beta(t)}^\natural\bigl(z_t, \{\beta(l):\,l\in
  \ov{t,\mathbf{n}}\}\bigl)\bigl] + f^\natural\bigl(\mathrm{pr}_2(z_\mathbf{n})\bigl)
  ,
  &
  \label{4.4.11}
\end{eqnarray}
получая неотрицательную величину.
Разумеется, (\ref{4.4.11})
нас будет интересовать в случае
\bfn
  \label{4.4.11`}
  \bigl(\beta,(z_i)_{i\in\ov{0,\mathbf{n}}}\bigl)\in \mathrm{SOL}
  ,
\efn
то есть в ситуации, когда оценивается ДР <<большой>> задачи.
Последняя с~учетом
(\ref{4.4.11})
примет следующий вид:
\bfn
  \label{4.4.12}
  \widehat{\mathfrak{C}}_\beta [(z_i)_{i\in \ov{0,\mathbf{n}}}] \longrightarrow
  \min,\ \beta \in \mathcal{A},\ (z_i)_{i\in \ov{0,\mathbf{n}}}\in \mathfrak{Z}_\beta
  .
\efn

Предположим, что по соображениям сложности вычислений
(число $\mathbf{n}$ достаточно велико)
решить задачу (\ref{4.4.12}) точно не удается:
мы не можем указать явно оптимальное
(в смысле (\ref{4.4.12}))
ДР, хотя структура его нам известна
(имеется в виду представление в терминах соответствующего варианта ДП; см. \ref{sect:4.3}).
Однако с использованием эвристических алгоритмов мы можем за приемлемое время определить
некоторое ДР (\ref{4.4.11`})
нашей <<большой>> задачи (\ref{4.4.12}).
Будем в настоящем разделе полагать
некоторое такое ДР уже найденным.
Рассмотрим вариант его улучшения на основе оптимизирующей вставки.

Итак, пусть
$\bigl(\la, (\mathbf{h}_i)_{i\in\ov{0,\mathbf{n}}}\bigl)\in \mathrm{SOL}$.
Тогда
\bfn
  \label{4.4.13}
  \la\in \ca
  ,
\efn
где $\la$ --- допустимый по предшествованию маршрут в <<большой>> задаче
\bfn
  \label{4.4.14}
  (\mathbf{h}_i)_{i\in\ov{0,\mathbf{n}}}\in \mathfrak{Z}_\la
  ,
\efn
где $(\mathbf{h}_i)_{i\in\ov{0,\mathbf{n}}}$ --- трасса, согласованная с маршрутом
$\la$ (\ref{4.4.13}).
Отметим, что из (\ref{4.4.13}) следует, в частности, свойство
$$
  \la \in \mathbf{P}
  ,
$$
означающее, что
$\la:\,\ov{1,\mathbf{n}}{\stackrel{\mbox{\footnotesize{на}}}{\longrightarrow}}\,
\ov{1,\mathbf{n}}$
и при этом
$\fa i_1\in \ov{1,\mathbf{n}}\ \ \fa i_2\in \ov{1,\mathbf{n}}$
$$
  \bigl(\la(i_1) = \la(i_2)\bigl) \Longrightarrow (i_1 = i_2)
  .
$$

Кроме того,
из (\ref{4.4.13}) имеем (см. (\ref{4.4.6})), что
\bfn
  \label{4.4.15}
  \la^{-1}\bigl(\mathrm{pr}_1(z)\bigl) < \la^{-1}\bigl(\mathrm{pr}_2(z)\bigl)\ \
  \fa z\in \mathfrak{K}
  .
\efn

Здесь $\la^{-1}\in \mathbf{P}$
есть перестановка, обратная к $\la$.
В частности
$$
  \la^{-1}:\ \ov{1,\mathbf{n}} {\stackrel{\mbox{\footnotesize{на}}}{\longrightarrow}}\,
  \ov{1,\mathbf{n}}
  ,
$$
при этом
$\fa k\in\ov{1,\mathbf{n}}$
\bfn
  \label{4.4.16}
  \la\bigl(\la^{-1}(k)\bigl) = \la^{-1}\bigl(\la(k)\bigl) = k
\efn
(свойство (\ref{4.4.16}) легко извлекается из (\ref{3.1.4})).
Свойство (\ref{4.4.15})
характеризует соблюдение условий предшествования.

В отношении кортежа (\ref{4.4.14}) заметим, что
$$
  \mathbf{h}_o\in \mathfrak{X}\times \mathfrak{X}, \mathbf{h}_1\in \mathfrak{X}\times
  \mathfrak{X}, \,\ldots,\mathbf{h}_\mathbf{n}\in \mathfrak{X}\times \mathfrak{X}
  .
$$

В виде (\ref{4.4.14}) имеем кортеж упорядоченных пар.
При этом (см. (\ref{4.4.8}))
\bfn
  \label{4.4.17}
  \mathbf{h}_o = (\mathbf{x}_o,\mathbf{x}_o)
\efn
и справедливо свойство
\bfn
  \label{4.4.18}
  \mathbf{h}_t \in \bbl_{\la(t)}\ \ \fa t\in \ov{1,\mathbf{n}}
  .
\efn

Свойства (\ref{4.4.17}), (\ref{4.4.18})
характеризуют (\ref{4.4.14})
в терминах (\ref{4.4.8})
исчерпывающим образом.
Согласно (\ref{4.4.11}), (\ref{4.4.13}) и (\ref{4.4.14})
получим
\begin{eqnarray}
  &\widehat{\mathfrak{C}}_\la[(\mathbf{h}_i)_{i\in \ov{o,\mathbf{n}}}] =
  \sum\limits_{t=1}^\mathbf{n} \bigl[\mathbf{c}^\natural\bigl(\mathrm{pr}_2(\mathbf{h}_{t-1}),
  \mathrm{pr}_1(\mathbf{h}_t),\{\la(l):\,l\in\ov{t,\mathbf{n}}\}\bigl) +
  &\nonumber\\
  &+c_{\la(t)}^\natural\bigl(\mathbf{h}_t, \{\la(l):\,l\in\ov{t,\mathbf{n}}\}\bigl)\bigl] + f^\natural
  \bigl(\mathrm{pr}_2(\mathbf{h}_\mathbf{n})\bigl)
  .
  &
  \label{4.4.19}
\end{eqnarray}

Разумеется (\ref{4.4.19}) определяет стоимость ДР
$\bigl(\la,(\mathbf{h}_i)_{i\in\ov{0,\mathbf{n}}}\bigl)$
<<большой>> задачи (\ref{4.4.12}).

В связи с (\ref{4.4.19})
условимся о следующем соглашении:
если $X$ и $Y$ --- непустые множества,
отвечающие условию
$$
  g: X \longrightarrow Y
  ,
$$
и $A\in \cp(X),$
то через $g^1(A)$
обозначаем образ множества $A$ при действии отображения $g$:
$$
  g^1(A) \df \{g(a):\,a\in A\}\in \cp(Y)
  .
$$

Отметим:
если $A \neq \emp,$ то и $g^1(A) \neq \emp.$
Среди прочих свойств операции
\bfn
  \label{4.4.20}
  A \longmapsto g^1(A):\,\cp(X) \longrightarrow\cp(Y)
\efn
(взятия образа) отметим следующее
\bfn
  \label{4.4.21}
  g^1(A_1 \cup A_2) = g^1(A_1) \cup g^1(A_2)\ \ \fa A_1\in \cp(X)\ \ \fa A_2\in \cp(X)
  .
\efn

Полезно также напомнить, что
$$
  g^1(B_1 \cap B_2) \subset g^1(B_1) \cap g^1(B_2)
  ,
$$
где $B_1\in \cp(X)$
и $B_2\in \cp(X)$.
Причем возможен случай, когда
$$
  g^1(B_1 \cap B_2) \neq g^1(B_1) \cap g^1(B_2)
  ;
$$
в этой части операция (\ref{4.4.20})
отличается от операции взятия прообраза.
Поскольку $\mathbf{P}$
содержится в множестве всех отображений,
действующих в $\ov{1,\mathbf{n}},$
то (\ref{4.4.20}) можно использовать при
$g=\lambda\in \mathbf{P}$,
$X= Y = \ov{1,\mathbf{n}}$, а потому из
(\ref{4.4.19}) имеем, что
\begin{eqnarray}
  &\widehat{\mathfrak{C}}_\la[(\mathbf{h}_i)_{i\in \ov{o,\mathbf{n}}}] =
  \sum\limits_{t=1}^\mathbf{n} \bigl[\mathbf{c}^\natural\bigl(\mathrm{pr}_2(\mathbf{h}_{t-1}),
  \mathrm{pr}_1(\mathbf{h}_t),\la^1(\ov{t,\mathbf{n}})\bigl) +
  &\nonumber\\
  &+ c_{\la(t)}^\natural\bigl(\mathbf{h}_t, \la^1(\ov{t,\mathbf{n}})\bigl)\bigl] + f^\natural
  \bigl(\mathrm{pr}_2(\mathbf{h}_\mathbf{n})\bigl)
  .
  &
  \label{4.4.22}
\end{eqnarray}

В дальнейшем выражения типа (\ref{4.4.22})
будут, в частности, использоваться в связи со свойством
(\ref{4.4.21}).

Приступим к изложению конструкции оптимизирующей вставки,
для чего вначале обсудим схему
\cite{Cha12`,Cha13`,Cha14`},
связанную с преобразованием условий предшествования.
Фиксируем
\bfn
  \label{4.4.23}
  N \in\ov{2,\mathbf{n}}
  ,
\efn
в котором число $N$ определяет <<размер окна>> вставки.
Кроме того, пусть
\bfn
  \label{4.4.24}
  \nu\in \ov{0,\mathbf{n}-N}
\efn
есть <<момент>> начала вставки.
Тогда
$$
  \La \df \bigl(\la(\nu + s)\bigl)_{s\in\ov{1,N}}
$$
есть отображение из
$\ov{1,N}$ в $\ov{1,\mathbf{n}},$
то есть
$$
  \La:\,\ov{1,N} \longrightarrow \ov{1,\mathbf{n}}
  .
$$

Полагая
$\Gamma \df \La^1(\ov{1,N}),$
получаем непустое п/м
$\ov{1,\mathbf{n}},$ то есть
$$
  \Gamma\in \cp^\prime(\ov{1,\mathbf{n}})
  ,
$$
для которого определена мощность $|\Gamma|\in \bbn$
(действительно, $\Gamma$ --- непустое конечное множество).
Более того, из определения $\Gamma$ следует, что
$$
  |\Gamma| = N
$$
и, кроме того,
$\La\in (\mathrm{bi})[\Gamma].$
Итак,
$$
  \Gamma = \{\La(s):\,s\in\ov{1,N}\} = \{\la(\nu+ s):\,s\in \ov{1,N}\} = \la^1(\ov{\nu+1,\nu+N})
$$
играет далее роль <<окна>> в маршруте $\la.$
В этой связи полагаем, что
\bfn
  \label{4.4.25}
  Q \df \{z\in \mathfrak{K}|\,\bigl(\mathrm{pr}_1(z)\in \Gamma\bigl)\,\&\,
  \bigl(\mathrm{pr}_2(z)\in \Gamma\bigl)\}
  ,
\efn
получая соответствующее <<окно>> в множестве адресных пар.

Теперь конструируем нужный вариант множества $\mathbf{K}$ из \ref{sect:3.3}.
Учитываем при этом тот очевидный факт,
что при $z\in Q$ в виде $\mathrm{pr}_1(z)$
и $\mathrm{pr}_2(z)$ имеем  элементы $\Gamma,$
а потому для биекции $\La^{-1},$
обратной к $\La,$ получаем, что
$$
  \Bigl(\La^{-1}\bigl(\mathrm{pr}_1(z)\bigl)\in \ov{1,N}\Bigl)\,\&\,
  \Bigl(\La^{-1}\bigl(\mathrm{pr}_2(z)\bigl)\in \ov{1,N}\Bigl)
  .
$$

С учетом этого полагаем далее, что
\bfn
  \label{4.4.26}
  \mathbf{K}\df \Bigl\{\Bigl(\La^{-1}\bigl(\mathrm{pr}_1(z)\bigl),\La^{-1}
  \bigl(\mathrm{pr}_2(z)\bigl)\Bigl):\,z\in Q\Bigl\}
  ,
\efn
получая множество, для которого
$$
  \mathbf{K}\subset \ov{1,N} \times \ov{1,N}
  ,
$$
что соответствует \ref{sect:3.3}.
Отметим, что для множества $\mathbf{K}$
(\ref{4.4.26}) выполняется условие (\ref{3.3.21}).

\begin{zam}
\label{z4.4.1`}
Проверим условие $(\ref{3.3.21})$, фиксируя
$\mathbf{K}_o\in \cp^\prime(\mathbf{K}).$
Тогда $\mathbf{K}_o\neq \emp$ и
$\mathbf{K}_o\subset \mathbf{K}.$
При этом
\bfn
  \label{4.4.27}
  Q_o \df \Bigl\{z\in Q \bigl|\,\Bigl(\La^{-1}\bigl(\mathrm{pr}_1(z)\bigl),
  \La^{-1}\bigl(\mathrm{pr}_2(z)\bigl)\Bigl)\in \mathbf{K}_o\Bigl\}\in \cp^\prime(Q)
  .
\efn

Из $(\ref{4.4.25})$ и $(\ref{4.4.27})$
получаем, в частности, что
$Q_o\in \cp^\prime(\mathfrak{K})$.
При этом $\fa z\in Q_o$
$$
  \bigl(\mathrm{pr}_1(z) \in \Gamma\bigl)\,\&\,\bigl(\mathrm{pr}_2(z) \in \Gamma\bigl)
  .
$$

С учетом $(\ref{4.4.5})$
можно указать $z_o\in Q_o$ со свойством
\bfn
  \label{4.4.28}
  \mathrm{pr}_1(z_o)\neq \mathrm{pr}_2(z)\ \ \fa z\in Q_o
  .
\efn

Тогда $z_o\in Q$ и
$\tilde{z}_o \df \Bigl(\La^{-1}\bigl(\mathrm{pr}_1(z_o)
\bigl),\La^{-1}\bigl(\mathrm{pr}_2(z_o)\bigl)\Bigl)\in \mathbf{K}_o$
согласно
$(\ref{4.4.27}).$
Выберем произвольно $z^o\in \mathbf{K}_o.$
Тогда, в
частности, $z^o\in \mathbf{K},$
а потому для некоторого $\tilde{z}^o\in Q$
$$
  z^o = \Bigl(\La^{-1}\bigl(\mathrm{pr}_1(\tilde{z}^o)\bigl), \La^{-1}\bigl(\mathrm{pr}_2(\tilde{z}^o)
  \bigl)\Bigl)
  .
$$

Тогда
$\Bigl(\La^{-1}\bigl(\mathrm{pr}_1(\tilde{z}^o)\bigl),\La^{-1}
\bigl(\mathrm{pr}_2(\tilde{z}^o)\bigl)
\Bigl)\in \mathbf{K}_o$
и с учетом $(\ref{4.4.27})$
$\tilde{z}^o\in Q_o,$ а потому, согласно
$(\ref{4.4.28})$,
$$
 \mathrm{pr}_1(z_o) \neq \mathrm{pr}_2(\tilde{z}^o)
 .
$$

В силу биективности $\La^{-1}$ получаем как следствие, что
$$
  \mathrm{pr}_1(\tilde{z}_o) = \La^{-1}\bigl(\mathrm{pr}_1(z_o)\bigl)
  \neq \La^{-1}\bigl(\mathrm{pr}_2(\tilde{z}^o)\bigl)= \mathrm{pr}_2(z^o)
  .
$$

Поскольку выбор $z^o$ был произвольным,
установлено, что $\tilde{z}_o\in \mathbf{K}_o$
таково, что
$\mathrm{pr}_1(\tilde{z}_o)  \neq \mathrm{pr}_2(z)\ \ \fa z\in \mathbf{K}_o.$
\hfill $\Box$
\end{zam}

Учитывая (\ref{3.3.24}), (\ref{3.3.25}), имеем, что
\begin{eqnarray}
  &\mathbf{A}\df \{\al\in \bbp|\,\al^{-1}\bigl(\mathrm{pr}_1(z)\bigl) <
  \al^{-1}\bigl(\mathrm{pr}_2(z)\bigl)\ \ \fa z\in \mathbf{K}\} =
  &\nonumber\\
  &=\{\al\in \bbp|\,
  \fa z\in \mathbf{K}\ \ \fa t_1\in \ov{1,N}
  \ \ \fa t_2\in \ov{1,N}\ \
  &\nonumber\\
  &\Bigl(\bigl(\al(t_1) =
  \mathrm{pr}_1(z)\bigl)\,\&\,\bigl(\al(t_2) = \mathrm{pr}_2(z)\bigl)\Bigl) \Longrightarrow
  (t_1 < t_2)\}\in \cp^\prime(\bbp)
  .
  &
  \label{4.4.29}
\end{eqnarray}

Итак, $\mathbf{A}$ --- непустое п/м $\bbp$.
Видно, что при $\al\in \bbp$
\bfn
  \label{4.4.30}
  \La\circ \al = \Bigl(\la\bigl(\nu + \al(s)\bigl)\Bigl)_{s\in\ov{1,N}}\in (\mathrm{bi})[\Gamma]
\efn
и, в частности,
$$
  \La \circ \al:\ov{1,N}{\stackrel{\mbox{\footnotesize{на}}}{\longrightarrow}} \Gamma
  .
$$
При этом
$(\La\circ \al)(s)  = \La\bigl(\al(s)\bigl) = \la\bigl(\nu + \al(s)\bigl)
\ \ \fa s\in \ov{1,N}.$
Вместе с тем
$$
  t-\nu\in \ov{1,N}\ \ \fa t\in \ov{\nu+1,\nu+ N}
  ,
$$
поэтому при $\tau\in \ov{\nu+1,\nu+ N}$ определен индекс
$$
  (\La\circ \al)(\tau - \nu) = \la\bigl(\nu + \al(\tau - \nu)\bigl)\in \Gamma
  .
$$

С учетом этого введем естественное правило преобразования маршрута
$\la$
\cite{Cha13`}.

\begin{opred}
\label{o4.4.1}
Если $\al\in \bbp,$ то отображение
$$
  (\al - \mathrm{sew})[\la;\nu]:\,\ov{1,\mathbf{n}}\longrightarrow \ov{1,\mathbf{n}}
$$
определяем следующими правилами
\begin{eqnarray}
  &\bigl((\al - \mathrm{sew})[\la;\nu](t) \df \la(t)\ \ \fa t\in \ov{1,\mathbf{n}}
  \setminus  \ov{\nu+1,\nu + N}\bigl)\,\&\,\Bigl((\al - \mathrm{sew})[\la;\nu](t) \df
  &\nonumber\\
  &\df (\La\circ \al)(t - \nu)\ \ \fa t \in \ov{\nu +1,\nu +N}\bigl)
  .
  &
  \label{4.4.31}
\end{eqnarray}
\end{opred}

\begin{pred}
\label{p4.4.1}
Если
$\al\in \bbp,$
то
$(\al - \mathrm{sew})[\la;\nu]\in \mathbf{P}.$
\end{pred}

{Д о к а з а т е л ь с т в о}.
Покажем сначала, что (при $\al\in \bbp)$
$\mu \df (\al - \mathrm{sew})[\la;\nu]$
есть сюръективное отображение.
Пусть $\theta \in \ov{1,\mathbf{n}}.$
Тогда по выбору $\la$ имеем,
что $\theta = \la(\tau),$
где $\tau\in \ov{1,\mathbf{n}}.$
При этом, согласно (\ref{4.4.31}),
$$
  (\tau \notin \ov{\nu + 1,\nu+ N}) \Longrightarrow \bigl(\mu(\tau) = \la(\tau)\bigl)
  .
$$

Как следствие получили импликацию
\bfn
  \label{4.4.32}
  (\tau \notin \ov{\nu + 1,\nu+ N}) \Longrightarrow \bigl(\exists\,t\in
  \ov{1,\mathbf{n}}:\,\theta = \mu(t)\bigl).
\efn

Допустим, что $\tau \in \ov{\nu + 1,\nu+ N}.$
В этом случае
$$
  \la^{-1}(\theta) \in \ov{\nu+1,\nu+ N}
  ,
$$
так как по выбору $\tau$ имеем равенство $\tau = \la^{-1}(\theta).$
Заметим,
что по определению $\La$
$$
  \theta = \la(\tau) = \la\bigl(\nu +(\tau - \nu)\bigl) = \La(\tau-\nu) \in \Gamma
  ,
$$
где $\tau-\nu\in \ov{1,N}.$
С учетом (\ref{4.4.30}) имеем, что для
некоторого $j\in \ov{1,N}$
\bfn
  \label{4.4.33}
  \theta = (\La \circ \al)(j) = \la\bigl(\nu + \al(j)\bigl)
  ,
\efn
где
$\nu+\alpha(j)\in \ov{\nu+1,\nu+N}.$
Видно, что
$$
  j=(\La \circ \al)^{-1}(\theta)
  ,
$$
а потому из (\ref{4.4.31}) вытекает, что
$$
  \mu(\nu +j) = (\La\circ \al)(j) = \theta
  .
$$

Поскольку, в частности, $\nu+j \in \ov{1,\mathbf{n}},$
то установлено,
что в рассматриваемом случае
$\tau\in \ov{\nu+1,\nu+N}\ \ \exists\, t\in
\ov{1,\mathbf{n}}:\,\theta = \mu(t).$
Итак,
\bfn
  \label{4.4.34}
  (\tau\in \ov{\nu+1,\nu+N})\Longrightarrow \bigl(\exists\, t\in
  \ov{1,\mathbf{n}}:\,\theta = \mu(t)\bigl)
  .
\efn

Из (\ref{4.4.32}) и (\ref{4.4.34}) получаем, что
$\theta\in \mu^1(\ov{1,\mathbf{n}}).$
Поскольку выбор $\theta$ был произвольным,
установлено, что $\ov{1,\mathbf{n}}\subset
\mu^1(\ov{1,\mathbf{n}})$
и, как следствие,
$\mu^1(\ov{1,\mathbf{n}}) = \ov{1,\mathbf{n}},$
то есть
$$
  \mu:\,\ov{1,\mathbf{n}}{\stackrel{\mbox{\footnotesize{на}}}{\longrightarrow}}{\ov{1,\mathbf{n}}}
  .
$$

Осталось проверить инъективность $\mu.$
Пусть $p\in \ov{1,\mathbf{n}}$ и $q\in \ov{1,\mathbf{n}}$
таковы, что
\bfn
  \label{4.4.34`}
  \mu(p) = \mu(q)
  .
\efn

Для краткости полагаем $u\df\mu(p),$
получая в силу (\ref{4.4.34`}),
что и $\mu(q) = u.$
Заметим, что
\bfn
  \label{4.4.35}
  (p\in \ov{\nu+1,\nu+N}) \vee (p\in \ov{1,\mathbf{n}}\setminus  \ov{\nu+1,\nu+ N})
  .
\efn

Оба случая в (\ref{4.4.35}) рассмотрим отдельно.

а) Пусть сначала $p\in \ov{\nu+1,\nu+N}.$ Тогда
$$
  p-\nu \in \ov{1,N}
$$
и, согласно (\ref{4.4.31}),
$u = (\La\circ \al)(p- \nu)\in \Gamma.$ В этом случае
\bfn
  \label{4.4.35`}
  q\in \ov{\nu+1,\nu+N}
  ,
\efn
поскольку в противном случае, то есть при
\bfn
  \label{4.4.36}
  q \notin \ov{\nu+1,\nu+N}
  ,
\efn
имеет место
(см. (\ref{4.4.31}))
$u\notin \Gamma$.
Действительно, пусть истинно (\ref{4.4.36}),
тогда $u=\mu(q) = \la(q),$
где в силу инъективности $\la$
$$
  \la(q) \notin \Gamma
  ,
$$
то есть $u\notin \Gamma$.
При условии (\ref{4.4.36})
получаем очевидное противоречие, доказывающее
(\ref{4.4.35`}).
Итак,
$$
  (p\in\ov{\nu+1,\nu+N})\,\&\,(q\in \ov{\nu+1,\nu+N})
  ,
$$
тогда в силу (\ref{4.4.31}) и (\ref{4.4.34`})
$$
  (\La\circ \al)(p-\nu) = (\La\circ \al)(q - \nu)
  .
$$

С учетом (\ref{4.4.30}) получаем равенство
$$
  p-\nu =q -\nu
  ,
$$
то есть $p = q.$
Итак, установлена импликация
\bfn
  \label{4.4.37}
  (p\in \ov{\nu+1,\nu+N}) \Longrightarrow (p= q)
  .
\efn

б) Пусть теперь
$p\notin \ov{\nu+1,\nu+N}.$
Тогда в силу (\ref{4.4.31}) $\mu(p) = \la(p) \notin \Gamma,$
то есть $u\notin \Gamma$
(учитываем инъективность $\la).$
Поэтому $\mu(q) \notin \Gamma,$ что означает
$$
  q\notin \ov{\nu+1,\nu+N}
  .
$$

В самом деле, при
$q\in \ov{\nu+1,\nu+N}$ в силу (\ref{4.4.31})
имеем, что
$\mu(q) = (\La \circ \al)
(q - \nu)\in \Gamma,$
а это исключено.
Тогда $\mu(q) =\la(q),$
а потому (см. (\ref{4.4.34`}))
$$
  \la(p) = \la(q)
$$
и, как следствие, $p=q$ при $p\notin \ov{\nu+1,\nu+N}.$
Итак,
$$
  (p\notin \ov{\nu+1,\nu+N}) \Longrightarrow (p = q)
  .
$$

С учетом (\ref{4.4.35}) и (\ref{4.4.37})
при условии (\ref{4.4.34`})
получаем во всех возможных случаях требуемое равенство $p =q,$
чем и завершается проверка импликации
$$
  \bigl(\mu(p) = \mu(q)\bigl) \Longrightarrow (p= q)
  .
$$

Поскольку выбор $p$ и $q$ был произвольным,
установлено, что
$\fa i_1\in \ov{1,\mathbf{n}}$
$\fa i_2\in \ov{1,\mathbf{n}}$
$$
  \bigl(\mu(i_1) = \mu(i_2)\bigl) \Longrightarrow (i_1 = i_2)
  .
$$

\begin{pred}
  \label{p4.4.2}
  Если $\al\in \mathbf{A},$ то
  $(\al - \mathrm{sew})[\la;\nu]\in \ca$.
\end{pred}

Д о к а з а т е л ь с т в о.
Пусть $\al\in \mathbf{A}.$
Тогда, в частности, $\al\in \bbp,$ то
есть $\al$ --- перестановка множества $\ov{1,N}.$
В силу предложения~\ref{p4.4.1}
имеем, что
\bfn
  \label{4.4.38}
  \mu \df (\al - \mathrm{sew}))[\la;\nu]\in \mathbf{P}
  .
\efn

Кроме того, по определению множества $\mathbf{A}$
имеем, что (см. (\ref{4.4.29}))
\bfn
  \label{4.4.39}
  \al^{-1}\bigl(\mathrm{pr}_1(z)\bigl) <  \al^{-1}\bigl(\mathrm{pr}_2(z)\bigl)\ \
  \fa z\in \mathbf{K}
  .
\efn

Покажем, что свойством, подобным (\ref{4.4.39}),
обладает и маршрут $\mu.$
Для проверки данного
свойства выберем и зафиксируем адресную пару
<<большой>> задачи
\bfn
  \label{4.4.40}
  \theta\in \mathfrak{K}
  .
\efn

Тогда,  в частности,
$\theta\in \ov{1,\mathbf{n}}\times \ov{1,\mathbf{n}},$
а потому
\bfn
  \label{4.4.41}
  \bigl(\mathrm{pr}_1(\theta)\in\ov{1,\mathbf{n}}\bigl)\,\&\,
  \bigl(\mathrm{pr}_2(\theta)\in\ov{1,\mathbf{n}}\bigl)
  .
\efn

По выбору $\la$
имеем из (\ref{4.4.15}), (\ref{4.4.40}) и (\ref{4.4.41}), что
\bfn
  \label{4.4.42}
  \la^{-1}\bigl(\mathrm{pr}_1(\theta)\bigl) < \la^{-1}\bigl(\mathrm{pr}_2(\theta)\bigl)
  ,
\efn
где $\mathrm{pr}_1(\theta)$ --- <<отправитель>>,
а $\mathrm{pr}_2(\theta)$ --- <<получатель>>
адресной пары $\theta$.
При этом
\bfn
  \label{4.4.43}
  (\theta \in Q) \vee (\theta\in \mathfrak{K}\setminus Q)
  .
\efn

Рассмотрим отдельно оба случая, упомянутых в (\ref{4.4.43}),
имея в виду (\ref{4.4.25}), (\ref{4.4.26}).

1) Пусть $\theta\in Q.$
Поэтому, согласно (\ref{4.4.25}),
\bfn
  \label{4.4.43`}
  \bigl(\mathrm{pr}_1(\theta)\in \Gamma\bigl)\,\&\,\bigl(\mathrm{pr}_2(\theta)\in \Gamma\bigl)
\efn
и, как следствие,
при (\ref{4.4.26})
реализуется включение
$$
  \Bigl(\La^{-1}\bigl(\mathrm{pr}_1(\theta)\bigl),\La^{-1}\bigl(\mathrm{pr}_2(\theta)\bigl)\Bigl)\in \mathbf{K}
  .
$$

Из (\ref{4.4.39}) имеем, что
$$
  \al^{-1}\Bigl(\La^{-1}\bigl(\mathrm{pr}_1(\theta)\bigl)\Bigl) <
  \al^{-1}\Bigl(\La^{-1}\bigl(\mathrm{pr}_2(\theta)\bigl)\Bigl)
  .
$$

Иными словами, справедливо неравенство
\bfn
  \label{4.4.44}
  (\La \circ \al)^{-1}\bigl(\mathrm{pr}_1(\theta)\bigl) <
  (\La \circ \al)^{-1}\bigl(\mathrm{pr}_2(\theta)\bigl)
  .
\efn

В (\ref{4.4.44}) учтено очевидное свойство
$$
  (\La \circ \al)^{-1}(t) = \al^{-1}\bigl(\La^{-1}(t)\bigl) \ \ \fa t\in \Gamma
  .
$$

В этой связи полезно иметь в виду (\ref{4.4.30}).
Отметим, что
(см. (\ref{4.4.38}), (\ref{4.4.41}))
$$
  \Bigl(\mu^{-1}\bigl(\mathrm{pr}_1(\theta)\bigl) \in \ov{1,\mathbf{n}}\Bigl)\,\&\,
  \Bigl(\mu^{-1}\bigl(\mathrm{pr}_2(\theta)\bigl) \in \ov{1,\mathbf{n}}\Bigl)
  .
$$

При этом справедливы очевидные равенства
$$
  \biggl(\mu\Bigl(\mu^{-1}\bigl(\mathrm{pr}_1(\theta)\bigl)\Bigl) = \mathrm{pr}_1 (\theta)\biggl)\,\&\,
  \biggl(\mu\Bigl(\mu^{-1}\bigl(\mathrm{pr}_2(\theta)\bigl)\Bigl) = \mathrm{pr}_2(\theta)\biggl)
  .
$$

Отметим здесь же, что по определению $\mu$
имеем из (\ref{4.4.31}) следующие импликации
$$
  \Bigl(\mu^{-1}\bigl(\mathrm{pr}_1(\theta)\bigl)\notin \ov{\nu+1,\nu+N}\Bigl) \Longrightarrow
  \biggl(\la\Bigl(\mu^{-1}\bigl(\mathrm{pr}_1(\theta)\bigl)\Bigl) = \mathrm{pr}_1(\theta)\biggl),
$$
$$
  \Bigl(\mu^{-1}\bigl(\mathrm{pr}_2(\theta)\bigl)\notin \ov{\nu+1,\nu+N}\Bigl) \Longrightarrow
  \biggl(\la\Bigl(\mu^{-1}\bigl(\mathrm{pr}_2(\theta)\bigl)\Bigl) = \mathrm{pr}_2(\theta)\biggl)
  .
$$

Тогда с учетом (\ref{4.4.43`})
получаем, как следствие, что
\bfn
  \label{4.4.45}
  \Bigl(\mu^{-1}\bigl(\mathrm{pr}_1(\theta)\bigl)\notin \ov{\nu+1,\nu+N}\Bigl)
  \Longrightarrow
  \biggl(\la\Bigl(\mu^{-1}\bigl(\mathrm{pr}_1(\theta)\bigl)\Bigl) \in \Gamma\biggl)
  ,
\efn
\bfn
  \label{4.4.46}
  \Bigl(\mu^{-1}\bigl(\mathrm{pr}_2(\theta)\bigl)\notin \ov{\nu+1,\nu+N}\Bigl)
  \Longrightarrow
  \biggl(\la\Bigl(\mu^{-1}\bigl(\mathrm{pr}_2(\theta)\bigl)\Bigl) \in \Gamma\biggl)
  .
\efn

Вместе с тем по определению $\Gamma$
с учетом инъективности  $\la$
имеем, что
$$
  \biggl(\la\Bigl(\mu^{-1}\bigl(\mathrm{pr}_1(\theta)\bigl)\Bigl) \in \Gamma\biggl)\Longrightarrow
  \Bigl(\mu^{-1}\bigl(\mathrm{pr}_1(\theta)\bigl)\in \ov{\nu+1,\nu+N}\Bigl)
  ,
$$
$$
  \biggl(\la\Bigl(\mu^{-1}\bigl(\mathrm{pr}_2(\theta)\bigl)\Bigl) \in \Gamma\biggl)\Longrightarrow
  \Bigl(\mu^{-1}\bigl(\mathrm{pr}_2(\theta)\bigl)\in \ov{\nu+1,\nu+N}\Bigl)
  .
$$

Комбинируя две последние импликации с (\ref{4.4.45}) и (\ref{4.4.46}),
получаем, что
$$
  \Bigl(\mu^{-1}\bigl(\mathrm{pr}_1(\theta)\bigl)\in \ov{\nu+1,\nu+N}\Bigl)\,\&\,
  \Bigl(\mu^{-1}\bigl(\mathrm{pr}_2(\theta)\bigl)\in \ov{\nu+1,\nu+N}\Bigl)
  .
$$

Последнее в силу (\ref{4.4.31})
и определения $\mu$ означает, что
$$
  \mathrm{pr}_1(\theta) = \mu\Bigl(\mu^{-1}\bigl(\mathrm{pr}_1(\theta)\bigl)\Bigl) =
  (\La\circ \al)\Bigl(\mu^{-1}\bigl(\mathrm{pr}_1(\theta)\bigl) - \nu\Bigl)
  ,
$$
$$
  \mathrm{pr}_2(\theta) = \mu\Bigl(\mu^{-1}\bigl(\mathrm{pr}_2(\theta)\bigl)\Bigl) =
  (\La\circ \al)\Bigl(\mu^{-1}\bigl(\mathrm{pr}_2(\theta)\bigl) - \nu\Bigl)
  .
$$

Учитывая (\ref{4.4.44})
и свойства обратной биекции,
получаем, что
$$
  \mu^{-1}\bigl(\mathrm{pr}_1(\theta)\bigl) - \nu  = (\La\circ \al)^{-1}\biggl((\La\circ\,\al)\Bigl(
  \mu^{-1}\bigl(\mathrm{pr}_1(\theta)\bigl) - \nu\Bigl)\biggl)= (\La\circ\,\al)^{-1}\bigl(\mathrm{pr}_1
  (\theta)\bigl) <
$$
$$
  < (\La\circ\, \al)^{-1}\bigl(\mathrm{pr}_2 (\theta)\bigl)
  = (\La\circ\, \al)^{-1}\biggl((\La\circ\, \al)\Bigl(\mu^{-1}\bigl(\mathrm{pr}_2(\theta)\bigl) -
  \nu\Bigl)\biggl)= \mu^{-1}\bigl(\mathrm{pr}_2(\theta)\bigl) - \nu
  .
$$

В итоге
$\mu^{-1}\bigl(\mathrm{pr}_1(\theta)\bigl) < \mu^{-1}\bigl(\mathrm{pr}_2(\theta)\bigl)$
в рассматриваемом сейчас случае 1).
Итак,
\bfn
  \label{4.4.47}
  (\theta \in Q) \Longrightarrow \Bigl(\mu^{-1}\bigl(\mathrm{pr}_1(\theta)\bigl)<
  \mu^{-1}\bigl(\mathrm{pr}_2(\theta)\bigl)\Bigl)
  .
\efn

2) Пусть теперь $\theta\in \mathfrak{K}\setminus Q.$
В силу (\ref{4.4.25}) имеем, что
\bfn
  \label{4.4.48}
  \bigl(\mathrm{pr}_1(\theta)\notin \Gamma\bigl) \vee
  \bigl(\mathrm{pr}_2(\theta)\notin \Gamma\bigl)
  .
\efn

С учетом определения $\Gamma$
имеем следующие две импликации
\bfn
  \label{4.4.49}
  \bigl(\mathrm{pr}_1(\theta)\notin \Gamma\bigl)\Longrightarrow \bigl(\la(k) \neq
  \mathrm{pr}_1(\theta)\ \ \fa  k\in \ov{\nu+1,\nu+ N}\bigl)
  ,
\efn
\bfn
  \label{4.4.50}
  \bigl(\mathrm{pr}_2(\theta)\notin \Gamma\bigl)\Longrightarrow \bigl(\la(k) \neq
  \mathrm{pr}_2(\theta)\ \ \fa  k\in \ov{\nu+1,\nu+ N}\bigl)
  .
\efn

Рассмотрим отдельно случаи, упомянутые в (\ref{4.4.48}).

2.1) Пусть $\mathrm{pr}_1(\theta) \notin \Gamma.$
Тогда в силу (\ref{4.4.49})
$$
  \la(k) \neq \mathrm{pr}_1(\theta)\ \ \fa k\in \ov{\nu+1,\nu+ N}
  .
$$

Если же $s\in \ov{1,N},$
то $\al(s) \in \ov{1,N}$
и $\nu +\al(s) \in \ov{\nu +1,\nu+N},$
а потому
$$
  \la\bigl(\nu +\al(s)\bigl) \neq \mathrm{pr}_1(\theta)
  .
$$

Как следствие
$(\La\circ \al)(s) \neq \mathrm{pr}_1(\theta).$
Поскольку выбор $s$ был
произвольным, установлено, что
$$
  (\La \circ \al)(k) \neq \mathrm{pr}_1(\theta)\ \ \fa k\in \ov{1,N}
  .
$$

С учетом (\ref{4.4.31})
и определения $\mu$ получаем теперь, что
$$
  \mu(j) \neq \mathrm{pr}_1(\theta)\ \ \fa j\in \ov{\nu+1,\nu+N}
  .
$$

Используя инъективность $\mu^{-1}$ получаем, что
$$
  \mu^{-1}\bigl(\mathrm{pr}_1(\theta)\bigl)\notin \ov{\nu+1,\nu+N}
  .
$$

Последнее означает, что
\bfn
  \label{4.4.51}
  \Bigl(\mu^{-1}\bigl(\mathrm{pr}_1(\theta)\bigl)\in \ov{1,\nu}\Bigl)  \vee
  \Bigl(\mu^{-1}\bigl(\mathrm{pr}_1(\theta)\bigl)\in \ov{\nu +  N+1,\mathbf{n}}\Bigl)
  .
\efn

Вновь используя (\ref{4.4.31}),
получаем следующее положение:
$$
  \la\Bigl(\mu^{-1}\bigl(\mathrm{pr}_1(\theta)\bigl)\Bigl) = \mu\Bigl(\mu^{-1}\bigl(\mathrm{pr}_1(\theta)\bigl)\Bigl)= \mathrm{pr}_1(\theta)
  .
$$

Как следствие реализуется цепочка равенств
\bfn
  \label{4.4.51`}
  \mu^{-1}\bigl(\mathrm{pr}_1(\theta)\bigl)= \la^{-1}\biggl(\la\Bigl(\mu^{-1}
  \bigl(\mathrm{pr}_1(\theta)\bigl)\Bigl)\biggl) =
  \la^{-1}\bigl(\mathrm{pr}_1(\theta)\bigl)
  ,
\efn
откуда с учетом (\ref{4.4.42})
получаем неравенство
\bfn
  \label{4.4.52}
  \mu^{-1}\bigl(\mathrm{pr}_1(\theta)\bigl) < \la^{-1}\bigl(\mathrm{pr}_2(\theta)\bigl)
  .
\efn

Пусть в случае 2.1,
(то есть при $\mathrm{pr}_1(\theta) \notin \Gamma$)
$\mathrm{pr}_2(\theta) \notin \Gamma$,
тогда подобно обоснованию (\ref{4.4.51`}) проверяется, что
$$
  \mu^{-1}\bigl(\mathrm{pr}_2(\theta)\bigl) = \la^{-1}\bigl(\mathrm{pr}_2(\theta)\bigl)
  ,
$$
а из (\ref{4.4.52})
следует неравенство
$\mu^{-1}\bigl(\mathrm{pr}_1(\theta)\bigl) <
\mu^{-1}\bigl(\mathrm{pr}_2(\theta)\bigl).$
Итак, в рассматриваемом случае 2.1,
то есть при
$\mathrm{pr}_1(\theta) \notin \Gamma$,
истинна импликация
\bfn
  \label{4.4.53}
  \bigl(\mathrm{pr}_2(\theta) \notin \Gamma) \Longrightarrow
  \Bigl(\mu^{-1}\bigl(\mathrm{pr}_1(\theta)\bigl) <
  \mu^{-1}\bigl(\mathrm{pr}_2(\theta)\bigl)\Bigl)
  .
\efn

Допустим теперь, что (при условии $\mathrm{pr}_1(\theta) \notin \Gamma)$
$\mathrm{pr}_2(\theta) \in \Gamma.$
Тогда определен индекс
$s_\theta \df \La^{-1}\bigl(\mathrm{pr}_2(\theta) \bigl)\in \ov{1,N},$
а потому
$$
  \La(s_\theta) = \la(\nu +s_\theta) = \mathrm{pr}_2(\theta)
  .
$$

Как следствие,
$\nu + s_\theta = \la^{-1}\bigl(\la(\nu+ s_\theta)\bigl) =
\la^{-1}\bigl(\mathrm{pr}_2(\theta) \bigl).$
С учетом (\ref{4.4.52}) имеем, что
$$
  \mu^{-1}\bigl(\mathrm{pr}_1(\theta)\bigl) < \nu + s_\theta \leqslant \nu+ N
  ,
$$
а потому в силу (\ref{4.4.51})
\bfn
  \label{4.4.54}
  \mu^{-1}\bigl(\mathrm{pr}_1(\theta)\bigl)\in \ov{1,\nu}
  .
\efn

С другой стороны, с учетом (\ref{4.4.31})
и определения $\mu$ имеем, что
$$
  \al^{-1}(s_\theta) = \al^{-1}\Bigl(\La^{-1}\bigl(\mathrm{pr}_2(\theta)\bigl)\Bigl) =
  (\La\circ \al)^{-1}\bigl(\mathrm{pr}_2(\theta)\bigl)\in \ov{1,N}
  ,
$$
следовательно, имеем цепочку равенств
$$
  \mu\bigl(\nu + \al^{-1}(s_\theta)\bigl) = (\La\circ \al)\bigl(\al^{-1}(s_\theta)\bigl) =
  \mathrm{pr}_2(\theta)
  ,
$$
из которой вытекает, что
$$
  \mu^{-1}\bigl(\mathrm{pr}_2(\theta)\bigl) = \nu + \al^{-1}(s_\theta)
  .
$$

В итоге
$\nu +1 \leqslant \mu^{-1}\bigl(\mathrm{pr}_2(\theta)\bigl),$
учитывая (\ref{4.4.54}),
$\mu^{-1}\bigl(\mathrm{pr}_1(\theta)\bigl) <
\mu^{-1}\bigl(\mathrm{pr}_2(\theta)\bigl),$
чем и завершается (в случае
$\mathrm{pr}_1(\theta)\notin \Gamma)$
проверка импликации
$$
  \bigl(\mathrm{pr}_2(\theta)\in \Gamma\bigl) \Longrightarrow
  \Bigl(\mu^{-1}\bigl(\mathrm{pr}_1(\theta)\bigl) <
  \mu^{-1}\bigl(\mathrm{pr}_2(\theta)\bigl)\Bigl)
  .
$$

С учетом (\ref{4.4.53})
в рассматриваемом сейчас случае 2.1
получаем неравенство
\bfn
  \label{4.4.54`}
  \mu^{-1}\bigl(\mathrm{pr}_1(\theta)\bigl) < \mu^{-1}\bigl(\mathrm{pr}_2(\theta)\bigl)
  .
\efn

Итак, установлена следующая импликация:
\bfn
  \label{4.4.55}
  \bigl(\mathrm{pr}_1(\theta)\notin \Gamma\bigl) \Longrightarrow
  \Bigl(\mu^{-1}\bigl(\mathrm{pr}_1(\theta)\bigl) < \mu^{-1}\bigl(\mathrm{pr}_2(\theta)\bigl)\Bigl)
  .
\efn

Проверка импликации
$$
  \bigl(\mathrm{pr}_2(\theta)\notin \Gamma\bigl) \Longrightarrow
  \Bigl(\mu^{-1}\bigl(\mathrm{pr}_1(\theta)\bigl) <
  \mu^{-1}\bigl(\mathrm{pr}_2(\theta)\bigl)\Bigl)
$$
осуществляется по аналогичной схеме
\cite[предложение~7.2]{Cha14`}.
С учетом (\ref{4.4.48}) и (\ref{4.4.55})
получаем требуемое неравенство
(\ref{4.4.54`}) во всех возможных случаях,
если только
$\theta \in \mathfrak{K}\setminus Q.$
Учитывая (\ref{4.4.43}) и (\ref{4.4.47}),
имеем окончательно (\ref{4.4.54`}).
Поскольку выбор $\theta$ был произвольным,
установлено, что (см. (\ref{4.4.40}))
$$
  \mu^{-1}\bigl(\mathrm{pr}_1(z)\bigl) < \mu^{-1}\bigl(\mathrm{pr}_2(z)\bigl)\ \ \fa z\in \mathfrak{K}
  .
$$

С учетом (\ref{4.4.38}) и (\ref{4.4.6}) получаем, что
$\mu\in \ca.$ \hfill $\Box$  \smallskip

Введем в рассмотрение тождественную перестановку
$\mathbf{e}\in \bbp$:
$\mathbf{e}(s) \df s$
$\fa s\in \ov{1,N}$.

\begin{pred}\label{p4.4.3}
Маршрут $\mathbf{e}$
допустим в смысле условий предшествования оптимизирующей вставки:
$\mathbf{e}\in \mathbf{A}$.
\end{pred}

Д о к а з а т е л ь с т в о.
Выберем произвольно $\zeta \in \mathbf{K},$
получая в силу (\ref{4.4.26}) представление
$$
  \zeta = \Bigl(\La^{-1}\bigl(\mathrm{pr}_1(q)\bigl), \La^{-1}\bigl(\mathrm{pr}_2(q)\bigl)\Bigl)
$$
для некоторого $q\in Q.$
Ясно, что
\bfn
  \label{4.4.56}
  \Bigl(\mathrm{pr}_1(\zeta) = \La^{-1}\bigl(\mathrm{pr}_1(q)\bigl)\Bigl)\,\&\,
  \Bigl(\mathrm{pr}_2(\zeta) = \La^{-1}\bigl(\mathrm{pr}_2(q)\bigl)\Bigl)
  .
\efn

При этом $q\in \mathfrak{K}$
обладает свойствами
$$
  \bigl(\mathrm{pr}_1(q) \in \Gamma\bigl) \,\&\,\bigl(\mathrm{pr}_2(q) \in \Gamma\bigl)
  .
$$

С учетом
$\mathfrak{K}$-допустимости маршрута $\la$
(см. (\ref{4.4.15})) имеем, что
\bfn
  \label{4.4.57}
  \la^{-1}\bigl(\mathrm{pr}_1(q)\bigl) < \la^{-1}\bigl(\mathrm{pr}_2(q)\bigl)
  .
\efn

Из (\ref{4.4.56})
получаем очевидные равенства
$$
  \Bigl(\La\bigl(\mathrm{pr}_1(\zeta)\bigl) = \mathrm{pr}_1(q)\Bigl)\,\&\,
  \Bigl(\La\bigl(\mathrm{pr}_2(\zeta)\bigl) = \mathrm{pr}_2(q)\Bigl)
  .
$$

Иными словами,
$$
  \Bigl(\la\bigl(\nu + \mathrm{pr}_1(\zeta)\bigl) = \mathrm{pr}_1(q)\Bigl)\,\&\,
  \Bigl(\la\bigl(\nu + \mathrm{pr}_2(\zeta)\bigl) = \mathrm{pr}_2(q)\Bigl)
  ,
$$
поэтому получаем, что
$$
  \Bigl(\la^{-1}\bigl(\mathrm{pr}_1(q)\bigl) = \nu +\mathrm{pr}_1(\zeta)\Bigl)\,\&\,
  \Bigl(\la^{-1}\bigl(\mathrm{pr}_2(q)\bigl) = \nu +\mathrm{pr}_2(\zeta)\Bigl)
  .
$$

С учетом  (\ref{4.4.57}) имеем неравенство
$$
  \nu +\mathrm{pr}_1(\zeta) < \nu +\mathrm{pr}_2(\zeta)
  ,
$$
тогда
$\mathrm{pr}_1(\zeta) < \mathrm{pr}_2(\zeta),$
то есть
$\mathbf{e}\bigl(\mathrm{pr}_1(\zeta) \bigl) < \mathbf{e}\bigl(\mathrm{pr}_2(\zeta)\bigl).$
Поскольку выбор $\zeta$
был произвольным,
то установлено, что
$\mathbf{e}\in \mathbf{A}.$
\hfill $\Box$

\begin{pred}
\label{p4.4.4}
Справедливо равенство
$(\mathbf{e}- \mathrm{sew})[\la;\nu] = \la$.
\end{pred}

Д о к а з а т е л ь с т в о.
Полагаем для краткости, что
$$
  \mu \df (\mathbf{e}- \mathrm{sew})[\la;\nu]
  .
$$

Тогда из двух последних предложений следует, что $\mu \in \ca,$
причем
$$
  \mu(t) = (\La \circ \mathbf{e})(t -\nu) = \la\Bigl(\nu + \bigl(\mathbf{e}(t) - \nu\bigl)\Bigl) =
  \la\bigl(\nu + (t-\nu)\bigl) = \la(t)\ \ \fa t\in \ov{\nu+1,\nu+N}
  .
$$

С учетом определения $\mu$
получаем теперь, что
$\mu(t) = \la(t)\ \ \fa t\in \ov{1,\mathbf{n}}.$
Иными словами,
$\mu = \la.$
\hfill $\Box$

\subsubsection*{Построение локальной задачи}

Полагаем до конца настоящего раздела, что
\bfn
  \label{4.4.58}
  x^o \df \mathrm{pr}_2(\mathbf{h}_\nu)
  ,
\efn
получая включение $x^o\in \mathfrak{X}.$
Обозначение (\ref{4.4.58}) сохраняем и в
следующем разделе.
Кроме  того,
полагаем в данном и последующих разделах главы, что
$\fa s\in \ov{1,N}$
\bfn
  \label{4.4.59}
  (M_s \df \mathbf{L}_{\La(s)})\,\&\,(\bbm_s \df \bbl_{\La(s)})
  .
\efn

Ясно, что
$\bbm_t\in \cp^\prime(M_t \times M_t)$
$\fa t\in \ov{1,N}.$
Итак,
$$
  \bbm_1\subset M_1 \times M_1,\,\ldots,\bbm_N\subset M_N \times M_N
  .
$$

Напомним, что (см. (\ref{3.3.3}))
$$
  \mathbf{M}_1 = \{\mathrm{pr}_2(z):\,z\in \bbm_1\},\,\ldots,\mathbf{M}_N = \{\mathrm{pr}_2(z):\,z\in \bbm_N\}
$$
суть непустые конечные множества,
$$
  \mathbf{M}_1\subset M_1,\,\ldots,\mathbf{M}_N\subset M_N
  .
$$

Действуя в соответствии с символикой, принятой в \ref{sect:4.2}, \ref{sect:4.3},
введем в рассмотрение множества $\bbx$ и $\mathbf{X}$,
см. (\ref{3.3.5}), (\ref{3.3.28}).
Напомним, что
$\bbx \hm \in \mathrm{Fin}(X)$,
$\mathbf{X}\in \mathrm{Fin}(\bbx)$ и
$x^o\in \mathbf{X}$.
Кроме того, из
(\ref{4.4.59}) следует, что в нашем случае
$$
  \bbx = \{x^o\} \cup \Bigl(\bigcup\limits_{s=1}^N M_s\Bigl)\in \mathrm{Fin}(\mathfrak{X})
  ,
$$
$$
  \mathbf{X}= \{x^o\} \cup \Bigl(\bigcup\limits_{s=1}^N \mathbf{M}_s\Bigl)\in \mathrm{Fin}(\mathfrak{X})
  ,
$$
$\mathbf{X}\subset \bbx$
и, как следствие,
$$
  \bbx \times \mathbf{X}\subset \bbx \times \bbx \subset \mathfrak{X}\times \mathfrak{X}
  .
$$

Теперь следуем (\ref{3.3.15})
в части определения трасс,
согласованных с тем или иным маршрутом,
используя конкретизацию (\ref{4.4.59}).
Таким образом, в частности,
каждому маршруту $\al\in \mathbf{A}$
сопоставляется непустое конечное множество
$\mathbf{Z}_\al$
(\ref{3.3.15}) всех трасс,
согласованных с данным маршрутом.
Поэтому в
виде $\widetilde{\mathbf{D}}$
(\ref{3.3.27})
получаем требуемое множество ДР, реализуемых
в рамках оптимизирующей вставки.

\subsubsection*{Функции стоимости локальной задачи}

Напомним, что $\mathfrak{N}$
есть семейство всех непустых п/м $\ov{1,N}.$
Если $K\in \mathfrak{N},$
то определено множество
$$
  \La^1(K) = \{\la(\nu+s):\,s\in K\}\in \cp^\prime(\Gamma)
  ,
$$
для которого, в частности, имеем, что
$$
  \La^1(K) \subset \ov{1,\mathbf{n}}
  .
$$

Всюду в дальнейшем полагаем, если не оговорено противное, что
\bfn
  \label{4.4.60}
  \nu + N + 1 \leqslant \mathbf{n}
\efn
(мы несколько усилили требование к выбору $\nu).$
Тогда
$$
  \ov{\nu+N+1,\mathbf{n}}\in \mathbf{N}
  ,
$$
то есть
$\ov{\nu+N+1,\mathbf{n}}$
есть непустое п/м $\ov{1,\mathbf{n}}.$
Как следствие,
имеем при
$K\in \mathfrak{N},$ что
$$
  \La^1(K) \cup \la^1(\ov{\nu+N+1,\mathbf{n}})\in \mathbf{N}
  .
$$

Если к тому же
$z\in \bbx \times \bbx,$ то определены значения
\begin{eqnarray}
  &\mathbf{c}^\natural\bigl(z,\La^1(K) \cup \la^1(\ov{\nu+N+1,\mathbf{n}})\bigl)\in
  &\nonumber\\
  &\in[0,\infty[,\,c_1^\natural\bigl(z,\La^1(K) \cup \la^1(\ov{\nu+N+1,\mathbf{n}})\bigl)\in [0,\infty[,
  \,\ldots,
  &\nonumber\\
  &c_N^\natural\bigl(z,\La^1(K) \cup \la^1(\ov{\nu+N+1,\mathbf{n}})\bigl)\in [0,\infty[
  .
  &
  \label{4.4.61}
\end{eqnarray}

Упорядоченная пара $z$ может конкретизироваться по-разному:
при рассмотрении значения функции $\mathbf{c}^\natural$
в качестве  $z$ используем,
как правило, пару $(x,y),$ где
$x= x^o$ или $x\in \mathbf{M}_j$ для некоторого
$j\in\ov{1,N}$
(а $y\in M_s$ при $s\in \ov{1,N})$;
если же рассматривается значение той или иной функции $c_s^\natural,$ где
$s\in \ov{1,N},$
то в~качестве $z$ обычно используется элемент $\bbm_s,$
что в конкретной задаче управления инструментом соответствует случаю, когда
$\mathrm{pr}_1(z)$ --- точка врезки, а~$\mathrm{pr}_2(z)$ --- точка выключения инструмента.

С учетом (\ref{4.4.61})
конкретизируем значения локальных функций стоимости (\ref{4.2.1}).

Итак, предположим, что функция
$\mathbf{c}\in \car_+[\bbx \times \bbx \times \mathfrak{N}]$
определяется по следующему правилу:
\bfn
  \label{4.4.62}
  \mathbf{c}(z,K) \df \mathbf{c}^\natural\bigl(z,\La^1(K) \cup \la^1(\ov{\nu+N+1,
  \mathbf{n}})\bigl)\ \ \fa z\in \bbx \times \bbx\ \ \fa K\in \mathfrak{N}
  .
\efn

Если $j\in \ov{1,N},$
то определяем функцию
$c_j\in \car_+[\bbx \times \bbx \times \mathfrak{N}]$
условиями
\bfn
  \label{4.4.63}
  c_j(z,K) \df c_{\La(j)}^\natural\bigl(z,\La^1(K) \cup \la^1(\ov{\nu+N+1,
  \mathbf{n}})\bigl)\ \ \fa z\in \bbx \times \bbx\ \ \fa K\in \mathfrak{N}
  .
\efn

Наконец, терминальную функцию $f\in \car_+[\bbx]$
определяем в виде
\bfn
  \label{4.4.64}
  f(x) \df \mathbf{c}^\natural\bigl(x,\mathrm{pr}_1(\mathbf{h}_{\nu+N+1}),
  \la^1(\ov{\nu+N+1,\mathbf{n}})\bigl)\ \ \fa x\in \bbx
  .
\efn

Итак, все функции затрат для оптимизирующей вставки построены и
можно ввести соответствующий аддитивный критерий, используя (\ref{4.2.4}):
каждому ДР
$\bigl(\al,(z_i)_{i\in\ov{0,N}}\bigl)\in \widetilde{\mathbf{D}}$
(это означает, что
$\al\in \mathbf{A}$ и $(z_i)_{i\in\ov{0,N}}\in \mathbf{Z}_\al$)
сопоставляем число
$$
  \mathfrak{B}_\al [(z_i)_{i\in\ov{0,N}}]\in [0,\infty[
  .
$$

В итоге получаем нужный вариант задачи (\ref{4.2.5}),
для решения которой можно использовать схему \ref{sect:4.2}
на основе широко понимаемого ДП.
Мы предположим,
что значение $N$ <<умеренно>> в том смысле,
что упомянутая схема позволяет провести все
необходимые вычисления и получить значение
$\mathbb{V}$ (\ref{4.2.6}),
а также определить оптимальное ДР
$\bigl(\al^o,(z_i^o)_{i\in\ov{0,N}}\bigl)$
(\ref{4.2.7}).
Итак, полагаем, что маршрут
\bfn
  \label{4.4.65}
  \al^o\in \mathbf{A}
\efn
и трасса (траектория)
\bfn
  \label{4.4.66}
  (z_i^o)_{i\in\ov{0,N}}\in \mathbf{Z}_{\al^o}
\efn
таковы, что при этом
\bfn
  \label{4.4.67}
  \mathfrak{B}_{\al^o}[(z_i^o)_{i\in\ov{0,N}}] = \mathbb{V}
  .
\efn

Последнее означает, что справедливы неравенства
$$
  \mathfrak{B}_{\al^o}[(z_i^o)_{i\in\ov{0,N}}] \leqslant \mathfrak{B}_\al[(z_i)_{i\in\ov{0,N}}]\ \
  \fa \al\in \mathbf{A}\ \ \fa (z_i)_{i\in\ov{0,N}}\in \mathbf{Z}_\al
  .
$$

Отметим простое следствие последнего положения:
с учетом предложения~\ref{p4.4.3} получаем, что
\bfn
  \label{4.4.68}
  \mathfrak{B}_{\al^o}[(z_i^o)_{i\in\ov{0,N}}]\leqslant
  \mathfrak{B}_\mathbf{e}[(z_i)_{i\in\ov{0,N}}]\ \ \fa
  (z_i)_{i\in\ov{0,N}}\in \mathbf{Z}_\mathbf{e}
  .
\efn

С целью последующего использования (\ref{4.4.68})
введем в рассмотрение кортеж
$(\tilde{\mathbf{h}}_i)_{i\in\ov{0,N}}\in \bbz,$
для которого
$$
  \bigl(\tilde{\mathbf{h}}_o \df (x^o,x^o)\bigl)\,\&\,(\tilde{\mathbf{h}}_t \df
  \mathbf{h}_{\nu+t}\ \ \fa t\in \ov{1,N})
  .
$$

\begin{pred}
\label{p4.4.5}
Кортеж $(\tilde{\mathbf{h}}_i)_{i\in\ov{0,N}}$
является трассой, согласованной с маршрутом $\mathbf{e}:$
$$
  (\tilde{\mathbf{h}}_i)_{i\in\ov{0,N}}\in \mathbf{Z}_\mathbf{e}
  .
$$
\end{pred}

Д о к а з а т е л ь с т в о.
Если $t\in \ov{1,N},$
то
$\tilde{\mathbf{h}}_t = \mathbf{h}_{\nu +t}\in \bbl_{\La(t)},$
а потому
$$
  \tilde{\mathbf{h}}_t \in \bbm_{\mathbf{e}(t)}
  ,
$$
так как $(\La \circ \mathbf{e})(t) = \La(t).$
Дальнейшее рассуждение очевидно
(см. (\ref{3.3.15})).
\hfill $\Box$

Из (\ref{4.4.68}) и предложения~{\ref{p4.4.5} получаем неравенство
$$
  \mathfrak{B}_{\al^o}[(z_i^o)_{i\in\ov{0,N}}]\leqslant
  \mathfrak{B}_\mathbf{e}[(\tilde{\mathbf{h}}_i)_{i\in\ov{0,N}}]
  .
$$

Как следствие, с учетом (\ref{4.4.67}), имеем, что
\bfn
  \label{4.4.69}
  \mathbb{V} \leqslant \mathfrak{B}_\mathbf{e}[(\tilde{\mathbf{h}}_i)_{i\in\ov{0,N}}]
  .
\efn

Учитывая (\ref{4.4.69}), получаем, что
\bfn
  \label{4.4.70}
  \kappa \df \mathfrak{B}_\mathbf{e}[(\tilde{\mathbf{h}}_i)_{i\in\ov{0,N}}] -
  \mathbb{V} = \mathfrak{B}_\mathbf{e}[(\tilde{\mathbf{h}}_i)_{i\in\ov{0,N}}]-
  \mathfrak{B}_{\al^o}[(z_i^o)_{i\in\ov{0,N}}]\in [0,\infty[
  .
\efn

Заметим, что величина (\ref{4.4.70})
<<привязана>> к исходному ДР
$\bigl(\la,(\mathbf{h}_i)_{i\in\ov{0,N}}\bigl)$
<<большой>> задачи.
В этой связи полезно заметить, что
(см. (\ref{4.4.58}))
$$
  \mathfrak{B}_\mathbf{e}[(\tilde{\mathbf{h}}_i)_{i\in\ov{0,N}}]=
  \sum\limits_{s=1}^N\bigl[\mathbf{c}\bigl(\mathrm{pr}_2(\tilde{\mathbf{h}}_{s-1}),\mathrm{pr}_1
  (\tilde{\mathbf{h}}_s),\{\mathbf{e}(t):\,t\in\ov{s,N}\}\bigl) +
$$
$$
  + c_{\mathbf{e}(s)}\bigl(\tilde{\mathbf{h}}_s,\{\mathbf{e}(t):\,t\in\ov{s,N}\}\bigl)\bigl] +
  f\bigl(\mathrm{pr}_2(\tilde{\mathbf{h}}_N)\bigl) = \bigl[\mathbf{c}\bigl(\mathrm{pr}_2
  (\tilde{\mathbf{h}}_o),\mathrm{pr}_1(\tilde{\mathbf{h}}_1),\ov{1,N}) +
$$
$$
  + c_1(\tilde{\mathbf{h}}_1,\ov{1,N})\bigl] +
  \sum\limits_{s=2}^N\bigl[\mathbf{c}\bigl(\mathrm{pr}_2(\tilde{\mathbf{h}}_{s-1}),
  \mathrm{pr}_1(\tilde{\mathbf{h}}_s),\ov{s,N}\bigl) +
  c_s(\tilde{\mathbf{h}}_s,\ov{s,N})\bigl] +
  f\bigl(\mathrm{pr}_2(\tilde{\mathbf{h}}_N)\bigl) =
$$
$$
  = \bigl[\mathbf{c}\bigl(x^o,\mathrm{pr}_1(\mathbf{h}_{\nu+1}),\ov{1,N}\bigl) +
  c_1(\mathbf{h}_{\nu+1},\ov{1,N})\bigl] +
  \sum\limits_{s=2}^N\bigl[\mathbf{c}\bigl(\mathrm{pr}_2(\mathbf{h}_{\nu+s-1}),
$$
$$
  \mathrm{pr}_1(\mathbf{h}_{\nu+s}),\ov{s,N}\bigl) + c_s(\mathbf{h}_{\nu+s},\ov{s,N})\bigl] +
  f\bigl(\mathrm{pr}_2(\mathbf{h}_{\nu+N})\bigl) =
$$
$$
  =\bigl[\mathbf{c}\bigl(\mathrm{pr}_2(\mathbf{h}_\nu),\mathrm{pr}_1(\mathbf{h}_{\nu+1}),
  \ov{1,N}\bigl) + c_1(\mathbf{h}_{\nu+1},\ov{1,N})\bigl] +
$$
$$
  + \sum\limits_{s=2}^N\bigl[\mathbf{c}\bigl(\mathrm{pr}_2(\mathbf{h}_{\nu+s-1}),\mathrm{pr}_1
  (\mathbf{h}_{\nu+s}),\ov{s,N}\bigl) + c_s(\mathbf{h}_{\nu+s},\ov{s,N})\bigl] +
$$
$$
  +f\bigl(\mathrm{pr}_2(\mathbf{h}_{\nu+N})\bigl) =
  \sum\limits_{s=1}^N\bigl[\mathbf{c}\bigl(\mathrm{pr}_2(\mathbf{h}_{\nu+s-1}),\mathrm{pr}_1
  (\mathbf{h}_{\nu+s}),\ov{s,N}\bigl)+
$$
  $$+ c_s(\mathbf{h}_{\nu+s},\ov{s,N})\bigl] + f\bigl(\mathrm{pr}_2(\mathbf{h}_{\nu+N})\bigl).
$$

Учтем теперь (\ref{4.4.62}) -- (\ref{4.4.64}).
Тогда последнее выражение сводится к~следующему
\begin{eqnarray}
  &\mathfrak{B}_\mathbf{e}[(\tilde{\mathbf{h}}_i)_{i\in\ov{0,N}}]=
  \sum\limits_{s=1}^N \bigl[\mathbf{c}^\natural\bigl(\mathrm{pr}_2(\mathbf{h}_{\nu+s-1}),
  \mathrm{pr}_1(\mathbf{h}_{\nu+s}),
  \La^1(\ov{s,N})\cup \la^1(\ov{\nu+N+1,\mathbf{n}})\bigl) +
  &\nonumber\\
  &+ c_{\La(s)}^\natural\bigl(\mathbf{h}_{\nu+s},\La^1(\ov{s,N}) \cup \la^1(\ov{\nu+N+1,\mathbf{n}})\bigl)\bigl]+
  &\nonumber\\
  &+ \mathbf{c}^\natural\bigl(\mathrm{pr}_2(\mathbf{h}_{\nu+N}),\mathrm{pr}_1(\mathbf{h}_{\nu+N+1}),
  \la^1(\ov{\nu+N+1,\mathbf{n}})\bigl)
  .
  &
  \label{4.4.71}
\end{eqnarray}

С учетом определения $\La$
получим, что при
$s\in\ov{1,N}$
$$
  \La^1(\ov{s,N}) \cup \la^1(\ov{\nu+N+1,\mathbf{n}}) = \{\la(\nu+t):\,t\in \ov{s,N}\} \cup
  \la^1(\ov{\nu+N+1,\mathbf{n}}) =
$$
$$
  =\{\la(k):\,k\in\ov{\nu+s,\nu+N}\} \cup \la^1(\ov{\nu+N+1,\mathbf{n}})=
$$
$$
  =\la^1(\ov{\nu+s,\nu+N}) \cup \la^1(\ov{\nu+N+1,\mathbf{n}}) =
$$
$$
  = \la^1(\ov{\nu+s,\nu+N} \cup\, \ov{\nu+N+1,\mathbf{n}}) = \la^1(\ov{\nu+s,
  \mathbf{n}})
  .
$$

Используем следующее простое свойство:
образ объединения двух множеств равен объединению их образов.
Тогда из (\ref{4.4.71}) получаем, что
\begin{eqnarray}
  &\mathfrak{B}_\mathbf{e}[(\tilde{\mathbf{h}}_i)_{i\in\ov{0,N}}]=
  \sum\limits_{s=1}^N \bigl[\mathbf{c}^\natural\bigl(\mathrm{pr}_2(\mathbf{h}_{\nu+s-1}),
  \mathrm{pr}_1(\mathbf{h}_{\nu+s}),
  \la^1(\ov{\nu+s,\mathbf{n}})\bigl) +
  &\nonumber\\
  &+c_{\la(\nu+s)}^\natural(\mathbf{h}_{\nu+s},
  \la^1(\ov{\nu+s,\mathbf{n}})\bigl)\bigl] +
  \mathbf{c}^\natural\bigl(\mathrm{pr}_2(\mathbf{h}_{\nu+N}),\mathrm{pr}_1(\mathbf{h}_{\nu+N+1}),
  \la^1(\ov{\nu+N+1,\mathbf{n}})\bigl) =
  &\nonumber\\
  &=\sum\limits_{t=\nu+1}^{\nu+N}\bigl[\mathbf{c}^\natural\bigl(\mathrm{pr}_2(\mathbf{h}_{t-1}),
  \mathrm{pr}_1(\mathbf{h}_t),\la^1(\ov{t,\mathbf{n}})\bigl)+
  c_{\la(t)}^\natural\bigl(\mathbf{h}_t,\la^1(\ov{t,\mathbf{n}})\bigl)\bigl] +
  &\nonumber\\
  &+ \mathbf{c}^\natural\bigl(\mathrm{pr}_2(\mathbf{h}_{\nu+N}),\mathrm{pr}_1(\mathbf{h}_{\nu+N+1}),
  \la^1(\ov{\nu+N+1,\mathbf{n}})\bigl)
  .
  &
  \label{4.4.72}
\end{eqnarray}

Из (\ref{4.4.11}) и (\ref{4.4.72}) видно, что
$\mathfrak{B}_\mathbf{e}[(\tilde{\mathbf{h}}_i)_{i\in\ov{0,N}}]$
является <<частью>>
$\widehat{\mathfrak{C}}_\la[(\mathbf{h}_i)_{i\in\ov{0,\mathbf{n}}}],$
а точнее говоря, слагаемым в сумме, определяющей последнее значение.
