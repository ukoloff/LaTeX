% !TeX root = ..

\clearpage
\section*{
  Заключительные замечания
  к части II
}

Задача управления инструментом при листовой резке на машинах с~ЧПУ
является весьма трудной с точки зрения вычислительной реализации методов,
развиваемых на основе математической теории.
При решении конкретных практических задач неизбежным
представляется применение эвристических алгоритмов.
Эти алгоритмы могут быть построены так,
что при этом учитываются весьма различные
и плохо формализуемые ограничения.
Последнее, наряду с достаточно высоким
быстродействием и возможностью решать <<большие>> задачи,
является ценным обстоятельством.
Вместе с тем при их применении могут возникать явные коллизии.
Для оперативного
вмешательства с целью <<исправления>> фрагментов ДР
и было предложено применять оптимизирующие вставки.

Данные вставки конструировались так,
что оказывалось возможным
(на локальном уровне)
использовать методы построения точных решений.
Это было достигнуто за счет применения вставок,
приводящих к маршрутным задачам умеренной размерности.
Таким образом, в задачах маршрутизации,
имеющих достаточно большую размерность,
удалось <<задействовать>> аппарат ДП, применяемый в режиме итераций,
причем сами итерации могут организовываться,
как показано в главе~5, по-разному.
В частности, в качестве правила,
определяющего конкретное расположение оптимизирующей вставки на
<<эвристическом>> решении,
может быть выбран принцип своеобразной максимизации условий предшествования,
<<захватываемых>> данной вставкой,
т. е. вовлечение в локальную задачу
как можно большего числа ограничений упомянутого типа.
Данный подход, реализованный
в \ref{sect:5.4},
показывает,
что учет некоторых ограничений может играть положительную роль
в вопросах снижения сложности вычислений.
Конечно, такой эффект достигается при должной
теоретической проработке,
что и было сделано в рамках ДП.

Хотя интуитивно очевидно,
что чем больше и богаче условия предшествования,
тем проще соответствующая маршрутная задача,
строгая зависимость теоретической границы сложности
от свойств ограничений предшествования еще недостаточно изучена.
В связи с этим отметим работы \cite{bib:x200,bib:x201}.
Существует два специальных типа ограничений предшествования,
для которых полиномиальная временная сложность
дискретных маршрутных задач доказана теоретически.
Первый тип ограничений был введен Э. Баласом \cite{bib:x202}
для классической задачи коммивояжёра {\it TSP}.
Эффективные точные алгоритмы для обобщенной задачи коммивояжера {\it GTSP}
с ограничениями предшествования этого типа
были предложены в недавних работах \cite{bib:x203,bib:x204}.
Эффективные параметризованные алгоритмы для {\it GTSP}
со вторым типом ограничений приоритета,
которые называются квази- и псевдопирамидальными,
описаны в \cite{bib:x205,bib:x206}.
Принимая во внимание вышеизложенное,
можно резюмировать,
что в области алгоритмического анализа
эвристические алгоритмы для оптимизации
маршрута инструмента машин листовой резки с ЧПУ
все еще остаются слабо изученными.
В частности, отсутствие эффективных моделей смешанного целочисленного линейного программирования
({\it MILP}) для {\it GTSP}
делает невозможным использование современных оптимизаторов,
таких как {\it Gurobi} \cite{bib:x207},
для построения нижних и верхних границ и изучения эвристических решений.
Эта проблема
(наряду с проблемой разработки эффективных алгоритмов маршрутизации для задач большой размерности)
также является предметом дальнейших исследований.
