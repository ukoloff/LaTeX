% !TeX root = ..

\clearpage
\section*{
  Заключительные замечания
  к части II
}

Задача управления инструментом при листовой резке на машинах с~ЧПУ
является весьма трудной с точки зрения вычислительной реализации методов,
развиваемых на основе математической теории.
При решении конкретных практических задач неизбежным
представляется применение эвристических алгоритмов.
Эти алгоритмы могут быть построены так,
что при этом учитываются весьма различные
и плохо формализуемые ограничения.
Последнее, наряду с достаточно высоким
быстродействием и возможностью решать <<большие>> задачи,
является ценным обстоятельством.
Вместе с тем при их применении могут возникать явные коллизии.
Для оперативного
вмешательства с целью <<исправления>> фрагментов ДР
и было предложено применять оптимизирующие вставки.

Данные вставки конструировались так,
что оказывалось возможным
(на локальном уровне)
использовать методы построения точных решений.
Это было достигнуто за счет применения вставок,
приводящих к маршрутным задачам умеренной размерности.
Таким образом, в задачах маршрутизации,
имеющих достаточно большую размерность,
удалось <<задействовать>> аппарат ДП, применяемый в режиме итераций,
причем сами итерации могут организовываться,
как показано в главе~5, по-разному.
В частности, в качестве правила,
определяющего конкретное расположение оптимизирующей вставки на
<<эвристическом>> решении,
может быть выбран принцип своеобразной максимизации условий предшествования,
<<захватываемых>> данной вставкой,
т. е. вовлечение в локальную задачу
как можно большего числа ограничений упомянутого типа.
Данный подход, реализованный
в \ref{sect:5.4},
показывает,
что учет некоторых ограничений может играть положительную роль
в вопросах снижения сложности вычислений.
Конечно, такой эффект достигается при должной
теоретической проработке,
что и было сделано в рамках ДП.
