% !TeX root = ..

\section{
  Динамическое программирование
  при~усложненных функциях стоимости
}
\label{sect:4.3}
\setcounter{equation}{0}

Мы без  дополнительных пояснений используем построения главы 3,
касающиеся маршрутов и трасс
(траекторий).
В частности, используем конструкции
\ref{sect:3.4},
связанные с редукцией системы ограничений
(замена допустимости по предшествованию допустимостью по вычеркиванию).
В этой связи, действуя
в духе (\ref{3.4.20}),
полагаем при
$x\in \bx,\,K\in \mathfrak{N},\,\al\in (\mathrm{bi})[K]$ и
$(z_i)_{i\in\ov{0,|K|}} \in \mathbf{Z}(x,K,\al),$
что
\begin{eqnarray}
  &\widehat{\mathfrak{B}}_\al[(z_i)_{i\in\ov{0,|K|}}\,\bigl|\,K]  \df
  \sum\limits_{t=1}^{|K|}\bigl[\mathbf{c}\bigl(\mathrm{pr}_2(z_{t-1}),\mathrm{pr}_1(z_t),\{\al(j):\,
  j\in \ov{t,|K|}\}\bigl) +
  &\nonumber\\
  &+ c_{\al(t)}\bigl(z_t, \{\al(j):\, j\in \ov{t,|K|}\}\bigl)\bigl] +
  f\bigl(\mathrm{pr}_2(z_{|K|})\bigl)
  .
  &
  \label{4.3.1}
\end{eqnarray}

Разумеется,
(\ref{4.3.1}) можно, в частности, рассматривать при
$x=x^o$, $K= \ov{1,N}$, $\al\in \bbp$ и
$(z_i)_{i\in\ov{0,N}}\in \mathbf{Z}_\al$
(учитываем (\ref{3.4.15})).
В этом случае мы приходим к
(\ref{4.2.4}):
\bfn
  \label{4.3.2}
  \widehat{\mathfrak{B}}_\al[(z_i)_{i\in\ov{0,N}}|\,\ov{1,N}] = \mathfrak{B}_\al
  [(z_i)_{i\in\ov{0,N}}]
  \ \ \fa\al\in \bbp\ \ \fa (z_i)_{i\in\ov{0,N}}\in \mathbf{Z}_\al
  .
\efn

Учитывая свойства, отмеченные в \ref{sect:3.4}
(\ref{3.4.6}),
при
$x\in \bx$ и $K\in \mathfrak{N}$
имеем экстремальную задачу
\bfn
  \label{4.3.3}
  \widehat{\mathfrak{B}}_\al[(z_i)_{i\in\ov{0,|K|}}|\,K]
  \longrightarrow \min,\ \al\in (\mathbf{I}-\mathrm{bi})[K],\,
  (z_i)_{i\in\ov{0,|K|}}\in \mathbf{Z}(x,K,\al)
  ,
\efn
у которой ограничения совместны, а
пространство решений составляет непустое конечное множество.
Поэтому определено значение
(экстремум)
\bfn
  \label{4.3.4}
  \mathbf{v}(x,K) \df \min\limits_{\al\in (\mathbf{I}-
  \mathrm{bi})[K]}\ \min\limits_{(z_i)_{i\in\ov{0,|K|}}\in
  \mathbf{Z}(x,K,\al)}\widehat{\mathfrak{B}}_\al[(z_i)_{i\in\ov{0,|K|}}|\,K]
  \in [0,\infty[
\efn
и непустое (конечное)
множество оптимальных решений данной задачи
(\ref{4.3.3}).
В частности, задачу (\ref{4.3.3})
можно рассматривать при
$x=x^o$ и $K=\ov{1,N}$ соответственно,
для этого случая определено значение (\ref{4.3.4}),
то есть величина
$\mathbf{v}(x^o,\ov{1,N}).$
С учетом (\ref{3.4.8}), (\ref{3.4.13}), (\ref{4.2.6}) и (\ref{4.3.4})
получим равенство
\bfn
  \label{4.3.5}
  \mathbb{V} = \mathbf{v}(x^o,\ov{1,N})
  ,
\efn
показывающее, что задачи (\ref{4.3.3})
в своей совокупности образуют расширение базовой задачи (\ref{4.2.5}).
Полагаем, наконец, что
\bfn
  \label{4.3.6}
  \mathbf{v}(x,\emp) \df f(x)\ \ \fa x\in \mathbf{X}
  ,
\efn
определяя тем самым краевое условие в схеме
построения функции Беллмана
(с учетом (\ref{3.4.25}) и (\ref{4.3.6}) имеем,
в частности, что
$v(x,\emp) = \mathbf{v}(x,\emp)$
при $x \hm \in \mathbf{X}$).
Таким образом,
в новых (усложненных)
условиях определена функция
$$
  \mathbf{v}:\ \bx \times \cp(\ov{1,N}) \longrightarrow [0,\infty[
  .
$$

Иными словами, в силу (\ref{4.3.4}) и (\ref{4.3.6})
имеем, что
$\mathbf{v}\in \car_+[\bx \times \cp(\ov{1,N})].$
Из
\cite[теорема~5.1]{Cha3`}
вытекает, в частности, следующее

\begin{pred}
\label{p4.3.1}
Если
$x\in \bx$ и $K\in \mathfrak{N},$
то
$$
  \mathbf{v}(x,K) = \min\limits_{j\in\mathbf{I}(K)}\ \min\limits_{z\in
  \bbm_j}\bigl[\mathbf{c}\bigl(x,\mathrm{pr}_1(z),K\bigl) + c_j(z,K) +
  \mathbf{v}\bigl(\mathrm{pr}_2(z),K\sm\{j\}\bigl)\bigl]
  .
$$
\end{pred}

С учетом (\ref{4.3.5}) и предложения~\ref{p4.3.1} получаем
\cite[следствие~5.1]{Cha3`} очевидное

\begin{cor}
\label{p4.3.1-too}
Справедливо равенство
$$
  \mathbb{V} = \min\limits_{j\in\mathbf{I}(\ov{1,N})}\ \min\limits_{z\in
  \bbm_j}\bigl[\mathbf{c}\bigl(x^o,\mathrm{pr}_1(z),\ov{1,N}\bigl) +
  c_j(z,\ov{1,N}) + \mathbf{v}\bigl(\mathrm{pr}_2(z),\ov{1,N}\sm\{j\}\bigl)\bigl]
  .
$$
\end{cor}

Теперь, как и в более идейно простой задаче \ref{sect:3.5},
мы рассмотрим слои функции Беллмана,
следуя в части определения слоев пространства
позиций обозначениям упомянутого  раздела.
В этой связи полезно иметь
в виду также пояснения \ref{sect:3.6}.

Итак, учитывая (\ref{3.5.13}),
предположим, что, как и в
\cite[(6.12)]{Cha3`},
при
$s\in \ov{0,N}$ функция
\bfn
  \label{4.3.6`}
  \mathbf{v}_s\in \car_+ [D_s]
\efn
определяется следующим условием:
\bfn
  \label{4.3.7}
  \mathbf{v}_s(x,K) \df \mathbf{v}(x,K)\ \ \fa (x,K)\in D_s
  .
\efn

Таким образом, определены
(см. (\ref{4.3.6}), (\ref{4.3.7}))
следующие функции
$$
  \mathbf{v}_o\in \car_+[D_o],\,\mathbf{v}_1\in \car_+[D_1],\,\cdots,
  \mathbf{v}_N\in \car_+[D_N]
  ,
$$
роль которых подобна той роли,
которую играли функции (\ref{3.5.19})
в построениях 3 главы.
Отметим, что из (\ref{4.3.6}) и (\ref{4.3.7})
следует, что функция
\bfn
  \label{4.3.8}
  \mathbf{v}_o\in \car_+[D_o]
  ,
\efn
однозначно определяется правилом
\bfn
  \label{4.3.9}
  \mathbf{v}_o(x,\emp) = f(x)\ \ \fa x\in
  \bigcup\limits_{i\in\ov{1,N}\sm \mathbf{K}_1}\mathbf{M}_i
\efn
(мы используем здесь (\ref{3.5.9})).
Из (\ref{3.5.20}) и (\ref{4.3.9})
имеем равенство $v_o = \mathbf{v}_o,$
что и следовало ожидать,
поскольку <<терминальная>> функция $f$ не изменялась,
см. (\ref{4.2.1}).
Далее, из определений раздела 3.5 и (\ref{4.3.7})
следует, что
$$
  \mathbf{v}_N(x^o,\ov{1,N}) = \mathbf{v}(x^o,\ov{1,N})
  .
$$

С учетом (\ref{4.3.5}) получаем важное равенство
\bfn
  \label{4.3.10}
  \mathbf{v}_N(x^o,\ov{1,N})= \mathbb{V}
  .
\efn

Посредством (\ref{4.3.9}) и (\ref{4.3.10})
определены крайние слои функции Беллмана.
При построении
промежуточных слоев следует учитывать (\ref{3.6.13}):
если
$s\in\ov{1,N},\, (x,K)\in D_s,\,j\in \mathbf{I}(K)$ и
$z\in \bbm_j,$
то определено значение
$$
  \mathbf{v}_{s-1}\bigl(\mathrm{pr}_2(z),K\sm\{j\}\bigl)\in [0,\infty[
  .
$$

Поэтому при
$s\in\ov{1,N}$ и $(x,K)\in D_s$
у нас определены значения
$\mathbf{v}_s(x,K)\in [0,\infty[$ и
$$
  \min\limits_{j\in \mathbf{I}(K)}\ \min\limits_{z\in\bbm_j}
  \bigl[\mathbf{c}\bigl(x,\mathrm{pr}_1(z),K\bigl) +
  c_j(z,K) + \mathbf{v}_{s-1}\bigl(\mathrm{pr}_2(z),K\sm\{j\}\bigl)
  \bigl]\in [0,\infty[
  .
$$

С учетом предложения~\ref{p4.3.1}
получаем частный случай \cite[предложение~6.1]{Cha3`}:
если $s\in\ov{1,N},$
то преобразование
\bfn
  \label{4.3.11}
  \mathbf{v}_{s-1}\longrightarrow \mathbf{v}_s
\efn
определяется правилом
\begin{eqnarray}
  &\mathbf{v}_s(x,K) =  \min\limits_{j\in \mathbf{I}(K)}\
  \min\limits_{z\in\bbm_j}\bigl[\mathbf{c}\bigl(x,\mathrm{pr}_1(z),K\bigl) +
  c_j(z,K) +
  &\nonumber\\
  &+\mathbf{v}_{s-1}\bigl(\mathrm{pr}_2(z),K\sm\{j\}\bigl)\bigl]\ \
  \fa (x,K)\in  D_s
  .
  \label{4.3.12}
\end{eqnarray}

Итак, мы получаем рекуррентную процедуру построения функций (\ref{4.3.8}):
\bfn
  \label{4.3.13}
  \mathbf{v}_o\longrightarrow  \mathbf{v}_1\longrightarrow\cdots
  \longrightarrow \mathbf{v}_{N-1}\longrightarrow \mathbf{v}_N
  ;
\efn
при этом функция
$\mathbf{v}_o$ в (\ref{4.3.13}) известна
(см. (\ref{4.3.9})),
а каждый регулярный шаг данной процедуры реализуется посредством
(\ref{4.3.12}).

В заключение раздела совсем кратко рассмотрим построение оптимального
решения в виде соответствующей пары маршрут-трасса.
Однако сначала отметим,
что согласно (\ref{4.3.10}) и (\ref{4.3.12})
\bfn
  \label{4.3.14}
  \mathbb{V} = \min\limits_{j\in \mathbf{I}(\ov{1,N})}\
  \min\limits_{z\in\bbm_j}\bigl[\mathbf{c}\bigl(x^o,\mathrm{pr}_1(z),\ov{1,N}\bigl) +
  c_j(z,\ov{1,N}) + \mathbf{v}_{N-1}\bigl(\mathrm{pr}_2(z),\ov{1,N}\sm\{j\}\bigl)\bigl]
  .
\efn

Напомним, что согласно (\ref{3.6.13})
$$
  \bigl(\mathrm{pr}_2(z),\ov{1,N}\sm\{j\}\bigl)\in D_{N-1}
$$
при
$j\in\mathbf{I}(\ov{1,N})$ и $z\in\bbm_j,$
что позволяет в силу
(\ref{4.3.6`}) определять и использовать
$$
  \mathbf{v}_{N-1}\bigl(\mathrm{pr}_2(z),\ov{1,N}\sm\{j\}\bigl)\in[0,\infty[
  ,
$$
что и было сделано в (\ref{4.3.14}).
Полагаем, как и в \ref{sect:3.5}, что
$\mathbf{z}^{(o)}\df (x^o,x^o),$
получая, в частности
$\mathbf{z}^{(0)}\in \bbx\times \mathbf{X}.$

С учетом (\ref{4.3.14})
выбираем
$\mathbf{j}_1\in \mathbf{I}(\ov{1,N})$ и
$\mathbf{z}^{(1)}\in \bbm_{\mathbf{j}_1}$
из условия
\bfn
  \label{4.3.15}
  \mathbb{V}= \mathbf{c}\bigl(x^o,\mathrm{pr}_1(\mathbf{z}^{(1)}),\ov{1,N}\bigl) +
  c_{\mathbf{j}_1}(\mathbf{z}^{(1)},\ov{1,N}) + \mathbf{v}_{N-1}
  \bigl(\mathrm{pr}_2(\mathbf{z}^{(1)}),\ov{1,N}\sm\{\mathbf{j}_1\}\bigl)
\efn
(решая локальную экстремальную задачу, связанную с (\ref{4.3.14})),
получаем включение
\bfn
  \label{4.3.16}
  \bigl(\mathrm{pr}_2(\mathbf{z}^{(1)}),\ov{1,N}\sm\{\mathbf{j}_1\}\bigl)\in D_{N-1}
  .
\efn

Из (\ref{3.6.13}) и (\ref{4.3.16})
имеем, свойство
\begin{eqnarray}
  &\bigl(\mathrm{pr}_2(z),\ov{1,N}\sm\{\mathbf{j}_1;j\}\bigl) =
  \bigl(\mathrm{pr}_2(z),(\ov{1,N}\sm
  \{\mathbf{j}_1\})\sm\{j\}\bigl)\in D_{N-2}\ \
  &\nonumber\\
  &\fa j\in
  \mathbf{I}(\ov{1,N}\sm\{\mathbf{j}_1\})\
  \ \fa z\in \bbm_j
  .
\end{eqnarray}

Согласно (\ref{4.3.12}) и (\ref{4.3.16})
получаем теперь следующее равенство:
\begin{eqnarray}
  &\mathbf{v}_{N-1}\bigl(\mathrm{pr}_2(\mathbf{z}^{(1)}),\ov{1,N}\sm\{\mathbf{j}_1\}
  \bigl)=
  &\nonumber\\
  &=\min\limits_{j\in\mathbf{I}(\ov{1,N}\sm\{\mathbf{j}_1\})}\
  \min\limits_{z\in\bbm_j}\bigl[\mathbf{c}\bigl(\mathrm{pr}_2(\mathbf{z}^{(1)}),\mathrm{pr}_1(z),
  \ov{1,N}\sm \{\mathbf{j}_1\}\bigl) +
  &\nonumber\\
  &+ c_j(z,\ov{1,N}\sm\{\mathbf{j}_1\}) +
  \mathbf{v}_{N-2}\bigl(\mathrm{pr}_2(z),\ov{1,N}\sm\{\mathbf{j}_1;j\}\bigl)\bigl]
  .
  &
  \label{4.3.17}
\end{eqnarray}

С учетом (\ref{4.3.17}) выбираем
$\mathbf{j}_2\in \mathbf{I}(\ov{1,N}\sm
\{\mathbf{j}_1\})$ и $\mathbf{z}^{(2)}\in \bbm_{\mathbf{j}_2}$
из условия
\begin{eqnarray}
  &\mathbf{v}_{N-1}\bigl(\mathrm{pr}_2(\mathbf{z}^{(1)}),\ov{1,N}\sm\{\mathbf{j}_1\}
  \bigl) = \mathbf{c}\bigl(\mathrm{pr}_2(\mathbf{z}^{(1)}, \mathrm{pr}_1(\mathbf{z}^{(2)},\ov{1,N}\sm
  \{\mathbf{j}_1\}\bigl) +
  &\nonumber\\
  &+ c_{\mathbf{j}_2}(\mathbf{z}^{(2)}, \ov{1,N}\sm\{\mathbf{j}_1\}) +
  \mathbf{v}_{N-2}\bigl(\mathrm{pr}_2(\mathbf{z}^{(2)}),\ov{1,N}\sm\{\mathbf{j}_1;\mathbf{j}_2\}\bigl)
  .
  &
  \label{4.3.18}
\end{eqnarray}

Отметим, что согласно (\ref{3.6.13})
$$
  \bigl(\mathrm{pr}_2(\mathbf{z}^{(2)}),\ov{1,N}\sm\{\mathbf{j}_1;\mathbf{j}_2\}\bigl)\in D_{N-2}
  .
$$

Данное построение следует продолжать вплоть до исчерпывания $\ov{1,N}.$
При этом должны последовательно выбираться упорядоченные пары
$(\mathbf{j}_t,\mathbf{z}^{(t)})$,
$t\in \ov{1,N}$
в соответствии с требованиями, подобными
(\ref{4.3.15}) и (\ref{4.3.18}).
В результате будут сформированы кортежи
$$
  \eta\df (\mathbf{j}_t)_{t\in\ov{1,N}}\in \mathbf{A},
  \ \ (\mathbf{z}^{(t)})_{t\in \ov{0,N}}\in \mathbf{Z}_\eta
  ,
$$
образующие в совокупности ДР
$$
  \bigl(\eta, (\mathbf{z}^{(t)})_{t\in \ov{0,N}}\bigl)\in \widetilde{\mathbf{D}}
  ,
$$
для которого
\bfn
  \label{4.3.19}
  \mathbb{V}= \mathfrak{B}_\eta[(\mathbf{z}^{(t)})_{t\in \ov{0,N}}]
  .
\efn

\begin{zam}
\label{z4.3.1}
В связи с $(\ref{4.3.19})$ отметим,
что из $(\ref{4.3.15})$ и $(\ref{4.3.18})$
вытекает, что
\begin{eqnarray}
  & \mathbb{V}=\mathbf{c}\bigl(x^o,\mathrm{pr}_1(\mathbf{z}^{(1)}),\ov{1,N}\bigl) +
  \mathbf{c}\bigl(\mathrm{pr}_2(\mathbf{z}^{(1)}), \mathrm{pr}_1(\mathbf{z}^{(2)}),\ov{1,N}\sm
  \{\mathbf{j}_1\}) +
  &\nonumber\\
  &+  c_{\mathbf{j}_1}(\mathbf{z}^{(1)},\ov{1,N}) +
  c_{\mathbf{j}_2}(\mathbf{z}^{(2)},
  \ov{1,N}\sm\{\mathbf{j}_1 \}) +
  &\nonumber\\
  &+ \mathbf{v}_{N-2}\bigl(\mathrm{pr}_2(\mathbf{z}^{(2)}),\ov{1,N}\sm\{\mathbf{j}_1;\mathbf{j}_2\}\bigl)
  .
  &
  \label{4.3.20}
\end{eqnarray}

Из $(\ref{4.3.20})$
при $N=2$
непосредственно следует равенство $(\ref{4.3.19})$,
см.
определение $\mathbf{v}_o$ и $(\ref{4.3.9})$.
\hfill $\Box$
\end{zam}

Таким образом,
оптимальное ДР задачи (\ref{4.2.5}) построено.
Характерной его особенностью является
<<синхронное>> построение элементов маршрута и трассы
(см. (\ref{4.3.15}), (\ref{4.3.18}))
при последовательном решении локальных задач на основе ДП.
