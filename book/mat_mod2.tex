\documentclass[11pt,twoside,openright]{report}

% !TeX root = ..

\usepackage[utf8]{inputenc}
\usepackage[T2A]{fontenc}
\usepackage[english,russian]{babel}

% !TeX root = ..

\usepackage[active]{srcltx} %связь dvi and tex файлов
\usepackage{amsmath,amsfonts,amssymb,amscd,euscript}
\usepackage{mathtext}

%\renewcommand{\renfname}{{\sc\footnotesize Библиографический~~список}}
%\usepackage{amsfonts}
%\usepackage{amssymb}
\usepackage{latexsym}
% \usepackage{cite}

\usepackage{graphicx}
\usepackage{subfigure}
\usepackage{float}
\usepackage{flafter}

%\usepackage{my}
\usepackage{multicol}
\usepackage{multirow}
%\usepackage[nottoc,numbib]{tocbibind}
%\usepackage[dvips,unicode]{hyperref}
%%%\usepackage[pdftex, unicode]{hyperref}
%%%\usepackage{literat}
%%%\usepackage[dvips]{graphicx}

\usepackage{bbding}

% !TeX root = ../mat_mod2.tex

\usepackage{geometry}

\geometry{
  paperheight=240mm,
  paperwidth=170mm,
  top=12mm,
  bottom=20mm,
  left=18mm,
  right=17mm,
  nomarginpar,
}

\linespread{1.2}

  %   \textwidth 16,5cm
  %  \textheight 24cm
  %    \topmargin -0.5cm
  %    \headheight 0pt
  %    \headsep 0pt
  %    \oddsidemargin 0pt
  %    \evensidemargin 0pt
  %    \baselineskip 25pt

% !TeX root = ../mat_mod2.tex

\usepackage{indentfirst}
\setlength{\parindent}{2.5em}

% Висячие строки
\clubpenalty=10000
\widowpenalty=10000
\brokenpenalty=4991

% Короткое тире для ненумерованых списков
% ГОСТ 2.105-95, пункт 4.1.7, требует дефиса, но так лучше смотрится
\renewcommand{\labelitemi}{\normalfont\bfseries{--}}

% Грязный трюк для уменьшения кол-ва переносов в Оглавлении
% https://tex.stackexchange.com/a/217902
\makeatletter
  \renewcommand{\@tocrmarg}{\@pnumwidth plus1fil}
\makeatother

% Надписи рисунков и таблиц
\usepackage{caption}

\captionsetup{
  font=small,
  labelsep=period,
  justification=centering,
}

% !TeX root = ..

\usepackage{csquotes}
\usepackage[%
  parentracker=true,
  style=gost-numeric,
  defernumbers=true,
  movenames=false,
  maxnames=100,
  % isbn=false,
  % doi=false,
  sorting=none,
]{biblatex}

% \toggletrue{bbx:gostbibliography}

\usepackage{url}
\urlstyle{same}

\addbibresource{book.bib}

% !TeX root = ../mat_mod2.tex

%%% Русская традиция начертания греческих букв (греческие буквы вертикальные, через пакет upgreek)
\usepackage{upgreek}
\renewcommand{\epsilon}{\ensuremath{\upvarepsilon}}   %  русская традиция записи
\renewcommand{\phi}{\ensuremath{\upvarphi}}
%\renewcommand{\kappa}{\ensuremath{\varkappa}}
\renewcommand{\alpha}{\upalpha}
\renewcommand{\beta}{\upbeta}
\renewcommand{\gamma}{\upgamma}
\renewcommand{\delta}{\updelta}
\renewcommand{\varepsilon}{\upvarepsilon}
\renewcommand{\zeta}{\upzeta}
\renewcommand{\eta}{\upeta}
\renewcommand{\theta}{\uptheta}
\renewcommand{\vartheta}{\upvartheta}
\renewcommand{\iota}{\upiota}
\renewcommand{\kappa}{\upkappa}
\renewcommand{\lambda}{\uplambda}
\renewcommand{\mu}{\upmu}
\renewcommand{\nu}{\upnu}
\renewcommand{\xi}{\upxi}
\renewcommand{\pi}{\uppi}
\renewcommand{\varpi}{\upvarpi}
\renewcommand{\rho}{\uprho}
%\renewcommand{\varrho}{\upvarrho}
\renewcommand{\sigma}{\upsigma}
%\renewcommand{\varsigma}{\upvarsigma}
\renewcommand{\tau}{\uptau}
\renewcommand{\upsilon}{\upupsilon}
\renewcommand{\varphi}{\upvarphi}
\renewcommand{\chi}{\upchi}
\renewcommand{\psi}{\uppsi}
\renewcommand{\omega}{\upomega}

% !TeX root = ..

\renewcommand{\thechapter}{\arabic{chapter}}
\renewcommand{\thesection}{\S\,\thechapter.\arabic{section}}

%\renewcommand\section{\@startsection {section}{1}{\parindent}%
%{-3.5ex \@plus -1ex \@minus -.2ex}%
%{2.3ex \@plus.2ex}%
%{\normalfont\large\bfseries\rightskip\@flushglue}}%

%\renewcommand\subsection{\@startsection{subsection}{2}{\parindent}%
%{-3.25ex\@plus -1ex \@minus -.2ex}%
%{1.5ex \@plus .2ex}%
%{\normalfont\large\bfseries\rightskip\@flushglue}}

%\renewcommand\subsubsection{\@startsection{subsubsection}{3}{\\parindent}%
%{-3.25ex\@plus -1ex \@minus -.2ex}%
%{1.5ex \@plus .2ex}%
%{\normalfont\normalsize\bfseries\rightskip\@flushglue}}



%\newcommand{\sect}[1]{\section{#1}\setcounter{equation}{0}}

% !TeX root = ..

\usepackage{ntheorem}
\theoremseparator{.}

\newcounter{lem}
\newcounter{theo}
%\newcounter{sled}
\newcounter{cor}
\newcounter{zam}
\newcounter{usl}
\newcounter{pred}
\newcounter{opred}

\newtheorem{lem}{Лемма}[section]
\newtheorem{usl}{Условие}[section]
\newtheorem{theo}{Теорема}[section]
\newtheorem{pred}{Предложение}[section]
\newtheorem{zam}{Замечание}[section]
\newtheorem{cor}{Следствие}[section]
\newtheorem{opred}{Определение}[section]

\renewcommand{\thelem}{\arabic{chapter}.\arabic{section}.\arabic{lem}}
\renewcommand{\thetheo}{\arabic{chapter}.\arabic{section}.\arabic{theo}}
\renewcommand{\theusl}{\arabic{chapter}.\arabic{section}.\arabic{usl}}
\renewcommand{\thepred}{\arabic{chapter}.\arabic{section}.\arabic{pred}}
\renewcommand{\thezam}{\arabic{chapter}.\arabic{section}.\arabic{zam}}
\renewcommand{\thecor}{\arabic{chapter}.\arabic{section}.\arabic{cor}}
\renewcommand{\theequation}{\arabic{chapter}.\arabic{section}.\arabic{equation}}
\renewcommand{\theopred}{\arabic{chapter}.\arabic{section}.\arabic{opred}}

%\newtheorem{prop}{Proposition}[section]
%\newtheorem{cond}{Condition}[section]

% !TeX root = ../mat_mod2.tex

\newcommand{\bmp}{{\bf p}}
\newcommand{\bmm}{{\bf m}}
\newcommand{\bmi}{{\bf i}}
\newcommand{\bmj}{{\bf j}}
\newcommand{\bmk}{{\bf k}}
\newcommand{\bml}{{\bf l}}
\newcommand{\bmn}{{\bf n}}
\newcommand{\bmr}{{\bf r}}
\newcommand{\bmq}{{\bf q}}
\newcommand{\bmb}{{\bf b}}
\newcommand{\st}[1]{\stackrel{-1}{#1}}
\newcommand{\cpl}{\bigcap\limits}

%\newcommand{\sml}{\sum\limits}
\newcommand{\cul}{\bigcup\limits}
\newcommand{\nc}[1]{(%\ref
{#1})}
\newcommand{\rc}[1]{(\ref{#1})}
%\newcommand{\rn}[1]{(\ref{\mbox{Д.#1}})}
\newcommand{\n}[1]{\eqno{(\mbox{Д.#1})}}
\newcommand{\e}{\emptyset}
\newcommand{\x}{\times}
\newcommand{\be}{\beta}
\newcommand{\sg}{\sigma}
\newcommand{\bcu}{\bigcup}
\newcommand{\bca}{\bigcap}
\newcommand{\fs}[1]{\mbox{$\forall_{#1}\,S[#1]$}}
\newcommand{\es}[1]{\mbox{$\exists_{#1}\,S[#1]$}}
\newcommand{\ese}[1]{\mbox{$\exists!_{#1}\,S[#1]$}}
\newcommand{\fas}[1]{\mbox{$\forall_{#1}\,S[#1 \ne \emptyset]$}}
\newcommand{\sne}[1]{\mbox{$S[#1 \ne \emptyset]$}}
\newcommand{\exs}[1]{\mbox{$\exists_{#1}\,S[#1 \ne \emptyset]$}}
\newcommand{\exse}[1]{\mbox{$\exists!_{#1}\,S[#1 \ne \emptyset]$}}
\newcommand{\fo}[1]{\mbox{$\forall_{#1}\,(ob)[#1]$}}
\newcommand{\eo}[1]{\mbox{$\exists_{#1}\,(ob)[#1]$}}
\newcommand{\eeo}[1]{\mbox{$\exists!_{#1}\,(ob)[#1]$}}
\newcommand{\fao}[1]{\mbox{$\forall_{#1}\,(ob)[#1 \ne \emptyset]$}}
\newcommand{\sneo}[1]{\mbox{$(ob)[#1 \ne \emptyset]$}}
\newcommand{\exo}[1]{\mbox{$\exists_{#1}\,(ob)[#1 \ne \emptyset]$}}
\newcommand{\exeo}[1]{\mbox{$\exists!_{#1}\,(ob)[#1 \ne \emptyset]$}}
\newcommand{\ff}[1]{\forall_{#1}\,(Fam)[#1]}
\newcommand{\ef}[1]{\exists_{#1}\,(Fam)[#1]}
\newcommand{\cfe}[1]{\exists!_{#1}\,(Fam)[#1]}
\newcommand{\ffn}[1]{\forall_{#1}\,(Fam)[#1\nee]}
\newcommand{\cfn}[1]{\exists_{#1}\,(Fam)[#1\nee]}
\newcommand{\cfen}[1]{\exists!_{#1}\,(Fam)[#1\nee]}
\newcommand{\fpr}[1]{\forall_{#1}\,(pair)[#1]}
\newcommand{\eepr}[1]{\exists!_{#1}\,(pair)[#1]}
\newcommand{\epr}[1]{\exists_{#1}\,  (pair)[#1]}
\newcommand{\ffu}[1]{\forall_{#1}\,(FUNC)[#1]}
\newcommand{\eefu}[1]{\exists!_{#1}\,(FUNC)[#1]}
\newcommand{\efu}[1]{\exists_{#1}\,(FUNC)[#1]}
\newcommand{\ffun}[1]{\forall_{#1}\,(FUNC)[#1\nee]}
\newcommand{\eefun}[1]{\exists!_{#1}\,(FUNC)[#1\nee]}
\newcommand{\efun}[1]{\exists_{#1}\,(FUNC)[#1\nee]}
\newcommand{\ffuo}[1]{\forall_{#1}\,(FUNC)_0 [#1]}
\newcommand{\eefuo}[1]{\exists!_{#1}\,(FUNC)_0 [#1]}
\newcommand{\efuo}[1]{\exists_{#1}\,(FUNC)_0 [#1]}
\newcommand{\frc}[1]{\frac{1}{#1}}
\newcommand{\spo}[1]{S\bigg[\bigcup_{#1\in{\cal #1}}\,#1\bigg]}
\newcommand{\spp}[1]{S\bigg[\bigcap\limits_{#1\in{\cal #1}}\,#1\bigg]}
\newcommand{\bI}{{\bf I}} \newcommand{\lb}{\linebreak}
\newcommand{\nee}{\ne\emptyset}
%\renewcommand{\thesection}{arabic{section}.}
%\renewcommand{\theequation}{\thesection.\arabic{equation}}
\newcommand{\beg}[1]{\begin{equation}\label{#1}}
\newcommand{\bfn}{\begin{equation}}
\newcommand{\efn}{\end{equation}}
\newcommand{\df}{\stackrel{\triangle}{=}}
\newcommand{\ts}{\thicksim}
\newcommand{\ov}{\overline}
\newcommand{\ovr}{\overrightarrow}
\newcommand{\bc}{\begin{center}}
\newcommand{\ec}{\end{center}}
\newcommand{\modd}{\raisebox{-2pt}{\RectangleThin}}

\newcommand{\li}[1]{(#1-\mathrm{LIM})}

\newcommand{\Dom}{\mathrm{Dom}}
%\newcommand{\Im}{\mathrm{Im}}
\newcommand{\TL}{\mbox{\bf{$\!\!$.}}}


\newcommand{\eps}{\varepsilon}
\newcommand{\Om}{\Omega}
\newcommand{\om}{\omega}
\newcommand{\La}{\Lambda}
\newcommand{\la}{\lambda}
\newcommand{\ti}{\tilde}
\newcommand{\al}{\alpha}
\newcommand{\su}{\subset}
\newcommand{\sm}{\setminus}
\newcommand{\sml}{\sum\limits}
\newcommand{\fa}{\forall}
\newcommand{\ex}{\exists}
\newcommand{\vp}{\varphi}
\newcommand{\vth}{\vartheta}
\newcommand{\vto}{\vartheta_0}
\newcommand{\ci}{{\cal I}}
\newcommand{\car}{{\cal R}}
\newcommand{\cj}{{\cal J}}
\newcommand{\cn}{{\cal N}}
\newcommand{\cp}{{\cal P}}
\newcommand{\cx}{{\cal X}}
\newcommand{\ca}{{\cal A}}
\newcommand{\cc}{{\cal C}}
\newcommand{\cw}{{\cal W}}
\newcommand{\co}{{\cal O}}
\newcommand{\cs}{{\cal S}}
\newcommand{\Ch}{{\cal H}}
\newcommand{\cv}{{\cal V}}
\newcommand{\cq}{{\cal Q}}
\newcommand{\cy}{{\cal Y}}
\newcommand{\cu}{{\cal U}}
\newcommand{\cf}{{\cal F}}
\newcommand{\cg}{{\cal G}}
\newcommand{\ce}{{\cal E}}
\newcommand{\cd}{{\cal D}}
\newcommand{\cb}{{\cal B}}
\newcommand{\cl}{{\cal L}}
\newcommand{\clm}{{\cal M}}
\newcommand{\clr}{{\cal R}}
\newcommand{\ck}{{\cal K}}
\newcommand{\ct}{{\cal T}}
\newcommand{\cz}{{\cal Z}}
\newcommand{\gk}{\mathfrak K}
\newcommand{\gz}{{\mathfrak Z}}
\newcommand{\gc}{{\mathfrak C}}
\newcommand{\gf}{{\mathfrak F}}
\newcommand{\gx}{\mathfrak X}
\newcommand{\ggh}{\mathfrak G}
\newcommand{\gn}{\mathfrak N}
\newcommand{\gb}{\mathfrak B}
\newcommand{\gd}{\mathfrak D}
\newcommand{\gj}{\mathfrak J}
\newcommand{\gm}{\mathfrak M}
\newcommand{\trl}{\tau_\bbr}
\newcommand{\bfi}{{\bf I}}
\newcommand{\bl}{{\bf L}}
\newcommand{\bn}{{\bf N}}
\newcommand{\bx}{{\bf X}}
\newcommand{\bh}{{\bf H}}
\newcommand{\by}{{\bf Y}}
\newcommand{\bd}{{\bf D}}
\newcommand{\br}{{\bf R}}
\newcommand{\bfc}{{\bf C}}
\newcommand{\ba}{{\bf A}}
\newcommand{\bu}{{\bf U}}
\newcommand{\bt}{{\bf T}}
\newcommand{\bq}{{\bf Q}}
\newcommand{\bs}{{\bf S}}
\newcommand{\bff}{{\bf F}}
\newcommand{\bz}{{\bf Z}}
\newcommand{\bb}{{\bf B}}
\newcommand{\bba}{{\mathbb A}}
\newcommand{\bbd}{{\mathbb D}}
\newcommand{\bbj}{{\mathbb J}}
\newcommand{\bbu}{{\mathbb U}}
\newcommand{\bbs}{{\mathbb S}}
\newcommand{\bbg}{{\mathbb G}}
\newcommand{\bbz}{{\mathbb Z}}
\newcommand{\bbl}{{\mathbb L}}
\newcommand{\bbc}{{\mathbb C}}
\newcommand{\bby}{{\mathbb Y}}
\newcommand{\bbw}{{\mathbb W}}
\newcommand{\bbq}{{\mathbb Q}}
\newcommand{\bbb}{{\mathbb B}}
\newcommand{\bbk}{{\mathbb K}}
\newcommand{\bbt}{{\mathbb T}}
\newcommand{\bbn}{{\mathbb N}}
\newcommand{\bbm}{{\mathbb M}}
\newcommand{\bbh}{{\mathbb H}}
\newcommand{\bbr}{{\mathbb R}}
\newcommand{\bbx}{{\mathbb X}}
\newcommand{\bbf}{{\mathbb F}}
\newcommand{\bbo}{{\mathbb O}}
\newcommand{\bk}{{\bf K}}
\newcommand{\bbp}{{\mathbb P}}
\newcommand{\bbv}{{\mathbb V}}
\newcommand{\bm}{{\bf M}}
%\newcommand{\m}{{\bf m}_\Om^0}

\newcommand{\mg}{\mbox{\boldmath$\Gamma$}}
\newcommand{\mbr}{\mbox{\boldmath$\bbr$}}
\newcommand{\mo}{\mbox{\boldmath$\Omega$}}
\newcommand{\mi}{\mbox{\boldmath$\imath$}}
\newcommand{\mt}{\mbox{\boldmath$\tau$}}
\newcommand{\mr}{\mbox{\boldmath$\rho$}}
\newcommand{\zc}{{\mathbf c}}
\newcommand{\nn}{{\mathbf n}}
\newcommand{\emp}{\varnothing}

% !TeX root = ../mat_mod2.tex

\usepackage{sectsty}

\allsectionsfont{\raggedright}
\chapterfont{\Large}
\partfont{\centering\thispagestyle{empty}}


\graphicspath{{media/}}

%\nofiles
\begin{document}

% !TeX root = ..

\thispagestyle{empty}

\setcounter{page}{1}
\begin{center}
{\footnotesize
Министерство науки и высшего образования РФ

Уральский федеральный университет
имени первого Президента России
Б. Н. Ельцина
}
\includegraphics[width=4cm]{urfu.pdf}

\vspace{0pt plus2fill}
А. А. Петунин,
А. Г. Ченцов,
П. А. Ченцов

\vspace{0pt plus1fill}
{\bf
Оптимальная маршрутизация инструмента машин фигурной
листовой резки с~числовым
программным управлением.\\
Математические модели и~алгоритмы}

\vspace{3cm}
Монография

\vspace{0pt plus6fill}
Екатеринбург

Издательство Уральского университета

2020
\end{center}
\newpage

% !TeX root = ../mat_mod2.tex

\textit{
\centering
Результаты исследований получены
при выполнении проекта создания и
развития научной лаборатории
<<Лаборатория оптимального раскроя промышленных материалов
и оптимальных маршрутных технологий>>
в рамках Программы повышения конкурентоспособности
Уральского федерального университета 5-100-2020
и при поддержке Российского Фонда Фундаментальных Исследований
(гранты №17-08-01385, №20-08-00873)
}


\newpage
\tableofcontents

% !TeX root = ..

\chapter*{ВВЕДЕНИЕ}
\addcontentsline{toc}{chapter}{ВВЕДЕНИЕ}
%\section{ВВЕДЕНИЕ}

В различных технических приложениях
возникают задачи моделирования маршрута и маршрутной оптимизации.
Большая часть таких задач обычно рассматривается современными
исследователями через призму различных комбинаторных моделей дискретной оптимизации.
Вместе с тем, при моделировании маршрута в реальных технических задачах
числовые значения некоторых параметров маршрута могут выбираться
из множества допустимых величин, имеющего континуальную мощность,
что усложняет математические модели оптимальной маршрутизации
в сравнении с классическими маршрутными постановками типа задачи коммивояжера (ЗК).
Кроме того, на множество допустимых решений могут накладываться дополнительные ограничения,
вызванные техническими особенностями задачи, например,
технологическими требованиями к маршруту,
порождаемыми спецификой конкретной предметной области.
В результате возникают новые математические постановки,
не охватываемые существующими методами решения.
К числу такого рода сложных задач относится
проблема оптимальной маршрутизации инструмента
машин фигурной листовой резки с числовым программным управлением (ЧПУ).
Эта проблема возникает на этапе разработки управляющих программ для машины с ЧПУ,
которые задают траекторию перемещения инструмента и ряд технологических команд,
определяющих параметры резки листового материала
для получения из него заготовок известных форм и размеров.
Необходимые данные для моделирования маршрута инструмента
машины с ЧПУ определяет информация о раскройных картах,
которые разрабатываются на этапе проектирования раскроя
и порождают известные задачи оптимизации раскроя листового материала.
С точки зрения геометрической оптимизации задачи раскроя относятся
к классу задач раскроя -- упаковки
(\textit{Cutting \& Packing}),
для которых, также как и для маршрутных оптимизационных проблем,
не известны алгоритмы решения полиномиальной сложности.
В данной работе задачи раскроя не рассматриваются.
Основное направление исследования в настоящей монографии
связано с моделированием маршрута инструмента
машин фигурной листовой резки с ЧПУ и
проблемой его оптимизации по временным и стоимостным параметрам.

В исходной задаче требуется осуществить последовательное
посещение всех контуров с целью осуществления резки по эквидистантам,
представляющим собой замкнутые кривые
(обсуждаются также и более сложные типы резки);
точки, определяющие начало и окончание реза,
могут при этом назначаться произвольно.
В интересах построения конкретных решений приходится,
однако, использовать дискретизацию эквидистант
и некоторые дополнительные преобразования последних в непустые конечные множества — мегаполисы,
что и делается в настоящей монографии
(см.~в этой связи \cite{intro01,intro02}).

Если рассматривать сформулированное научное направление в его полной общности,
то приходится признать, что адекватной математической теории здесь не разработано.
Имеются отдельные направления, среди которых особо отметим проблему полиномиальной разрешимости
для отдельных классов оптимизационных задач,
которые могут использоваться в качестве подзадач рассматриваемой проблемы.
Известные результаты, которые получены в последние годы в предметных областях,
связанных с разработкой алгоритмов дискретной оптимизации
и исследованием проблемы полиномиальной разрешимости,
при всей своей значимости не охватывают проблемы <<диапазонных>>
(в смысле размерности) задач и особенно задач,
осложненных ограничениями.
В монографии авторы исследуют вопросы разработки
теоретических и методологических основ решения проблемы
оптимальной маршрутизации инструмента для машин фигурной листовой резки с ЧПУ,
включая разработку адекватных математических моделей
и алгоритмов решения для исследуемой прикладной задачи.
Результаты работы могут быть использованы и для решения
других прикладных задач,
описываемых предложенными в монографии математическими моделями.

Монография структурно состоит из двух частей и пяти глав.

В первой главе рассмотрены основные понятия
фигурной листовой резки на машинах с ЧПУ,
формулируется содержательная постановка исследуемой проблемы,
приводятся общие постановки и классификация
возникающих оптимизационных маршрутных задач.
Здесь же приведена <<первичная>> математическая формализация
рассматриваемой проблемы и описана дискретная модель
некоторых сформулированных ранее оптимизационных задач,
основанная на использовании модели мегаполисов.

Во второй главе рассматриваются отдельные
практические аспекты оптимизации траектории
инструмента для машин листовой резки с ЧПУ:
описываются способы уменьшения термических деформаций
материала при оптимальной маршрутизации инструмента,
исследуется проблема точного вычисления целевых функций
на примере машины лазерной резки
{\it ByStar 3015}
и эффективность применения специальных техник резки
в сравнении со стандартной техникой <<резки по контуру>>.

Третья глава содержит описание математических моделей и методов,
используемых при решении задачи последовательного обхода
мегаполисов с условиями предшествования.

В четвертой главе исследованы задачи маршрутизации
с ограничениями и усложненными функциями стоимости.
Рассматриваются вопросы, связанные с
локальным улучшением эвристических решений.

В пятой главе приводятся описание разработанных авторами
алгоритмов для решения задач маршрутизации,
а также результаты вычислительных экспериментов,
содержащих данные решения некоторых практических
задач оптимизации маршрута инструмента для машин
фигурной листовой резки с ЧПУ.

Две первые главы образуют в своей совокупности
первую часть настоящей работы,
непосредственно связанную с решением инженерных задач,
относящихся к листовой резке на машинах с ЧПУ.
Здесь обсуждаются конкретные варианты весьма общей постановки,
указываются характерные
особенности
и обозначаются
на идейном уровне основные элементы этой общей постановки.
Особую значимость приобретает обсуждение
различных вариантов осуществления резки,
включая многие подробности,
важные в инженерном отношении,
а также характерные ограничения.
Последние существенно влияют на математическую постановку;
учет некоторых ограничений оказывается весьма затруднительным.

В первой главе подробно обсуждается
стандартная техника резки (резка по замкнутому контуру),
которая, как представляется,
более близка к известным математическим постановкам задач
о последовательном обходе мегаполисов с условиями предшествования
(данное обстоятельство существенно используется во второй части работы).
Упомянутые условия играют важную роль как на этапе инженерной постановки,
так и на этапе математического исследования.
Их конкретный вариант состоит (в данной задаче)
в необходимости более раннего вырезания внутренних контуров
деталей и <<внутренних>> деталей, то есть деталей,
располагаемых (после раскроя) внутри других (объемлющих) деталей,
что соответствует размещению по схеме <<матрешки>>.
Само решение задачи является многоэтапным,
и упомянутые
условия предшествования касаются всей совокупности упомянутых этапов.
В то же время сам характер этих условий оказывается
до некоторой степени удобным для их последующего учета
на этапе общей постановки;
они касаются выбора очередности достаточно
крупных фрагментов решения и имеют комбинаторный характер.

В первой части монографии обсуждаются также
различные варианты нестандартной техники резки
(цепная резка, резка с перемычками, резка <<змейкой>> и др.).
Вводятся важные понятия сегмента резки и базового сегмента резки,
определяющие общий взгляд на проблему классификации вариантов резки
(резка по замкнутому контуру, мультисегментная и мультиконтурная резки).
Понятия сегмента резки и базового сегмента являются
по сути объединяющими различные варианты резки в естественные классы,
допускающие исследование соответствующих конкретных вариантов
с единых позиций и существенно расширяющие имеющуюся
классификацию задач маршрутизации инструмента
для машин листовой резки с ЧПУ.

Особое внимание уделено в монографии вопросам,
связанным с формализацией и математической постановкой
рассматриваемых инженерных задач.
Частично эти вопросы затрагиваются в первой части,
где проблемы формализации обсуждаются с позиций инженерного исследования;
решения трактуются как маршруты резки,
являющиеся объектами выбора исследователя с целью
максимального улучшения (совокупного)
результата при соблюдении комплекса разнообразных ограничений.
Такой подход позволяет сформулировать определенные ориентиры,
которые особенно полезны при разработке эффективных эвристических алгоритмов.
Само же применение эвристических методов для решения
практических задач представляется неизбежным.
Здесь же рассматривается задача точного вычисления целевых функций,
в рамках решения которой исследуются практические
вопросы определения зависимости фактической скорости резки
от числа кадров управляющей программы
(на примере машины лазерной резки
{\it ByStar 3015}),
описывается методика определения параметров
для целевой функции стоимости лазерной резки
с вычислением стоимостных параметров этой функции
для различных марок и толщин листовых материалов.

В результате вышеупомянутой и,
по смыслу, <<первичной>> формализации проблемы,
проведенной в первых двух главах монографии,
мы получаем дискретные задачи нелинейного программирования
большой размерности, представляющие в своей исходной постановке
серьезные затруднения как для качественного исследования,
так и для процедур поиска конкретных решений.
Определенные возможности для теоретического исследования
подобных задач открывает,
как представляется, весьма общий подход,
последовательно развиваемый во второй части
(третья, четвертая и пятая главы монографии)
и связанный с применением аппарата широко понимаемого
динамического программирования (ДП),
реализуемого в условиях ограничений исходной задачи.
Данный подход, естественно связываемый с идеями
Р. Беллмана и широко используемый, в частности,
в современной теории управления, требует, однако,
определенного переосмысливания самой постановочной части.
Так, выбор решения (маршрут резки в первой части)
полезно трактовать как выбор пары маршрут -- трасса,
где маршрут связывается уже с перестановкой индексов,
используемых для нумерации контуров вырезаемых деталей,
а трасса имеет смысл, подобный маршруту резки первой части.
При этом возникает определенная иерархия:
маршрут (в виде перестановки индексов)
определяет пучок согласованных с ним и
потому подчиненных ему трасс или траекторий,
которые уже перестановками, вообще говоря, не являются.
Маршрут позволяет занумеровать контуры, подлежащие резке,
а трасса определяет конкретный вариант их посещения
(точнее, посещения эквидистант, соответствующих данным контурам).
Имеется целый ряд обстоятельств, мотивирующих упомянутую иерархию.
Сейчас отметим только одно:
условия предшествования относятся,
строго говоря, к способу нумерации контуров.
Таким образом, эти условия порождают ограничения
именно в выборе перестановки индексов,
то есть в выборе маршрута, понимаемого в традиционном для ЗК смысле.
Это важное обстоятельство позволяет затем
использовать условия предшествования в <<положительном>>
направлении в смысле снижения сложности вычислений
(имеется в виду процедура на основе ДП).

Итак, во второй части монографии само понятие
решения определенным образом структурируется;
выделяются две компоненты:
маршрут
(как перестановка индексов)
и трасса,
или траектория.
Такая логика естественна с точки зрения теории управления,
элементы которой (имеются в виду задачи управления с дискретным временем)
используются в построениях второй части монографии.
При этом реализация трассы осуществляется в пределах пучка,
однозначно определяемого маршрутом.
Критерий качества предполагается аддитивным.
Это означает, что для каждого конкретного решения
значение критерия получается суммированием стоимостей,
характеризующих все этапы перемещений,
связанных с реализацией упомянутого
решения в виде пары маршрут -- трасса.

Для задач, связанных с листовой резкой,
исключительно важным является учет ограничений,
связанных с тепловыми деформациями материала
и порождаемыми этими деформациями эвристическими правилами
(т.~н. <<жесткостью>> листа и деталей),
сформулированными в первой главе монографии.
Характерной особенностью таких ограничений
является то, что все они формируются по мере
развития процесса резки и, по большому счету,
зависят от истории последнего, что определяет
принципиальное отличие рассматриваемых задач от
оптимизационных задач с фиксированными ограничениями.
Здесь опять-таки оказывается уместным двухуровневое представление решения,
поскольку целый ряд данных <<динамических>> ограничений удается представить
в терминах зависимостей от маршрута,
определяемого в виде перестановки индексов.

Учет динамических ограничений
осуществляется в настоящей монографии
посредством введения специальных функций стоимости,
которые объективно играют роль штрафов.
При этом, однако, возникают функции стоимости,
включающие зависимость от списка заданий,
уже выполненных на момент соответствующего перемещения.
Данная особенность существенно осложняет конструкции на основе ДП;
в этой связи в третьей главе рассматривается случай,
когда вышеупомянутая зависимость от списка заданий отсутствует,
что позволяет привлечь для целей качественного исследования
более простую и понятную версию ДП.

Более общий случай,
когда зависимость функций стоимости от
списка заданий уже допускается,
рассматривается в четвертой главе.
С точки зрения применения аппарата ДП
оказывается более удобным использовать
при формализации задачи функции стоимости,
допускающие зависимость от списка еще не выполненных заданий.
Кроме того, по постановке допускаются условия предшествования,
которые в задачах, связанных с листовой резкой,
имеют ясный содержательный смысл:
внутренние контуры деталей должны вырезаться раньше внешних;
в случае расположения одних деталей <<внутри>> других
резка <<внутренних>> деталей должна осуществляться раньше,
чем резка <<внешних>>.

Для вышеупомянутой общей постановки
в рамках концепции двухуровневого решения
(определяемого всякий раз в виде пары маршрут -- трасса)
осуществляется построение специального расширения исходной задачи.
Потребность в данном расширении связана с учетом условий предшествования,
которые порождают <<неудобные>> ограничения на маршрут в целом.
Эти ограничения удается, однако,
эквивалентным образом преобразовать к условиям,
определяемым некоторым естественным правилом вычеркивания заданий из списка.
Итак, допустимость по предшествованию эквивалентным образом
заменяется допустимостью по вычеркиванию.
Последняя более удобна для целей применения аппарата ДП,
поскольку связывается с условиями на отдельные этапы процесса перемещений.
Одним словом,
такая допустимость нужным образом локализуется,
что и позволяет затем задействовать конструкции
широко понимаемого ДП и получить затем уравнение Беллмана.

В связи с трудностями вычислительной реализации
на основе этого уравнения конструируется
система преобразования так называемых слоев функции Беллмана.
Речь идет о том, чтобы при условиях предшествования
(а это типичный случай в рассматриваемом классе задач)
ограничиться частичным построением массива функции Беллмана,
а точнее, системы ее слоев.
Последние, в свою очередь, определяются
соответствующими слоями пространства позиций,
в определении которых задействуются
так называемые существенные списки заданий.

Разумеется, даже при использовании 
(вышеупомянутым способом) усеченного массива значений
функции Беллмана практическое использование
(оптимальной) процедуры на основе ДП возможно лишь
в задачах умеренной размерности.
В то же время представляют интерес методы
локального улучшения маршрутных решений
посредством применения оптимизирующих вставок,
при построении которых удается уже задействовать схему на основе ДП.

Важно отметить, что само применение оптимизирующих вставок
в задаче маршрутизации с условиями предшествования и стоимостями,
зависящими от списка заданий,
потребовало серьезного теоретического обоснования,
которое приведено в четвертой главе.

В целях более глубокого воздействия на
исходное эвристическое решение
(имеется в виду решение задачи достаточно большой размерности)
предлагается использовать итерационные процедуры с варьированием начала вставки.
Конкретные варианты построения таких процедур
приведены в пятой главе,
в которой также содержатся соответствующие результаты
вычислительного эксперимента.

В целом использование аппарата ДП
на уровне вставок,
в том числе применение режима итераций,
представляется реальной возможностью включения упомянутого
(теоретического) аппарата в процесс решения маршрутных задач,
имеющих практический интерес.
Здесь особенно важным кажется
разработка методов и алгоритмов решения задач с ограничениями разных типов.
В частности, это касается динамических ограничений,
которые складываются по мере развития процесса.
Данный тип ограничений <<обрабатывается>> в настоящей монографии
(это уже отмечалось ранее)
посредством введения функций стоимости с зависимостью от списка заданий,
что требует конструирования таких функций
и насчитывания соответствующих массивов их значений.
Последнее существенно осложняет вычисления
(особенно при использовании ДП).
Поэтому представляется важной разработка
эффективных эвристических алгоритмов,
для которых предварительное глобальное построение
вышеупомянутых массивов значений функций стоимости не делается;
вместо этого осуществляется построение локальных массивов,
реализующихся по мере развития процесса.
Один из таких алгоритмов приведен в пятой главе.

Оценивая содержание монографии,
можно отметить основательную инженерную и математическую проработку материала.
Обсуждаются различные варианты фигурной резки и
намечены обобщения известных понятий,
позволяющие применять специальные математические методы.
В частности, предлагается при описании процесса резки
использовать естественную модель мегаполисов,
в рамках которой на каждом этапе допускается
возможность выбора точки врезки из заданной
и достаточно представительной совокупности.
Это позволяет, с одной стороны,
свести трудно
решаемую непре\-рывно-дис\-крет\-ную
задачу нелинейного программирования
к задаче дискретной оптимизации, а с другой --
существенно расширить возможности получения оптимальной
(или близкой к ней)
управляющей программы резки в сравнении
с тем случаем,
когда точка врезки фиксирована для каждого контура.

Отдельного обсуждения заслуживает вопрос о применении ДП.
Прежде всего, следует отметить, что ДП в изложении,
принятом в настоящей монографии,
является теоретическим методом.
На его основе, конечно, может быть построен алгоритм,
применимый для построения оптимальных решений
в задачах малой размерности. Но все же это уже следствие.
Роль ДП как общего метода решения экстремальных задач
очень велика.
Но, пожалуй, в наибольшей степени эта роль проявляется
в задачах теории управления, что связано прежде всего
с работами Р. Беллмана.
В настоящей монографии конструкции широко понимаемого ДП
идейно соответствуют взгляду на данный метод,
принятому в теории управления.
В частности, значительное место занимает
получение уравнения Беллмана и следствий этого уравнения,
связанных с использованием условий предшествования
в положительном направлении.
В то же время вывод уравнения Беллмана
опирается на специальную процедуру расширения исходной задачи,
в основе которой находится эквивалентное преобразование системы ограничений.
Итак, широко понимаемое ДП является
(в настоящей монографии)
прежде всего теоретическим методом,
позволяющим изучать структуру очень сложных задач маршрутизации.
Одним словом, ДП <<справляется>> с разнообразными ограничениями,
проявляя при этом большую универсальность.
Так, например, данный метод без каких-либо
существенных изменений идейного характера
удается использовать при неаддитивном агрегировании затрат и,
в частности, в маршрутных задачах <<на узкие места>>.

В то же время в дискретной оптимизации ДП
нередко воспринимается только как алгоритм;
здесь имеется в виду прежде всего применение
ДП для решения ЗК
(в англоязычной редакции -- \textit{TSP}).
Вполне естественным является тот факт,
что в такой <<простой>> по постановке задаче,
как ЗК, алгоритм на основе ДП нередко проигрывает
другим алгоритмам
(например, методу ветвей и границ).
Это и неудивительно в силу определенной <<всеядности>> ДП.
Однако вопрос о месте ДП в решении
сложных задач маршрутизации с ограничениями
все же стоит достаточно остро.
В настоящей монографии наряду с организацией оптимизирующих вставок
с применением ДП
развивается также следующий взгляд
на упомянутую проблему.
Речь идет о тестировании эвристик на
задачах маршрутизации умеренной размерности,
но при тех же ограничениях,
что и реальная исходная постановка
(таким образом, реализуется своеобразная
<<дрессировка>> эвристик; при этом,
конечно, требуется достаточно представительная выборка решенных задач).
Итак, принимая точку зрения о неизбежности эвристик
в маршрутных задачах большой размерности,
мы с помощью ДП стараемся <<наладить>>
сравнение эвристик на выборках задач умеренной размерности.

Сейчас мы совсем кратко коснемся имеющихся источников,
обозначая тем самым сложившиеся направления исследований.

В связи с конкретной задачей оптимизации управления
режущим инструментом машин листовой резки с ЧПУ
отметим работы
\cite{intro03,intro04,intro05,intro06,intro07,intro08,intro09,Cha10`,intro11,intro12}
и обзор \cite{intro13}.
В целом ряде российских и зарубежных исследований предполагалось,
что точка врезки инструмента в листовой материал
выбрана заранее для каждого вырезаемого контура.
Это позволяет использовать модель ЗК,
но снижает практическую ценность,
поскольку уже на постановочном уровне
исключает из рассмотрения основную часть
полезных вариантов решения.
Еще одна группа новых зарубежных публикаций
описывает алгоритмы решения задач,
в которых точки врезки для каждого контура
выбираются из некоторого конечного множества
(что было предложено авторами монографии ранее),
но применяется только стандартная техника резки
(резка по замкнутому контуру – задача
\textit{GTSP}).
В качестве математической модели оптимизационной задачи
в этом случае используется модель обобщенной задачи коммивояжера.
Более общий случай -- задача резки с конечным набором точек врезки:
резка может начаться только в одной из заранее заданных точек на контуре,
однако контур может быть вырезан за несколько подходов,
по частям.
Некоторые алгоритмы для решения частных случаев
этой задачи описаны, например,
в работах~\cite{intro12,intro14}.
Следует отметить, что для задачи т. н. <<произвольной резки>>,
когда не накладывается никаких ограничений
на выбор точек начала и конца резки,
а также на последовательность резки контуров и их частей,
пока не предложено формальных математических моделей
и каких-либо алгоритмов решения.
Кроме того, во всех современных исследованиях
остаются практически не рассмотренными
вопросы учета технологических требований резки,
связанных с упомянутой выше <<жесткостью>> материала,
порождающей ограничения в выборе точек врезки в материал
и последовательность резки контуров вырезаемых деталей.
На практике эти вопросы часто решаются
с использованием интерактивных методов проектирования,
когда пользователь системы автоматизированного проектирования
управляющих программ для машин листовой резки с ЧПУ
в диалоговом режиме определяет и набор сегментов резки,
и точки врезки для каждого сегмента.
Кажущаяся естественной идея использования результатов
моделирования тепловых полей для соблюдения
технологических требований термической резки
пока не получила адекватной для практики реализации.

Другой особенностью публикаций
по рассматриваемой оптимизационной проблеме
является отсутствие разработок точных алгоритмов.

В связи с исследованиями ЗК отметим сейчас обстоятельный обзор
\cite{intro15,intro16,intro17},
работы \cite{Cha4`,Cha16`},
связанные с применением ДП для решения ЗК,
а также более поздние монографии \cite{intro20,intro21}.
Отметим, что в обзоре
\cite{intro15,intro16,intro17}
обсуждаются также задачи типа ЗК
(то есть варианты ЗК с теми или иными особенностями);
в этой связи см. также исследование \cite{intro22}.
Имеется и много других работ, идейно ориентированных на подходы,
сложившиеся в связи с решением ЗК.
Это касается, например,
использования метода ветвей и границ
\cite{Cha17`},
который находит широкое применение и в других задачах дискретной оптимизации,
в частности, в задачах с условиями предшествования
\cite{intro24}.
Традиционно много публикаций
появляется в последнее время в связи с разработкой
различных вариантов метаэвристик \cite{intro25,intro26,intro27,intro28},
однако они ориентированы в основном
на решение ЗК без дополнительных ограничений.

Несколько слов о работах авторского коллектива монографии
и его соавторов по теме исследования.
Решение задач оптимизации управления
инструментом для машин листовой резки с ЧПУ,
помимо уже упомянутых публикаций,
рассматривалось авторами, в частности,
в работах~\cite{intro29,intro30,intro31,intro32,Cha15`,intro35,intro36,intro37}.
В исследованиях~\cite{intro29,intro30}
были сформулированы эвристические правила
(правила <<жесткости>>) резки фигурных заготовок
на машинах для термической резки листовых материалов.
В~\cite{intro31}
для формализации задачи оптимизации маршрута
для случая стандартной техники резки
предложено при программировании в CAM
(системе
управляющих программ резки)
использовать
математическую модель обобщенной задачи коммивояжера
с дополнительными ограничениями.
В работе~\cite{intro32} применена модель ДП
для решения задачи о последовательном
обходе мегаполисов А.~Г.~Ченцова,
позволяющая разрабатывать точные алгоритмы
решения маршрутной задачи со сложными видами ограничений.
Для задач большой размерности
был разработан ряд приближенных алгоритмов
(см., например, \cite{Cha15`,intro34}).
Впоследствии на основе введенных понятий <<сегмента резки>>
и <<базового сегмента резки>> \cite{intro35,intro36}
проведено обобщение полученных результатов
для случая задач с заранее определенным набором сегментов резки,
а в работе~\cite{intro37} реализован алгоритм,
учитывающий динамические ограничения жесткости
детали при выборе точек врезки.
В работах \cite{intro38,intro39}
было показано, что этот выбор
может быть сделан на основе моделирования температурных полей
при термической резке материала.
Вопросы оптимальности разрабатываемых алгоритмов
при применении метода ДП были рассмотрены в
\cite{intro40,intro41,intro42,intro43}.
В работах \cite{intro44,intro45,intro46,intro47}
исследованы вопросы точного вычисления
целевых функций и эффективность применения
специальных техник резки при решении
практических оптимизационных задач
лазерной резки деталей на машинах с ЧПУ.

Построения, связанные с используемым в монографии
вариантом метода ДП, восходят к
\cite{Cha1`} и последующей большой серии
журнальных статей,
среди которых сейчас отметим лишь некоторые
(см.~\cite{Cha8`,intro51,intro52,intro53,intro54,intro55,intro56}).
Упомянутые публикации
\cite{Cha8`,intro51,intro52,intro53,intro54,intro55,intro56}
в основном посвящены решению абстрактных задач маршрутизации,
но математический аппарат, разработанный в этих работах,
оказался полезным и для решения различных прикладных задач.
В числе последних
необходимо
отметить практические задачи первой части монографии,
связанные с разработкой УП для машин с ЧПУ.
С другой стороны, развиваемые в этих исследованиях подходы
нашли применение в некоторых задачах атомной энергетики,
связанных с проблемой снижения облучаемости работников АЭС
при выполнении комплекса работ.
Одна из постановок такого рода касается
актуальной проблемы демонтажа энергоблока АЭС,
выведенного из эксплуатации.
Применение метода возможнои и при решении задач,
возникающих при аварийных ситуациях,
подобных Чернобылю и Фукусиме.
В этой связи обратим внимание на монографию \cite{Cha2`}.

Полезно отметить, что существует много других прикладных задач
с элементами маршрутизации и ограничениями,
подобными рассмотренным в монографии.
Примером могут служить задачи о морских и авиационных перевозках,
где также возникают условия предшествования,
определяющие, в частности, порядок перевозки грузов
между промежуточными пунктами (портами, аэродромами).
Элементы маршрутизации присутствуют и в задаче
авиапожарного патрулирования лесных массивов.

% Монография подготовлена при финансовой поддержке РФФИ
% (проект 20-08-00873).

% Вступление от Буслаевой 17.07.2016
%^^^^^^^^^^^^^^^^^^^^^^^^^^^^^^^^^^^^^^^^^^^^^^^^^^^^^^^^^^^^^^^^^^^^^^^^^^^^^^^

% !TeX root = ../mat_mod2.tex

% Часть I

\part{
  Инженерные задачи
  маршрутизации инструмента машин листовой резки.
  Общие постановки
  и~подходы к~их~решению
}

% !TeX root = ../mat_mod2.tex

% Часть I Глава 1

\chapter{
  МОДЕЛИРОВАНИЕ МАРШРУТА ИНСТРУМЕНТА ДЛЯ МАШИН ФИГУРНОЙ ЛИСТОВОЙ РЕЗКИ
  С~ЧИСЛОВЫМ ПРОГРАММНЫМ УПРАВЛЕНИЕМ.
  ОСНОВНЫЕ ПОНЯТИЯ И~ЗАДАЧИ
}
%\setcounter{section}{1}\setcounter{subsection}{1}
\setcounter{chapter}{1}
\setcounter{equation}{0}

% !TeX root = ..

\section{
  Технологии и техники листовой резки
  на~машинах с~ЧПУ
}
\label{sect:1.1}
\setcounter{equation}{0}

В машиностроении,
производстве металлоконструкций
и других отраслях промышленности существенная часть продукции
изготавливается из заготовок,
получаемых из листовых материалов на различном технологическом оборудовании.
К такому оборудованию относятся, в частности,
используемые на предприятиях отечественные и зарубежные
системы автоматизированного проектирования (САПР),
предназначенные для разработки управляющих программ (УП)
для машин листовой резки с ЧПУ
(т. н. \textit{Computer-Aided Manufacturing},
\textit{CAM}-системы),
которые
обеспечивают автоматизацию процесса разработки УП,
однако не позволяют решить многие оптимизационные задачи.
При этом при моделировании маршрута инструмента пользователям
САПР часто приходится применять интерактивные методы проектирования УП,
поскольку алгоритмы генерации УП,
реализованные в автоматическом режиме проектирования,
во многих случаях не позволяют генерировать оптимальные управляющие программы,
а также обеспечить соблюдение некоторых технологических требований листовой резки.
В качестве критериев оптимизации имеются в виду время резки и
некоторые другие стоимостные характеристики процесса листовой резки.
Проблема разработки методов, алгоритмов и соответствующего программного обеспечения,
позволяющих в автоматическом режиме оптимизировать параметры
процесса резки заготовок из листовых материалов на машинах с ЧПУ,
включая алгоритмы маршрутизации движения инструмента,
которые бы обеспечивали минимизацию времени резки и стоимости процесса,
остается актуальнейшей задачей раскройно-заготовительного производства.

Рассмотрим понятие маршрута инструмента (маршрута резки)
применительно к некоторым технологиям фигурной листовой резки.
В настоящее время в промышленном производстве
единичного и мелкосерийного типа для раскроя листовых материалов
используются в основном следующие технологии:
лазерная, плазменная, газовая и гидроабразивная.
Целесообразность их применения определяется различными технологическими факторами,
например, свойствами раскраиваемого материала,
экономическими требованиями к процессу резки,
требованиями к качеству реза и пр.
Эти и некоторые другие технологии резки предполагают,
что для сохранения требуемой геометрии заготовки
траектория движения режущего инструмента не совпадает
с граничным контуром заготовки,
а задается некоторой эквидистантой этого контура,
поскольку часть материала вырезается (<<сгорает>>, <<вымывается>> и пр.)
в процессе резки.
Как правило, дистанция между эквидистантным контуром,
по которому осуществляется резка, и граничным контуром заготовки определяется величиной,
равной половине ширины реза.
Эта величина зависит от выбранной технологии резки,
толщины и марки материала, заданной скорости резки
и особенностей конкретного технологического оборудования,
используемого для резки.

Еще одна особенность листовой резки –
необходимость предварительной врезки (пробивки)
материала перед процессом резки непосредственно
по эквидистантному контуру заготовки.
Пробивка материала сопровождается дополнительными
деформациями материала в точке врезки,
поэтому производится на расстоянии (дистанции)
от контура заготовки большем,
чем дистанция до эквидистантного контура за исключением случаев,
когда для точек врезки в листовом материале механическим способом
готовятся (например, просверливаются)
отверстия.
Врезка может также осуществляться
непосредственно на границе материала
(<<врезка с края листа>>).
В этом случае достигается уменьшение
деформаций материала и сокращается время врезки.

\noindent На рис. \ref{standard-cutting}.
показан
один из способов резки заготовки
(стандартная техника).

Если используется стандартная техника резки,
то в этом случае каждый замкнутый контур вырезается целиком,
и после резки одного контура переход к следующей точке врезки
происходит с выключенным инструментом на холостом ходу.
При этом точка выключения инструмента
в общем случае
может не совпадать с точкой входа в эквидистантный контур заготовки,
по которому осуществляется резка, и также,
как и точка врезки,
может лежать вне заданного эквидистантного контура.
Во многих случаях допускается программирование точки выключения
инструмента непосредственно на эквидистантном контуре.

\begin{figure}[H]
  \centering
  \includegraphics[width=0.9\textwidth]{cutting-path.png}
  \caption{
    Схема стандартной техники резки (резка по замкнутому контуру)
  }
  \label{standard-cutting}
\end{figure}

Стратегия минимизации тепловых деформаций при термической резке
и требования к качеству реза порождают необходимость управления
не только выбором точек врезки,
но и управлением траекторией подхода к контуру
(\textit{lead-in})
и способом выхода из контура
(\textit{lead-out}).
В зависимости от конкретных условий
(вида термической резки, марки и толщины материала,
скорости резки, геометрической формы контура и пр.)
подход к контуру может осуществляться по дуге окружности,
касательная к которой совпадает с касательной к контуру в точке входа,
либо производиться по прямой линии
(например, по наикратчайшему расстоянию до контура).
Соответственно, и после завершения резки выход из контура
также может осуществляться с включенным инструментом
(либо по дуге, либо по прямой линии).
Необходимость выхода из контура с включенным
инструментом может быть вызвана тем,
что в точке выключения инструмента может возникнуть
<<вырыв>> или оплавление части материала,
что приводит к искажению геометрии заготовки.
Уменьшение эффекта деформации заготовок обеспечивает
также врезка в <<угловые>> точки заготовок,
рис.~\ref{corner}.

\begin{figure}[H]
  \centering
  \includegraphics[width=0.4\textwidth]{corner.png}
  \caption{
    Пример врезки <<в угол>>
  }
  \label{corner}
\end{figure}

Примером нестандартной техники
может служить <<цепная>> резка,
которая заключается в резке нескольких контуров с
использованием одной точки врезки.
При этом каждый контур,
как и в случае применения стандартной техники резки,
вырезается целиком.
На рис.~\ref{chain}
показан пример схемы резки двух заготовок,
в которой резка внешних контуров обеих заготовок
производится без выключения инструмента
с использованием только одной точки врезки.

\begin{figure}[H]
  \centering
  \includegraphics[width=0.75\textwidth]{chain.png}
  \caption{
    Пример схемы резки двух заготовок
    с~использованием стандартной
    и~<<цепной>> техники резки
  }
  \label{chain}
\end{figure}

Перемещение инструмента в точку врезки
в этом примере начинается из начальной точки на холостом ходу,
а после завершения резки последнего контура
предусмотрен возврат инструмента в начальную точку.

На практике применяется также техника резки
замкнутого контура заготовок по частям
с использованием нескольких точек врезки
с целью формирования т. н. <<перемычек>>
(рис.~\ref{jumper}),
а также используются другие специальные приемы,
целью которых является оптимизация различных параметров,
характеризующих процесс резки,
и соблюдение необходимых технологических требований резки.
Техника резки <<перемычка>> предусматривает
оставление невырезанной части контура заготовки
обычно небольшого прямолинейного отрезка или нескольких отрезков
с резкой этих отрезков, после завершения резки оставшейся части контура.
Этот прием применяется с целью уменьшения деформаций материала
при термической резке заготовок, склонных к термическим деформациям,
в частности, длинномерных заготовок.

\begin{figure}[H]
  \centering
  \includegraphics[width=0.75\textwidth]{jumper.png}
  \caption{
    Схема формирования перемычки на контуре при резке полосы
  }
  \label{jumper}
\end{figure}

На рис. \ref{saber} показан пример искажения геометрической формы
(получения т. н. формы <<сабли>>)
и~изменения размера длинномерной прямоугольной заготовки,
вырезаемой без использования техники <<перемычка>>.

\begin{figure}[H]
  \centering
  \includegraphics[width=0.75\textwidth]{saber.png}
  \caption{
    Результат изменения формы
    и~размера прямоугольной заготовки
    при~термической резке
  }
  \label{saber}
\end{figure}

На рис. \ref{bridge}
приведен пример использования техники <<мост>>,
предполагающей  частичную резку замкнутого контура
заготовки с последующим завершением резки контура
после резки контура другой заготовки или
группы контуров других заготовок.
Эта техника используется при резке двух или
нескольких рядом расположенных заготовок и
предусматривает переход по короткой траектории (<<мосту>>)
к резке другой заготовки и возврат к первому контуру
по этой же траектории для завершения процесса резки.
Так же, как и <<перемычки>>,
мосты существенно уменьшают тепловые деформации материала,
особенно при резке длинномерных заготовок,
кроме того, сокращают число точек врезки.

\begin{figure}[H]
  \centering
  \includegraphics[width=0.75\textwidth]{bridge.png}
  \caption{
    Схема резки двух полос с использованием техники <<мост>>
  }
  \label{bridge}
\end{figure}

Разновидностью техники <<мост>> можно считать технику <<змейка>>,
показанную на рис.~\ref{snake},
в которой также используется прием
частичной резки контура и резки
нескольких заготовок без выключения инструмента.

\begin{figure}[H]
  \centering
  \includegraphics[width=0.85\textwidth]{snake.png}
  \caption{
    Схема резки <<змейка>>
  }
  \label{snake}
\end{figure}

Для уменьшения длины рабочего хода инструмента
применяют т. н. <<совмещенный>> рез.
Он используется для вырезки заготовок,
которые содержат прямолинейные отрезки в контуре
и которые в процессе раскроя размещаются таким образом,
что имеют общую границу по одному из таких прямолинейных отрезков.
Общая прямолинейная граница позволяет размещать
заготовки с половинным припуском на рез
(т. е. на ширину реза),
поскольку режется только один раз,
что экономит материал и сокращает суммарную
длину резки на величину совмещенного реза.
Совмещенный рез реализован, в частности,
в технике резки <<восьмерка>>,
применяемой для резки двух одинаковых заготовок
(рис.~\ref{8}).
В этой технике используется также идея цепной резки.

\begin{figure}[H]
  \centering
  \includegraphics[width=0.6\textwidth]{8.png}
  \caption{
    Схема резки <<восьмеркой>> двух заготовок
  }
  \label{8}
\end{figure}

Основные технологические требования
фигурной резки на машинах с ЧПУ обусловлены
необходимостью учета возникающих деформаций
материала и искажения геометрических размеров
вырезаемых заготовок при использовании
термических технологий резки.
Применение специальных техник позволяет
уменьшить эффект искажения геометрии,
который особенно значителен при
использовании газовой и плазменной технологий.

При использовании любой техники резки маршрут инструмента
машины с ЧПУ для фигурной листовой резки
включает в себя следующие компоненты:
\begin{itemize}
  \item точки врезки;
  \item рабочий ход инструмента;
  \item точки выключения инструмента;
  \item линейное перемещение инструмента на холостом ходу
  между точкой выключения инструмента и следующей точкой врезки.
\end{itemize}

При разработке управляющей программы
первое перемещение инструмента обычно
программируется, как на рис.~\ref{chain},
из начальной точки.

Отметим, что некоторые машины фигурной листовой
резки с ЧПУ могут быть укомплектованы
специальным видом инструмента,
т. н. трехрезаковым блоком для вырезания
из листа заготовок с одновременной разделкой
кромок поверхности реза для последующей сварки.
Врезка в материал для такого инструмента
программируется специальными способами.

Введем некоторые определения,
касающиеся понятия маршрута резки.
В дальнейшем при формальном обозначении
математических и геометрических категорий
мы будем использовать стандартную
теоретико-множественную символику.
Ее детальное описание дано в
\ref{sect:3.1}.
Введем следующее определение.

\begin{opred}
\label{def:cutting-segment}
{\bf Сегментом резки}
$S=MM^*$
будем называть траекторию рабочего хода
инструмента между точкой врезки
$M$
и соответствующей ей точкой выключения инструмента
$M^*$.
Геометрически сегмент резки представляет собой
определенную на эвклидовой плоскости
$\mathbb R \times \mathbb R$
кривую.
$(S \subset \mathbb R \times \mathbb R;
M=(x,y) \in \mathbb R \times \mathbb R,
M^* =(x^*,y^*)\in \mathbb R \times \mathbb R)$.
Будем также полагать,
что в каждой точке траектории определено направление движения инструмента.
Заметим, что если сегмент резки не содержит замкнутых контуров,
то направление движения резки в каждой точке траектории
однозначно определяется начальной точкой сегмента
(точкой врезки).
Замкнутые контуры в траектории рабочего хода инструмента
могут появляться не только в результате резки контуров заготовок,
но и при программировании т. н. петель,
которые используются для повышения качества реза.
\end{opred}

Используя понятие сегмента резки,
все техники фигурной резки на машинах с ЧПУ
можно разделить на три класса.
\begin{enumerate}
  \item
  {\it Резка по замкнутому контуру (стандартная техника)}:
  в этом случае сегмент резки содержит
  ровно один замкнутый эквидистантный контур заготовки,
  который вырезается целиком.
  \item
  {\it Мультисегментная резка контура}:
  в этом случае для вырезки одного контура
  используются не менее двух сегментов резки.
  \item
  {\it Мультиконтурная резка}:
  резка предполагает вырезку нескольких
  контуров в одном сегменте.
\end{enumerate}

Примерами мультиконтурной резки являются,
в частности, приведенные выше техники резки:
<<цепная резка>>, <<мост>>, <<змейка>> и <<восьмерка>>,
а примером мультисегментной резки --
резка с перемычкой.
На практике используются и другие специальные техники резки,
но все они являются разновидностями техник,
относящихся к одному из определенных выше классов.

При разработке управляющих программ для
машин фигурной листовой резки с ЧПУ чаще всего
применяется стандартная техника резки.
Вместе с тем нередко используются и
комбинации различных техник резки.
Применение той или иной техники резки
при проектировании маршрута резки в
каждом конкретном случае, как правило,
обусловлено либо технологическими требованиями резки,
либо стремлением оптимизировать некоторые
параметры листовой резки.
Подробнее эти вопросы рассмотрены далее.

% !TeX root = ../mat_mod2.tex

\section{
  Маршрут резки и оптимизационные задачи
  маршрутизации инструмента машин листовой
  резки с~ЧПУ
}
\label{sect:1.2}
\setcounter{equation}{0}

Для формального определения маршрута резки
введем следующие обозначения.
Пусть
$A_1, A_2, \,\dots, A_n$
– двумерные геометрические объекты (точечные замкнутые множества),
представляющие собой односвязные или
многосвязные области эвклидовой плоскости
$\mathbb R \times \mathbb R$,
ограниченные одной или несколькими замкнутыми кривыми
(граничными контурами)
$C_1, C_2, \,\dots, C_N$
$(A_i, C_J \subset \mathbb R \times \mathbb R;
i \in \overline{1,n};
j \in \overline{1, N};
N \geqslant n)$.
Объекты
$A_1, A_2, \,\dots, A_n$
являются геометрическими моделями плоских заготовок/деталей
({\it в дальнейшем в книге термин <<деталь>>,
которая вырезается из листового материала,
будет использоваться как синоним термина <<заготовка>>}).

Пусть также определена область размещения объектов
$B \subset \mathbb R \times \mathbb R$,
которая является геометрической моделью листового материала,
из которого вырезаются детали.
В общем случае область размещения
может содержать несколько кусков материала
(не обязательно прямоугольной формы),
но для решения оптимизационных задач
маршрутизации инструмента целесообразно рассматривать
в качестве области размещения одно замкнутое точечное множество,
ограниченное (как и деталь)
одним внешним контуром.
При этом допустимо наличие отверстий в материале
(внутренних контуров).
Будем полагать, что зафиксирован некоторый вариант размещения
объектов в области размещения,
при этом выполнены условия взаимного непересечения объектов.
Полагаем также, что выполнены другие дополнительные условия,
обусловленные технологическими требованиями резки деталей
на конкретном технологическом оборудовании с ЧПУ,
в частности, условие соблюдения необходимой ширины реза.
Другими словами, фиксированный вариант размещения объектов
является допустимым вариантом раскроя листового материала
для заданного набора $n$ деталей.

Пример размещения в прямоугольной области 24 объектов
($n=24$),
описываемых 30 замкнутыми контурами
($N=30$)
с заданным минимальным расстоянием между объектами,
приведен на рис.~\ref{nesting}.

\begin{figure}[h]
  \begin{center}
  \includegraphics[width=0.9\textwidth]{nesting.png}
  \caption{Пример раскроя листа $2000 \times 1000$ мм с заданным минимальным расстоянием между деталями 10 мм}
  \label{nesting}
  \end{center}
\end{figure}

Раскройная карта
на рис.~\ref{nesting}
получена с помощью подсистемы автоматического раскроя
\textit{CAD/CAM}
системы <<Сириус>>.

Предположим, что для вырезки деталей было использовано
$K$
сегментов резки
$S_k=M_kM^*_k; k \in \overline{1,K}$.
Тогда маршрут резки деталей можно определить
в терминах сегментов резки как кортеж
\begin{equation}
  ROUTE = \left<
    M_0, M_1, S_1, M_1^*, M_2, S_2, M_2^*, \,\dots, M_K, S_K, M_K^*,
    i_1, i_2, \,\dots, i_K
  \right>
  ,
  \label{tuple}
\end{equation}
где
$M_0$
-- начальная точка положения инструмента,
$i_1, i_2, \,\dots, i_K$
– последовательность, в которой вырезаются используемые сегменты резки
$S_1, S_2, \,\dots, S_K$.
Линейное перемещение инструмента на холостом ходу
между точкой выключения инструмента и следующей точкой врезки
однозначно определяется этой последовательностью.
Если применить комбинаторную терминологию,
то последовательность однозначно задается перестановкой порядка
$K$,
т. е. упорядоченным набором натуральных чисел от $1$ до $K$
(биекцией на множестве $\overline{1,K}$),
которая числу
$k \in \overline{1,K}$
ставит в соответствие элемент
$i_k$ из набора.
Отметим, что термин <<маршрут резки>> является
общепринятым технологическим понятием.
В главах 3 -- 5 при описании математических моделей оптимизации
маршрута резки мы будем использовать термин <<маршрут>>
применительно к перестановке
$i_1, i_2, \,\dots, i_K$,
что, в свою очередь, соответствует устоявшейся
терминологии в задаче о последовательном обходе мегаполисов.

На рис.~\ref{cutting}
показана схема одного из возможных маршрутов резки для примера,
приведенного на рис.~\ref{nesting}.

\begin{figure}[h]
  \begin{center}
  \includegraphics[width=0.9\textwidth]{cutting.png}
  \caption{Пример маршрута резки, содержащего 24 сегмента резки}
  \label{cutting}
  \end{center}
\end{figure}

Маршрут резки содержит 24 сегмента.
Для резки внешних контуров трех групп деталей
с точками врезки $M_1$
(три детали в группе),
$M_7$
(четыре детали в группе) и
$M_{11}$
была использована мультиконтурная резка
(указанные группы деталей выделены коричневым цветом).
Все остальные контуры вырезаны с применением стандартной техники резки.
Последовательность резки сегментов соответствует
номерам точек врезки $M_J$ ($J=1,2,\,\dots, 24$).
После вырезки последнего сегмента
возврат инструмента в начальную точку $M_0$
не программировался.

На приведенном рис.~\ref{cutting}
визуализация траектории инструмента
осуществляется точно по граничным контурам деталей,
а не по их эквидистантным контурам, хотя,
как отмечено выше,
траектория реза должна отстоять от
граничного контура на половину ширины реза.
Это связано с тем, что в большинстве
\textit{CAM}-систем
(\textit{Computer Aided Manufacturing})
программирование движения инструмента первоначально
осуществляется по граничным контурам деталей,
а вычисление реальной траектории производится
либо непосредственно самой системой ЧПУ,
либо специальной программой-постпроцессором,
предназначенной для конвертирования информации о
маршруте резки из внутреннего формата системы в
формат команд конкретного технологического оборудования с ЧПУ.
В этом случае величину припуска на рез
устанавливает оператор на станке перед запуском
управляющей программы резки.

В дальнейшем без ограничения общности мы будем полагать,
что траектория инструмента в маршруте резки $ROUTE$
программируется по граничным контурам,
и сегменты резки
$S_k=M_kM^*_k; k \in \overline{1,K}$
содержат все граничные контуры деталей
$C_1, C_2, \,\dots, C_N$,
т. е.
$$
\bigcup_{j=1}^N C_j \subset \bigcup_{k=1}^K S_k
$$

Соответственно, все точки входа в эквидистантный контур
(и выхода из эквидистантного контура)
также лежат на граничных контурах,
см. рис.~\ref{standard-cutting}.

На рис.~\ref{control-program}
показан фрагмент управляющей программы
($G$-кода) для машины листовой газовой резки
типа <<Комета>> с системой ЧПУ 2Р32М.

\begin{figure}
\begin{multicols}{3}

  \%УП\_2Р32М\_01
  N1G91 \\
  N2G00X7662Y9909F6000 \\
  N3M70T1 \\
  N4M71T1 \\
  N5G01X-141Y-48F460 \\
  N6X-2400  \\
  N7X-40  \\
  N8X-67  \\
  N9X-2400  \\
  N10X-40 \\
  N11X-67 \\
  N12X-2400 \\
  N13Y-700  \\
  N14X2400  \\
  N15Y700 \\
  N16Y40  \\
  N17X107Y-40 \\
  N18Y-700  \\
  N19X2400  \\
  N20Y700 \\
  N21Y40  \\
  N22X107Y-40 \\
  N23Y-700  \\
  N24X2400  \\
  N25Y700 \\
  N26Y40  \\
  N27M74T1  \\
  N28G00X817Y-8745F6000 \\
  N29M71T1  \\
  N30G03X-108Y0I-54J0F460 \\
  N31G01Y-1048  \\
  N32X-1740 \\
  N33Y400 \\
  N34X940Y900 \\
  N35X800 \\
  N36Y-252  \\
  N37X20Y-30  \\
  … \\
  N314M71T1 \\
  N315G02X-130Y0I-65J0F460  \\
  N316G01Y267 \\
  N317G03X-50Y50I-50J0  \\
  N318G01X-1366 \\
  N319G03X-46Y-31I0J-50 \\
  N320G01X-384Y-960 \\
  N321G03X-4Y-19I46J-19 \\
  N322G01Y-1120 \\
  N323G03X14Y-35I50J0 \\
  N324G01X122Y-121  \\
  N325G03X37Y-14I35J35  \\
  N326G01X1627Y-1 \\
  N327G03X50Y50I0J50  \\
  N328G01Y1933  \\
  N329X20Y30  \\
  N330M74T1 \\
  N331M75T1 \\
  N332M02 \\
  M30
\end{multicols}
\caption{Фрагмент УП для машины листовой резки <<Комета>> с ЧПУ 2Р32М }
\label{control-program}
\end{figure}

Программа сгенерирована на основе маршрута резки
(спроектированного в интерактивном режиме в
\textit{CAD/CAM} системе <<Сириус>>
и показанного на рис.~\ref{cutting})
соответствующим постпроцессором со
следующими числовыми параметрами резки:
\begin{itemize}
  \item	число строк в УП – 333;
  \item	количество точек врезки (пробивки листа) – 24;
  \item	путь инструмента на рабочей скорости – 27,36 м;
  \item	путь инструмента на быстром (холостом) ходу – 8,39 м;
  \item	время движения на рабочей скорости – 62,04 мин;
  \item	время движения на быстром (холостом) ходу – 1,64 мин;
  \item	общее время резки: 63,68 мин.
\end{itemize}

В зависимости от выбранного маршрута резки
числовые параметры резки могут существенно различаться.
Таким образом, при разработке управляющих программ
для машин фигурной листовой резки с ЧПУ возникают
различные задачи оптимизации маршрута инструмента.
В качестве критерия оптимизации (целевой функции)
в этих задачах чаще всего рассматривается общее время резки.
При термической и гидроабразивной резке для сформированного
маршрута резки общее время резки
$T_{cut}$
рассчитывается по следующей формуле:
\begin{equation}
  T_{cut} = \frac{L_{on}}{V_{on}} + \frac{L_{off}}{V_{off}} +N_{pt} \cdot t_{pt}
  ,
  \label{cutting-time}
\end{equation}
где
$L_{on}$ -- длина реза с включенным режущим инструментом;
$V_{on}$ -- скорость рабочего хода режущего инструмента;
$L_{off}$ -- длина переходов с выключенным режущим инструментом (холостой ход);
$V_{off}$ -- скорость холостого хода;
$N_{pt}$ -- количество точек врезки;
$t_{pt}$ -- время, затрачиваемое на одну точку врезки.
При этом подразумевается,
что получаемое в результате врезки отверстие
расположено внутри материала листа.
Однако при резке заготовок, как отмечалось,
могут быть использованы и другие типы врезки,
что приводит к изменению времени врезки
$t_{pt}$
в этих случаях.
Если при резке деталей было использовано несколько типов врезки,
то формула~(\ref{cutting-time}) запишется в более общем виде:
\begin{equation}
  T_{cut} = \frac{L_{on}}{V_{on}} + \frac{L_{off}}{V_{off}} +
  \sum_{j=1}^p N_{pt}^j \cdot t_{pt}^j
  ,
  \label{cutting-time-multi}
\end{equation}
где
$p$ -- число использованных типов врезки,
$N_{pt}^j$ -- количество точек врезки типа $j$;
$t_{pt}^j$ -- время, затрачиваемое на одну точку врезки типа $j$.

И в (\ref{cutting-time})
и в (\ref{cutting-time-multi})
значение скорости холостого хода инструмента
$V_{off}$ -- константа,
определяемая техническими характеристиками
используемого технологического оборудования.
Значение скорости рабочего хода инструмента
$V_{on}$
программируется при разработке управляющей программы
в соответствии с используемой технологией резки
и параметрами листового материала
(марка материала и толщина).
Предполагается, что заданная величина
$V_{on}$  в (\ref{cutting-time}) и в (\ref{cutting-time-multi})
также является константой,
однако на практике фактическая скорость резки
может меняться в зависимости от различных технологических факторов,
а также характеристик спроектированной управляющей программы.
Это диктует необходимость проведения исследований для
определения поправочного коэффициента для величины
$V_{on}$.
В \ref{sect:1.4}
приведены результаты такого рода исследований
применительно к машине лазерной резки с ЧПУ
\textit{ByStar 3015}
для листового материала
\textit{АМг3М} толщиной от 1,5 до 5 мм.

Важнейшей экономической характеристикой качества
разработанной управляющей программы является стоимость
(себестоимость) резки деталей на машине с ЧПУ.
Это сложный интегрированный показатель,
который включает в себя произведенные во время
резки затраты на электроэнергию и расходные материалы,
на обслуживание машины с ЧПУ,
а также другие эксплуатационные затраты.
Отметим, что стоимость резки не всегда
пропорциональна времени резки,
поскольку зависит еще и от различных режимов резки.
По аналогии с формулой времени резки (\ref{cutting-time})
показатель стоимости резки можно определить по следующей формуле:
\begin{equation}
  F_{cost}=
  L_{on} \cdot C_{on} +
  L_{off} \cdot C_{off} +
  N_{pt} \cdot C_{pt}
  ,
  \label{cutting-cost}
\end{equation}
где
$C_{on}$ -- стоимость единицы пути с включенным режущим инструментом;
$C_{off}$ -- стоимость единицы пути с выключенным режущим инструментом;
$C_{pt}$ -- стоимость одной точки врезки,
а $L_{on}, L_{off}, N_{pt}$
имеют тот же смысл, что и в формуле~(\ref{cutting-time}).
При этом $C_{on}, C_{off}, C_{pt}$ --
величины, зависящие от типа машины с ЧПУ,
технологии резки, используемой скорости рабочего хода инструмента,
толщины и марки материала.
Функциональная зависимость
$C_{on}, C_{off}, C_{pt}$
от перечисленных параметров
может задаваться либо табличными функциями,
либо аналитически.
При этом абсолютные значения этих величин
на российских предприятиях, использующих машины с ЧПУ,
определяются экономическими службами с учетом многих факторов
и могут существенно различаться на разных предприятиях.
Зачастую стоимость резки вообще не учитывается
в раскройно-заготовительном производстве,
либо рассчитывается на основании специальных нормативов,
не зависящих от величин
$L_{on}, L_{off}, N_{pt}$.

Очевидно, что необходимость расчета стоимости резки
для каждой управляющей программы резки возникает на предприятиях,
оказывающих услуги сторонним организациям по резке материала.
Однако и на многих таких предприятиях для определения
стоимости резки учитывается только длина рабочего хода инструмента
$L_{on}$,
которая принимается равной суммарному периметру граничных контуров вырезаемых деталей
$C_1, C_2, \,\dots, C_N$,
что, естественно, приводит к неадекватной оценке эффективности процесса резки.
В дальнейшем при оптимизации стоимостных параметров резки
мы будем применять теоретически обоснованный способ определения
показателя стоимости резки, задаваемый формулой (\ref{cutting-cost}).

На рис.~\ref{cost} представлен расчет стоимости резки
$F_{cost}$
для рассматриваемого примера при резке деталей на машине газовой резки.
Значения величин
$C_{on}, C_{off}, C_{pt}$
взяты из таблицы, используемой для расчета себестоимости фигурной
листовой резки на одном из предприятий Свердловской области.

\begin{figure}[h]
  \begin{center}
  \includegraphics[width=0.9\textwidth]{cost.png}
  \caption{
    Пример расчета стоимости резки $F_{cost}$
    при резке деталей из~углеродистой стали
    толщины 20~мм на~машине газовой резки}
  \label{cost}
  \end{center}
\end{figure}

Следует отметить,
что задача правильного определения величин
$C_{on}$, $C_{off}$, $C_{pt}$
для конкретного технологического оборудования
и конкретного материала сама по себе является малоисследованной проблемой.
В \ref{sect:1.4}
приведены результаты исследования,
позволяющего точно вычислять себестоимость
лазерной резки применительно для машины с ЧПУ
\textit{ByStar3015}
при резке углеродистой и нержавеющей
стали различных толщин
(на примере Ст10кп и 12Х18Н10Т),
а также при резке алюминия и его сплавов
(на примере \textit{Амг3М}).

В случае использования нескольких типов врезки формула~(\ref{cutting-cost}) примет вид:
\begin{equation}
  F_{cost}=
  L_{on} \cdot C_{on} +
  L_{off} \cdot C_{off} +
  \sum_{j=1}^p N_{pt}^j \cdot C_{pt}^j
  ,
  \label{cutting-cost-multi}
\end{equation}
где $C_{pt}^j$ -- стоимость одной точки врезки типа $j$.

Как легко заметить,
значения целевых функций (\ref{cutting-time}) -- (\ref{cutting-cost-multi})
однозначно определяются маршрутом резки,
задаваемым кортежем (\ref{tuple}),
поскольку геометрия сегментов резки
$S_1, S_2, \,\dots, S_K$
позволяет вычислить длину рабочего хода инструмента  $L_{on}$,
а координаты точек
$M_0$, $M_1$, $M_1^*$, $M_2$, $M_2^*$ \,\dots, $M_K$, $M_K^*$
и перестановка
$i_1$, $i_2$, \,\dots, $i_K$
(последовательность, в которой вырезаются используемые сегменты резки)
задают набор холостых перемещений инструмента,
который определяет суммарную длину холостого хода
$L_{off}$.

Таким образом, сформулированные задачи оптимизации
маршрута инструмента для машин фигурной листовой резки с ЧПУ
можно представить в самом общем виде
как задачу минимизации некоторой числовой функции $F$,
заданной на множестве $G$ допустимых кортежей $ROUTE$,
т. е.
\begin{equation}
  F(ROUTE) \to \min_{ROUTE \in G}
  .
  \label{problem-statement}
\end{equation}

Поскольку элементы кортежа содержат
(помимо последовательности резки
$i_1, i_2, \,\dots, i_K$,
выбираемой из дискретного множества перестановок)
точки врезки и точки выключения инструмента
$M_kM_k^*, k \in \overline{1,K}$,
которые, в свою очередь,
могут быть выбраны из континуальных подмножеств евклидовой плоскости
$\mathbb R \times \mathbb R$,
даже в случае наложения существенных ограничений
на возможность выбора допустимых сегментов
$S_k$
оптимизационная задача (\ref{problem-statement})
может быть отнесена к классу очень сложных задач
непрерывно-дискретной оптимизации.
Некоторые вопросы формирования допустимых сегментов резки мы рассмотрим в части 2
настоящей монографии при математической формализации задачи (\ref{problem-statement})
и ее сведении к задаче о последовательном обходе мегаполисов.
В следующем параграфе мы сформулируем основные ограничения
на допустимые значения элементов последовательности резки
$i_1, i_2, \dots i_K$
и на значения
$M_kM_k^*, k \in \overline{1,K}$,
вызванные особенностями технологии листовой резки на машинах с ЧПУ.

% !TeX root = ../mat_mod2.tex

%!!! Проверить в финальном варианте !!!
%\newpage
\section{
  Технологические ограничения параметров маршрута инструмента
  машин листовой резки с~ЧПУ
}
\label{sect:1.3}
\setcounter{equation}{0}

\input{txt/1.3.1.tex}
\input{txt/1.3.2.tex}
\input{txt/1.3.3.tex}

% !TeX root = ..


\section{
  Классификация оптимизационных задач
  маршрутизации инструмента
  машин фигурной листовой резки с~ЧПУ
}
\label{sect:1.4}
\setcounter{equation}{0}

Существующая  классификация задач маршрутизации инструмента
машин фигурной листовой резки с ЧПУ определяется
типом использованной техники резки и способом задания
возможных точек входа инструмента в контур.
В \cite{intro13}
все задачи маршрутизации разбиты на пять основных классов
(см. рис.~\ref{dewil}).

\begin{itemize}
\item Задача непрерывной резки
({\it CCP, Continuous Cutting Problem}):
каждый контур вырезается целиком,
и резка может начаться в любой точке контура (и в ней же завершиться).
Переход к следующему контуру осуществляется на холостом ходу инструмента машины с ЧПУ.

\item Задача коммивояжера
({\it TSP, Traveling Salesman Problem}):
самый простой частный  случай задачи {\it CCP} --
каждый контур вырезается целиком,
и резка может начаться в только в одной заранее определенной точке контура
(и в ней же завершиться).

\item Обобщенная задача коммивояжера
({\it GTSP, Generalized Traveling Salesman Problem}),
также частный случай задачи {\it CCP} --
резка может начаться только в одной из заранее
заданных точек на контуре (их может быть несколько),
контур также должен быть вырезан целиком.

\item Задача резки с конечным набором точек
({\it ECP, Endpoint Cutting Problem}):
резка может начаться только в одной из
заранее заданных точек на контуре,
однако контур может быть вырезан за несколько подходов, по частям.

\item Задача произвольной резки
({\it ICP, Intermittent Cutting Problem})
-- наиболее общая формулировка задачи моделирования траектории резки,
когда не накладывается никаких ограничений на выбор точек начала и конца резки,
а также на последовательность резки контуров и их частей:
контуры могут резаться по частям,
за несколько подходов,
и резка может быть
начата и продолжена в любой точке контура.

\end{itemize}

\begin{figure}[h]
  \begin{center}
  \includegraphics[width=0.9\textwidth]{dewil.png}
  \caption{Основные классы задач маршрутизации инструмента для машин фигурной листовой резки }
  \label{dewil}
  \end{center}
\end{figure}

Обычно предполагается также,
что точки врезки в материал,
которые из-за технологических требований резки
(\ref{pierce-constraint})
не совпадают с точками входа в контур,
однозначно определяются выбранными точками входа в контуры (и наооборот)
и находятся от контуров на фиксированном расстоянии.
Отметим, что число сегментов резки
$K$ в кортеже (\ref{tuple})
$ROUTE = \left<
  M_0, M_1, S_1, M_1^*, M_2, S_2, M_2^*, \,\dots, M_K, S_K, M_K^*,
  i_1, i_2, \,\dots, i_K
\right>
$
для первых трех классов задач маршрутизации всегда равно количеству вырезаемых контуров
$N$.

Введем следующее определение.

\begin{opred}
  \label{def:base-segment}
  {\bf Базовым сегментом резки}
  $B^S$
  для сегмента резки
  $S=MM^*$
  будем называть часть траектории сегмента,
  не содержающую траектории входа в контур
  {\it lead-in} и выхода из контура {\it lead-out},
  т.е.
  \begin{equation}
    S=MM^* = M \, lead_{in} \, B^S \, lead_{out} \, M^*
    .
    \label{base-segment}
  \end{equation}
\end{opred}

Будем полагать,
что базовый сегмент,
в отличие от сегмента резки,
не имеет направления,
но тогда если базовый сегмент содержит
один или более замкнутых контуров,
то при определении сегмента резки $S$
нам необходимо для каждого контура задать направление резки
(<<+>> при резке по часовой стрелке,
<<->> против часовой).

Таким образом,
каждый базовый сегмент резки
содержит список
$L(B^S)$
своих замкнутых контуров (может быть пустой).
Пусть
$|L(B^S)|$ --
длина этого списка, тогда кортеж $ROUTE$ (\ref{tuple})
в терминах базового сегмента резки запишется в следующем виде:

\begin{multline}
  ROUTE =  \langle
    M_0, M_1 B^{S_1} M_1^* p_1^1 \,\dots, p_1^{|L(B^{S_1})|},
    M_2 B^{S_2} M_2^* p_2^1 \,\dots, p_2^{|L(B^{S_2})|},
    \\ \dots, \\
    M_K B^{S_K} M_K^* p_K^1 \,\dots, p_K^{|L(B^{S_K})|},
    \\
    i_1, i_2, \,\dots, i_K
  \rangle
  ,
  \label{tuple-base-segments}
\end{multline}
где $p_t^r=\pm 1$
(направление резки в контуре с номером $r$ базового сегмента $B^{S_t}$),
$r=\overline{1, |L(B^{S_t})|}$.

Формула (\ref{tuple-base-segments})
дает наиболее общее формальное описание маршрута резки (траектории интрумента)
машины листовой резки с ЧПУ.

В примере резки двух деталей
на рис.~\ref{cut2-1}
два базовых сегмента выделены пунктирными
желтой и коричневой линиями
(использована мультиконтурная техника резки).
Пример сегмента резки, в котором использована техника совмещенного реза,
и базового сегмента, содержащего четыре контура,
приведен на рис.~\ref{cut4-3}.
Стрелками отмечено направление резки контуров.

\begin{figure}[h]
  \begin{center}
  \includegraphics[width=0.9\textwidth]{cut2-1.png}
  \caption{
    Пример схемы резки трех замкнутых контуров
    с использованием двух~сегментов резки
    }
  \label{cut2-1}
  \end{center}
\end{figure}

\begin{figure}[h]
  \begin{center}
  \includegraphics[width=0.7\textwidth]{cut4-3.png}
  \caption{
    Сегмент резки, включающий базовый сегмент
    с использованием техники~резки <<совмещенный рез>>
    }
  \label{cut4-3}
  \end{center}
\end{figure}

На основе концепции базового сегмента введем еще
два класса задач маршрутизации инструмента машин
фигурной листовой резки с ЧПУ

\begin{opred}
  \label{def:SCCP}
  {\it SCCP (Segment Continuous Cutting Problem)}
  --
  задача с фиксированным числом $K$
  сементов резки
  (и базовых сегментов резки
  $B^{S_k}, k = \overline{1, K}$).
\end{opred}

{\it Замечание}:
Если все граничные контуры деталей
$C_1, C_2 \,\dots, C_N$
это базовые сегменты $B^{S_k}, k = \overline{1,K}$,
и $N=K$,
тогда $SCCP$ эквивалентна $CCP$.

Предположим, что для исходной задачи маршрутизации
определен конечный набор (ансамбль)
базовых сегментов резки размерности $T$.
Этот ансамбль соответствует ансамблю задач
$\{SCCP_i, i =\overline{1,T}\}$.

\begin{opred}
  {\it GSCCP (Generalized SCCP)} -- есть
  $\{SCCP_i, i =\overline{1,T}\}$.
\end{opred}

Как нетрудно видеть,
введя классы $SCCP$ и $GSCCP$,
мы значительно расширили существующую
классификацию задач маршрутизации инструмента
для машин листовой резки с ЧПУ.
Фактически $SCCP$ и $GSCCP$ являются подклассами $ICP$,
содержащими все задачи с конечным набором базовых сегментов резки,
т. е.
$CCP \subset SCCP \subset GSCCP \subset ICP$.
Таким образом, в классе $ICP$ выбран большой подкласс
задач маршрутизации,
для которых можно разработать эффективные алгоритмы оптимизации.

На рис.~\ref{x-classify}
приведена расширенная классификация этих задач.
Как видно из данного  рисунка,
все классы сгруппированы в три группы,
различающиеся мощностью множеств,
из которых выбираются точки входа в контуры
(точки врезки).

\begin{figure}[h]
  \begin{center}
  \includegraphics[width=0.9\textwidth]{x-classify.png}
  \caption{Расширенная классификация задач маршрутизации инструмента машин листовой резки }
  \label{x-classify}
  \end{center}
\end{figure}

Далее рассмотрим подход,
основанный на  дискретизации трех клаcсов задач маршрутизации
первой группы ({\it CCP, SCCP, GSCCP})
и сведении их к задаче о последовательном обходе мегаполисов,
в которой  используется математическая модель А. Г. Ченцова,
описанная подробно во второй части настоящей монографии
(главы 3 -- 5).

Эта модель может быть интерпретирована
как математическая модель обобщенной задачи коммивояжера ($GTSP$)
с дополнительными ограничениями.
(Следует различать модель $GTSP$
и~задачу маршрутизации $GTSP$ из второй группы,
которая представляеи собой дискретный вариант задачи $CCP$).
В отличие от классического $GTSP$ эта модель
предусматривает учет так называемой внутренней работы
(в данном случае -- процесса резки).
Кроме того, модель мегаполисов с
использованием специальной схемы динамического программирования
учитывает сложные типы целевых функций и сложные ограничения,
в том числе динамические.
Наконец, принимая во внимание ограничения предшествования,
можно получить точные решения для дискретных вариантов $SCCP$
достаточно большой размерности.

В качестве примера рассмотрим задачу $GSCCP$,
которая содержит две
задачи $SCCP$ с разными наборами сегментов,
показанными на рис.~\ref{gsccp-both}
(21 базовый сегмент -- \ref{gsccp-a}
и 18 -- \ref{gsccp-b} соответственно).

\begin{figure}[h]
  \centering
  \subfigure[21 базовый сегмент]{
    \includegraphics[width=0.9\textwidth]{gsccp-a.png}
    \label{gsccp-a}
  }
  \subfigure[18 базовых сегментов]{
    \includegraphics[width=0.9\textwidth]{gsccp-b.png}
    \label{gsccp-b}
  }
  \caption{Пример задачи GSCCP с набором из двух комплектов базовых сегментов}
  \label{gsccp-both}
\end{figure}

Первый набор базовых сегментов,
рис.~\ref{gsccp-a},
задан всеми граничными контурами деталей, т. е.
$C_j = B^{S_j}, j=\overline{1,N}, N=K=21$.
В этом случае мы имеем классическую задачу $CCP$.
На рис.~\ref{gsccp-b} восемь
базовых сегментов заданы внешними граничными контурами <<серых>> деталей.
Три дополнительных базовых сегмента заданы шестью
внешними граничными контурами <<цветных>> деталей
(один базовый сегмент состоит из двух внешних контуров плюс перемычка между ними).
Эти контуры будут вырезаться <<цепной>> резкой попарно в одном базовом сегменте.
Наконец семь базовых сегментов заданы внутренними
граничными контурами всех деталей,
в которых имеются отверстия.
В целом, в этом случае мы имеем набор из восемнадцати
базовых сегментов.
Все они выделены цветом.
В качестве целевой функции для данного примера было выбрано
время процесса резки~(\ref{cutting-time}):
$$
  T_{cut} = \frac{L_{on}}{V_{on}} + \frac{L_{off}}{V_{off}} +N_{pt} \cdot t_{pt}
  .
$$

Чтобы свести непрерывные задачи $SCCP$
к дискретной модели,
каждый базовый сегмент делится с определенным шагом по точкам,
которые будут претендентами на точку входа инструмента в контур.
Каждая такая точка входа однозначно определяет точку врезки.
В то же время каждая возможная точка врезки должна
удовлетворять технологическим ограничениям процесса резки
(\ref{pierce-constraint}),
то есть многие точки базовых сегментов будут удалены.
Напомним, что для каждого базового сегмента
такая точка может быть только одна.
На рис.~\ref{discrete21}
показано конечное множество возможных точек врезки
(выделены зеленым цветом)
для первой задачи
(набор из двадцати одного базового сегмента).
Фактически мы свели первую задачу $SCCP$
(в данном конкретном случае эквивалентную $CCP$)
(рис.~\ref{gsccp-a})
к задаче о последовательном обходе мегаполисов (модели $GTSP$).
Обратите внимание,
что в этом случае число возможных маршрутов резки $ROUTE$
становится конечным.
Процесс дискретизации второй задачи SCCP,
рис.~\ref{gsccp-b},
производится аналогичным образом.
Для решения обеих задач применен метод динамического программирования,
использующий специальную схему Беллмана,
которая описана во второй части монографии.

\begin{figure}[h]
  \begin{center}
  \includegraphics[width=0.9\textwidth]{discrete21.png}
  \caption{Дискретизация задачи CCP, приведенная на рисунке \ref{gsccp-a}}
  \label{discrete21}
  \end{center}
\end{figure}

На рис. ~\ref{path21}
и~\ref{path18}
показаны оптимальные маршруты резки для двух задач SCCP
с двацать одним и восемнадцатью
базовыми сегментами соответственно.

\begin{figure}[h]
  \begin{center}
  \includegraphics[width=0.9\textwidth]{path21.png}
  \caption{
    Схема оптимальной траектории инструмента
    для 21 базового сегмента
    }
  \label{path21}
  \end{center}
\end{figure}

\begin{figure}[h]
  \begin{center}
  \includegraphics[width=0.9\textwidth]{path18.png}
  \caption{
    Схема оптимальной траектории инструмента
    для 18 базовых сегментов
    }
  \label{path18}
  \end{center}
\end{figure}

В первом случае время процесса резки
$T_{cut}$
составляет 2255 с,
во втором -- 2244 с.
Обратите внимание, что в первом случае длина рабочего хода инструмента
$L_{on}$
(20567 мм) меньше, чем во втором -- (20727 мм),
но из-за уменьшения количества точек врезки
для восемнадцати сегментов общее время резки также уменьшилось.
Еще раз отметим, что оба решения являются оптимальными
для выбранных наборов базовых сегментов.
Таким образом, оптимальное значение целевой
функции для выбранной задачи $GSCCP$ составляет 2244 с.

При решении задач были учтены необходимые <<статические>> ограничения:
условия предшествования и ограничения для координат
точек врезки (\ref{pierce-constraint}).
Динамические ограничения в этом модельном примере не рассматривались.

Описанный подход позволяет решать задачи из
наиболее сложного класса задач маршрутизации
траектории инструмента -- $ICP$,
который не ограничивает выбор точки входа
инструмента в контур детали и использование
любой техники резки.
Наиболее важной особенностью подхода
является возможность для одной задачи оптимизации
формировать разные наборы базовых сегментов и
применять разные алгоритмы оптимизации,
используя как дискретные, так и,
в некоторых случаях, непрерывные модели.



% !TeX root = ../mat_mod2.tex

% Часть II

%Part2
\part{
  Математические модели и~методы решения задач маршрутизации,
  связанных с~листовой резкой на машинах с~ЧПУ
}
%Часть II. Математические модели и методы решения задач %маршрутизации, связанных с листовой резкой на машинах с ЧПУ.



\chapter{
  АЛГОРИТМЫ РЕШЕНИЯ ЗАДАЧ МАРШРУТИЗАЦИИ С~ОГРАНИЧЕНИЯМИ
}
%\setcounter{section}{1}\setcounter{subsection}{1}
\setcounter{chapter}{5}
\setcounter{equation}{0}



\section{
  Введение
}
\label{sect:5.1}

В данной главе рассматриваются точные и эвристические алгоритмы решения задач
маршрутизации с ограничениями, а также различные комбинации таких алгоритмов.

Сначала приводятся точные алгоритмы на основе ДП, позволяющие получать
оптимальный результат. Размерность решаемых задач ограничена несколькими десятками
контуров, что, в принципе, соответствует некоторым типам раскроя. Проблема с
размерностью заключается в том, что данный алгоритм имеет экспоненциальный
рост времени счета и размера требуемой оперативной памяти.
Кроме того, в задачах оптимизации движения режущего инструмента важную роль
играет точное соблюдение всех ограничений;

Важный момент, который следует отметить особо, состоит в том,
что время счета и память сильно зависят от количества вложенных контуров, приводящих
к условиям предшествования. Здесь речь идет о деталях с внутренними отверстиями
и областями, и о других, меньших делалях, размещенных внутри этих областей.
В случае простых деталей, не имеющих врутренних вырезаемых областей или
отверстий, размерность решаемой задачи снижается, как показывает вычислительный
эксперимент, на несколько контуров за счет снижения количества условий
предшествования.

Принципиально иной подход к решению рассматриваемых задач состоит в применении
эвристических алгоритмов. Данные алгоритмы строятся на основе эмпирических
правил и не подтверждаются оценками. Следовательно, говорить об оптимальности
здесь невозможно. Более того, нельзя сказать, насколько тот или иной результат
отличается от оптимального значения. Можно лишь производить сравнение с результатами
для точных алгоритмов на примерах задач, имеющих сравнительно малую размерность.
Конечно, результаты работы эвристических алгоримов на примерах <<малых>>
задач могут сильно отличаться от аналогичных результатов на <<больших>> примерах
(сотни и тысячи контуров). Следовательно, упомянутое сравнение может дать лишь
общее представление о работе того или иного эвристического алгоритма.
Сильной стороной эвристических алгоритмов является их высокая производительность
и, во многих случаях, полиномиальная зависимость времени счета и объема
требуемой памяти от размерности задачи. Кроме того, эвристические алгоритмы
позволяют в полной мере учитывать детально проработанные тепловые и геометрические
ограничения.

Третий подход к решению задач реализации раскроя состоит в использовании
вставок на основе ДП. Сначала производятся вычисления
с использованием эвристического алгоритма, после чего участки маршрута
реорганизуются с использованием ДП (данные вставки охватывают участки маршрута
с размером, приемлемым для ДП). Такой подход уже не может претендовать на
оптимальность, точных оценок здесь также нет, что автоматически причисляет
его к разряду эвристических. Основная идея состоит в том, что мы применяем
к участкам маршрута и трассы гарантированно неухудшающие преобразования.
Здесь опять возникают проблемы, присущие методу ДП, --- малая размерность
вставок и большое время счета.

Для учета тепловых и геометрических ограничений могут быть применены
два подхода (в обоих используется зависимость от списка вырезанных к данному
моменту контуров). В первом реализуется дополнительное жесткое ограничение,
запрещающее перемещение на тот или иной контур, не согласующийся с тепловыми
или геометрическими ограничениями. Во втором подходе такое перемещение разрешено,
но к функции стоимости перемещения добавляется дополнительный штраф (большое
число). В настоящей монографии используется второй подход.
Рассматриваемая в предыдущих главах схема решения на основе ДП
допускает применение функций стоимости такого типа и вполне реализуема для
задач малой размерности. При решении же практических задач, связанных с
листовой резкой и имеющих ощутимую размерность, серьезную проблему
представляет само построение функций стоимости с упомянутой особенностью
(фактическая размерность резко возрастает при учете зависимости от списков
заданий). В настоящей главе приведен эффективный эвристический алгоритм, не
предусматривающий априорного построения упомянутых усложненных функций
стоимости и реализующий построение их фрагментов по мере развития процесса.
Данный алгоритм пригоден для решения маршрутных задач с ограничениями
различных типов, имеющих достаточно большую размерность.

Все вычисления производились на ПЭВМ с процессором Intel i7-2630QM с 8Гб
оперативной памяти, работающей под управлением Windows 7 (64-bit). Для
разработки программы была использована среда Microsoft Visual C++ 2013.

\section{
  Задача маршрутизации перемещений (частная постановка)
}
\label{sect:5.2}
\setcounter{equation}{0}

В настоящем разделе мы следуем конструкциям главы~3 и рассматриваем далеко не самый
общий случай постановки, ориентированной на применение  в задаче маршрутизации
движения инструмента при листовой резке на машинах с ЧПУ. Содержательное обсуждение
задачи приведено в главе~3 и в первой части монографии. Сейчас мы остановимся только
на одном моменте, связанном с ограничениями, придерживаясь содержательного способа
изложения.

Мы полагаем (как и в главе~3) заданными мегаполисы $M_1,\ldots,M_N,$ которые в
нашей конкретной задаче являются непустыми конечными множествами на плоскости,
располагающимися на некоторых <<вторичных>> эквидистантах. Имеются  отношения
$\bbm_1,\ldots,\bbm_N;$ при этом $\bbm_j\subset M_j\times M_j.$ Элементами
$\bbm_j,$ где $j\in \ov{1,N},$ являются упорядоченные пары $(x,y),$ где $x$ ---
возможная точка врезки, а $y$ --- точка выключения инструмента; разумеется, $x$
и $y$ --- суть плоские векторы. В этой связи отметим, что в качестве объемлющего
множества $X$ далее используется плоскость:
$$X = \bbr\times \bbr$$
(в качестве $X$ может использоваться также достаточно большой прямоугольник на
плоскости). Сохраняем определения множеств $\mathbf{M}_1,\ldots,\mathbf{M}_N,
\bbx$ и $\mathbf{X},$ которые были введены в главе~3. В настоящем разделе
предполагаем заданными операторы (\ref{3.3.9}); каждый из этих операторов
сопоставляет точке множества $\mathbf{X}$ непустое конечное множество на
плоскости, причем, как уже отмечалось в главе~3 (см. замечание после
(\ref{3.3.11})) в определении $A_1,\ldots,A_N$ допускается определенная
избыточность; существенная <<часть>> данных определений реализована (для
одного из возможных вариантов) в замечании~\ref{z3.3.2}.

В непосредственных расчетах использовался несколько иной вариант определения
$A_1,\ldots,A_N$ (ограничиваемся определением значений, существенных для
рассматриваемой маршрутной задачи).
Пусть
$$\rho:\,X\times X \longrightarrow [0,\infty[$$
есть обычная евклидова метрика на $X;$ при $x_1\in X$ и $x_2\in X$ число
$$\rho(x_1,x_2)\in [0,\infty[$$
определяет, следовательно, евклидово   расстояние между плоскими векторами
$x_1$ и $x_2.$ Рассмотрим нужный вариант, фиксируя $j\in\ov{1,N}$ и полагая
(это достаточно для всех наших целей), что $x\in \mathbf{X}\setminus
\mathbf{M}_j.$ По самому смыслу оператора $A_j$ (\ref{3.3.9}) имеем, что
множество $A_j(x)$ исчерпывает возможности выбора новой точки врезки при
условии перемещения в мегаполис из состояния $x,$ которое не принадлежит
$\mathbf{M}_j:$ мы имеем возможность выбирать только векторы $y\in A_j(x).$
Одно из соображений, определяющих упомянутое ограничение, может состоять в
том, чтобы новая точка врезки выбиралась в отдалении от предыдущей. Здесь,
конечно, надо иметь в виду, что реально у нас либо $x=x^o,$ либо $x$ является
точкой выключения инструмента и может не совпадать с предыдущей точкой врезки.
Однако в рассматриваемой задаче <<парные>> точка врезки и точка выключения
(инструмента) близки, а потому можно считать, что предыдущая точка врезки
находится вблизи $x$ и, удаляясь от $x,$ мы удаляемся и от упомянутой предыдущей
точки врезки.

Будем полагать заданным некоторое число $\mathbf{r}_j(x)\in [0,\infty[$
(пороговый уровень) и весовой коэффициент $\xi_j(x) \in ]0,1];$ введем
значение
$$d_j(x) \df \max\limits_{h\in \bbm_j}\rho\bigl(x,\mathrm{pr}_1(h)\bigl);$$
в нашем случае $d_j(x) > 0.$ Определяем множество $A_j(x)$ следующими тремя
выражениями:
$$\bigl(d_j(x) < \mathbf{r}_j(x)\bigl)\Longrightarrow \Bigl(A_j(x) \df
\bigl\{\mathrm{pr}_1(y):\,y\in \bbm_j,\ \xi_j(x) d_j(x) \leqslant \rho
\bigl(x,\mathrm{pr}_1(y)\bigl)\bigl\}\Bigl),
$$
$$\biggl(\Bigl(\min\limits_{h\in \bbm_j}\rho\bigl(x,\mathrm{pr}_1(h)\bigl)
\leqslant \mathbf{r}_j(x)\Bigl)\,\&\,\bigl(\mathbf{r}_j(x)\leqslant d_j(x)
\bigl)\biggl) \Longrightarrow $$ $$\Longrightarrow\Bigl(A_j(x) \df \bigl
\{\mathrm{pr}_1(z):\,z\in \bbm_j,\ \mathbf{r}_j(x) \leqslant \rho\bigl(x,
\mathrm{pr}_1(z)\bigl)\bigl\}\Bigl),
$$
$$\Bigl(\mathbf{r}_j(x) < \min\limits_{h\in\bbm_j}\rho\bigl(x,
\mathrm{pr}_1(h)\bigl)\Bigl)\Longrightarrow \bigl(A_j(x) \df
\{\mathrm{pr}_1(h):\,h\in \bbm_j\}\bigl).
$$
В первом из соотношений говорится о <<хоть каком-то>> уклонении от предыдущей
точки врезки (число $\xi_j(x)$ логично выбирать близким к 1); речь идет об
обработке <<плохих>>  в упомянутом смысле случаев. Второе соотношение
характеризует пороговое разделение множества новых точек врезки на два
подмножества, одно из которых и трактуется как $A_j(x).$ Наконец, третье
соотношение отвечает <<хорошему>> в смысле упомянутой удаленности случаю,
когда любые точки врезки нового мегаполиса доступны для перемещения.

Заметим, что упомянутое <<тройственное>> определение обеспечивает условие
$A_j(x) \neq \emp,$ хотя при $d_j(x) < \mathbf{r}_j(x)$ за это условие
приходится <<расплачиваться>> качеством реализации порогового уровня
$\mathbf{r}_j(x);$ по сути дела мы действуем в случае упомянутого строгого
неравенства по остаточному принципу. Тем не менее определенные меры в части
осуществления удаленности новой точки врезки от предыдущей удается реализовать
и обеспечить тем самым постановку задачи (\ref{3.3.31}) в ее конкретном варианте.

При построении алгоритма использовались соглашения:
$$\bigl(\mathbf{r}_j(x) = \mathbf{r}\ \ \fa j\in \ov{1,N}\ \ \fa x\in
\mathbf{X}\setminus \mathbf{M}_j\bigl)\,\&\,\bigl(\xi_j(x) = \xi\ \ \fa
j\in \ov{1,N}\ \ \fa x\in \mathbf{X}\setminus \mathbf{M}_j\bigl)
$$
(напомним, что при доопределении вышеупомянутых <<тройственных>> правил до
отображений на всем множестве $\mathbf{X}$ использовались соображения, подобные
изложенным в разделе~3.3 после соотношения (\ref{3.3.11})); здесь $\mathbf{r}>0$
и $1 \geqslant\xi >0.$

Рассматриваемые в книге примеры связаны с машинами резки металла на станках с ЧПУ.
Функции стоимости представляют из себя не расстояния, а времена движения инструмента.
Все перемещения осуществляются либо на скорости холостого хода $V_i,V_i>0$,
либо на скорости реза
$V_c, V_c > 0$.
Скорость холостого хода превышает скорость
реза в десятки и сотни раз. В примерах $V_i=500$ мм/с, а $V_c=10$ мм/с.
Движение предполагается безинерционным.
Точка $x^0$ задается в виде начала координат: $x^0=(0,0)$.
Для $x_1\in \mathbf{X}$ и $x_2\in \mathbb{X}$ полагаем
\begin{equation}\label{ExtPrice}
\mathbf{c}(x_1,x_2)=\frac{\rho (x_1,x_2)}{V_i}.
\end{equation}

Функции стоимости внутренних работ вычисляются с учетом скорости реза.
Именно, для $i\in \overline{1,N}$, $z\in \mathbb{M}_i$
$$
c_i(z)=\frac{\rho(\mbox{pr}_1(z),u(z))+\rho(u(z),\mbox{pr}_2(z))}{V_c},
$$
где вектор $u(z)$ находится на эквидистанте и сопоставляется $z$ однозначно.
Функция $f$ также определяется временем движения в финишную точку.
\begin{equation}\label{TerminalPrice}
f(x)=\frac{\rho (x,x^0)}{V_i}.
\end{equation}
%^^^^^^^^^^^^^^^^^^^^^^^^^^^^^^^^^^^^^^^^^^^^^^^^^^^^^^^^^^^^^^^^^^^^^^^^^^^^^^^

%*******************************************************************************
В примере рассматривался случай $N=29$ и $|\mathbf{K}| =30$. Использовалось
<<тройственное>> определение операторов $A_1,\ldots,A_N$ при следующей
конкретизации параметров:
$$
\mathbf{r}=100,\ \xi=0,75.
$$
Точное описание мегаполисов не приводится по соображениям объема; отметим
только, что мощности $|M_i|, i\in \overline{1,29}$, имеют значение от 3 до 30.

\begin{figure}
  \begin{center}
  \includegraphics[width=0.9\textwidth]{route_29_DP_AD.png}
  \caption{Траектория, построенная на основе ДП}
  \label{DP_Result}
  \end{center}
\end{figure}

Результат счета 100,61, время счета 13 мин. 13 сек. Маршрут и трасса показаны
на рис. \ref{DP_Result}.

\section{
  Итерационный режим с комбинированием оптимизирующих вставок разной <<длины>>
}
\label{sect:5.3}
\setcounter{equation}{0}

Начиная с настоящего раздела мы рассматриваем конструкции решения задачи
(\ref{4.4.13}) с использованием оптимизирующих вставок. Основное внимание
здесь и далее уделяется итерационным алгоритмам на основе подходов,
сформулированных в заключении предыдущей главы. Будем придерживаться
обозначений раздела~4.4 в части описания элементов <<большой>> задачи
(см. (\ref{4.4.13})), а, в части описания локальных задач, --- обозначений
раздела~4.3. Итак, схема ДП раздела~4.3 используется при построении
оптимизирующих вставок, направленных на обеспечение   локального улучшения
ДР <<большой>> задачи. Последняя рассматривалась на уровне построения
алгоритмов и программ в реализации, ориентированной на применение для
целей маршрутизации движения инструмента при листовой резке на станках
с ЧПУ.

Сначала совсем кратко напомним итерационную процедуру, в рамках которой
предполагалось сочетание оптимизирующих вставок разной <<длины>>. Как и
в разделе~4.4, полагаем, что $\nn\in \bbn$ определяет количество мегаполисов
<<большой>> задачи; значение $\nn$ полагается достаточно большим, а потому
непосредственное применение ДП (см. раздел~4.3) для целей <<глобальной>>
оптимизации практически невозможно. Пусть, однако, задано число
$N\in \bbn,$ определяющее размер или <<длину>> вставки; при этом
в последующих конструкциях полагается, что при данном значении $N$ в
локальной задаче маршрутизации, возникающей при построении вставки,
процедура на основе ДП уже позволяет осуществить вычисления, приводящие
к построению оптимального локального решения (имеется в виду вариант
задачи (\ref{4.3.3})) и нахождению значения (\ref{4.3.5}), то есть
экстремума локальной задачи.

Некоторую проблему составляет выбор начала вставки, которое в разделе~4.4
было обозначено через $\nu$ (см. (\ref{4.4.24})). Конечно, данный параметр
можно выбрать как угодно, соблюдая (\ref{4.4.24}); при этом, как показано в
главе~4, значение критерия <<большой>> задачи не ухудшится. Однако желательно
устраивать вставку там, где она обеспечит реальное улучшение совокупного
критерия; таким образом, <<момент>> начала вставки надо искать, исходя из
соображений, связанных с улучшением критерия. В конце главы~4 были намечены
некоторые подходы к выбору параметра $\nu.$ Все эти подходы связывались с
организацией той или иной итерационной процедуры. Мы начнем сейчас с
рассмотрения того варианта данной процедуры, когда для поиска $\nu$
привлекается система вставок меньшего размера. Последнее существенно, так
как на каждом этапе основной итерационной процедуры  приходится (в рамках
данного подхода) проигрывать реализацию поисковых вставок многократно.
Цель этих построений, как  уже отмечалось в главе~4, состоит в определении
такого значения $\nu,$ для которого исходное эвристическое ДР было бы
наиболее <<податливо>> к улучшению. Последнее же предлагается осуществлять
с помощью вставки большего размера (большей <<длины>>). Итак, мы рассматриваем
вставки разной <<длины>>, придавая значению $N$ одно из двух значений
$$(N = N_1) \vee (N=N_2),
$$
где $2 \leqslant N_1 < N_2 <\nn.$ Соответственно у нас на каждом этапе
возникают (зондирующие) $N_1$-вставки и одна рабочая $N_2$-вставка.

При этом, как уже отмечалось в заключительной части главы~4, значение
$N_1$ подбирается таким, чтобы было возможно <<прорешать>> все локальные
маршрутные задачи в диапазоне $\ov{0,\nn-N_2}.$ Правда при решении упомянутых
локальных задач не требуется находить оптимальные маршрут и трассу; требуется
только <<быстро>> определять экстремумы; в настоящем варианте для этого
использовалось ДП в духе процедуры (\ref{4.3.13}), финалом которой является
определение экстремума локальной маршрутной задачи.

%*******************************************************************************
В примере рассматривался случай $\mathbf{n}=75$ и $|\mathbf{K}|=37$. Размеры вставок
были выбраны следующими: $N_1=12$, $N_2=24$.
Функции стоимости выбраны такими же, как в предыдущем примере (из раздела
5.2). Количество итераций равнялось
15. Начальный вариант посчитан с использованием эвристического итерационного
алгоритма, значение полученного критерия равно 250,79. Результат, достигнутый
после применения вставок ДП равен 249,3. Общее время счета составляет 17
мин. 3 сек. Маршрут и трасса показаны на рис. \ref{DP_Inserts_Result}.

\begin{figure}
  \begin{center}
  \includegraphics[width=0.9\textwidth]{routing_75_checking_ins.png}
  \caption{
    Траектория, полученная с использованием эвристического алгоритма
    с~улучшающими вставками ДП
    }
  \label{DP_Inserts_Result}
  \end{center}
\end{figure}

\section{
  Итерационный режим с элементами оптимизации локальных условий
  прешествования
}
\label{sect:5.4}
\setcounter{equation}{0}

В настоящем разделе обсудим вариант итерационной процедуры раздела~4.8,
предложенный А.А.Ченцовым и связанный с использованием задачи (\ref{4.8.5}).
Мы сохраняем символику главы~4 как в части описания <<большой>> задачи, так
и в части описания локальных задач типа (\ref{4.2.5}). Как уже отмечалось
в разделе~4.8, данный вариант ориентирован на построение оптимизирующих вставок,
в которых количество адресных пар должно быть достаточно большим; это позволяет
в определенной степени снижать сложность вычислений по методу ДП.

Введем в рассмотрение множество
\bfn\label{5.4.1}\bbd \df \bigl\{\bigl(\al,(z_i)_{i\in\ov{0,\nn}})\in \ca
\times \mathfrak{Z}|\,(z_i)_{i\in\ov{0,\nn}}\in \mathfrak{Z}_\al\bigl\}
\efn
всех ДР <<большой>> задачи, после чего на множестве
$$\bbd \times \ov{0,\mathbf{k}},
$$
где $\mathbf{k}\in \bbn,\,\mathbf{k}\leqslant \nn-N,$ фиксировано,
определим многозначное  отображение
\bfn\label{5.4.2}\Phi:\,\bbd\times \ov{0,\mathbf{k}}\longrightarrow
\cp^\prime(\ov{0,\mathbf{k}}).
\efn
Отображение (\ref{5.4.2}) может, в принципе, определяться по разному, но
мы пока зафиксируем какой-либо вариант $\Phi$ с тем, чтобы изложить логику
возникающей итерационной процедуры, после чего конкретизируем $\Phi$
(\ref{5.4.2}).

Итак, пусть сейчас $\Phi$ --- любое многозначное отображение (мультифункция)
(\ref{5.4.2}). Допустим, что тем или иным способом мы смогли определить
некоторое начальное ДР
\bfn\label{5.4.3}
(\la_o,h_o)\in \bbd
\efn
(для построения данного ДР может использоваться тот или иной эвристический
(в частности, жадный) алгоритм); выберем, кроме того, $\nu_o\in \ov{0,\mathbf{k}}.$
В результате реализуется (непустое) множество
$$\Phi(\la_o,h_o,\nu_o) = \Phi\bigl((\la_o,h_o),\nu_o\bigl)\in
\cp^\prime(\ov{0,\mathbf{k}}),
$$
после чего  осуществляется выбор индекса
\bfn\label{5.4.3`}\nu_1\in \Phi(\la_o,h_o,\nu_o),
\efn
определяющего конкретное начало вставки. Заметим, что в (\ref{5.4.3`}) может
использоваться то или иное оговоренное заранее правило выбора. Так, например,
мы можем действовать по правилу: в качестве $\nu_1$ выбираем наименьший (наибольший)
элемент множества $\Phi(\la_o,h_o,\nu_o).$ После того, как выбор (\ref{5.4.3`})
осуществлен, используем процедуру, изложенную в разделах~4.4--4.7, при условиях
\bfn\label{5.4.4}\la = \la_o, (\mathbf{h}_i)_{i\in\ov{0,\nn}}= h_o, \nu = \nu_1
\efn
((\ref{4.4.13}), (\ref{4.4.14}) и (\ref{4.4.24}) применяем в условиях (\ref{5.4.4})).
После реализации процедуры разделов~4.4--4.7 (при условиях (\ref{5.4.4})) получаем
новое ДР
\bfn\label{5.4.5}\bigl(\eta,(\hat{\mathbf{h}}_t)_{t\in\ov{0,\nn}}\bigl)\in \bbd
\efn
<<большой>> задачи, которое мы принимаем за $(\la_1,h_1);$ итак, полагаем
\bfn\label{5.4.6}\la_1 = \eta,\,h_1 = (\hat{\mathbf{h}}_t)_{t\in\ov{0,\nn}},
\efn
что соответствует построениям разделов~4.5-- 4.7. Напомним, что при этом (см.
(\ref{4.7.49}) и теорему~\ref{t4.6.1})
$$
\widehat{\mathfrak{C}}_\eta[(\hat{\mathbf{h}}_t)_{t\in\ov{0,\nn}}] \leqslant
\widehat{\mathfrak{C}}_\la[(\mathbf{h}_i)_{i\in\ov{0,\nn}}],
$$
что при наших условиях (\ref{5.4.6}) приводит к неравенству
\bfn\label{5.4.7}\widehat{\mathfrak{C}}_{\la_1}[h_1] \leqslant
\widehat{\mathfrak{C}}_{\la_o}[h_o]
\efn
(конкретная степень <<улучшения>> значений критерия указана в (\ref{4.7.49})
и теореме~\ref{t4.6.1}). Теперь мы получаем множество
$$\Phi(\la_1,h_1,\nu_1) \in \cp^\prime(\ov{0,\mathbf{k}}).
$$
Поскольку, в частности, $\Phi(\la_1,h_1,\nu_1)\neq \emp,$ осуществляем выбор
\bfn\label{5.4.8}\nu_2\in \Phi(\la_1,h_1,\nu_1),
\efn
используя то или иное правило (см. обсуждение после (\ref{5.4.3`})). Теперь
применяем процедуру, изложенную в разделах~4.4--4.7 при условиях, что
$$\la =\la_1,(\mathbf{h}_i)_{i\in\ov{0,\nn}}= h_1, \nu= \nu_2,
$$
получая в результате новый вариант ДР (\ref{5.4.5}) <<большой>> задачи,
обозначаемый сейчас через $(\la_2,h_2),$ что соответствует конкретизации
\bfn\label{5.4.9}
\eta = \la_2,(\hat{\mathbf{h}}_t)_{t\in\ov{0,\nn}}= h_2.
\efn
Итак, $(\la_2,h_2)$ получается в виде ДР, улучшающего $(\la_1,h_1):$
\bfn\label{5.4.10}
\widehat{\mathfrak{C}}_{\la_2}[h_2] \leqslant
\widehat{\mathfrak{C}}_{\la_1}[h_1].
\efn
Далее процесс повторяется заданное (или  выбираемое по мере реализации
алгоритма) число раз; обозначая данное число через $\mathbf{r},$ мы можем
говорить о кортеже
$$
s \longmapsto (\la_s,h_s):\,\ov{1,\mathbf{r}}\longrightarrow \bbd
$$
ДР <<большой>> задачи, получаемом в результате повторения процедуры
разделов~4.4--4.7 $\mathbf{r}$ раз, где $\mathbf{r}\in \bbn.$ Ясно, что
\bfn\label{5.4.11}
\widehat{\mathfrak{C}}_{\la_\mathbf{r}}[h_\mathbf{r}] \leqslant
\widehat{\mathfrak{C}}_{\la_o}[h_o]
\efn
(действительно, имеем, что $\widehat{\mathfrak{C}}_{\la_j}[h_j]\leqslant
\widehat{\mathfrak{C}}_{\la_{j-1}}[h_{j-1}]\ \ \fa j\in \ov{1,\mathbf{r}}).$
Разумеется, выбирая тот или иной вариант отображения $\Phi,$ мы получаем
(в смысле (\ref{5.4.11})) различные результаты.

Теперь мы рассмотрим конкретный вариант $\Phi,$ имея в виду применение
в алгоритме, предложенном А.А. Ченцовым. Для этого сначала введем при
$(\la,h)\in \bbd$ и $\nu\in \ov{0,\mathbf{k}}$ множество
\bfn\label{5.4.12}
\Psi_\nu[\la;h] \df \bigl\{z\in \mathfrak{K}|\,\exists\,t_1\in
\ov{1,N}\ \ \exists\,t_2\in \ov{1,N}:\,z = \bigl(\la(\nu+t_1),
\la(\nu+t_2)\bigl)\bigl\},
\efn
где $\mathfrak{K}$ --- множество всех адресных пар <<большой>>
задачи, а число $N$ соответствуют разделам~4.4--4.7 (в рассматриваемом
ниже алгоритме число $N\in \bbn$ не изменяется, в отличие от построений
предыдущего раздела). Поскольку (\ref{5.4.12}) --- конечное множество,
определена мощность
$$
|\Psi_\nu[\la;h]|\,\in \bbn_o.
$$
Теперь мы при $\mathbf{k}\geqslant 2$ конкретизируем выбор отображения
$\Phi$ следующим образом: полагаем при $(\la,h)\in \bbd$ и $\nu\in
\ov{0,\mathbf{k}},$ что
\bfn\label{5.4.13}
\Phi(\la,h,\nu) = \bigl\{\nu_o\in \ov{0,\mathbf{k}}\setminus
\{\nu\}\bigl|\,|\Psi_{\bar{\nu}}[\la;h]|\leqslant |\Psi_{\nu_o}[\la;h]|\ \
\fa \bar{\nu}\in \ov{0,\mathbf{k}}\setminus \{\nu\}\bigl\}.
\efn
Правило (\ref{5.4.13}) состоит очевидно в следующем: мы при заданном
ДР $(\la,h)\in \bbd$ назначаем такие изменения
\bfn\label{5.4.14}\nu\longrightarrow \nu_o,
\efn
при которых (см. (\ref{5.4.12})) <<окно>>, определяемое началом $\nu_o,$
содержит наибольшее число адресных пар исходной <<большой>> задачи.
При изменениях (преобразованиях) (\ref{5.4.14}) мы стремимся обеспечить
лучшую <<просчитываемость>> локальной задачи, имея в виду использование
условий предшествования <<в положительном направлении>>.



Пусть $X=\mathbb{R}\times\mathbb{R}$, $\mathbf{n}=60$,
$\mathbf{x}_0=(0,0)$,
$\vert \mathfrak{K} \vert=47$. Функцию $\mathbf{c}^{\natural}$ определяем посредством евклидова расстояния,
а значения функций $c_j^{\natural},\;j \in \overline{1,\mathbf{n}}$, --- в виде суммы евклидова расстояния
от пункта прибытия до заданного (и зависящего от $j$) плоского вектора и аналогичного расстояния от данного
вектора до пункта отправления. Алгоритмические конструкции были реализованы
А.А.Ченцовым в виде программы для ЭВМ, написанной на языке программирования C++
и работающей под управлением 64-х разрядной операционной
системы семейства Windows; вычислительная часть программы реализована в отдельном от интерфейса пользователя потоке.
По соображениям объема ограничимся изложением результатов вычислений на ПЭВМ
(в ходе вычислительного эксперимента применялся ноутбук с центральным процессором Core i7, объемом ОЗУ 6 гБ
с установленной операционной системой Windows 7 Максимальная Sp1), опуская описание мегаполисов и адресных пар;
начальное решение (пара маршрут--трасса) определялось жадным алгоритмом, соблюдающим условия предшествования
и реализующим значения критерия 4858.68 в <<незамкнутой>> задаче (функция $f^{\natural}$ тождественно равна нулю)
и 4988.18 в <<замкнутой>> задаче (значения $f^{\natural}$ определялись евкидовым расстоянием до <<нуля>>).

Итак, исходному решению и ранее используемому <<моменту>> начала
вставки сопоставляется отличное от данного <<момента>> начало вставки, максимизирующее
число адресных пар, занумерованных в соответствии с прежним маршрутом и попадающих в
окно предполагаемой вставки. В процессе счета для <<незамкнутой>> задачи
реализовались следующие показатели:

при $N=20$ результативных итераций 4, стабилизация наступила при значении критерия 4500.71;

при $N=25$ результативная итерация одна, стабилизация наступила при достижении результата 4456.51;

при $N=27$ результативная итерация одна, а достигнутое значение критерия $4446.77$;

при $N=30$ имеем одну результативную итерацию и значение критерия 4446.77.

Среднее время счета последовательно нарастало (при $N=20$ время одной итерации 36--45 сек, при $N=25$ ---
чуть более 10 минут, при $N=27$ --- уже 46 мин. 11 сек., а при $N=30$ время счета составило 5 час. 4 мин. 16 сек.).


В <<замкнутой>> задаче аналогичные показатели были следующими:

при $N=20$ результативных итераций 4, стабилизация наступила по достижении результата $4630.21$;

при $N=25$ результативная итерация одна, результат 4586.01;

при $N=27$ результативных итераций две, а достигнутый при этом результат 4576.26;

при $N=30$ имеем одну результативную итерацию и значение критерия 4576.26.

Среднее время счета сохраняло тенденцию <<незамкнутой>>, задачи (27-45 сек, 10-11 минут,
46 мин. 22 сек., 38 мин. 34 сек. в третьем случае, т.е. при $N=27$, и 5 час. 3 мин. 2 сек при $N=30$).

Таким образом, увеличивая $N$ (длина вставки), мы последовательно улучшаем
результат, затрачивая, однако, все большее время. Еще одно обстоятельство,
проявившееся в ходе вычислительного эксперимента, состоит в том, что при
$N=30$ (сравнительно большая длина вставки) получается всего одна результативная
итерация, после чего итерационная процедура стабилизируется. Это обстоятельство
можно рассматривать как некоторый косвенный <<индикатор>> приближения к
оптимальности.

\section{
  Итерационный режим со случайным расположением вставок фиксированной <<длины>>
}
\label{sect:5.5}
\setcounter{equation}{0}

Один из возможных подходов к решению проблемы выбора расположения оптимизирующей
вставки связан с организацией случайных испытаний. Точнее, имеется в виду
случайный выбор начала вставки. На этой основе был построен вариант итерационного
алгоритма, краткое изложение которого приведено в настоящем разделе.

Итак, фиксируем $\mathbf{k}\in \overline{0,\mathbf{n}-N}$ и допускаем возможность
случайного выбора начала вставки из <<интервала>> $\overline{0,\mathbf{k}}$. Такой
выбор осуществляется многократно, что приводит к реализации итерационной
процедуры. Количество итераций может задаваться заранее или определяться
по мере достижения заданного качества.

Рассмотрим несколько этапов процедуры, полагая, что на каждом из них выбор
начала вставки осуществляется на основе испытаний при использовании равномерного
распределения; значение $N$ фиксировано. Итак, пусть число $\nu_1\in \overline{0,\mathbf{k}}$
получено как результат случайного испытания. Далее используем процедуру,
изложенную в разделах 4.4-4.7 при условиях (\ref{5.4.4}), где $(\lambda_0,h_0)$
соответствует (\ref{5.4.3}) и является ДР, определенным с помощью какого-либо
эвристического алгоритма. Мы используем (\ref{4.4.13}), (\ref{4.4.14}) и
(\ref{4.4.24}) в условиях (\ref{5.4.4}); получаем ДР (\ref{5.4.5}) исходной
<<большой>> задачи, которое принимается за $(\lambda_1,h_1)$ (см. (\ref{5.4.6})),
причем реализуется неравенство (\ref{5.4.7}).

Теперь осуществляется случайное испытание, результатом которого является
$\nu_2\in \overline{0,\mathbf{k}}$. Вновь применяется процедура, изложенная
в разделах 4.4-4.7 при условиях $\lambda=\lambda_1$,
$(\mathbf{h}_i)_{i\in \overline{0,\mathbf{n}}}=h_1$ и $\nu=\nu_2$ и доставляющая
новое ДР $(\lambda_2,h_2)$ <<большой>> задачи. При этом реализуется неравенство
(\ref{5.4.10}).

Далее процедура повторяется, доставляя в конце концов неравенство (\ref{5.4.11}),
где $\mathbf{r}\in \mathbb{N}$ определяет количество итераций.

В примере рассматривался случай $\mathbf{n}=75$ и $|\mathbf{K}|=42$. Размер вставок
был выбран следующим: $N=23$. Функции стоимости предполагались такими же, как в разделе
5.2. Количество итераций равнялось 15. Начальный вариант был посчитан с использованием
эвристического итерационного алгоритма, значение полученного критерия равно 259,51.
Результат, достигнутый после применения вставок на основе ДП равен 257,16. Общее время
счета составляет 16 мин. 11 сек. Маршрут и трасса показаны на рис.
\ref{DP_Random_Inserts_Result}.

\begin{figure}
  \begin{center}
  \includegraphics[width=0.9\textwidth]{route_random_dp_insertions.png}
  \caption{
    Траектория, полученная с использованием эвристического алгоритма
    с~улучшающими вставками на основе ДП (случайное расположение вставок)}
  \label{DP_Random_Inserts_Result}
  \end{center}
\end{figure}

\section{
  Один вариант жадного эвристического алгоритма
}
\label{sect:5.6}
\setcounter{equation}{0}

В задачах раскроя количество контуров нередко превышает несколько сотен.
Для решения таких задач предлагается использовать эвристические методы.
Кроме того, в задачах раскроя важную роль играют тепловые и
геометрические ограничения. Несоблюдение этих ограничений делает бессмысленной
оптимизацию длины пути.

Рассмотрим один из подходов к построению эвристического алгоритма для решения
задач большой размерности при наличии упомянутых ограничений. В данном подходе
оптимизирующие вставки не используются.

В качестве $X$ предполагается заданным достаточно большой прямоугольник
на плоскости, соответствующий листу металла. Процедура раскроя полагается
выполненной. В результате в пределах $X$ размещены контура деталей,
подлежащие резке по некоторым эквидистантам в виде замкнутых кривых.
С внешней по отношению к деталям стороны вблизи каждой эквидистанты
размещены конечные множества --- мегаполисы. Точки мегаполисов группируются
в упорядоченный пары. Элементами каждой такой упорядоченной пары являются точка
врезки и точка выключения инструмента. Каждой такой упорядоченной паре
сопоставляется также точка на эквидистанте, определяющая начало и завершение
реза соответствующего контура.

Как и в предыдущих разделах, фиксируются скорости холостого хода
и реза $V_i>0$ и $V_c>0$ соответственно. Значения функции $\mathbf{c}$
определяются соответствующим временем внешних перемещений. При этом точка
$x^0$ задается в виде начала координат: $x^0=(0,0)$. Функция $\mathbf{c}$
та же, что и в (\ref{ExtPrice}). Функция $f$ та же, что и в
(\ref{TerminalPrice}).

Перейдем к рассмотрению функций стоимости внутренних работ $c_1,...,c_N$,
в которых используется зависимость от списка заданий.

Пусть $j\in \overline{1,N}$, $z=(x,y)\in \mathbb{M}_j$ и
$u=u(z)$ --- отвечающая паре $z$ точка на $j$-й эквидистанте.
Кроме того, через $\tilde{D}$, $\tilde{D}\stackrel{\triangle}{=}\{0;1\}$,
обозначим множество, соответствующее возможным направлениям реза, причем $1$
определяет направление реза по часовой стрелке, $0$ --- против
часовой стрелки. Скорость реза здесь также совпадает с $V_c, V_c>0$.

Важным ограничением при резке металла является требование достаточного
количества металла возле завершающего участка реза --- рассеивание тепла
особенно важно на финишном участке.
В свете этого ограничения большую значимость приобретает направление реза.
Дело в том, что при резе в разных направлениях участок завершения реза
также оказывается с разных сторон от точки начала реза; при этом ситуация
с количеством металла с этих разных сторон может сильно различаться.

Условимся относительно следующих обозначений, применяемых ниже в целях формализации
содержательных понятий <<много металла>> и <<мало металла>>. Итак, если
$j\in \overline{1,N}$, $z\in \mathbb{M}_j$, $d\in \tilde{D}$ и
$\mathbb{K}\in \mathcal{P}(\overline{1,N}\setminus \{j\})$, то через
$S_1(j,z,d,\mathbb{K})$ обозначаем площадь пересечения заданной
области вокруг участка завершения реза и области имеющегося металла, а через
$S_2(j,z,d,\mathbb{K})$ --- площадь пересечения упомянутой области
и пустот, образовавшихся после вырезания контуров с номерами из $\mathbb{K}$,
а также внешнего пространства, выходящего за края листа. Под областью возле
участка завершения реза подразумевается геометрическая фигура, лежащая вне
детали, и такая, что расстояние от любой ее точки до ближайшей точки участка
завершения реза не превышает заданной величины (см. рис. \ref{FinishCutArea}).

\begin{figure}
  \begin{center}
  \includegraphics[width=0.9\textwidth]{CutFinishArea.png}
  \caption{Область металла возле участка завершения реза; резка против часовой
стрелки здесь более предпочтительна}
  \label{FinishCutArea}
  \end{center}
\end{figure}

Пусть $\tilde{K}, \tilde{K}\subset \overline{1,N}\setminus \{j\},$
--- множество номеров вырезанных к текущему моменту контуров,
и $k\in \tilde{K}$. На рисунке 1 изображены
два варианта реза --- по часовой стрелке и против часовой стрелки.
На рисунке область завершения реза пересекается только с вырезанным
контуром с номером $k$.

Стоимость внутренних работ по вырезанию контура с номером $j,j\in \overline{1,N},$
при использовании пары $z$ точек врезки и выключения инструмента,
$z\in \mathbb{M}_j$, в направлении $d,d\in \tilde{D},$ при условии,
что к настоящему моменту уже вырезаны контура с номерами из
$\tilde{K},\tilde{K}\subset \overline{1,N}\setminus \{j\}$, будет
вычисляться следующим образом: при $K=\overline{1,N}\setminus \tilde{K}$

\begin{equation}\label{IntPrice}
\begin{array}{c}
c_j(z,d,K)=\frac{\rho(\mbox{pr}_1(z),u(z))}{V_c}+
\frac{\rho(u(z),\mbox{pr}_2(z))}{V_c}+\\
+\frac{S_2(j,z,d,\overline{1,N}\setminus K)}{S_1(j,z,d,\overline{1,N}\setminus K)+
S_2(j,z,d,\overline{1,N}\setminus K)}\cdot\tilde{P}.
\end{array}
\end{equation}

Коэффициент $\tilde{P}$ определяет влияние штрафа
в зависимости от расположения финишной области реза. Если
$\tilde{P}=0$, то расположение финишной области никак не
сказывается на стоимости маршрута. Если $\tilde{P}$ есть
большая величина, то основная часть стоимости внутренних
работ в основном определяется вышеупомянутым штрафом, а
не временем реза на подходе к контуру.

Эвристический алгоритм показан в виде ряда шагов.
Изначально считается, что маршрут пуст. Далее к нему на
каждом шаге добавляется номер одного мегаполиса.

1. Находим мегаполис с номером $i,i\in \overline{1,N},$ для которого
$i\neq \mbox{pr}_2(z)$ $\forall z\in \mathbf{K}$, и упорядоченную пару
$(x,y)\in \mathbb{M}_i$, а также $d\in \tilde{D}$,
обеспечивающие минимум значения
$\mathbf{c}(x^0,x)+c_i((x,y),d,\overline{1,N})$.
Добавляем мегаполис $M_i$ в маршрут, а пару точек $(x,y)$ --- на трассу.
Отмечаем мегаполис $M_i$ как посещенный с направлением обхода $d$.

2. Через $V,V\subset \overline{1,N},$ обозначим множество номеров уже
посещенных мегаполисов. Пусть $j,j\in V,$ является номером
последнего мегаполиса частичного маршрута,
а $(x,y)\in \mathbb{M}_j$ его пара точек входа и выхода.
Далее находим такой номер мегаполиса $k,k\in I(\overline{1,N}\setminus V)$,
а также $(\tilde{x},\tilde{y})\in \mathbb{M}_k$ и $d\in \tilde{D}$,
которые обеспечивают минимальное значение выражения
$\mathbf{c}(y,\tilde{x})+c_k((\tilde{x};\tilde{y}),d,\overline{1,N}\setminus V)$.
Добавляем мегаполис $M_k$ в маршрут, а пару точек $(\tilde{x},\tilde{y})$
--- на трассу. Отмечаем мегаполис с номером $k$ как посещенный с направлением
обхода $d$.

3. Выполняем шаг 2 до тех пор, пока весь маршрут не будет построен.

Данный алгоритм позволяет получить некоторое допустимое решение. Так как алгоритм
жадный, то при его реализации значительный поиск глобального экстремума не производится.
Алгоритм может выдать некоторый локальный экстремум. В целях улучшения результата в
данной работе предлагается вариант многократного выполнения алгоритма с запретами
определенных перемещений, обеспечивающими возможность получения всякий раз новых
маршрута и трассы. Данные запреты осуществляются посредством изменения функций
стоимости $c_1,...,c_N$. Именно, для заданного мегаполиса значение этой функции
принимает очень большую величину, если он попадает в заранее оговоренную позицию
маршрута. Следовательно, алгоритм будет избегать посещения упомянутого мегаполиса
в данной выбранной позиции.

Для реализации этого подхода вводится специальная матрица $D$ коррекции стоимостей
внутренних работ: $D_{i,j}\in \{0;1\}$, $i\in \overline{1,N}$, $j\in \overline{1,N}$.
Если $D_{i,j}=1$, то
$$
c_i(h,d,K)=1000000\;\; \forall h\in \mathbb{M}_i,\;\;
\forall d\in \tilde{D},\;\;\forall K\in \mathcal{P}(\overline{1,N}\setminus \{i\})
$$
при условии, что мегаполис с номером $i$ помещен на маршруте в позицию $j$.
Если же $D_{i,j}=0$, то оценка внутренних
работ вычисляется обычным способом (см. (\ref{IntPrice})). Данный способ вычисления
внутренних стоимостей используется только во время построения маршрута и
трассы, для их оценки он не используется.

Работа итерационного алгоритма определяется двумя параметрами: $N_1$ и $N_2$.
$N_1$ определяет общее количество итераций, $N_2$ определяет количество
итераций в одном цикле. Смысл цикла итераций заключается в постепенном добавлении
новых коррекций (единиц в матрице $D$) к уже существующим. После завершения
цикла все значения матрицы $D$ сбрасываются в ноль. Далее итерационный метод
использования алгоритма приводится в виде ряда шагов.

1. Выполнение алгоритма без каких-либо коррекций.

2. Выбираем случайным образом позицию $i,i\in \overline{1,N}$, на маршруте.
Пусть в данной позиции располагается мегаполис с номером $j,j\in \overline{1,N}$.
Принимаем $D_{j,i}=1$.

3. Выполняем описанный ранее алгоритм. Вычисляем оценку маршрута и трассы
без коррекций. Если значение оценки улучшилось, то запоминаем данное значение,
а также полученные маршрут, трассу и направления обхода мегаполисов.

4. Выполняем шаги 2 и 3 $N_2-1$ раз ($N_2-2$ раз для первого цикла итераций,
так как в самом начале был произведен счет без коррекций).

5. Обнуляем матрицу $D$.

6. Выполняем шаги 2-5, пока общее количество итераций не достигнет $N_1$.

7. Выбираем лучший результат, а также соответствующие ему маршрут, трассу
и направления обхода мегаполисов.



{\bf Пример 1.} Скорость холостого хода $V_i=500$ мм/с, скорость реза $V_c=10$ мм/с.
Длина финишного участка реза равна 150 мм, ширина области завершения реза 50 мм.
Параметры итерационного метода в обоих примерах $N_1=100$, $N_2=10$,
коэффициент штрафа для внутренних работ $\tilde{P}=1000000$.
Количество контуров $N=170$, количество адресных пар $|\mathbf{K}|=80$.

\begin{figure}
  \begin{center}
  \includegraphics[width=0.9\textwidth]{route_170_approximate_CA.png}
  \caption{Маршрут и трасса обхода множеств. Пример 1. Эвристический алгоритм}
  \label{Sample1Heuristic}
  \end{center}
\end{figure}

Результат счета 532,99, результат на первой итерации 556,15.
Время счета 8 мин. 23 сек.
Маршрут и трасса показаны на рис. \ref{Sample1Heuristic}.

{\bf Пример 2.} Скорость холостого хода $V_i=500$ мм/с, скорость реза $V_c=10$ мм/с.
Длина финишного участка реза равна 150 мм, ширина области завершения реза 50 мм.
Параметры итерационного метода в обоих примерах $N_1=100$, $N_2=10$,
коэффициент штрафа для внутренних работ $\tilde{P}=1000000$.
Количество контуров $N=201$, количество адресных пар $|\mathbf{K}|=56$.

\begin{figure}
  \begin{center}
  \includegraphics[width=0.9\textwidth]{Sample_201_sets_approximate.png}
  \caption{Маршрут и трасса обхода множеств. Пример 2. Эвристический алгоритм}
  \label{Sample2Heuristic}
  \end{center}
\end{figure}

Результат счета 2255679,293, результат без учета штрафов 531,631 (совпадает
со значением на первой итерации). Время счета 1 мин. 40 сек.
Маршрут и трасса показаны на рис. \ref{Sample2Heuristic}.
В данном случае максимальный штраф составил 387001. Учитывая тот факт, что
максимально возможный штраф имеет значение 1000000 (если вся область завершения
реза приходится на пустоты в металле), то можно сделать вывод, что 38\% области
завершения реза пришлось на пустоты в металле для данного контура. Такой
результат может быть естественным следствием очень плотной компановки деталей
и, в принципе, вряд ли является критичным.

{\bf Пример 3.} Скорость холостого хода $V_i=500$ мм/с, скорость реза $V_c=10$ мм/с.
Длина финишного участка реза равна 150 мм, ширина области завершения реза 50 мм.
Решение получено с использованием метода динамического программирования
и эвристического итерационного алгоритма. Коэффициент штрафа для внутренних
работ $\tilde{P}=1000000$. В случае эвристического алгоритма параметры итерационного
метода имели стедующие значения $N_1=10000$, $N_2=10$.
Количество контуров $N=27$, количество адресных пар $|\mathbf{K}|=12$.

\begin{figure}
  \begin{center}
  \includegraphics[width=0.9\textwidth]{Sample_27_sets_DP.png}
  \caption{Маршрут и трасса обхода множеств. Пример 3. Динамическое программирование}
  \label{Sample3DP}
  \end{center}
\end{figure}

Результат счета методом ДП 62,815, результат без учета штрафов 62,815.
Время счета 7 мин. 25 сек.
Маршрут и трасса показаны на рис. \ref{Sample3DP}.

\begin{figure}
  \begin{center}
  \includegraphics[width=0.9\textwidth]{Sample_27_sets_approximate.png}
  \caption{Маршрут и трасса обхода множеств. Пример 3. Эвристический алгоритм}
  \label{Sample3Heuristic}
  \end{center}
\end{figure}

Результат счета эвристическим итерационным методом 243112,454, результат
без учета штрафов 68,495 (для первой итерации 243113,235 и 69,276
соответственно).
Маршрут и трасса показаны на рис. \ref{Sample3Heuristic}.
Максимальный из штрафов составил 176823,72. Можно
сделать вывод, что 17\% области завершения реза пришлось на пустоты для
данного контура, что достаточно неплохо. Время счета 25 сек.

Особенность метода ДП состоит в том, что размерность решаемых с его
использованием примеров ограничена приблизительно четырьмя десятками контуров.
В принципе, такие задачи уже представляют практический интерес (возможны
раскрои такой размерности). Кроме того, метод ДП может использоваться для
грубой оценки приведенного эвристического алгоритма. Именно, предлагается
выполнить вычисления на нескольких десятках примеров эвристическим методом
и методом ДП, после чего сравнить результаты. Ниже приведены результаты
данного сравнения.

Количество примеров равно 25. Высота области, занимаемой контурами, приблизительно
равна одному метру. Ширина колеблется в пределах от одного до двух метров.
Матрицы, используемая для построения значений функций стоимости,
имеет 1500 столбцов и 700 строк.

Скорость холостого хода $V_i=500$ мм/с, скорость реза $V_c=10$ мм/с.
Длина финишного участка реза равна 100 мм, ширина области завершения реза 25 мм.
Параметры итерационного метода во всех примерах $N_1=100$, $N_2=10$,
коэффициент штрафа для внутренних работ $\tilde{P}=1000$.
Количество контуров в каждом из примеров $N=20$.
Количество адресных пар $|\mathbf{K}|$ принимало значение от 2 до 22.
Далее через $T_h$ будем обозначать значение критерия для эвристического
алгоритма, а через $T_{DP}$ - для метода ДП.
В ходе вычислительного эксперимента получены следующие значения критерия.\newline
Пример 4, $T_h=71.41, T_{DP}=66.42$.\newline
Пример 5, $T_h=372.35, T_{DP}=136.89$.\newline
Пример 6, $T_h=67.40, T_{DP}=66.30$.\newline
Пример 7, $T_h=263.41, T_{DP}=66.68$.\newline
Пример 8, $T_h=64.29, T_{DP}=63.18$.\newline
Пример 9, $T_h=70.26, T_{DP}=67.60$.\newline
Пример 10, $T_h=64.41, T_{DP}=62.87$.\newline
Пример 11, $T_h=68.16, T_{DP}=67.07$.\newline
Пример 12, $T_h=139.32, T_{DP}=136.64$.\newline
Пример 13, $T_h=68.30, T_{DP}=66.58$.\newline
Пример 14, $T_h=114.10, T_{DP}=65.53$.\newline
Пример 15, $T_h=70.10, T_{DP}=67.85$.\newline
Пример 16, $T_h=69.00, T_{DP}=66.61$.\newline
Пример 17, $T_h=68.06, T_{DP}=66.83$.\newline
Пример 18, $T_h=66.39, T_{DP}=64.94$.\newline
Пример 19, $T_h=69.83, T_{DP}=68.35$.\newline
Пример 20, $T_h=69.69, T_{DP}=66.46$.\newline
Пример 21, $T_h=68.28, T_{DP}=66.38$.\newline
Пример 22, $T_h=171.85, T_{DP}=65.25$.\newline
Пример 23, $T_h=68.11, T_{DP}=66.39$.\newline
Пример 24, $T_h=308.90, T_{DP}=135.72$.\newline
Пример 25, $T_h=69.05, T_{DP}=67.29$.\newline
Пример 26, $T_h=68.27, T_{DP}=66.16$.\newline
Пример 27, $T_h=221.97, T_{DP}=66.86$.\newline
Пример 28, $T_h=68.78, T_{DP}=66.92$.\newline

{\bf Выводы.} Сравнивая результаты работы эвристического алгоритма и
алгоритма на основе ДП можно отметить следующие моменты. В трех примерах
из 24 найти точное решение без штрафов не удалось. Напомним, штрафы возникают,
если область вокруг участка завершения реза пересекается с отверстиями в металле
или выходит за края листа. Причем в одном случае эвристический алгоритм дал
близкий к оптимальному результат, а в двух других случаях штраф был заметно
больше, чем в случае с ДП. Еще в четырех примерах эвристический алгоритм дал
результат со штрафами.

Максимальное значение критерия для одного из примеров для эвристического алгоритма
равно 372,35. Оптимальный результат для этого примера 136,89 (есть штрафы, величина
которых составляет 68,95). Для ДП стоимость маршрута без учета штрафов 68.09, для
эвристического алгоритма --- 68.95. Таким образом, без учета штрафов результаты счета
очень близки. В случае с эвристическим алгоритмом величина штрафа равна 303,4. Напомним,
что коэффициент штрафа для внутренних работ $\tilde{P}=1000$. Если вся область вокруг
участка завершения реза попадает на пустоты, то штраф для данного контура будет равен
1000. Даже в том случае, если вся величина штрафа 303,4 попала на один контур, менее
трети области вокруг участка завершения реза пересекается с пустотами в металле.
Такой результат представляется вполне приемлемым с практической точки зрения.

Таким образом, данный эвристический алгоритм представляется весьма эффективным.
Что касается алгоритма на основе ДП --- он дает оптимальный результат, что
исключительно важно, но размерость решаемых этим методом примеров лежит в пределах
четырех десятков контуров. Данное обстоятельство накладывает значительные
ограничения на практическое его использование.


\clearpage
\section*{
  Заключительные замечания к разделу II
}

Задача управления инструментом при листовой резке на машинах с ЧПУ является весьма
трудной с точки зрения вычислительной реализации методов, развиваемых на основе
математической теории. При решении конкретных практических задач неизбежным
представляется применение эвристических алгоритмов. Эти алгоритмы могут быть
построены (см. предыдущий раздел) так, что при этом учитываются весьма различные
и плохо формализуемые ограничения. Последнее является, наряду с достаточно высоким
быстродействием и возможностью решать <<большие>> задачи, ценным обстоятельством.
Вместе с тем при их применении могут возникать явные коллизии. Для оперативного
вмешательства с целью <<исправления>> фрагментов ДР и было предложено применять
оптимизирующие вставки.

Данные вставки конструировались так, что оказывалось возможным (на локальном уровне)
использовать методы построения точных решений. Это было достигнуто за счет применения
вставок, приводящих к маршрутным задачам умеренной размерности. Таким образом, в
задачах маршрутизации, имеющих достаточно большую размерность, удалось
<<задействовать>> аппарат ДП, применяемый в режиме итераций, причем сами итерации
могут организовываться, как показано в настоящей главе, по-разному. В частности, в
качестве правила, определяющего конкретное расположение оптимизирующей вставки на
<<эвристическом>> решении, может быть выбран принцип своеобразной максимизации условий
предшествования, <<захватываемых>> данной вставкой, т.е. вовлечение в локальную задачу
как можно большего числа ограничений упомянутого типа. Данный подход, реализованный
в \S 5.4, показывает, что учет некоторых ограничений может играть положительную роль
в вопросах снижения сложности вычислений. Конечно, такой эффект достигается при должной
теоретической проработке, что и было сделано в рамках ДП.

% !TeX root = ../mat_mod2.tex

\chapter*{ЗАКЛЮЧЕНИЕ}
\addcontentsline{toc}{chapter}{ЗАКЛЮЧЕНИЕ}

В монографии приведены постановки и математические модели инженерных задач,
связанных с листовой резкой на машинах с ЧПУ.
В первой части изложены содержательные конструкции,
относящиеся к вопросам оптимизации в задачах маршрутизации,
касающихся управления режущим инструментом.
Подробно обсуждаются общие подходы и соображения
по различным вариантам осуществления листовой резки на машинах с ЧПУ
(авторы не ограничиваются здесь стандартным вариантом резки по замкнутому контуру).
Введенные в первой части понятия определяют широкий взгляд на задачу,
что позволяет с использованием доступного перебора вариантов резки достигать лучшего качества.
Вместе с тем, из конструкций первой части (главы 1 и 2)
естественным образом возникает необходимость в построении адекватной математической теории,
позволяющей
(при соответствующей формализации)
учитывать различные осложняющие факторы и,
прежде всего, различные ограничения,
возникающие из соображений технологического характера.
Эта цель достигается
(возможно, не в полной мере)
во второй части книги.

Задачи маршрутизации перемещений имеют своим прототипом известную труднорешаемую задачу коммивояжера
(ЗК или {\it TSP} в англоязычной литературе).
Вместе с тем, и это видно уже из содержательных построений первой части,
имеются существенные различия исследуемых постановок с ЗК,
причем это различия не только количественного, но и качественного характера.
В этой связи возникает потребность в формализации,
которая существенно отличается от аналогичной формализации для ЗК и ей подобных.
Используется модель мегаполисов с условиями предшествования и функциями стоимости,
допускающими зависимость от списка заданий.
Последнее обстоятельство естественно для задачи минимизации дозовой нагрузки в атомной энергетике
(см. \cite{Cha2`}),
здесь оно связано с использованием штрафов за нарушение ограничений.
Итак, нарушать ограничения
(в том числе ограничения динамического характера)
формально разрешается, но ценой существенного проигрыша в качестве.

Для реальных задач управления инструментом при листовой резке на машинах с ЧПУ
типичным является случай достаточно большой размерности,
что крайне затрудняет реализацию оптимального алгоритма на основе ДП
даже при наличии большого количества условий предшествования.
В этой связи во второй части последовательно развиваются
конструкции локального улучшения эвристических решений
посредством оптимизирующих вставок и итерационных процедур с применением таких вставок.
Здесь речь идет всякий раз об оптимизации <<в окне>>.
Объектом применения оптимизирующих вставок является
эвристическое решение со свойством допустимости
в смысле соблюдения полной системы ограничений.
При построении оптимизирующих вставок <<умеренной>> размерности
активно используется аппарат ДП.
Итак, ДП находит свое применение и в задачах большой размерности.
В самое последнее время упомянутый подход получил дальнейшее развитие
в конструкциях мультивставок
(см. \cite{ChenGrig, ChenChenGrig}).

Другое <<недавнее>> направление исследований,
не затрагиваемое в настоящей книге,
связано с вопросом оптимизации точки старта в задачах последовательного обхода мегаполисов
(см. \cite{StartPoint,StartFinishPoint,ChenChen}).
Здесь также используется аппарат ДП.
В этих постановках объектом оптимизации звляется триплет
$(\alpha,(z_t)_{t\in \overline{0,N}},x)$,
где $\alpha$ -- маршрут (перестановка индексов),
а $(z_t)_{t\in \overline{0,N}}$ -- траектория, стартующая из точки $(x,x)$,
где $x$ -- точка старта, которая может выбираться из заданного множества $X^0$.

Таким образом,
исследования авторов и их коллег, работающих в ИММ УрО РАН и УрФУ,
находятся в постоянном развитии.
Получаемые при этом результаты важны,
как представляется,
не только в теоретическом аспекте,
но и в многочисленных приложениях.


\newpage
\printbibliography[heading=bibintoc,title=Список используемых источников]

% !TeX root = ../mat_mod2.tex

% https://tex.stackexchange.com/a/365249
\makeatletter
\newcommand*{\cleartoleftpage}{%
  \clearpage
    \if@twoside
    \ifodd\c@page
      \hbox{}\thispagestyle{empty}\newpage
      \if@twocolumn
        \hbox{}\newpage
      \fi
    \fi
  \fi
}
\makeatother
\cleartoleftpage

\thispagestyle{empty}
\centering

\vspace{0pt plus2fill}
{\small\it
Научное издание
}

\vspace{0pt plus2fill}

Петунин Александр Александрович,
Ченцов Александр Георгиевич,
Ченцов~Павел~Александрович

\vspace{0pt plus1fill}

{\bf
Оптимальная маршрутизация инструмента машин фигурной
листовой резки с~числовым
программным управлением.
Математические модели и~алгоритмы
}

\vspace{0pt plus2fill}

Редактор {\it
И. О. Фамилия
}

Корректор {\it
И. О. Фамилия
}

Компьютерная верстка {\it
% В форме родительного падежа
С. С. Уколова
}

\vspace{0pt plus1fill}

{\small
Подписано в печать \underline{\hspace{1cm}}.
Формат 70$\times$100/64.

Бумага офсетная.
Цифровая печать.
Усл. печ. л. \underline{\hspace{1cm}}.

Уч.-изд. л. \underline{\hspace{1cm}}.
Тираж \underline{\hspace{1cm}} экз.
Заказ \underline{\hspace{1cm}}

\vspace{0pt plus1fill}

Издательство Уральского университета

Редакционно-издательский отдел ИПЦ УрФУ

620049, Екатеринбург, ул. С. Ковалевской, 5

Тел.: +7 (343) 375-48-25, 375-46-85, 374-19-41

E-mail: rio@urfu.ru

\vspace{0pt plus1fill}

Отпечатано в Издательско-полиграфическом центре УрФУ

620000, Екатеринбург, ул. Тургенева, 4

Тел.: +7 (343) 358-93-06, 350-58-20, 350-90-13

Факс: +7 (343) 358-93-06

http://print.urfu.ru
}

\vspace{0pt plus1fill}


\end{document}
