\documentclass{article}
    \usepackage[utf8]{inputenc}
    \usepackage[T2A]{fontenc}
    \usepackage[english,russian]{babel}

    % Related to math
    \usepackage{amsmath,amssymb,amsfonts,amsthm}

    \usepackage{graphicx}
    \graphicspath{{media/}}


\begin{document}

\section{Практические аспекты оптимизации траектории инструмента для машин лазерной резки с ЧПУ}

\subsection{Точное вычисление целевых функций в задаче оптимизации маршрута резки на примере машины лазерной резки ByStar3015}

Как мы отмечали в Параграфе 1.2
%%% TODO fix reference  ^^^^^^^^^^^^^^^
для задачи оптимизации маршрута резки (\ref{problem-statement})
проблема точного вычисления целевых функций времени резки и стоимости резки,
определяемых, в частности, формулами (\ref{cutting-time}) и (\ref{cutting-cost}):
$$
T_{cut} = \frac{L_{on}}{V_{on}} + \frac{L_{off}}{V_{off}} +N_{pt} \cdot t_{pt}
$$
$$
F_{cost}=
L_{on} \cdot C_{on} +
L_{off} \cdot C_{off} +
N_{pt} \cdot C_{pt}
$$
является малоисследованной.
Ниже будут приведены результаты исследований,
проведенных А.Ф.Таваевой на предприятии
АО «Производственное объединение “Уральский оптико – механический завод”
имени Э.С. Яламова» (Екатеринбург)
на машине лазерной резки ByStar3015.
Более подробно результаты этих исследований изложены в
\cite{intro45,intro46,intro47}.

\subsubsection{Вычисление фактического времени лазерной резки машины с ЧПУ
в зависимости от параметров управляющей программы и технологических факторов процесса резки}

Неточность вычисления фактического времени резки
$T_{cut}$
связана с тем, что скорость рабочего хода машины с ЧПУ
$V_{on}$,
программируемая в управляющей программе как константа,
фактически таковой не является и может меняться
в зависимости от различных технологических факторов,
а также характеристик спроектированной управляющей программы.
В частности, было установлено,
что при увеличении числа кадров в управляющих программах
резки разных наборов заготовок,
имеющих один и тот же суммарный периметр контуров,
фактическая средняя скорость резки падает.
Причины, по которым УП могут содержать большое количество кадров,
в основном, связано с тем, что контуры со сложной геометрией
(например, сплайны) при конвертации из CAD системы в CAM
модуль из-за разницы в геометрических форматах файлов
разбиваются на большое число геометрических примитивов
(например, на отрезки прямых и дуги окружностей),
т.е. аппроксимируются более простыми геометрическими примитивами.
Разница в форматах, в свою очередь,
вызвана тем, что практически все системы ЧПУ
оснащаются только линейными и круговыми интерполяторами.
Как правило, аппроксимация сложной геометрии сводится
именно к линейной аппроксимации.
Иногда конвертеры CAD файлов аппроксимируют отрезками прямых
даже дуги окружностей, хотя в этом нет необходимости,
если система ЧПУ поддерживает круговую интерполяцию.

Ниже приведены некоторые практические результаты
по определению зависимости скорости рабочего хода
инструмента лазерного комплекса ByStar3015
от количества кадров управляющей программы.

Исследования были проведены для следующих материалов:
10кп ($\Delta$=1-10мм) и АМг3М ($\Delta$=1-5мм).
Для проведения вычислительных экспериментов были разработаны
150 тестовых УП для резки различных фигурных заготовок с числом кадров
$n=\overline{10,5000}$
для материала 10кп и 150 УП – для материала АМг3М с числом кадров
$n=\overline{10,2000}$.

Статистический материал был обработан в программе “Mathcad”
и с помощью метода наименьших квадратов были построены
аппроксимирующие функции для зависимости скорости
рабочего хода инструмента
$V_{on}$
от количества кадров в спроектированной УП.
По результатам эксперимента были сделаны следующие выводы:

\begin{enumerate}
\item Фактическая средняя скорость рабочего хода режущего инструмента
$V_{on}$
является монотонно убывающей функцией от числа кадров УП
(рис. \ref{amg3m}, \ref{10kp});

\item Заданная в УП скорость
$V_{on}$
совпадает с фактической средней скоростью
при достижении числа кадров некоторого порогового значения $N$.
Когда количество кадров в УП меньше порогового значения $n<N$,
то фактическая скорость выше заданной,
а при увеличении числа кадров больше порогового $n>N$
– может существенно снижаться
(в проведенных экспериментах снижение средней
фактической скорости режущего инструмента по сравнению
с заданным в УП значением доходило до 70\%);

\item Пороговое значение различно для разных марок материала и толщин.

\end{enumerate}

Для изложения результатов вычислительных экспериментов
введём следующие обозначения:
пусть
$n$  – число кадров в УП,
$V_\text{факт}$  – фактическая средняя скорость режущего инструмента при заданной скорости $V_{on}$,
$N$ - число кадров (пороговое значение), для которого $V_\text{факт}=V_{on}$;
$\sum \varepsilon_n^2$  - сумма квадратов отклонений исходных значений
скорости режущего инструмента и значений аппроксимирующей функции $V_{on}(n)$
в этих точках.

При аппроксимации точечных графиков
зависимости фактической скорости
$V_\text{факт}$
от числа кадров $n$
в УП аппроксимирующими кривыми в “Mathcad”
для всех значений исследуемых марок материала и толщин материала было установлено,
что значения
$\sum \varepsilon_n^2 \to 0$
достигаются при аппроксимации экспериментальных данных логарифмической функцией.

Аналогичные результаты были получены для материала АМг3М $\Delta$=2-5мм и 10кп $\Delta$=1-10мм.
Обобщенные результаты для всех исследованных марок материала и толщин приведены в табл. \ref{v-formulae}.

При использовании материала других марок
необходимо проведение дополнительных исследований,
либо использование имеющихся данных по материалу
с близкими физическими свойствами.

\begin{figure}
  \begin{center}
  \includegraphics[width=0.7\textwidth]{amg3m.png}
  \caption{Изменение скорости режущего инструмента на рабочем ходе для АМг3М, $\Delta$=1мм ($n=\overline{10,2000}$)}
  \label{amg3m}
  \end{center}
\end{figure}


\begin{figure}
  \begin{center}
  \includegraphics[width=0.7\textwidth]{10kp.png}
  \caption{Изменение скорости режущего инструмента на рабочем ходе для 10кп, $\Delta$=3мм ($n=\overline{10,5000}$)}
  \label{10kp}
  \end{center}
\end{figure}

\begin{table}
  \begin{tabular}{lll}
    Материал & Толщина, мм & Формула расчёта $V_{on}$ \\
    \hline
    \multicolumn{3}{c}{$n=\overline{10,5000}$} \\
    10кп & 1 & $V_{on} = -0.024 \ln n+0.245$ \\
    10кп & 2 & $V_{on} = -0.015 \ln n+0.1686$ \\
    10кп & 3 & $V_{on} = -0.009 \ln n+0.1078$ \\
    10кп & 3.5 & $V_{on} = -0.006 \ln n+0.0756$ \\
    10кп & 4 & $V_{on} = -0.006 \ln n+0.0709$ \\
    10кп & 8 & $V_{on} = -0.003 \ln n+0.0442$ \\
    10кп & 10 & $V_{on} = -0.002 \ln n+0.0365$ \\
    \multicolumn{3}{c}{$n=\overline{10,2000}$} \\
    АМг3М & 1 & $V_{on} = -0.014 \ln n+0.1589$ \\
    АМг3М & 1.5 & $V_{on} = -0.001 \ln n+0.011$ \\
    АМг3М & 3 & $V_{on} = -0.004 \ln n+0.0672$ \\
    АМг3М & 4 & $V_{on} = -0.001 \ln n+0.0301$ \\
    АМг3М & 5 & $V_{on} = -6\cdot 10^{-4} \ln n+0.0177$ \\
  \end{tabular}
  \label{v-formulae}
  \caption{Обобщенная таблица формул для вычисления рабочей скорости инструмента на лазерном комплексе ByStar3015}
\end{table}

Рассмотрим пример оптимизации времени резки
$T_{cut}$
(\ref{cutting-time})
при резке 15 фигурных заготовок для задачи
(материал АМг3М $\Delta$=1мм).
Раскройная карта (рис. \ref{amg-cutting})
содержит 15 заготовки двух типоразмеров,
при этом количество граничных контуров заготовок равно 19.
Каждый контур вырезается с помощью резки
«по замкнутому контуру».
С целью сокращения множества допустимых решений
задачи множество возможных точек врезки было
ограничено конечным множеством (задача GTSP),
состоящим из 55 точек
(обозначены квадратами зеленого цвета;
соответствующие точки выключения инструмента обозначены крестиками).
Для решения задачи использован точный алгоритм ДП.
УП резки для данного примера содержат 120 команд или кадров
(т.е. $n=120$),
которые включают команды перемещения инструмента
для резки контуров
(с учетом разбиения каждого контура на несколько геометрических примитивов)
на рабочем ходе,
команды перемещения инструмента на холостом ходе
и ряд технологических команд.
Скорость рабочего хода инструмента, заданная в УП,
$V_{on}=0.1$ м/с.

\begin{figure}
  \begin{center}
  \includegraphics[width=0.5\textwidth]{amg-cutting.png}
  \caption{Раскройная карта и оптимальный по времени маршрут
  перемещения режущего инструмента для 15 заготовок (материал АМг3М $\Delta$=1мм) при условии, что
  $V_{on}=const=0.1$м/с}
  \label{amg-cutting}
  \end{center}
\end{figure}

На рис. \ref{amg-cutting}
показан маршрут резки
(перемещение инструмента на холостом ходе показаны стрелками синего цвета),
для которого значение целевой функции
$T_{cut}$ (\ref{cutting-time})
при
$V_{on}=0.1$ м/с
составляет
$T_{cut}=126.27$ сек.
Однако фактическое время резки по управляющей программе,
составленной для этого маршрута, оказалось (как и ожидалось)
значительно больше,
поскольку число кадров в программе ($n=120$)
значительно больше порогового значения $N=70$
для материала АМг3М $\Delta$=1мм.

При использовании значения
$V_{on}=-0.014 \ln n + 0.1589$
(табл.6) в целевой функции (\ref{cutting-time})
оптимизационная процедура ДП
даёт другое оптимальное решение задачи,
которое показано на рис. \ref{amg-optimal}.
Тогда среднее фактическое значение рабочей скорости инструмента при $n=120$
составило
$V_{on}=0.0919$ м/с.
В свою очередь для оптимального маршрута резки значение времени резки составило
$T_{cut}=141.38$ сек.

Таким образом,
точное вычисление целевой функции для
данного примера обеспечило не только
точное вычисления значения экстремума
целевой функции, но и другой (правильный)
результат поиска оптимального маршрута резки,
полученный  с учетом числа кадров УП.


\begin{figure}
  \begin{center}
  \includegraphics[width=0.5\textwidth]{amg-optimal.png}
  \caption{Оптимальный по времени маршрут перемещения режущего инструмента при условии, что
  $V_{on}=-0.014 \ln n + 0.1589$
  $(n=120)$}
  \label{amg-optimal}
  \end{center}
\end{figure}

Данный пример иллюстрирует необходимость
получения табли+ц типа Таблицы \ref{v-formulae}
при решении конкретных оптимизационных задач
маршрутизации инструмента машин листовой резки с ЧПУ.

\subsubsection{Вычисление стоимости резки заготовок на машине машине с ЧПУ в режиме моделирования процесса резки}

Другая проблема точного вычисления целевой функции
при оптимизации маршрута резки связано
с поиском адекватных значений стоимости в формуле (\ref{cutting-cost})
$$
F_{cost}=
L_{on} \cdot C_{on} +
L_{off} \cdot C_{off} +
N_{pt} \cdot C_{pt}
$$

Напомним:
$C_{on}$ – стоимость единицы пути с включенным режущим инструментом;
$C_{off}$ – стоимость единицы пути с выключенным режущим инструментом;
$C_{pt}$ – стоимость одной точки врезки,
$L_{off}$ – длина переходов с выключенным режущим инструментом (холостой ход);
$L_{on}$ – длина реза с включенным режущим инструментом;
$N_{pt}$ – количество точек врезки.

Рассмотрим вопрос точного вычисления
стоимости лазерной резки в задаче
оптимизации маршрута режущего инструмента
применительно к машине лазерной резки (тип лазера: СО$_2$)
с ЧПУ на примере машины ByStar3015.

Проблема точного вычисления целевой функции
при оптимизации маршрута резки связана с
поиском адекватных значений стоимости
$F_{cost}$,
вычисление которой зависит от параметров
$C_{on}, C_{off}, C_{pt}$.

Для расчета
$C_{on}$
введем следующие обозначения для стоимостных параметров,
вычисляемых на 1 м рабочего хода инструмента:
$C_\text{расх}$   - стоимость расходных материалов (например, сопло, защитное стекло, газовые трубки);
$C_\text{тех}$   - стоимость технологического газа (азот или кислород в зависимости от типа обрабатываемого материала);
$C_\text{лаз}$ - стоимость лазерного газа (при работе на машине с ЧПУ на проточном газовом лазере),
$C_\text{э/э}^{on}$ - стоимость электроэнергии;
$C_\text{зп}^{on}$ - затраты, связанные с заработной платой сопровождающего персонала;
$C_\text{А}^{on}$ - амортизация оборудования.
Тогда в общем виде
$C_{on}$
будем вычислять по следующей формуле:

\begin{equation}
  C_{on} =
  C_\text{э/э}^{on} +
  C_\text{тех} +
  C_\text{лаз} +
  C_\text{расх} +
  C_\text{зп}^{on} +
  C_\text{А}^{on}
  \label{c-on}
\end{equation}

Для вычисления значений
$C_{on}, C_\text{э/э}^{on}, C_\text{тех}, C_\text{лаз}, C_\text{расх}, C_\text{зп}^{on}, C_\text{А}^{on}$
введем дополнительные обозначения:
$t_{on}$ – время, затрачиваемое на один метр рабочего хода инструмента, час;
$P_{on}$ – затраты электроэнергии за один час работы лазерного комплекса на рабочем ходе, кВт/ч;
$V_\text{тех}$ – расход технологического газа, м$^3$/ч;
$V_\text{лаз}$ – расход лазерного газа, м$^3$/ч;
$C_\text{э/э}$ - стоимость электроэнергии за 1 кВт;
$C_{\text{лазМ}^3}$- стоимость 1м$^3$ лазерного газа;
$C_{\text{техМ}^3}$ - стоимость 1м$^3$ технологического газа;
$C_\text{расхЕд}$- стоимость единицы расходных материалов;
$t_\text{расхСрок}$- срок службы расходных материалов;
$C_\text{зп}$ - стоимость 1ч работы обслуживающего персонала;
$A$ – амортизация за 1 час работы лазерного комплекса, руб;
$N$ – срок полезного использования оборудования, год;
$C_\text{оборуд}$- первоначальная стоимость лазерного комплекса. Тогда
$C_{on}, C_\text{э/э}^{on}, C_\text{тех}, C_\text{лаз}, C_\text{расх}, C_\text{зп}^{on}, C_\text{А}^{on}$
вычислим по следующим формулам:

\begin{equation}
  C_\text{э/э}^{on} =
  P_{on} t_{on}   C_\text{э/э}
  \label{c-on-ee}
\end{equation}

\begin{equation}
  C_\text{тех} =
  V_\text{тех} C_{\text{техМ}^3} t_{on}
  \label{c-on-teh}
\end{equation}

\begin{equation}
  C_\text{лаз} =
  V_\text{лаз} C_{\text{лазМ}^3} t_{on}
  \label{c-on-laz}
\end{equation}

\begin{equation}
  C_\text{расх} =
  \frac{C_\text{расхЕд}}{t_\text{расхСрок}}
  \label{c-on-rasx}
\end{equation}

\begin{equation}
  C_\text{зп}^{on} =
  C_\text{зп} t_{on}
  \label{c-on-zp}
\end{equation}

\begin{equation}
  C_\text{А}^{on} =
  \frac{1}N \frac{C_\text{оборуд}}{1920} t_{on}
  \label{c-on-A}
\end{equation}

Параметр
$C_\text{тех}$
необходимо учитывать при расчете стоимости резки
только в тех случаях,
когда применяется вспомогательный рабочий газ
(кислород, азот в зависимости от типа обрабатываемого материала)
для увеличения скорости резки,
возможности обработки материалов более высоких толщин
и для сокращения затрат электроэнергии.
Расход газа зависит от диаметра используемого сопла и давления газа.

Для расчета
$C_{off}$
введем следующие обозначения параметров,
вычисляемых на 1 м холостого хода режущего инструмента:
$P_{off}$ – затраты электроэнергии за один час работы лазерного комплекса на холостом ходе, кВт/ч;
$t_{off}$ – время, затрачиваемое на один метр холостого хода инструмента, час.
Тогда

\begin{equation}
  C_{off} =
  P_{off} t_{off} C_\text{э/э}
  + C_\text{зп} t_{off}
  + \frac{1}N \frac{C_\text{оборуд}}{1920} t_{off}
  \label{c-off}
\end{equation}

Аналогично для расчета
$C_{pt}$
введем следующие обозначения для стоимостных параметров,
вычисляемых на одну точку врезки:
$C_\text{э/э}^{pt}$ - стоимость электроэнергии;
$C_\text{расх}^{pt}$ - стоимость расходных материалов;
$C_\text{лаз}^{pt}$– стоимость лазерного газа;
$C_\text{тех}^{pt}$- стоимость технологического газа,
$C_\text{зп}^{pt}$- затраты, связанные с заработной платой сопровождающего персонала;
$C_\text{А}^{pt}$- амортизация оборудования.
Тогда

\begin{equation}
  C_{pt} =
  C_\text{э/э}^{pt} +
  C_\text{расх}^{pt} +
  C_\text{лаз}^{pt} +
  C_\text{тех}^{pt} +
  C_\text{зп}^{pt} +
  C_\text{А}^{pt}
  \label{c-pt}
\end{equation}

Для вычисления значений  ,  ,
$C_\text{э/э}^{pt}, C_\text{расх}^{pt}, C_\text{лаз}^{pt}, C_\text{тех}^{pt}$
введем дополнительные параметры:
$P_{pt}$ - затраты электроэнергии на одну точку врезки, кВт/ч;
$t_{pt}$ – время, затрачиваемое на одну точку врезки, час. Тогда

\begin{equation}
  C_\text{э/э}^{pt} =
  P_{pt} t_{pt}   C_\text{э/э}
  \label{c-pt-ee}
\end{equation}

\begin{equation}
  C_\text{тех}^{pt} =
  V_\text{тех} C_{\text{техМ}^3} t_{pt}
  \label{c-pt-teh}
\end{equation}

\begin{equation}
  C_\text{лаз}^{pt} =
  V_\text{лаз} C_{\text{лазМ}^3} t_{pt}
  \label{c-pt-laz}
\end{equation}

\begin{equation}
  C_\text{расх}^{pt} =
  \frac{C_\text{расхЕд}}{t_\text{расхСрок}}
  \label{c-pt-rasx}
\end{equation}

\begin{equation}
  C_\text{зп}^{pt} =
  C_\text{зп} t_{pt}
  \label{c-pt-zp}
\end{equation}

\begin{equation}
  C_\text{А}^{pt} =
  \frac{1}N \frac{C_\text{оборуд}}{1920} t_{pt}
  \label{c-pt-A}
\end{equation}

При расчете стоимости одной точки врезки параметр
$C_\text{лаз}^{pt}$
необходимо учитывать только при обработке материала
на проточном газовом лазере.
Параметр
$C_\text{тех}^{pt}$
необходимо учитывать при расчете себестоимости резки только в тех случаях,
когда применяется вспомогательный рабочий газ.

Тогда целевую функцию стоимости резки (\ref{cutting-cost})
можно записать в следующем виде:
\begin{multline}
  F_{cost} =
  L_{on} \Big(
    C_\text{э/э}^{on} +
    C_\text{тех} +
    C_\text{лаз} +
    C_\text{расх} +
    C_\text{зп}^{on} +
    C_\text{А}^{on}
      \Big)
  \\
  +L_{off} C_{off} +
  \\
  N_{pt} \Big(
    C_\text{э/э}^{pt} +
    C_\text{расх}^{pt} +
    C_\text{лаз}^{pt} +
    C_\text{тех}^{pt} +
    C_\text{зп}^{pt} +
    C_\text{А}^{pt}
      \Big)
  \label{c-full}
\end{multline}

К основным расходным материалам и запчастям
для газового лазера можно отнести:
поворотные зеркала, фокусирующие линзы,
защитные стекла, сопла, юстировочные узлы,
газовые трубки.
К основным расходным материалам для
волоконного лазера можно отнести:
сопла, защитные стекла, фокусирующие линзы.
А для случая применения твердотельных лазеров
выделяют следующие основные расходные материалы и запчасти:
лампы оптической накачки, защитные стекла, зеркала,
квантрон, активный элемент.
Следует отметить, что стоимость расходных материалов
может изменяться в зависимости от фактических сроков
службы расходных материалов,
которые зависят от качества используемого газа,
опыта персонала, эксплуатирующего лазерный станок.
Следует отметить, что
$C_\text{расхЕд}$
зависит от ценообразования, курса доллара (USD) и евро (EUR),
а параметры
$C_\text{э/э}$,
$C_{\text{лазМ}^3}$ и
$C_{\text{техМ}^3}$
зависят от цен, которые устанавливает поставщик услуг,
поэтому при расчете
$F_{cost}$
для конкретных производственных задач,
изменения цен целесообразно учитывать,
используя изменяющиеся в зависимости от перечисленных
факторов таблицы стоимостных параметров в MS Excel.
В частности, была создана сводная таблица в MS Excel
для расчета себестоимости лазерной резки по разработанной
выше методике для газового СО$_2$
лазерного комплекса ByStar 3015 для следующих материалов:

\begin{itemize}
\item нержавеющая сталь (на примере 12Х18Н10Т) толщиной $\Delta$=1-10мм;

\item углеродистая сталь (на примере 10кп) толщиной $\Delta$=1-15мм;

\item алюминий и его сплавы (на примере АМг3М) толщиной $\Delta$=1-5мм.
\end{itemize}

Были определены значения основных стоимостных характеристик
$C_{on}$, $C_{off}$, $C_{pt}$
с учетом всех перечисленных параметров, приведенных в
(\ref{c-on})-(\ref{c-pt-A}).
В табл. \ref{c-table} приведены значения стоимости
одного погонного метра лазерного реза при максимальной
$C_{on}^{max}$
и минимальной
$C_{on}^{min}$
возможной рабочей скорости перемещения режущего инструмента
$V_{on}$
в зависимости от требуемого качества изготовления деталей.

\begin{table}
  \begin{tabular}{crrrrr}
    Материал & Толщина, мм & $C_{on}^{max}$, руб & $C_{on}^{min}$, руб & $C_{off}$, руб & $C_{pt}$, руб \\
    \hline
    10кп	& 1	& 5,3	& 7,5	& 0,42	& 0,7 \\
    10кп	& 1,2	& 6,6	& 9,5	& 0,42	& 1,0 \\
    10кп	& 1,5	&  6,6	& 9,5	& 0,42	& 1,1 \\
    10кп	& 2	& 8,1	& 11,7	& 0,42	& 1,3 \\
    10кп	& 2,5	&	9,7	& 14,0	& 0,42	& 1,5 \\
    10кп	& 3	& 12,0	& 17,4	& 0,42	& 1,6 \\
    10кп	& 3.5	&	13,3	& 19,0	& 0,42	& 1,6 \\
    10кп	& 3.9	&	13,3	& 19,0	& 0,42	& 1,9 \\
    10кп	& 4	&	14,8	& 21,0	& 0,42	& 2,2 \\
    10кп	& 5	&	17,9	& 26,1	& 0,42	& 2,7 \\
    10кп	& 8	&	26,1	& 38,2	& 0,42	& 3,4 \\
    10кп	& 10	&	31,8	& 44,1	& 0,42	& 5,1 \\
    10кп	& 15	&	52,1	& 71,7	& 0,42	& 6,0 \\
    АМг3М	& 1	& 11,1	& 18,6	& 0,42	& 3,7 \\
    АМг3М	& 2	& 18,0	& 30,0	& 0,42	& 5,6 \\
    АМг3М	& 3	& 56,8	& 92,8	& 0,42	& 14,2 \\
    АМг3М	& 5	& 193,0	& 328,2	& 0,42	& 32,2 \\
    12Х18Н10Т	& 1	& 14,9	& 24,9	& 0,42	& 2,5 \\
    12Х18Н10Т	& 1,5	& 18,7	& 31,4	& 0,42	& 3,8 \\
    12Х18Н10Т	& 2	& 25,3	& 42,4	& 0,42	& 4,5 \\
    12Х18Н10Т	& 2,5	& 38,1	& 63,5	& 0,42	& 6,8 \\
    12Х18Н10Т	& 3	& 46,4	& 76,1	& 0,42	& 8,6 \\
    12Х18Н10Т	& 4	& 87,2	& 143,7	& 0,42	& 13,1 \\
    12Х18Н10Т	& 5	& 122,6	& 198,1	& 0,42	& 18,9 \\
    12Х18Н10Т	& 6	& 241,5	& 386,5	& 0,42	& 31,7 \\
    12Х18Н10Т	& 8	& 475,5	& 856,0	& 0,42	& 42,2 \\
    12Х18Н10Т	& 10	& 1038,7	& 2077,3	& 0,42	& 72,0 \\
  \end{tabular}
  \label{c-table}
  \caption{Значения основных стоимостных параметров при вычислении целевой функции для CO$_2$ лазерного комплекса ByStar3015}
\end{table}

Изложенная выше методика является универсальной
для такогокласса лазерного обороддования с ЧПУ и,
следовательно, может применяться для вычисления значений
целевой функции стоимости резки
$F_{cost}$,
а также для создания таблиц стоимостных параметров в формуле (\ref{cutting-cost})
для других марок стали и толщин материала.
Аналогичный подход следует использовать и при создания
стоимостных парметров целевой функции стоимости резки
для другого технологического оборудования термической
резки листового материала с ЧПУ.

\subsection{Стратегии формирования маршрута режущего инструмента для типовых заготовок на машиностроительном производстве}

\end{document}
