\documentclass{article}
    \usepackage[utf8]{inputenc}
    \usepackage[T2A]{fontenc}
    \usepackage[english,russian]{babel}

    % Related to math
    \usepackage{amsmath,amssymb,amsfonts,amsthm}

    \usepackage{graphicx}
    \graphicspath{{media/}}


\begin{document}

\section{Практические аспекты оптимизации траектории инструмента для машин лазерной резки с ЧПУ}

\subsection{Точное вычисление целевых функций в задаче оптимизации маршрута резки на примере машины лазерной резки ByStar3015}

Как мы отмечали в Параграфе 1.2
%%% TODO fix reference  ^^^^^^^^^^^^^^^
для задачи оптимизации маршрута резки (\ref{problem-statement})
проблема точного вычисления целевых функций времени резки и стоимости резки,
определяемых, в частности, формулами (\ref{cutting-time}) и (\ref{cutting-cost}):
$$
T_{cut} = \frac{L_{on}}{V_{on}} + \frac{L_{off}}{V_{off}} +N_{pt} \cdot t_{pt}
$$
$$
F_{cost}=
L_{on} \cdot C_{on} +
L_{off} \cdot C_{off} +
N_{pt} \cdot C_{pt}
$$,
является малоисследованной.
Ниже будут приведены результаты исследований,
проведенных А.Ф.Таваевой на предприятии
АО «Производственное объединение “Уральский оптико – механический завод”
имени Э.С. Яламова» (Екатеринбург)
на машине лазерной резки ByStar3015.
Более подробно результаты этих исследований изложены в
\cite{intro45,intro46,intro47}.

\subsubsection{Вычисление фактического времени лазерной резки машины с ЧПУ
в зависимости от параметров управляющей программы и технологических факторов процесса резки}

Неточность вычисления фактического времени резки
$T_{cut}$
связана с тем, что скорость рабочего хода машины с ЧПУ
$V_{on}$,
программируемая в управляющей программе как константа,
фактически таковой не является и может меняться
в зависимости от различных технологических факторов,
а также характеристик спроектированной управляющей программы.
В частности, было установлено,
что при увеличении числа кадров в управляющих программах
резки разных наборов заготовок,
имеющих один и тот же суммарный периметр контуров,
фактическая средняя скорость резки падает.
Причины, по которым УП могут содержать большое количество кадров,
в основном, связано с тем, что контуры со сложной геометрией
(например, сплайны) при конвертации из CAD системы в CAM
модуль из-за разницы в геометрических форматах файлов
разбиваются на большое число геометрических примитивов
(например, на отрезки прямых и дуги окружностей),
т.е. аппроксимируются более простыми геометрическими примитивами.
Разница в форматах, в свою очередь,
вызвана тем, что практически все системы ЧПУ
оснащаются только линейными и круговыми интерполяторами.
Как правило, аппроксимация сложной геометрии сводится
именно к линейной аппроксимации.
Иногда конвертеры CAD файлов аппроксимируют отрезками прямых
даже дуги окружностей, хотя в этом нет необходимости,
если система ЧПУ поддерживает круговую интерполяцию.

Ниже приведены некоторые практические результаты
по определению зависимости скорости рабочего хода
инструмента лазерного комплекса ByStar3015
от количества кадров управляющей программы.

Исследования были проведены для следующих материалов:
10кп ($\Delta$=1-10мм) и АМг3М ($\Delta$=1-5мм).
Для проведения вычислительных экспериментов были разработаны
150 тестовых УП для резки различных фигурных заготовок с числом кадров
$n=\overline{10,5000}$
для материала 10кп и 150 УП – для материала АМг3М с числом кадров
$n=\overline{10,2000}$.

Статистический материал был обработан в программе “Mathcad”
и с помощью метода наименьших квадратов были построены
аппроксимирующие функции для зависимости скорости
рабочего хода инструмента
$V_{on}$
от количества кадров в спроектированной УП.
По результатам эксперимента были сделаны следующие выводы:

\begin{enumerate}
\item Фактическая средняя скорость рабочего хода режущего инструмента
$V_{on}$
является монотонно убывающей функцией от числа кадров УП
(рис. \ref{amg3m}, \ref{10kp});

\item Заданная в УП скорость
$V_{on}$
совпадает с фактической средней скоростью
при достижении числа кадров некоторого порогового значения $N$.
Когда количество кадров в УП меньше порогового значения $n<N$,
то фактическая скорость выше заданной,
а при увеличении числа кадров больше порогового $n>N$
– может существенно снижаться
(в проведенных экспериментах снижение средней
фактической скорости режущего инструмента по сравнению
с заданным в УП значением доходило до 70\%);

\item Пороговое значение различно для разных марок материала и толщин.

\end{enumerate}

Для изложения результатов вычислительных экспериментов
введём следующие обозначения:
пусть
$n$  – число кадров в УП,
$V_\text{факт}$  – фактическая средняя скорость режущего инструмента при заданной скорости $V_{on}$,
$N$ - число кадров (пороговое значение), для которого $V_\text{факт}=V_{on}$;
$\sum \varepsilon_n^2$  - сумма квадратов отклонений исходных значений
скорости режущего инструмента и значений аппроксимирующей функции $V_{on}(n)$
в этих точках.

При аппроксимации точечных графиков
зависимости фактической скорости
$V_\text{факт}$
от числа кадров $n$
в УП аппроксимирующими кривыми в “Mathcad”
для всех значений исследуемых марок материала и толщин материала было установлено,
что значения
$\sum \varepsilon_n^2 \to 0$
достигаются при аппроксимации экспериментальных данных логарифмической функцией.

Аналогичные результаты были получены для материала АМг3М $\Delta$=2-5мм и 10кп $\Delta$=1-10мм.
Обобщенные результаты для всех исследованных марок материала и толщин приведены в табл. \ref{v-formulae}.

При использовании материала других марок
необходимо проведение дополнительных исследований,
либо использование имеющихся данных по материалу
с близкими физическими свойствами.

\begin{figure}
  \begin{center}
  \includegraphics[width=0.7\textwidth]{amg3m.png}
  \caption{Изменение скорости режущего инструмента на рабочем ходе для АМг3М, $\Delta$=1мм ($n=\overline{10,2000}$)}
  \label{amg3m}
  \end{center}
\end{figure}


\begin{figure}
  \begin{center}
  \includegraphics[width=0.7\textwidth]{10kp.png}
  \caption{Изменение скорости режущего инструмента на рабочем ходе для 10кп, $\Delta$=3мм ($n=\overline{10,5000}$)}
  \label{10kp}
  \end{center}
\end{figure}

\begin{table}
  \begin{tabular}{lll}
    Материал & Толщина, мм & Формула расчёта $V_{on}$ \\
    \hline
    \multicolumn{3}{c}{$n=\overline{10,5000}$} \\
    10кп & 1 & $V_{on} = -0.024 \ln n+0.245$ \\
    10кп & 2 & $V_{on} = -0.015 \ln n+0.1686$ \\
    10кп & 3 & $V_{on} = -0.009 \ln n+0.1078$ \\
    10кп & 3.5 & $V_{on} = -0.006 \ln n+0.0756$ \\
    10кп & 4 & $V_{on} = -0.006 \ln n+0.0709$ \\
    10кп & 8 & $V_{on} = -0.003 \ln n+0.0442$ \\
    10кп & 10 & $V_{on} = -0.002 \ln n+0.0365$ \\
    \multicolumn{3}{c}{$n=\overline{10,2000}$} \\
    АМг3М & 1 & $V_{on} = -0.014 \ln n+0.1589$ \\
    АМг3М & 1.5 & $V_{on} = -0.001 \ln n+0.011$ \\
    АМг3М & 3 & $V_{on} = -0.004 \ln n+0.0672$ \\
    АМг3М & 4 & $V_{on} = -0.001 \ln n+0.0301$ \\
    АМг3М & 5 & $V_{on} = -6\cdot 10^{-4} \ln n+0.0177$ \\
  \end{tabular}
  \label{v-formulae}
  \caption{Обобщенная таблица формул для вычисления рабочей скорости инструмента на лазерном комплексе ByStar3015}
\end{table}

\end{document}
