\documentclass{article}

\usepackage[english]{babel}

\usepackage{amssymb,amsmath}

\begin{document}

We note that,
for problems of moderate dimension,
it is possible to obtain optimal solution taking into account
the essential part of real constraints
(precedence conditions,
dynamic constraints etc.)
Now,
we consider general scheme that can be used not only
under control of sheet cutting on CNC machines.

Fix nonempty sets
$X$ and $X^0$
for which
$X^0 \subset X$
(under sheet cutting,
$X$ is a rectangle on the plane).
Suppose that
$X^0$ is a finite set.
Let
$N \in \mathbb N$,
where
$\mathbb N \triangleq \{1; 2; \dots \}$
(here and below
$\triangleq$
is the equality by definition);
suppose that
$2 \leqslant N$.
Fix nonempty sets
$M_1, \dots M_N$:

\begin{equation}
  \label{eq-i.1}
  M_1 \subset X,
  \dots ,
  M_N \subset X,
\end{equation}

Using results of
[99],  % TODO
we suppose that
$M_1, \dots M_N$
are finite sets
(these sets can be obtained from
continuous equidistants by digitization
for computing modelling).
We call the sets
(\ref{eq-i.1})
megacities.
Those sets are visiting objects
(this is a significant difference from TSP).
Moreover,
for
$j \in \overline{1, N}$,
where
$\overline{1,N} \triangleq \{k \in \mathbb N|k \leqslant N\}$,
a nonempty relation
$\mathbb M_j,
\mathbb M_j \subset M_j \times M_j$
is defined.
So, we have procession
$(\mathbb M_j)_{j \in \overline{1,N}}$
of nonempty sets connected with megacities.
Suppose that
$\mathbb P$
is the set og all permutations of the set
$\overline{1,N}$.
We consider processes

\begin{align*}
  (x\in X^0) \to
  (x_1^{(1)} \in M_{\alpha(1)} \rightsquigarrow x_2^{(1)} \in M_{\alpha(1)}) \to \\
  \dots \to \\
  (x_1^{(N)} \in M_{\alpha(N)} \rightsquigarrow x_N^{(2)} \in M_{\alpha(N)})
\end{align*}

\begin{equation}
  \label{eq-i.2}
  \alpha \in \mathbb P,
  (x_1^{(1)}, x_2^{(1)}) \in \mathbb M_{\alpha(1)},
  \dots ,
  (x_1^{(N)}, x_2^{(N)}) \in \mathbb M_{\alpha(N)}
\end{equation}

The choice of
$\alpha$
restrict oneself to precedence conditions defined in terms of a set
$\mathbf K$,
$\mathbf K \subset \overline{1,N} \times \overline{1,N}$,
of so-called address pairs.
For this,
we suppose that,
for ordered pair (OP)
$Z$
of every objects
$x$ and $y$
(so,
$z=(x,y)$),
by
$\mathrm{pr}_1(z)$
and
$\mathrm{pr}_2(z)$
are denoted the first and the second elements of
$z$
respectively:
$\mathrm{pr}_1(z)=x$
and
$\mathrm{pr}_2(z)=y$.
If
$z \in \mathbf K$,
then
$\mathrm{pr}_1(z)$
is called sender
and
$\mathrm{pr}_2(z)$
is recepient of
$z$
respectively.
And sender must be visited earlier than
the corresponding recepient
(these are the precedence conditions).
Then,

\begin{equation*}
  \mathbf A \triangleq
  \{ \alpha \in \mathbb P |
  \forall t_1 \in \overline{1,N},
  \forall t_2 \in \overline{1,N},
  \big(
    (\alpha(t_1), \alpha(t_2)) \in \mathbf K
  \big)
  \Rightarrow
  \big(t_1 < t_2 \big)
\}
\end{equation*}

We suppose that
[101, Condition 2.2.1]  % TODO
holds.
Then
$\mathbf A \neq \varnothing$.
So,
admissible by precedence routes exist
(we consider a route as element of
$\mathbb P$);
in (\ref{eq-i.2}),
only routes
$\alpha \in \mathbf A$
can be used.
But, according to (\ref{eq-i.2}),
the choice of
$\alpha \in \mathbf A$
not yet determined our process.
It is required to determinate trajectories.
But, at first,
we introduce finite phase space
(instead of $X$).

Under
$j \in \overline{1, N}$
we suppose that
$
\mathfrak M_j \triangleq
\{\mathrm{pr}_1(z): z \in \mathbb M_j \}
$
and
$
\mathbf M_j \triangleq
\{ \mathrm{pr}_2(z): z \in \mathbb M_j \}
$;
of course,
we obtain two nonempty subsets of
$M_j$.
By
$\mathfrak M$
we denote the union of all sets
$\mathfrak M_j$,
$j \in \overline{1,N}$.
Analogously,
by
$\mathbf M$
we denote the union of all sets
$\mathbf M_j$,
$j \in \overline{1,N}$.
Let
$\mathbb X \triangleq
X^0 \bigcup \mathfrak M$
and
$
\mathbf X \triangleq
X^0 \bigcup \mathbf M
$;
we consider
$\mathbb X \times \mathbf X$
as phase space of our process.
Let
$\overline{0, N} \triangleq \{0\} \bigcup \overline{1,N}$
($\{0\}$ is singleton containing 0).
Then, by
$\mathbb Z$
we denote the set of all processions
$(z_i)_{i\in\overline{0,N}}$,
$z_j \in \mathbb X \times \mathbf X
\; \forall j \in \overline{0,N}$.
So, elements of
$\mathbb Z$
are mappings from
$\overline{0,N}$
into
$\mathbb X \times \mathbf X$
and only they.
For
$x \in X^0$
and
$\alpha \in \mathbf A$

\begin{equation}
  \label{eq-i.3}
  \mathbb Z_\alpha[x] \triangleq
  \{(z_i)_{i \in \overline{0,N}} \in \mathbb Z
  |
  (z_0 =(x,x))
  \&
  (z_t \in \mathbb M_{\alpha(t)}
  \;\forall t \in \overline{1,N})
  \}
\end{equation}
is a nonempty finite set.
Elements of (\ref{eq-i.3})
are trajectories starting from
$(x,x)$
and coordinated with route
$\alpha$
under
$x \in X^0$,
elements of nonempty finite set

\begin{equation}
  \label{eq-i.4}
\tilde {\mathbf D}[x] \triangleq
\{
  (\alpha,\mathbf z) \in \mathbf A \times \mathbb Z
|
  \mathbf z \in \mathbb Z_\alpha[x]
\}
\end{equation}
are admissible solutions for starting point
$x$
and only they.
Moreover,
we introduce the
(nonempty finite)
set

\begin{equation}
  \label{eq-i.5}
  \mathbf D \triangleq
  \{
    (\alpha,\mathbf z, x) \in \mathbf A \times \mathbb Z \times X^0
  |
    (\alpha,\mathbf z) \in \tilde {\mathbf D}[x]
  \}
\end{equation}
of ``complete'' solutions.
Elements of (\ref{eq-i.5})
are triplets
$(\alpha,(z_t)_{t \in \overline{0,N}}, x)$,
where
$x \in X^0$
and
$(\alpha,(z_t)_{t \in \overline{0,N}}) \in \tilde {\mathbf D}[x]$.

Now, we discuss the interpretation of our constructions
for the case of sheet cutting.
Then, our megacities are realized as ``second''
discrete equidistants of real detail contours to be cut.
The relation
$\mathbb M_j$,
where$j \in \overline{1,N}$,
is the set of all OP
$z=(x,y)$,
where
$x$ is cut-in point and
$y$ is cut-off point corresponding to $x$.
Moreover,
under every detail,
for interior contour with number $i$
and exterior contour with number $j$,
the property
$(i,j) \in \mathbf K$
is realized
(of course, in $\mathbf K$
address pairs of a different nature
may be contained).

\section*{Cost functions}

Let
$\mathbb R$
be real line and
$\mathbb R_+ \triangleq \{ \xi \in \mathbb R | 0 \leqslant \xi\}$.
If $T$ is a nonempty set,
then
$\mathfrak R_+[T]$
is the set of all functions from $T$
into $\mathbb R_+$.
So,
$\mathfrak R_+[T]$
is the set of all real-valued nonnegative functions on $T$.
By
$\mathfrak N$
we denote the family of all nonempty subsets of
$\overline{1,N}$.
We fix

\begin{multline}
  \label{eq-i.6}
\mathbf c \in \mathfrak R_+[\mathbf X \times \mathbb X \times \mathfrak N], \\
\mathbf c_1 \in \mathfrak R_+[\mathbb M_1 \times \mathfrak N],
\dots ,
\mathbf c_N \in \mathfrak R_+[\mathbb M_N \times \mathfrak N], \\
f \in \mathfrak R_+[\mathbf M]
\end{multline}

The functions
$\mathbf c$
is used for estimation of the exterior movements
(movements between megacities and from $X^0$ to a megacity).
In problem connected with sheet cutting,
the values of
$\mathbf c$
can be time of the corresponding movement
(the dependence from
$K \in \mathfrak N$
can be connected with penalty for violation of dynamic constraints;
the analogous remark can be refered to functions
$\mathbf c_1, \dots, \mathbf c_N$).
The functions
$\mathbf c_1, \dots, \mathbf c_N$
are used for estimation of interior works
connected with visiting to a megacity.

In connection with employment of ordered pairs
(OP),
we note the known rule:
under
$x_1 \in X, x_2 \in X$
and
$K \in \mathfrak N$,
we suppose that
$(x_1, x_2, K) \triangleq \big( (x_1, x_2), K\big)$.
If
$j \in \overline{1,N}$,
then
$\mathbf M \subset X \times X$;
therefore, under
$\mathbf z \in mathbf M$
and
$K \in \mathfrak N$
$(z_1,z_2, K)=(\mathbf z, K)$,
where
$z_1 \in \mathfrak M_j$
and
$z_ \in \mathbf M_j$
are first and second elements of
$\mathbf z$
respectively.
As a corollary,
$\mathbf c_j(z_1, z_2, K) = \mathbf c_j(\mathbf z, K)$.
Under
$x \in X^0$,
$\alpha \in \mathbf A$
and
$(z_t)_{t \in \overline{0,N}} \in \mathbb Z_\alpha[x]$,
we suppose that

\begin{multline}
  \label{eq-i.7}
  \mathfrak C_\alpha[(z_t)_{t \in \overline{0,N}}]
  \triangleq \\
  \sum_{t=1}^N
  [
    \mathbf c(
      \mathrm{pr}_2(z_{t-1}),
      \mathrm{pr}_1(z_t),
      \{ \alpha(k) : k \in \overline{t, N} \}
      +
      \mathbf c_{\alpha(t)}(z_t, \{ \alpha(k) : k \in \overline{t,N}\})
      )
  ]
  + \\
  f(\mathrm{pr}_2(z_N)
\end{multline}

In fact,
(\ref{eq-i.7})
realizes the estimation of (\ref{eq-i.2})
under
$z_0 = (x,x)$
and
$z_t^{(1)} =(x_1^{(t)}, x_2^{(t)})
\;
\forall t \in \overline{1, N}$
For
$x \in X^0$,
we consider the next
$x$-problem

\begin{equation}
  \label{eq-i.8}
  \mathfrak C_\alpha[(z_t)_{t \in \overline{0,N}}]
  \to \min,
  (\alpha, (z_t)_{t \in \overline{0,N}}) \in \tilde{\mathbf D}[x]
\end{equation}

with extremum
$\tilde{V}[x]$
and nonempty set

\begin{equation}
  \label{eq-i.9}
  (sol)[x]
  \triangleq
  \{
    (\alpha^0, (z_t^0)_{t \in \overline{0,N}}) \in \tilde{\mathbf D}[x]
  |
    \mathfrak C_{\alpha^0}[(z_t^0)_{t \in \overline{0,N}}] = \tilde{V}[x]
  \}
\end{equation}

Now, we consider tje next complete problem:

\begin{equation}
  \label{eq-i.10}
  \mathfrak C_\alpha[(z_t)_{t \in \overline{0,N}}]
  \to \min,
  (\alpha, (z_t)_{t \in \overline{0,N}}, x) \in \mathbf D
\end{equation}

This problem is characterized by extremum
$\mathbb V \in \mathbb R_+$
and by the (nonempty) set

\begin{equation}
  \label{eq-i.11}
  \mathbf{SOL}
  \triangleq
  \{
    (\alpha^0, (z_t^0)_{t \in \overline{0,N}}, x^0) \in \mathbf D
  |
    \mathfrak C_{\alpha^0}[(z_t^0)_{t \in \overline{0,N}}] = \mathbb V
  \}
  \neq \varnothing
\end{equation}

of optimal solutions.
Our goal consists in determination of
$\mathbb V$
and some solution of the set (\ref{eq-i.11}).
Finally, we consider the problem

\begin{equation}
  \label{eq-i.12}
  \tilde{V}[x] \to \min,
  x \in X^0
\end{equation}

for which
(see (\ref{eq-i.8}) and (\ref{eq-i.10}))
$\mathbb V$
is extremum and

\begin{equation*}
  X^0_{\mathrm{opt}}
  \triangleq
  \{
    x \in X^0
  |
    \tilde{V}[x] = \mathbb V
  \}
  \neq \varnothing
\end{equation*}

is the set of optimal starting points.
Of course,
(\ref{eq-i.12})
is connected with problems
(\ref{eq-i.8}), (\ref{eq-i.10}).
Now, we note that determination of
$\mathbb V$
and
$x^0 \in X^0_{\mathrm{opt}}$
can be useful for the heuristics testing
under big dimension of our problem.

We note, that solution methods are considered in
[100 - 105],       % TODO
where along with theoretical constructions,
algorithms and results of computing experiments are reduced.
Therefore,
we restrict oneself to consideration of investigation logic.

At first,
the main theoretical method is widely understood dynamic programming
(DP).
In addition,
we use variant of DP that is a development of Bellman
construction for TSP
(see [106])   % TODO
and not use the Help-Karp variant from
[107].        % TODO
As a result,
we obtain the DP procedure universal with respect to
a starting point; see
[108, 109].   % TODO
Therefore,
we realize construction of layers of Bellman function at once
for all possible starting points from the set
$X^0$.
Of course,
at first,
we establish the Bellman equation
[102, (4.5)].  % TODO
This equation defines the extremum evolution
(we keep in mind extremums of partial routing problems).

The next step connected with constructing the position set
sufficient for realize solutions
under precedence conditions;
see
[102, Section 4].  % TODO
Namely,
we construct layers of position space:
the sets

\begin{equation}
  \label{eq-i.13}
  D_0, D_1, \dots , D_N
\end{equation}

in
[102, Section 4]  % TODO
for consequent construction of the Bellman function layers.

Under construction of these sets-layers,
precedence conditions are indirectly used.
As a result,
the used position set defined as union of all sets (\ref{eq-i.13})
is essentially smaller than all position set.

The relation
[102, (4.9)]  % TODO
shows that

\end{document}
