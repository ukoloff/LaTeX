\documentclass{article}

\usepackage[english]{babel}

\usepackage{amssymb,amsmath}

\begin{document}

We note that,
for problems of moderate dimension,
it is possible to obtain optimal solution taking into account
the essential part of real constraints
(precedence conditions,
dynamic constraints etc.)
Now,
we consider general scheme that can be used not only
under control of sheet cutting on CNC machines.

Fix nonempty sets
$X$ and $X^0$
for which
$X^0 \subset X$
(under sheet cutting,
$X$ is a rectangle on the plane).
Suppose that
$X^0$ is a finite set.
Let
$N \in \mathbb N$,
where
$\mathbb N \triangleq \{1; 2; \dots \}$
(here and below
$\triangleq$
is the equality by definition);
suppose that
$2 \leqslant N$.
Fix nonempty sets
$M_1, \dots M_N$:

\begin{equation}
  \label{eq-i.1}
  M_1 \subset X,
  \dots ,
  M_N \subset X,
\end{equation}

Using results of
[99],  % TODO
we suppose that
$M_1, \dots M_N$
are finite sets
(these sets can be obtained from
continuous equidistants by digitization
for computing modelling).
We call the sets
(\ref{eq-i.1})
megacities.
Those sets are visiting objects
(this is a significant difference from TSP).
Moreover,
for
$j \in \overline{1, N}$,
where
$\overline{1,N} \triangleq \{k \in \mathbb N|k \leqslant N\}$,
a nonempty relation
$\mathbb M_j,
\mathbb M_j \subset M_j \times M_j$
is defined.
So, we have procession
$(\mathbb M_j)_{j \in \overline{1,N}}$
of nonempty sets connected with megacities.
Suppose that
$\mathbb P$
is the set og all permutations of the set
$\overline{1,N}$.
We consider processes

\begin{align*}
  (x\in X^0) \to
  (x_1^{(1)} \in M_{\alpha(1)} \rightsquigarrow x_2^{(1)} \in M_{\alpha(1)}) \to \\
  \dots \to \\
  (x_1^{(N)} \in M_{\alpha(N)} \rightsquigarrow x_N^{(2)} \in M_{\alpha(N)})
\end{align*}

\begin{equation}
  \label{eq-i.2}
  \alpha \in \mathbb P,
  (x_1^{(1)}, x_2^{(1)}) \in \mathbb M_{\alpha(1)},
  \dots ,
  (x_1^{(N)}, x_2^{(N)}) \in \mathbb M_{\alpha(N)}
\end{equation}

The choice of
$\alpha$
restrict oneself to precedence conditions defined in terms of a set
$\mathbf K$,
$\mathbf K \subset \overline{1,N} \times \overline{1,N}$,
of so-called address pairs.
For this,
we suppose that,
for ordered pair (OP)
$Z$
of every objects
$x$ and $y$
(so,
$z=(x,y)$),
by
$\mathrm{pr}_1(z)$
and
$\mathrm{pr}_2(z)$
are denoted the first and the second elements of
$z$
respectively:
$\mathrm{pr}_1(z)=x$
and
$\mathrm{pr}_2(z)=y$.
If
$z \in \mathbf K$,
then
$\mathrm{pr}_1(z)$
is called sender
and
$\mathrm{pr}_2(z)$
is recepient of
$z$
respectively.
And sender must be visited earlier than
the corresponding recepient
(these are the precedence conditions).
Then,

\begin{equation*}
  \mathbf A \triangleq
  \{ \alpha \in \mathbb P |
  \forall t_1 \in \overline{1,N},
  \forall t_2 \in \overline{1,N},
  \big(
    (\alpha(t_1), \alpha(t_2)) \in \mathbf K
  \big)
  \Rightarrow
  \big(t_1 < t_2 \big)
\}
\end{equation*}

We suppose that
[101, Condition 2.2.1]  % TODO
holds.
Then
$\mathbf A \neq \varnothing$.
So,
admissible by precedence routes exist
(we consider a route as element of
$\mathbb P$);
in (\ref{eq-i.2}),
only routes
$\alpha \in \mathbf A$
can be used.
But, according to (\ref{eq-i.2}),
the choice of
$\alpha \in \mathbf A$
not yet determined our process.
It is required to determinate trajectories.
But, at first,
we introduce finite phase space
(instead of $X$).

Under
$j \in \overline{1, N}$
we suppose that
$
\mathfrak M_j \triangleq
\{\mathrm{pr}_1(z): z \in \mathbb M_j \}
$
and
$
\mathbf M_j \triangleq
\{ \mathrm{pr}_2(z): z \in \mathbb M_j \}
$;
of course,
we obtain two nonempty subsets of
$M_j$.
By
$\mathfrak M$
we denote the union of all sets
$\mathfrak M_j$,
$j \in \overline{1,N}$.
Analogously,
by
$\mathbf M$
we denote the union of all sets
$\mathbf M_j$,
$j \in \overline{1,N}$.
Let
$\mathbb X \triangleq
X^0 \bigcup \mathfrak M$
and
$
\mathbf X \triangleq
X^0 \bigcup \mathbf M
$;
we consider
$\mathbb X \times \mathbf X$
as phase space of our process.
Let
$\overline{0, N} \triangleq \{0\} \bigcup \overline{1,N}$
($\{0\}$ is singleton containing 0).
Then, by
$\mathbb Z$
we denote the set of all processions
$(z_i)_{i\in\overline{0,N}}$,
$z_j \in \mathbb X \times \mathbf X
\; \forall j \in \overline{0,N}$.
So, elements of
$\mathbb Z$
are mappings from
$\overline{0,N}$
into
$\mathbb X \times \mathbf X$
and only they.
For
$x \in X^0$
and
$\alpha \in \mathbf A$

\begin{equation}
  \label{eq-i.3}
  \mathbb Z_\alpha[x] \triangleq
  \{(z_i)_{i \in \overline{0,N}} \in \mathbb Z
  |
  (z_0 =(x,x))
  \&
  (z_t \in \mathbb M_{\alpha(t)}
  \;\forall t \in \overline{1,N})
  \}
\end{equation}
is a nonempty finite set.
Elements of (\ref{eq-i.3})
are trajectories starting from
$(x,x)$
and coordinated with route
$\alpha$
under
$x \in X^0$,
elements of nonempty finite set

\begin{equation}
  \label{eq-i.4}
\tilde {\mathbf D}[x] \triangleq
\{
  (\alpha,\mathbf z) \in \mathbf A \times \mathbb Z
|
  \mathbf z \in \mathbb Z_\alpha[x]
\}
\end{equation}
are admissible solutions for starting point
$x$
and only they.
Moreover,
we introduce the
(nonempty finite)
set

\begin{equation}
  \label{eq-i.5}
  \mathbf D \triangleq
  \{
    (\alpha,\mathbf z, x) \in \mathbf A \times \mathbb Z \times X^0
  |
    (\alpha,\mathbf z) \in \tilde {\mathbf D}[x]
  \}
\end{equation}
of ``complete'' solutions.
Elements of (\ref{eq-i.5})
are triplets
$(\alpha,(z_t)_{t \in \overline{0,N}}, x)$,
where
$x \in X^0$
and
$(\alpha,(z_t)_{t \in \overline{0,N}}) \in \tilde {\mathbf D}[x]$.

Now, we discuss the interpretation of our constructions
for the case of sheet cutting.
Then, our megacities are realized as ``second''
discrete equidistants of real detail contours to be cut.
The relation
$\mathbb M_j$,
where$j \in \overline{1,N}$,
is the set of all OP
$z=(x,y)$,
where
$x$ is cut-in point and
$y$ is cut-off point corresponding to $x$.
Moreover,
under every detail,
for interior contour with number $i$
and exterior contour with number $j$,
the property
$(i,j) \in \mathbf K$
is realized
(of course, in $\mathbf K$
address pairs of a different nature
may be contained).

\section*{Cost functions}

Let
$\mathbb R$
be real line and
$\mathbb R_+ \triangleq \{ \xi \in \mathbb R | 0 \leqslant \xi\}$.
If $T$ is a nonempty set,
then
$\mathfrak R_+[T]$
is the set of all functions from $T$
into $\mathbb R_+$.
So,


\end{document}
