\documentclass{article}

\usepackage[english]{babel}

\usepackage{amssymb,amsmath}

\begin{document}

We note that,
for problems of moderate dimension,
it is possible to obtain optimal solution taking into account
the essential part of real constraints
(precedence conditions,
dynamic constraints etc.)
Now,
we consider general scheme that can be used not only
under control of sheet cutting on CNC machines.

Fix nonempty sets
$X$ and $X^0$
for which
$X^0 \subset X$
(under sheet cutting,
$X$ is a rectangle on the plane).
Suppose that
$X^0$ is a finite set.
Let
$N \in \mathbb N$,
where
$\mathbb N \triangleq \{1; 2; \dots \}$
(here and below
$\triangleq$
is the equality by definition);
suppose that
$2 \leqslant N$.
Fin nonempty sets
$M_1, \dots M_N$:

\begin{equation}
  \label{eq-i.1}
  M_1 \subset X,
  \dots ,
  M_N \subset X,
\end{equation}

Using results of
[99],  % TODO
we suppose that
$M_1, \dots M_N$
are finite sets
(these sets can be obtained from
continuous equidistants by digitization
for computing modelling).
We call the sets
(\ref{eq-i.1})
megacities.
Those sets are visiting objects
(this is a significant difference from TSP).
Moreover,
for
$j \in \overline{1, N}$,
where
$\overline{1,N} \triangleq \{k \in \mathbb N|k \leqslant N\}$,
a nonempty relation
$\mathbb M_j,
\mathbb M_j \subset M_j \times M_j$
is defined.
So, we have procession
$(\mathbb M_j)_{j \in \overline{1,N}}$
of nonempty sets connected with megacities.
Suppose that
$\mathbb P$
is the set og all permutations of the set
$\overline{1,N}$.
We consider processes

\begin{align*}
  (x\in X^0) \to
  (x_1^{(1)} \in M_{\alpha(1)} \rightsquigarrow x_2^{(1)} \in M_{\alpha(1)}) \to \\
  \dots \to \\
  (x_1^{(N)} \in M_{\alpha(N)} \rightsquigarrow x_N^{(2)} \in M_{\alpha(N)})
\end{align*}

\begin{equation}
  \label{eq-i.2}
  \alpha \in \mathbb P,
  (x_1^{(1)}, x_2^{(1)}) \in \mathbb M_{\alpha(1)},
  \dots ,
  (x_1^{(N)}, x_2^{(N)}) \in \mathbb M_{\alpha(N)}
\end{equation}

The choice of
$\alpha$
restrict oneself to precedence conditions defined in terms of a set
$\mathbf K$,
$\mathbf K \subset \overline{1,N} \times \overline{1,N}$,
of so-called address pairs.
For this,
we suppose that,
for ordered pair (OP)
$Z$
of every objects
$x$ and $y$
(so,
$z=(x,y)$),
by


\end{document}
