\documentclass{../download/tPRS2e}

    \usepackage[english]{babel}
    \usepackage{tikz}
    \usetikzlibrary{patterns}
    \usepackage{graphicx}
    \graphicspath{{media/}}
    
\begin{document}

\title{On the Problem of Tool Air Move Minimization for Sheet Cutting CNC Machine}

\author{
\name{
Petunin A.A.\textsuperscript{a},
Polishuk E.G.\textsuperscript{a},
Ukolov S.S.\textsuperscript{a}$^{\ast}$\thanks{$^\ast$Corresponding author. Email: s.s.ukolov@urfu.ru}
}
\affil{
\textsuperscript{a}Ural Federal University, Yekaterinburg, Russia;
}}

\maketitle

\begin{abstract}
The problem of finding pierce points to get
minimal tool air move path length is considered
in case of standard cutting technique
(closed contours cutting).
Discrete approximation is discussed,
then a new algorigthm is described,
which perform no discretization,
ie. any point on a contour can be 
used for piercing operation.
\end{abstract}

\begin{keywords}
    contour;
    pierce point;
    minimal tool path;
\end{keywords}

\section{Introduction}
    
\end{document}
