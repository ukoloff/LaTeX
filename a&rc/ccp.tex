\documentclass[12pt]{a&t}

\usepackage{cmap}
\usepackage{graphicx}
\usepackage{amssymb,amsmath}
\usepackage{subfig}
% \usepackage{tikz}
% \usetikzlibrary{patterns,intersections}

\begin{document}

\year{2021}
\title{
  Новый алгоритм построения кратчайшего пути обхода
  конечного множества непересекающихся контуров на плоскости
}%
\thanks{
  Работа выполнена при финансовой поддержке
  Российского фонда фундаментальных исследований
  (грант № 20-08-00873)
}

\authors{
  А.А.~Петунин, д-р~техн.~наук\\
  (Институт математики и механики им. Н.Н.Красовского УрО РАН),\\
  Е.Г.~Полищук, канд.~физ.-мат.~наук,\\
  С.С.~Уколов\\
  (Уральский Федеральный университет
  имени первого Президента России Б.Н. Ельцина)
}

\maketitle

\begin{abstract}
Рассматривается проблема маршрутизации режущего
инструмента машин термической резки с ЧПУ
когда точки врезки расположены
на границах деталей,
ограниченных отрезками прямых и дугами окружностей
с использованием техники непрерывной резки
(CCP),
то есть каждый контур вырезается целиком.
Общая задача минимизации длины маршрута
в этом случае сводится к минимизации длины холостого хода.
Показано, что она эквивалентна поиску
кратчайшей ломаной с вершинами,
расположенными на контурах.
Представлен алгоритм построения
такой ломаной для заданного порядка
обхода контуров,
доставляющий локальный минимум
а также предложены достаточные условия
глобального минимума.
Предложен эвристический алгоритм
маршрутизации на основе
метода переменных окрестностей
(VNS).
Рассмотрены результаты численных экспериментов
в сравнении с точным решением задачи GTSP.
\end{abstract}

\section{Введение}

\AdditionalInformation{Петунин А.А.}{Уральский Федеральный университет
имени первого Президента России Б.Н. Ельцина,
профессор}{a.a.petunin@urfu.ru}

\AdditionalInformation{Полищук Е.Г.}{Уральский Федеральный университет
имени первого Президента России Б.Н. Ельцина,
доцент}{e.g.polishchuk@urfu.ru}

\AdditionalInformation{Уколов С.С.}{Уральский Федеральный университет
имени первого Президента России Б.Н. Ельцина,
старший преподаватель}{s.s.ukolov@urfu.ru}

\end{document}
